\section{Die \textit{Josephi}{}-Messe}

Die \textit{Josephi}{}-Messe ist ein Spätwerk Högns. Der genaue
Zeitraum, wann das Werk begonnen und wann es vollendet wurde, ist nicht
bekannt. Doch wissen wir, dass das nach der Messe geschriebene
\textit{Marienlied Nr. 12 F-Dur op. 63} 1953 fertiggestellt wurde.
Wahrscheinlich arbeitete er beim Komponieren der Messe gleichzeitig
auch an einem oder mehreren Geschichtswerken.

Namensgeber der Messe war Högns Hausarzt Dr. Joseph Stern. Als Högn die
Messe fertig hatte, bot er Stern mit Verweis auf ihre gute Freundschaft
an, der neuen Messe einen Namen zu geben, da er zwar eine Messe
komponiert hatte, aber keinen Namen für sie hatte. Stern machte den
Vorschlag, die Messe nach seinem Namenspatron zu nennen. Möglicherweise
beeinflusste der Liebhaber alter Sprachen Högn auch darin, den Titel
der Josephi-Messe lateinisch zu formulieren, sie also \textit{Missa in
honorem Sancti Josephi} zu nennen.

Eine Reihe von Dokumenten belegt eine rege Aufführungspraxis über
Jahrzehnte hinweg. Die erste uns bekannte Aufführung fand am 14. Juni
1953 zur Installation des neuen Pfarrers Franz Seraph Reicheneder
statt. Ein Zeitungsartikel bescheinigte dem
\zitat{klangvollen Chor} unter der Leitung des Komponisten
den \zitat{ergreifenden} Vortrag der \zitat{Missa
St. Josephi.}

Wie es zu der Aufführung der Messe in der Kirche St. Magdalena in
Plattling am 19. März 1957 kam, ist nicht bekannt. In einer
Ankündigung am Tag der Aufführung in der Plattlinger Zeitung wird Högns
\zitat{romantisch, liebenswerter Kompositionsstil} gelobt.
Weiter heißt es: \zitat{Das Benediktus für Sopransolo und
Chor wird sicher allen gefallen, die die Musik nicht erst über den
Verstand, sondern gleich ins Herz fließen lassen wollen.} Auf einer
Postkarte bedankte sich der Chorleiter Gustl Gudmer bei Högn für das
Ausleihen des Notenmaterials und versicherte ihm: \zitat{Sie
hat am Josephsfest gut gefallen. Ich habe verschiedene Stimmen darüber
gehört. Unser} \zitat{geistlicher Rat hier hat sich sehr
gefreut über die Aufführung.}

Dem Engagement von Högns ehemaliger Chorsängerin Maria Schröck ist es zu
verdanken, dass die Messe am 19. März 1959 in Deggendorf vom Chor an
St. Martin aufgeführt wurde. Maria Schröck, die 1950 nach Deggendorf
geheiratet hatte und seitdem Mitglied im Kirchenchor von St. Martin
war, machte ihrem Chorregenten Fritz Goller, dem Neffen von Vinzenz
Goller, den Vorschlag, eine Messe von August Högn einzustudieren.
Goller ging auf diesen Vorschlag anfangs nicht ein. Erst als Maria
Schröck ihm drohte, eine Aufführung der Messe in der Pfarrkirche Mariä
Himmelfahrt anzustreben, ging Goller auf ihr Anliegen ein.

Zur „Josephi“-Messe äußerte sich Fritz Goller in einem Zeitungsartikel,
der einen Tag vor der Aufführung erschien: \zitat{}Die
Messe \zitat{verrät gediegene Handwerkskunst, die sich in
einer sauberen satztechnischen Handschrift äußert, Sinn für
harmonische Farbigkeit hat, ohne in spätromantische Chromatik
abzugleiten, und sich der Mittel des Kontrastes in der
Gegenüberstellung kraftvoller Unisoni und motivisch aufgelockerter
Chorsätze bedient.} August Högn war bei der Aufführung in Deggendorf
anwesend. Man gratulierte ihm zu seiner Komposition.

Beim Pfarrerwechsel in Ruhmannsfelden 1974 kam die
\textit{„Josephi“-Messe} zweimal zum Einsatz. Sie erklang unter der
Leitung von Karl Geiger zur Verabschiedung des alten Pfarrers
Reicheneder am 4. August 1974 und zur Installation des neuen Pfarrers
Otto Krottenthaler am 29. Sepember 1974.

In dieser Messe ist Högn ein besonderes Kunststück gelungen: Die
gesamte Komposition baut auf ein einziges Motiv aus vier Tönen auf.
Dieses sogenannte Soggetto ist in allen Sätzen der Messe außer im
Benedictus zu finden. Die Messe klingt deshalb keinesfalls
gleichförmig, denn es gelang Högn, jedem Satz einen unverwechselbaren
musikalischen Ausdruck zu verleihen, obwohl er immer auf dieses eine
Soggetto zurückgriff. So klingt es im Kyrie ruhig und fließend, im
Sanctus feierlich und erhaben. Fast nebenbei und sich nicht in den
Vordergrund spielend wird es im Gloria und Credo in eine Vielzahl von
anderen musikalischen Einfällen ungezwungen eingebaut. Melancholisch
und beinahe düster lässt es dagegen die Moll-Eintrübung zu Beginn des
Agnus Dei erscheinen.

Viele verschiedene stilistische Ebenen lassen sich in der
\textit{„Josephi“-Messe} beobachten. Man gewinnt fast den Eindruck,
Högn wollte mit dieser Messe am Ende sein bisheriges Werk noch einmal
in einer Komposition zusammenfassen. Die umfangreiche imitatorische
Arbeit im Kyrie verweist auf sein Frühwerk im Stil des strengen
Cäcilianismus. Die harmonisch farbige und chromatisch angehauchte
mittlere Kompositionsphase schimmert an vielen chromatischen
Stimmführungen besonders im Kyrie und Agnus Dei durch. Hier lässt sich
Högns Vorbild Peter Griesbacher, der ein großer Wagner-Verehrer war,
erkennen. Manche Passage des Credo, der Anfang des \textit{Pleni sunt
coeli} im Sanctus und das gesamte Benedictus sind stark von der
Volksmusik beeinflusst, die Högn sein ganz Leben lang beschäftigt
hat. Die Messe ist Högns reifste Komposition, die einen Vergleich mit
Werken bedeutender Komponisten nicht zu scheuen braucht.

\begin{figure}
\begin{subfigure}[b]{0.5\linewidth}
\centering
\img[height=7cm]{August-Hoegn2}
\caption{August Högn}
\end{subfigure}
\begin{subfigure}[b]{0.5\linewidth}
\centering
\img[height=7cm]{Mathilde-Glasschroeder}
\caption{Mathilde Glasschröder}
\end{subfigure}
\end{figure}

\begin{figure}
\begin{subfigure}[b]{0.5\linewidth}
\centering
\img[height=6cm]{Barbara-Essigmann}
\caption{Barbara Essigmann}
\end{subfigure}
\begin{subfigure}[b]{0.5\linewidth}
\centering
\img[height=6cm]{Theres-Raster}
\caption{Theres Raster}
\end{subfigure}
\end{figure}
