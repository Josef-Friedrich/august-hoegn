\section{Chorregent in schwieriger Zeit}

Am 25. Januar 1940 starb der erst 36 Jahre alte Chorregent und
Gemeindesekretär Albert Schroll, der zehn Jahre lang in
Ruhmannsfelden den Kirchenchor geleitetet hatte. Er erkrankte schon
1939, so dass der Lehrer Ertl für Schroll als Klavierbegleiter bei
einem Singspiel einspringen musste. Högn vertrat Schroll als
Chorregent und Organist während seiner Krankheit und nach dessen Tod
übernahm er seine Stelle aushilfsweise, wie den Chorabrechnungen zu
entnehmen ist. Der 61-jährige und daher nicht mehr wehrfähige Högn
blieb aber Chorregent und Organist während des gesamten Zweiten
Weltkriegs, da aufgrund der Kriegssituation kein hauptamtlicher
Kirchenmusiker als Nachfolger für Schroll verfügbar war. Zu Beginn von
Högns dritter Chorregentenzeit kehrte man zur gewohnten Praxis
zurück, bei Beerdigungen den Unterricht ausfallen zu lassen. Diese in
der Weimarer Republik geduldete Regelung wurde von den Behörden des
religionsfeindlichen NS-Regimes schließlich ab 1944 unterbunden.
Josef Brunner – er war zu jung für den Kriegseinsatz – und eine
Mallersdorfer Schwester, die an der Kinderbewahranstalt arbeitete,
übernahmen deshalb den Organistendienst bei Beerdigungen an Stelle von
Högn.

Der Beginn des Krieges stellte auch für die Streichorchestertradition in
Ruhmannsfelden einen scharfen Einschnitt dar. Schon 1939 wurden die
Lehrer und Orchestermitglieder Gruber, Schultz, Friedrich und
Kestlmeier eingezogen. Einheimische Musiker des Orchesters, wie
Lorenz Schlagintweit, blieben von einem Kriegseinsatz ebenso wenig
verschont. Ein weiterer Grund, der zum Niedergang des Streichorchesters
beitrug, waren die radikalen Einschränkungen in der Lehrerausbildung
durch die Reform im Jahr 1935. Große Aufführungen mit Orchester an
Festtagen, wie von mehreren Zeitzeugen berichtet, dürften daher eher
vor Kriegsbeginn, also zur Zeit von Chorregent Schroll, stattgefunden
haben. Als Ersatz für das Streichorchester etablierte sich eine
Blechbläserformation, die sich aus Musikern der umliegenden
Blaskapellen zusammensetzte.

Die Männerstimmen des Kirchenchores waren den Umständen entsprechend
mit nur zwei Sängern, nämlich Schwannberger und Holzfurtner, dünn
besetzt. Auch von der Ruhmannsfeldener Orgel forderte der Krieg seinen
Tribut. Angesichts des geringen Metallgewinns, den die am 12. Juli 1944
zum Ausbau bestimmten dünnwandigen Pfeifen und Leitungen erbrachten,
wird einerseits die aussichtslose Lage der Kriegswirtschaft,
andererseits die bewusste Schikane der Nazis gegenüber der Kirche
deutlich. Die Orgel – ihr fehlten die Register Quintatön, Vox coelestis
des ersten und alle metallischen Pfeifen einschließlich der
Windleitungen des zweiten Manuals – erlitt in der Nachkriegszeit
durch eine lange Trockenheit zusätzlich großen Schaden, ehe die Orgel
1947 vom Orgelbaumeister Kratochwill aus Plattling restauriert werden
konnte.

Angesichts der vielen Toten, die der Zweite Weltkrieg forderte,
verwundert es nicht, dass Högn in dieser Zeit vor allem
Kompositionen, die bei Beerdigungen eingesetzten werden konnten,
schrieb. So entstanden in dieser Zeit drei der insgesamt vier
Grablieder Högns, nämlich das \textit{Grablied Nr. 1 Es-Dur op. 35},
das \textit{Grablied Nr. 2 Es-Dur} und das \textit{Grablied Nr. 3
Es-Dur op. 44}. Das erste Grablied komponierte Högn auf einen Text, der
explizit auf den Tod eines Soldaten Bezug nimmt. Erst in einer späteren
Version erhielt das \textit{Grablied Nr. 1} einen Text, der bei jedem
Todesfall passte. Weitere Stücke aus dieser Zeit, die bei Trauerfeiern
aufgeführt werden konnten, waren das \textit{Libera e-moll op. 50} und
der \textit{Weihegesang Es-Dur} mit nationalsozialistischem Text.
Ebenso entstanden zu der Zeit zwei Marienlieder, nämlich das
\textit{Marienlied Nr. 6 F-Dur op. 41} und das \textit{Marienlied Nr. 7
G-Dur op. 45}, zwei Tantum ergo, nämlich das \textit{Tantum ergo Nr. 4
A-Dur op. 47} und das \textit{Tantum ergo Nr. 3 Es-Dur} \textit{op. 49}
und mit dem \textit{Pange lingua F-Dur op. 43} und dem \textit{Pange
lingua Es-Dur op. 46} zwei Pange lingua. Auch das \textit{Kommunionlied
C-Dur op. 37 b}, das \textit{Lied von Gotteszell G-Dur op}.
\textit{42}, das \textit{Offertorium G-Dur op. 48} und das
\textit{Benedictus G-Dur op. 50} sind Stücke, die während des Kriegs
komponiert wurden.

Ob Högns Probenpraxis – die Proben fanden in seiner Wohnung statt – eine
Anpassung an den durch den Krieg geschmälerten Chor war oder ob der
Probenort schon in seiner ersten und zweiten Chorregentenzeit die
Wohnung war, kann aus Mangel an Zeitzeugen der früheren
Chorregentenzeiten nicht geklärt werden. Die Probenarbeit in seiner
Dienstwohnung hatte für Högn einige praktische Vorteile: Hätte der Chor
in der Kirche geprobt, wäre es im Winter sehr kalt gewesen, da auf der
Orgelempore ein Fenster undicht war. Einen beheizten Proberaum konnte
die Pfarrgemeinde damals nicht zu Verfügung stellen, so dass Högns
große Lehrerwohnung eine willkommene Alternative zur Kirche war.
Außerdem besaß Högn keinen Kirchenschlüssel, er hätte also dem Mesner
Bescheid geben müssen, damit die Kirche abends offen blieb. Nach der
Probe hätte Högn den Schlüssel im Pfarrhof zurückgeben müssen, was
relativ umständlich gewesen wäre. Einige Zeitzeugen behaupten, dass
nicht nur eine einzige Stimmgruppe oder der komplette Chor in der
Wohnung probte, sondern auch das Blechbläserquartett. Nur die
Generalprobe fand in der Kirche statt. Die Proben wurden auch nicht
regelmäßig abgehalten, beispielsweise jede Woche an einem bestimmten
Wochentag, sondern fanden nur dann statt, wenn ein neues Repertoire
für große Festtage erarbeitet werden musste. Eine regelmäßige
Probenarbeit machte anscheinend wenig Sinn, wenn man sich den täglichen
Einsatz der Hauptsänger und -sängerinnen in den Gottesdiensten
vergegenwärtigt. Durch die vielen Aufführungen hatte der Chor
außerdem Routine.

Vielleicht ist es aber auch auf die Proben in der privaten Wohnung
zurückzuführen, dass die Kirchenmusik in Ruhmannsfelden dem
deutschlandweiten Kulturaufschwung nach Ende des Krieges nicht
folgte, sondern eher einen Niedergang erlebte. Die in Högns Wohnung
abgehaltenen Proben schienen zu einer Privatveranstaltung verkommen zu
sein. Nur für die Hauptsängerinnen hielt Högn Proben. Die Sängerin
Maria Freisinger hat jahrelang sonntags an den Gottesdiensten
mitgewirkt, doch an Proben konnte sie sich nicht erinnern. Der
öffentliche Chor wurde zu einem nach außen abgeschlossenen Zirkel von
eng befreundeten Hauptsängerinnen, zu dem nur schwer neue Mitglieder
stoßen konnten.

Dass der Zustrom von neuen Sängerinnen und Sängern ausblieb, lag sicher
auch am Verhalten der Hauptsängerinnen. Sie hatten aus einem bestimmten
Grund kein Interesse an einem größeren Chor: Geld. Denn je mehr
Personen im Chor mitgesungen hätten, desto öfter wäre die finanzielle
Entschädigung, die dem Hauptchor zustand, aufzuteilen gewesen. Diese
war etwa genau so hoch wie die des Chorregenten. Das „Gehalt“ der
einzelnen Sängerinnen wäre somit immer kleiner ausgefallen. Aus
dieser Sichtweise ist das Verhalten der Sängerin Theres Raster
gegenüber neuen Chormitgliedern durchaus nachvollziehbar. Raster war
für den Notenschrank zuständig und bestimmte sogar öfter als Högn,
welches Stück gesungen werden sollte. Waren zu wenige Singstimmen
vorhanden, und das war bei den meist handgeschriebenen
Notenmaterialien oft der Fall, gab sie insbesondere den jungen
Sängerinnen keine Noten. Neue Chormitglieder fühlten sich
logischerweise wenig akzeptiert, wenn sie wiederholt kein Notenblatt
bekamen, noch dazu, wenn ihnen nicht klargemacht wurde, warum sie keine
Noten bekommen hatten.

Der Hauptgrund, weshalb der Chor der Nachkriegszeit so wenige
Mitglieder zählte, ist aber in der Person von August Högn zu sehen.
Mit 70 Jahren hatte er nicht mehr den Elan, neue Sänger zu integrieren
oder sich dem intriganten Verhalten der Hauptsängerinnen entgegen zu
stellen. Vielleicht hat sich nicht nur sein hohes Alter, sondern auch
die Tatsache, dass er zeitlebens den Chorregentendienst nur
provisorisch oder aushilfsweise, paradoxerweise aber fast zwanzig Jahre
lang ausübte, wie in jedem Protokoll ausdrücklich erwähnt wird, und
immer nur als \textit{Notnagel} angesehen wurde, auf seine
Einsatzbereitschaft negativ ausgewirkt. Seinen drei großen
heimatkundlichen Abhandlungen, die alle in der Nachkriegszeit
entstanden sind, ist zu entnehmen, dass Högn ab 1945 sein
Hauptbetätigungsfeld nicht mehr in der Kirchenmusik sah.

War der Einsatz des Blechbläserquartetts zu Beginn des Weltkrieges noch
aus der Not geboren, so machen Högns Kompositionen, die in der
Nachkriegszeit entstanden sind, und die vielen Arrangements, die er
für Bläserquartett gesetzt hat, den Eindruck, dass Högn Gefallen an
dem Quartett gefunden hatte und gar nicht mehr daran dachte, an die
Streichorchestertradition anzuknüpfen. Die \textit{Blaskapelle
Ruhmannsfelden} war ein Ensemble, in dem sich kurz nach dem Krieg
ehemalige Militärmusiker, vertriebene Profimusiker und gute Amateure
zu einer hervorragenden und durch zahlreiche Rundfunkaufnahmen
ausgezeichneten Bläserformation zusammenfanden. Aus diesem Ensemble
ließ sich für den Einsatz in der Kirchenmusik ein höchst
leistungsfähiges Quartett ausgliedern, dessen Qualität das
Streichorchester auch zu seinen besten Zeiten bei weitem nicht erreicht
hätte.

Darüber hinaus war der Aufwand, Stimmen für ein Bläserquartett zu
schreiben, wesentlich geringer, als Notenmaterial für ein
Streichquintett anzufertigen. Im Gegensatz zum Streichorchester, das
nur bei „a-capella“-Stellen pausierte, beschränkte sich die Beteiligung
des Bläserquartetts auf kürzere und sehr effektvolle Einwürfe. Das
„colla-parte“-Spiel der Blechblasinstrumente deutete Högn in gedruckten
Partituren, die er arrangierte, lediglich durch eine Einrahmung der
entsprechenden Stellen mit Balken an. Da Högn gewöhnlich die Stimmen
für Streicher im Gegensatz zu den Blechbläsern eigenständig und nicht
„colla-parte“ mit den Singstimmen führte, wäre bei einem Stück mit
Streichorchesterbegleitung das Anfertigen einer neuen Partitur
unvermeidbar gewesen.

Anlässlich einer Firmung nach dem Zweiten Weltkrieg sind Högns
Kompositionen \textit{Ecce sacerdos F-Dur op. 57} und \textit{Juravit
Dominus B-Dur op.} \textit{58} entstanden. Zu jeder der zwei seltenen
geistlichen Kompositionen ließ sich im Noten-Archiv in Ruhmannsfelden
genau ein gedrucktes Werk mit derselben liturgischen Bestimmung finden,
nämlich das \textit{Ecce sacerdos As-Dur op. 12} von Richard Kempf und
das \textit{Juravit Dominus op. 19} von G. M. Alt.

Allein schon der Vergleich des Bassverlaufs zu Beginn des \textit{Ecce
sacerdos} beider Komponisten macht deutlich, dass Högn offenbar
stellenweise von Kempf „abgekupfert“ hat.

Auch beim \textit{Juravit Dominus B-Dur op. 58} zog Högn ein
gleichnamiges Stück eines anderen Komponisten als Quelle heran, nämlich
die Komposition von G. M. Alt. Er hat das Titelblatt wortwörtlich bei
Alt abgeschrieben und nur seinen Namen anstelle des Namens von Alt
gesetzt. Musikalisch geht Högn im Gegensatz zum \textit{Ecce sacerdos}
in seinem \textit{Juravit Dominus} einen ganz eigenständigen Weg.

Da die gedruckten Bezugskompositionen, nach dem Zustand des
Notenpapiers zu urteilen, kein einziges Mal gesungen wurden, liegt
der Grund, weshalb Högn eigene Werke zur Firmung schrieb, auf der
Hand: Die gekauften Werke waren für die Ruhmannsfeldener Verhältnisse
anscheinend nicht geeignet.

Mit der anspruchsvollen Komposition von Kempf wäre der Kirchenchor
sicher an seine Grenzen gestoßen. Dies verhinderte Högn, indem er eine
eigene, leichtere Version des \textit{Ecce sacerdos} anfertigte. Alts
Stück ist im Gegensatz dazu eine sehr einfache, aber auch
einfallslose Komposition, an der Högn offenbar wenig Gefallen gefunden
hat. Arrangieren hätte er die Kompositionen von Kempf und Alt sowieso
müssen, da das Blechbläserquartett bei so großen Anlässen nicht fehlen
durfte. Also lag die Entscheidung gleich „maßgeschneiderte“ eigene
Werke für die Firmung zu schreiben, eigentlich auf der Hand.

In gewisser zeitlicher Nähe zu den zwei oben erwähnten Werken dürften
auch das \textit{Pange lingua Es-Dur op. 51}, die
\textit{Fronleichnams-Prozessionsgesänge} \textit{Es-Dur op. 52}, das
\textit{Marienlied Nr. 8 G-Dur op. 54}, das \textit{Marienlied Nr. 10
F-Dur op. 56}, das \textit{Marienlied Nr. 11 F-Dur op. 59}, das
\textit{Pange lingua G-Dur} mit deutschem Text und das \textit{Ehre sei
Gott C-Dur} entstanden sein. Eine wichtige Komposition aus dieser Zeit
ist leider verloren gegangen: Die \textit{Herz-Jesu-Litanei}. Högn
versuchte 1947 die Litanei drucken zu lassen. Sowohl der
Sebaldus-Verlag als auch der Gregorius-Verlag lehnten aber mit Verweis
auf die \textit{gegenwärtige schwierige Lage auf dem Papiermarkt} ab.