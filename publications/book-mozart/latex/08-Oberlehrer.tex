\section{Ernennung zum Oberlehrer}

Am 1. November 1929 wurde August Högn vom Hauptlehrer zum Oberlehrer
befördert. Diese neue Amtsbezeichnung klingt weit weniger spektakulär
als der Titel „Rektor“, den Högn ab dem 1. April 1940 führte. Die
eigentliche Beförderung jedoch war die von 1929. 1940 fand eher eine
Umbenennung des Dienstgrades statt, ohne Auswirkung auf die
Gehaltsbezüge. Mit der Ernennung zum Oberlehrer wurde Högn in die
Besoldungsgruppe A 4 a befördert und sein jährliches Grundgehalt stieg
auf 5800 Reichsmark. Sein Aufgabenfeld als Schulleiter dürfte sich
seit 1. Oktober 1921, als er die Nachfolge des Bezirksoberlehrers Auer
antrat – Högn wurde dafür schon am 1. April 1920 zum Hauptlehrer
ernannt – also weder 1929 noch 1940 wesentlich verändert haben. Ein
weiteres Zeichen, dass die Beförderung zum Oberlehrer im Jahr 1929 eine
größere Tragweite hatte, als die anderen Beförderungen, war der
Besuch des Bezirksschulrats Jungwirth in Högns Klasse am 5. Oktober
1929, also knapp einen Monat vor der Ernennung. Der Bericht über diese
Schulbesichtigung des Schulrats gewährt uns einen kleinen Einblick in
Högns Unterricht.

\begin{figure}
\img{Klassenfoto-1932}
\caption{Klassenfoto von 1932 mit August Högn (links) und Pfarrer
Fahrmeier (rechts)}
\end{figure}

Im Sachunterricht stand Heimatgeschichte auf dem Programm. Zur
Vorbereitung auf den durchzunehmenden Pfingstritt in Kötzting ließ
Högn seine 37 anwesenden Schüler den schwarzen und weißen Regen bis zur
Mündung bei Kötzting gedanklich durchwandern. Die anschließende
Stoffdarbietung über den Pfingstritt befand der Schulrat als gut,
bemängelte jedoch, dass Högn die Selbsttätigkeit der Schüler durch
häufige Zwischenfragen hemmte. In der darauf folgenden Deutschstunde
lasen die Schüler nach dem Urteil des Sachverständigen mit wenigen
Ausnahmen das Kapitel Nr. 154 \textit{Der furchtsame Hase} aus dem
Lesebuch gut. Als Anschluss an das Lesestück wurde ein Aufsatz
vorbereitet. In der Rechenstunde erklärte Högn die dezimale
Schreibweise und den dezimalen Rest anschaulich und übte seine
praktische Umsetzung mit seinen Schülern.

So kam der Schulrat über Högns Unterrichtsstil zu folgendem Urteil:
\textit{Der Stand des Unterrichts entspricht, trotz der vielen sehr
schwachbegabten Schüler. Die Schulzucht ist sehr gut. Der Fleiß des
Hauptlehrers Högn verdient Lob. Högn bereitet sich gewissenhaft vor;
er ist stoffsicher. Er verfährt eingehend, anschaulich und gründlich
und sonst mit großem Fleiß, auch die Schwachen mitzubringen. Högn muss
jedoch noch mehr Gewicht auf die Selbsttätigkeit der Schüler legen.}

Er erteilte Högn die Note 2 für Fleiß, die Note 2 für die
Lehrbefähigung, die Note 3 für den Unterrichtserfolg und die Note 2 für
die erzieherische Wirksamkeit, was in der Summe die Note 2 ergab. Im
Protokoll wurde auch lobend erwähnt, dass sich Högn an der
Erforschung der Heimatgeschichte beteiligt. Högn zeigte also bestimmt
keine Glanzleistung, doch zur Beförderung reichte es.

Ob nun als Hauptlehrer, Oberlehrer oder Rektor, es war sicher zu keiner
Zeit eine leichte Aufgabe für Högn, die Volksschule zu leiten. Zum
einen war die finanzielle Lage der beteiligten Gemeinden, die für die
Schulgebäude und die Lehrmittel zuständig waren, immer angespannt.
Nicht einmal die notwendigsten Lehrmittel konnten zu manchen Zeiten
im erforderlichen Maße bereitgestellt werden. Die finanziellen
Engpässe wirkten sich natürlich auch auf den Zustand der Schulgebäude
aus. So mussten beispielsweise im Jahr 1934 die Toiletten im so
genannten Mädchenschulhaus eine Zeit lang gesperrt werden, weil sie,
laut Högn, derart baufällig waren, dass ihre Benutzung für die Kinder
lebensgefährlich gewesen wäre.

Zum anderen schlug sich die schwierige Erwerbslage vieler Familien
negativ auf das Lernklima nieder. Besonders zur Erntezeit mussten
Kinder in der Landwirtschaft ihrer Eltern mithelfen, da sich kaum eine
Familie zusätzliche Arbeitskräfte leisten konnte. Aus diesem Grund
wurde in den Sommermonaten zeitweise der Unterricht gekürzt.

Eine dauerhafte Belastung für den Schulbetrieb stellten die langen
Schulwege dar, die die Schüler aus weit entfernten Ortschaften und
Höfen zu Fuß zurücklegen mussten. Über eine Stunde dauernde Märsche zur
Schule waren keine Seltenheit und wurden bei Regen oder Schnee für die
Schüler zur Tortur.

Wahrscheinlich lag es an den damals geringeren Anforderungen, die man an
eine Landschule stellte – überspitzt formuliert, war man schon froh,
wenn die Schüler regelmäßig am Unterricht teilnahmen – und an der
Position Högns als Schulleiter, dass er sich gewisse Dinge erlauben
konnte, die nicht regelkonform waren und in der heutigen Zeit undenkbar
wären. So ließ er beispielsweise die Schüler nicht nur für schulische
Zwecke, sondern auch für seine Privatangelegenheiten kleine Dienste
und Arbeiten während der Unterrichtszeit erledigen. Zu diesen
Arbeiten gehörten unter anderem einen Zettel von Klassenzimmer zu
Klassenzimmer tragen, Lehrmittel aus dem entsprechenden Zimmer holen,
den Schulhof aufräumen, das Pflaster des der Schule nahe gelegenen
Kriegerdenkmals ausgrasen und Holzaufrichten. Diese Dienste hatten mit
der Schule einen gewissen Zusammenhang, aber die Teppiche aus Högns
Wohnung klopfen, von Högn geschossenes Wild zum Bahnhof bringen, sein
Rad putzen oder es sogar reparieren waren Arbeiten, die rein ins
Private fielen und jeden Bezug zur Schule entbehrten. Diese Tätigkeiten
wurden aber nicht als Strafarbeit, sondern als Zeichen besonderen
Vertrauens verstanden, wie mehrere Zeitzeugen bestätigten.

\begin{figure}
\img{Schulhaus-1908}
\caption{Das Schulhaus von 1908}
\end{figure}

Seitdem 1920 der Kirchendienst der Lehrer abgeschafft wurde, war der
Einsatz Högns als Organist bei Beerdigungen während der Unterrichtszeit
ein eindeutiger Verstoß gegen das damalige Schulrecht. Dass diese
Praxis auch noch zur NS-Zeit von den Behörden und den Eltern toleriert
wurde, lag an der besonders großen Macht der Institution Kirche auf dem
Land. Eine gewöhnliche Beerdigung dauerte von 9 bis 11 Uhr, so
genannte \textit{levitierte} Beerdigungen für wohlhabende Verstorbene
dauerten wesentlich länger. Högns Schüler mussten während seiner
Abwesenheit in \textit{Stillarbeit} Aufgaben erledigen, ein paar
Schüler wurden zu Aufpassern ernannt und die Lehrer der benachbarten
Klassen statteten der verwaisten Klasse hin und wieder einen Besuch
ab. Natürlich verhielten sich die Schüler nicht nach Högns Vorstellung.
Tumult im Klassenzimmer war noch das kleinere Übel. Manchmal kam es
sogar vor, dass sich einige Schüler bereits auf den Nachhauseweg
gemacht hatten, ehe Högn von der Kirche ins Klassenzimmer zurückkehrte.
Högn soll bei derartigen Vorkommnissen des Öfteren versucht haben,
diese Schüler mit dem Fahrrad einzuholen und zur Schule
zurückzubringen.

Die alltäglichen Beschwerlichkeiten, mit denen Lehrer damals überall in
Bayern zu kämpfen hatten – man bedenke die große Schüleranzahl in einer
Klasse – und die besonderen Schwierigkeiten in Ruhmannsfelden scheinen
sich unter anderem in der Art niedergeschlagen zu haben, wie Högn in
der Klasse für Disziplin sorgte. Viele der befragten Zeitzeugen
bezeichnen August Högn als strengen Lehrer. Dass er unfolgsame Schüler
mit Ohrfeigen, \textit{Tatzen}, Ziehen an den Haarspitzen und Schlägen
auf den Hintern bestrafte, ist daher kaum verwunderlich und außerdem
üblich für die damalige Zeit. Noch gewalttätigere Maßnahmen sind von
ehemaligen Schülern als Zeichen einer gewissen Hilflosigkeit verstanden
worden, die sie beim alternden und überforderten Lehrer beobachten
konnten. Hierzu gehörten Strafen, wie zum Beispiel Schülern die
Schiefertafel so stark über den Kopf zu schlagen, dass sie zerbrach,
einen Bund mit vielen Schlüsseln über den Kopf zu schlagen, den Kopf
der Schüler an die Tafel zu stoßen, oder sogar mit einem Blumentopf
nach einem Schüler zu werfen. Ebenso wenig zimperlich ging Högn mit
gewissen Schülern verbal um, als er sie \textit{ordinärer Hund},
\textit{dreckiger Hundschuft} oder \textit{blöder Hammel} nannte. Doch
nicht nur mit harten Strafen, die sich fester ins Gedächtnis ehemaliger
Schüler eingruben, als mancher pädagogische Kniff, versuchte er für
Ruhe im Klassenzimmer zu sorgen. Als geschickter Lehrer zeigte er sich,
als er die Schüler am Ende jeder Unterrichtsstunde zur Entspannung und
Auflockerung aufstehen ließ, um ihre überschüssigen Energien vor und
nicht während der Stunde loszuwerden. Das negative Bild vom strafenden
Lehrer hellt sich auf, wenn Zeitzeugen ihren ehemaligen Lehrer trotz
all seiner Wutausbrüche als sehr gerecht bezeichnen.

Als Entschädigung für manche Schwierigkeiten im Schulalltag galt für den
leidenschaftlichen Musiker sicher der Unterricht in diesem Fach. Dieser
hatte den größten Stellenwert unter allen Fächern, die Högn
unterrichtete. Zu manchen Zeiten wurde jeden Tag eine Stunde
gesungen. Selbst als kurz vor Ende des Zweiten Weltkriegs ein Lehrer
zwei Klassen unterrichten musste – eine Klasse hatte täglich nur drei
Stunden Unterricht – wurde jeden Tag vor Beginn des Unterrichts statt
des Gebets ein Lied gesungen. Da die Schule kein Klavier besaß,
begleitete Högn die Volkslieder normalerweise auf der Violine. Es kam
aber auch vor, dass die Lieder unbegleitet gesungen wurden. Nach
Aussagen mehrerer ehemaliger Schüler sang man unter August Högn
durchaus mehrstimmig. Ein anfangs dreistimmiges, zum Schluss hin
sogar teilweise fünfstimmiges Arrangement Högns von dem Lied
\textit{Tauet Himmel}, das mit \textit{aus meinen Kinderliedern}
überschrieben ist, bestätigt dies. Eine ehemalige Schülerin kann sich
sogar an den Fall erinnern, als Högn speziell für ihre Banknachbarin
eine zweite Stimme arrangierte: Die Schülerin sang einmal spontan zu
einem einstimmigen Volkslied eine zweite Stimme. Schon am nächsten
Schultag legte ihr Högn eine ausgeschriebene zweite Stimme vor. Beim
Singen legte August Högn durchaus Wert auf Stimmbildung. Er verlangte
zum Beispiel, dass mit Bruststimme und im Stehen gesungen wurde, da
sich so die \zitat{Lungen besser weiten} könnten. Den
Schülern standen Liederbücher zur Verfügung, aus denen sie sich selbst
Lieder zum Singen aussuchen durften. Es handelte sich wahrscheinlich
um ein damals gängiges Repertoire an allgemein bekannten
Volksliedern, das gemeinsam musiziert wurde. Aber auch politische
Lieder kamen im Singunterricht vor. Eine Zeitzeugin berichtete, dass um
1930, also zur Zeit der Rheinlandbefreiung und dem Aufkommen der
nationalsozialistischen Bewegung, häufiger patriotische Lieder wie zum
Beispiel das Lied \textit{Kräftiger Zweig} \textit{der deutschen Eiche}
gesungen wurden. Ab 28. Dezember 1938 war der \textit{Singkamerad}
das einzige an den bayerischen Volksschulen zugelassene Liederbuch und
von diesem Zeitpunkt an konnte man vor allem Lieder mit
NS-Gedankengut aus den Klassenzimmern hören. Högns Klassen bildeten
hier keine Ausnahme.

\begin{figure}
\img{Arrangement-fuer-Schulgebrauch}
\caption{Arrangement von August Högn für den Schulgebrauch}
\end{figure}