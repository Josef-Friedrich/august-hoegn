\section{Werkverzeichnis}

\subsection{Geistliche Musik}

\newcommand{\klein}[1]{\tiny#1}

\newenvironment{tabelle}{\begin{tabular}{p{8cm}l}}{\end{tabular}}

\subsubsection{Messen}

\begin{tabelle}
„Laurentius“-Messe C-Dur op. 14 &
\klein{4-st. gem. Chor, Blechbläserquartett, Orgel}\\

„Mater-Dei“-Messe F-Dur op. 16 &
\klein{4-st. gem. Chor, Streichquintett, Orgel}\\

„Josephi“-Messe F-Dur op. 62 &
\klein{4-st. gem. Chor, Soli, 2 Vl., Blechbläserquartett, Orgel}\\
\end{tabelle}

\subsubsection{Tantum ergo}

\begin{tabelle}
Tantum ergo Nr. 1 Es-Dur op. 11 &
\klein{4-st. gem. Chor, Streichquintett, Orgel}\\

Tantum ergo Nr. 2 F-Dur op. 32 &
\klein{4-st. gem. Chor, Streichquintett, Orgel}\\

Tantum ergo Nr. 3 Es-Dur op. 49 &
\klein{4-st. gem. Chor , Streichquintett, Orgel}\\

Tantum ergo Nr. 4 A-Dur op. 47 &
\klein{4-st. gem. Chor, Streichquintett, Orgel}\\
\end{tabelle}

\subsubsection{Pange lingua}

\begin{tabelle}
Pange lingua G-Dur (deutsch) &
\klein{4-st. gem. Chor, Blechbläserquartett}\\

Pange lingua F-Dur op. 43 &
\klein{4-st. gem. Chor, Orgel}\\

Pange lingua Es-Dur op. 46 &
\klein{4-st. gem. Chor, Orgel}\\

Pange lingua Es-Dur op. 51 &
\klein{4-st. gem. Chor, Orgel}\\
\end{tabelle}

\subsubsection{Marienlieder}

\begin{tabelle}
Ave Maria F-Dur op. 4 &
\klein{Unter- und Oberstimme, Orgel}\\

Marienlied Nr. 1 F-Dur op. 13 a &
\klein{4-st. gem. Chor, Orgel}\\

Marienlied Nr. 2 e-moll op. 19 &
\klein{Sopran-Solo, 4-st. gem. Chor, Orgel}\\

Marienlied Nr. 3 F-Dur op. 22 &
\klein{Sopran-Solo, 4-st. gem. Chor, Orgel}\\

Marienlied Nr. 4 G-Dur op. 23 &
\klein{Sopran-Solo, 4-st. gem. Chor, Orgel}\\

Marienlied Nr. 5 F-Dur op. 28 &
\klein{4-st. gem. Chor, Orgel}\\

Marienlied Nr. 6 F-Dur op. 41 &
\klein{4-st. Frauenchor, Orgel}\\

Marienlied Nr. 7 G-Dur op. 45 (fragmentarisch) &
\klein{Sopran-Solo, 4-st. gem. Chor, Orgel}\\

Marienlied Nr. 8 G-Dur op. 54 &
\klein{Sopran-Solo, 4-st. gem. Chor, Orgel}\\

Marienlied Nr. 9 G-Dur op. 34 &
\klein{4-st. Männerchor}\\

Marienlied Nr. 10 F-Dur op. 56 &
\klein{2 Sopran- und Alt-Solo, 4-st. gem. Chor, Orgel}\\

Marienlied Nr. 11 F-Dur op. 59 &
\klein{Bariton-Solo, 4-st. gem. Chor, Orgel}\\

Marienlied Nr. 12 F-Dur op. 63 &
\klein{Sopran- und Alt-Solo, Orgel}\\

Marienlied (Nr. 13) C-Dur &
\klein{Sopran-Solo, Klavier o. Harmonium}\\
\end{tabelle}

\subsubsection{Grablieder}

\begin{tabelle}
Grablied für gefallene Soldaten Es-Dur op. 35 &
\klein{4-st. gem. Chor, Blechbläserquartett}\\

Grablied Nr. 1 Es-Dur op. 35 &
\klein{4-st. gem. Chor, Blechbläserquartett}\\

Grablied Nr. 2 Es-Dur &
\klein{4-st. gem. Chor, Blechbläserquartett}\\

Grablied Nr. 3 Es-Dur op. 44 &
\klein{4-st. gem. Chor, Blechbläserquartett}\\

Grablied Nr. 4 F-Dur op. 20 &
\klein{Sopran-Solo, 4-st. gem. Chor, Orgel}\\
\end{tabelle}

\subsubsection{Offertorien}

\begin{tabelle}
Offertorium D-Dur op. 26 &
\klein{4-st. gem. Chor, Orgel}\\

Offertorium C-Dur op. 30 &
\klein{4-st. gem. Chor, Orgel}\\
\end{tabelle}

\subsubsection{Kommunionlieder}

\begin{tabelle}
Kommunionlied Es-Dur op. 12 &
\klein{4-st. gem. Chor, Orgel}\\

Kommunionlied G-Dur op. 21 a &
\klein{4-st. gem. Chor, Orgel}\\

Kommunionlied G-Dur op. 21 b &
\klein{4-st. gem. Chor, Orgel}\\

Kommunionlied C-Dur op. 37 b &
\klein{4-st. gem. Chor, Orgel}\\
\end{tabelle}

\subsubsection{Veni creator Spiritius}

\begin{tabelle}
Veni creator Spiritus B-Dur &
\klein{4-st. Männerchor}\\

11 Veni creator Spiritus op. 15 &
\klein{4-st. gem. Chor}\\
\end{tabelle}

\subsubsection{Adjuva nos}

\begin{tabelle}
Adjuva nos Es-Dur op. 8 &
\klein{4-st. gem. Chor, Streichquintett, Orgel}\\

8 Adjuva nos op. 15 &
\klein{4-st. gem. Chor}\\
\end{tabelle}

\subsubsection{verschiedene Genre}

\begin{tabelle}
Cäcilienlied E-Dur op. 12 b &
\klein{3-st. Frauenchor, Orgel}\\

Libera e-moll op. 50 &
\klein{4-st. gem. Chor}\\

Benedictus G-Dur op. 50 &
\klein{4-st. gem. Chor}\\

Fronleichnams-Prozessionsgesänge Es-Dur op. 52 &
\klein{4-st. gem. Chor, Blechbläserquartett}\\

Ecce sacerdos F-Dur op. 57 &
\klein{4-st. gem. Chor, Blechbläserquartett, Orgel}\\

Juravit Dominus B-Dur op. 58 &
\klein{4-st. gem. Chor, Blechbläserquartett, Orgel}\\

„Ehre sei Gott“ C-Dur &
\klein{4-st. gem. Chor}\\

{\itshape Herz-Jesu-Litanei (verloren)} &
\klein{}\\
\end{tabelle}

\subsection{Weltliche Musik}

\begin{tabelle}
Marsch „In Treue fest!“ D-Dur &
\klein{Klavier}\\

Weihegesang Es-Dur (fragmentarisch) &
\klein{4-st. gem. Chor, Blechbläserquartett}\\

Lied von Gotteszell G-Dur op. 42 (Arrangement) &
\klein{4-st. Männerchor}\\
\end{tabelle}

\subsection{Geschichtswerk}

\textit{Geschichte von Ruhmannsfelden,} Michael Laßleben, Kallmünz, 1949


\textit{Geschichte der Gemeinde Zachenberg,} Manuskript, 1954
(Högn/Trellinger)

\textit{Geschichte und Chronik der freiwilligen Feuerwehr
Ruhmannsfelden,} Manuskript, 1951

Zeitungsartikel:

\textit{Geschichtliches vom Markt Ruhmannsfelden: Der Name
„Ruhmannsfelden“ – Die Bezeichnung „Markt“,} Durch Gäu und Wald –
Beilage zum Deggendorfer Donauboten, 6.11.1926

\textit{Geschichtliches vom Markt Ruhmannsfelden: Das Wappen von
Ruhmannsfelden – Schloss und Schlossberg Ruhmannsfelden,} Durch Gäu und
Wald – Beilage zum Deggendorfer Donauboten, 15.12.1926

\textit{Geschichtliches vom Markt Ruhmannsfelden: Von der Schule in
Ruhmannsfelden in früherer Zeit bis 1835,} Durch Gäu und Wald – Beilage
zum Deggendorfer Donauboten, 20.8.1927

\textit{Geschichtliches vom Markt Ruhmannsfelden: Von der Schule in
Ruhmannsfelden ab 1835}, Erscheinungsort und -datum ungekannt, gefunden
in der Chronik der Volksschule Ruhmannsfelden

\textit{Das Wallfahrtskirchlein Osterbrünnl bei Ruhmannsfelden,} Durch
Gäu und Wald - Beilage zum Deggendorfer Donauboten, 1927/Nr. 23 und
1928/Nr. 2

\textit{Was Ruhmannsfelden für Jubiläen feiern könnte?,} Viechtacher
Tagblatt, 9.9.1928

\textit{Wie hat es um Ruhmannsfelden herum ausgesehen vor seiner
Entstehung?,} Viechtacher Tagblatt, 25.10.1928

\textit{Pfarrkirche St. Laurentius Ruhmannsfelden,} Viechtacher
Tagblatt, 1928/29 (in drei Artikeln erschienen)

\section{Interviews mit Zeitzeugen von Högn}

\begin{itemize}
\item Interview mit Wilhelm Ederer, Aug. 2002
\item Interview mit Barbara Essigmann, 27.12.2002
\item Interview mit Ida Högn, 29.12.2002
\item Interview mit Maria Schröck, 30.12.2002
\item Interview mit Barbara Essigmann, 2.1.2003
\item Interview mit Wilhelm Ederer, 2.1.2003
\item Interview mit Ida Högn, 3.1.2003
\item Interview mit Josef Brunner, 3.1.2003
\item Interview mit Dr. Doraliesa Wiegmann, 19.1.2003
\item Interview mit Mathilde Beischmied, 21.1.2003
\item Interview mit Dr. Josef Stern, 21.2.2003
\item Interview mit Emilie Seidl, 23.4.2003
\item Interview mit Lorenz Schlagintweit, 29.11.2003
\item Interview mit Johann Freisinger, 29.12.2003
\item Interview mit Stephan Leitner, 19.2.2004
\item Interview mit Maria Freisinger, 25.8.2004
\item Interview mit Max Holler, 26.8.2004
\item Interview mit Centa Schwannberger, 14.9.2004
\item Interview mit Mathilde Beischmied, 14.9.2004
\item Interview mit Gertraud von Molo, 23.11.2004
\item Interview mit Lilo Leuze, 2.12.2004
\item Interview mit Wilhelm Ederer, 28.12.2004
\item Interview mit Josef Raster, 28.12.2004
\item Interview mit Johann Glasschröder, 28.12.2004
\item Interview mit Lilo Leuze, 14.1.2005
\item Interview mit Eva Ertl, 9.2.2005
\end{itemize}

\section{Quellenverzeichnis}

\subsection{Literatur}

Dantl, Georg\textbf{,} \textit{Vom Schullehrling zum Schulmeister –
Geschichte der Lehrerbildung im 19. Jahrhundert,} in: Oberpfälzer
Raritäten, Band 5, Verlag der Buchhandlung Taubald, Weiden, 1989

Gärtner, Helmut\textbf{,} \textit{Deggendorfer Originale – Originelles
Deggendorf,} Morsak Verlag, Grafenau, 2. Auflage, 1995

Geyer, Otto\textbf{,} \textit{Schule und Lehrer in Niederbayern,} Hrsg.
Otto Glaser, Niederbayerischer Bezirkslehrerverein im BLLV,
Neue-Presse-Verlag-GmbH, Passau, 1964, 235 Seiten

Goller, Martina\textbf{,} \textit{Die Musik in der Lehrerbildung
Niederbayerns und ihre Ausstrahlung am Beispiel niederbayerischer
Lehrerkomponisten dargestellt an der Präparandenschule Deggendorf und
am Lehrerbildungsseminar Straubing,} Zulassungsarbeit zur ersten
Staatsprüfung für das Lehramt an Hauptschulen in Bayern eingereicht
bei Prof. Dr. Eckard Nolte im Fach Musikpädagogik, Deggendorf, April
1988

Högn, August\textbf{,} \textit{Geschichte und Chronik der freiwilligen
Feuerwehr Ruhmannsfelden,} Manuskript, 1951 (Feuerwehr)

Högn, August\textbf{,} \textit{Geschichte von Ruhmannsfelden,} Michael
Laßleben, Kallmünz, 1949 (Ruhmannsfelden)

Högn, August\textbf{,} \textit{Heimat-Geschichte der Gemeinde
Zachenberg,} Manuskript, 1954 (Zachenberg)

Lippert, Heinrich\textbf{,} \textit{Die Präparandenschule Deggendorf
(1866-1924) – Zur Geschichte einer niederbayerischen
Lehrerbildungsanstalt,} in: Deggendorfer Geschichtsblätter 17,
Deggendorf, 1996, S. 153 – 192

Proft, Hans \textbf{\textit{,}}\textit{„Immer froh und heiter bleibt der
Kutschenreuter“ – Leben und Werk des niederbayerischen Komponisten
Erhard Kutschenreuter,} Verlag Karl Stutz, Passau, 2004

Stengel, Georg Josef\textbf{,} \textit{Geschichte der
Lehrerbildungsanstalt Straubing von 1824-1924,} Manz, Straubing, 1925

\subsection{Quellen aus Archiven}

\noindent Reicheneder-Chronik:

\begin{itemize}

\item Die Seelsorger der Pfarrei (Seelsorger)
\item Auszüge a. d. Protokollbüchern: Ehrenbürger von Ruhmannsfelden
(Ehrenbürger)
\item Schul- und Bildungswesen in Ruhmannsfelden (Schulwesen)
\item Der Pfarrmesner von Ruhmannsfelden (Pfarrmesner)
\item Religiöse Feiern in der Pfarrei (Feiern)
\item Die Pfarrkirche, C. Nach 1820, II. Einrichtung – Die Orgel (Orgel)
\item Chronik der Volksschule Ruhmannsfelden
\item Protokollbuch der Feuerwehr Ruhmannsfelden

\end{itemize}

\section{Dank}

Vielen Dank an die vielen Mithelfer, durch die diese Arbeit erst möglich
wurde:

Franz Aichinger (Auskunft),
Mathilde Beischmied (Interview),
Hennes Berger (Turnvereins-Chronik),
Gerhard Bielmeier (Zugang zum Gemeindearchiv),
Josef Brunner (Interview),
Franz jun. Danziger (Kompositionen, Dokumente, Auskünfte),
Wilfried Deisser (juristische Beratung),
Wilhelm Ederer (3 Interviews, Fotos),
Eva Ertl (Interview, Fotos),
Barbara Essigmann (2 Interviews),Johann Freisinger (Interview, Ausleihen der Manuskripte des Geschichtswerk, Fotos),
Lotte Freisinger (Hilfe bei Recherchen im Pfarrarchiv),
Maria Freisinger (Interview),
Martin Friedrich (Beratung),
Helmuth Gärtner (Auskunft),
Gemeinde Baiersdorf (Auskunft),
Gemeinde Oberaudorf (Auskunft),
Johann Glasschröder (Interview, Fotos, Komposition),
Martina Goller (Suchen nach Kompositionen, Ausleihen ihrer Zulassungsarbeit),
Martha Grotz (Entziffern von Dokumenten),
Ida Högn (2 Interviews),
Peter Högn (Auskünfte, Buchpräsentation Deggendorf),
Michael Hüttinger (Auskünfte),
Max Holler (Interview),
Alfons, HV-Straubing Huber (Literatur-Tipps),
Michael, P. Dr. Kaufmann (Auskunft),
Kirchenchor Ruhmannsfelden (Aufführungen von Kompositionen),
Kloster Metten (Auskunft),
August Lankes (Aufführungen von Kompositionen),
Stefan Leitner (Interview, Dokumente),
Lilo Leuze (Interview, Fotos),
Michael Maimer (technische Beratung (Computer)),
Martina Masarwa (Foto),
Helmuth, Pfarrer Meier (Zugang zum Pfarrarchiv, Überlassen des Kopierers, Beratung),
Gertraud von Molo (Interview, Dokumente, Fotos),
Pfarramt Kollnburg (Auskunft),
Pfarramt Mariä Himmelfahrt, Deggendorf (Auskunft),
Pfarramt St. Martin, Deggendorf (Auskunft),
Thanos Prapas (Lektorat),
Hans Proft (Dokument),
Rudolf Radlbeck (telefonische Auskunft, Buchpräsentation Ruhmannsfelden),
Josef Raster (Interview, Fotos),
Irmgard Rebhahn (Vermittlung eines Interviews, Fotos, Auskünfte),
Rektor Roßmeißl (Ausleihen der Schulchronik),
Lorenz Schlagintweit (Interview, Foto),
Maria Schröck (Interview, Fotos, Dokumente),
Centa Schwannberger (Interview, Fotos),
Johannes Schwarz(technische Beratung (Computer)),
Emilie Seidl (Interview),
Karin Stadler (Auskunft),
Josef Steinbauer (Jäger-Chronik),
Josef Stern (Interview),
Karl Stutz (Verlag),
Peter Voit (Nachforschungen um Kompositionen),
Doraliesa Wiegmann (Interview),
und allen, die vergessen wurden …