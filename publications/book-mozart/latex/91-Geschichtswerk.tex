\subsection{Geschichtswerk}

\textit{Geschichte von Ruhmannsfelden,} Michael Laßleben, Kallmünz, 1949

\textit{Geschichte der Gemeinde Zachenberg,} Manuskript, 1954
(Högn/Trellinger)

\textit{Geschichte und Chronik der freiwilligen Feuerwehr
Ruhmannsfelden,} Manuskript, 1951

Zeitungsartikel:

\textit{Geschichtliches vom Markt Ruhmannsfelden: Der Name
„Ruhmannsfelden“ – Die Bezeichnung „Markt“,} Durch Gäu und Wald –
Beilage zum Deggendorfer Donauboten, 6.11.1926

\textit{Geschichtliches vom Markt Ruhmannsfelden: Das Wappen von
Ruhmannsfelden – Schloss und Schlossberg Ruhmannsfelden,} Durch Gäu und
Wald – Beilage zum Deggendorfer Donauboten, 15.12.1926

\textit{Geschichtliches vom Markt Ruhmannsfelden: Von der Schule in
Ruhmannsfelden in früherer Zeit bis 1835,} Durch Gäu und Wald – Beilage
zum Deggendorfer Donauboten, 20.8.1927

\textit{Geschichtliches vom Markt Ruhmannsfelden: Von der Schule in
Ruhmannsfelden ab 1835}, Erscheinungsort und -datum ungekannt, gefunden
in der Chronik der Volksschule Ruhmannsfelden

\textit{Das Wallfahrtskirchlein Osterbrünnl bei Ruhmannsfelden,} Durch
Gäu und Wald - Beilage zum Deggendorfer Donauboten, 1927/Nr. 23 und
1928/Nr. 2

\textit{Was Ruhmannsfelden für Jubiläen feiern könnte?,} Viechtacher
Tagblatt, 9.9.1928

\textit{Wie hat es um Ruhmannsfelden herum ausgesehen vor seiner
Entstehung?,} Viechtacher Tagblatt, 25.10.1928

\textit{Pfarrkirche St. Laurentius Ruhmannsfelden,} Viechtacher
Tagblatt, 1928/29 (in drei Artikeln erschienen)
