\section{Die letzten Lebensjahre}

Um seine Pensionierung musste Högn kämpfen. Ehe sein Fall im
Entnazifizierungsprozess nicht vom Spruchkammergericht verhandelt
wurde, konnte er weder auf Wiedereinstellung nach auf Pensionierung
hoffen. Als er am 20. Februar 1947 endlich „entnazifiziert“ war, musste
erst ein „Pensionierungsgesuch“ gestellt werden, bevor er in den
Ruhestand treten konnte, was erfahrungsgemäß lange dauerte. In dieser
Übergangszeit sollte er wieder als Lehrer arbeiten, möglichst an einer
Schule außerhalb Ruhmannsfeldens. Eine Wiedereinstellung an einer
anderen als der Volksschule Ruhmannsfelden wollte Högn – zumal er im
69. Lebensjahr stand – unbedingt vermeiden. Anträge von Högn und dem
Bürgermeister Muhr auf Wiedereinstellung an der Volksschule
Ruhmannsfelden hatten schließlich Erfolg und Högn übernahm vom 19. März
bis zum 2. April 1947 dreißig Wochenstunden des erkrankten Lehrers
Hans-Georg Dutsfeld. Zum 1. September 1947 wurde Högn endgültig in den
Ruhestand versetzt. Sein Ruhestand konnte vor 1953 wohl kaum als
solcher bezeichnet werden, wenn man seine Aktivitäten für die Kirche
und die Heimatkunde betrachtet. Einen tiefen Einschnitt in Högns
unruhiges Rentnerleben stellte sein Schlaganfall Ende 1953 dar.

Vieles spricht dafür, dass Högn trotz seines Schlaganfalls auch nach
1953 ein aktives Leben führte. Er war sogar noch schöpferisch tätig,
wie sein \textit{Marienlied Nr. 13} beweist, das zur 300-Jahrfeier
der Wallfahrtskapelle Osterbrünnl im September 1960 entstanden ist.
Den Text hat Högn ursprünglich für ein Lied zum Marianischen Jahr 1954
geschrieben. Die Kündigung durch Reicheneder verhinderte aber die
Fertigstellung des Stücks. Das Jubiläum der Wallfahrtskirche bot einen
neuen Anlass den selbstverfassten Text zu vertonen.

Högn verwendete nur bei den Marienliedern Nr. 12 und Nr. 13 eigene
Texte. Beide weisen Högn als eher mäßigen Dichter aus. Im Text zum
\textit{Marienlied Nr. 12 F-Dur op. 63} kann in drei Verszeilen eine
Unterbrechung des jambischen Versmaßes beobachtet werden. Insgesamt
sind die Zeilen viel zu lang. Einerseits verblasst dadurch der Effekt
des Reimes, andererseits konnten die langen Zeilen nicht immer mit
einer sinnvollen Aussage gefüllt werden. Beim Text des
\textit{Marienlieds Nr. 13} können zwar keine Fehler im Metrikfluss
oder im Reimschema festgestellt werden, dafür verstärkt der sehr
einfache Aufbau aus Paarreim und trochäischem Versmaß den trivialen
Gesamteindruck des Gedichtes.

\settowidth{\versewidth}{Singt das Lob dem Gnadenstern!}

\begin{verse}[\versewidth]
\itshape
Kommt herbei ihr Christen all \\
von dem Berg und aus dem Tal! \\
Kommt zu ihr, der Königin – \\
Aller Welt Beherrscherin!\\
Kommet all und rufet laut – \\
bis herab die Mutter schaut – \\
und dann Euch den Segen gibt – \\
weil Euch ja die Mutter liebt! \\!

Kommt herzu aus nah u. fern! \\
Singt das Lob dem Gnadenstern! \\
Ruft um Hilf zur Mittlerin! \\
Sagt den Dank der Helferin! \\
Allezeit und immerfort \\
stehet ihr in sicherem Hort! \\
Himmelsglück als ewgen Lohn \\
schenket Euch der Gottessohn! \\!
\end{verse}

Der Umstand, dass Högn bei der Entstehung des Marienliedes kein Chor zur
Verfügung stand, schlug sich in der Besetzung nieder. Waren die meisten
früheren Marienlieder für Chor mit großen solistischen Passagen
gesetzt, so ist dieses ein reines Sololied für eine hohe Stimme. Die
Begleitung kann von einem Klavier, Harmonium oder einer Orgel
übernommen werden. Nach dem Aufbau zur urteilen, ist die Begleitstimme
mit seiner permanenten Achtelbewegung, mehr für Klavier als für Orgel
gedacht und ähnelt manchen klassischen Kunstliedern. Zu groß ist der
Unterschied zu den choralartigen Begleitsätzen für Orgel der
früheren, so dass Högn wohl kaum an eine Ausführung mit Orgel gedacht
hatte. Möglicherweise spiegelt die Klavierbegleitung auch die
krankheitsbedingt eingeschränkte Spielfähigkeit Högns wieder. Die durch
die linksseitige Lähmung beeinträchtige Hand ist in diesem Lied
wesentlich einfacher gesetzt als die rechte.

\begin{figure}
\centering
\img[width=6cm]{Hoegn-letztes-Foto.jpg}
\caption{Das letzte Foto von August Högn}
\end{figure}

Mit Sicherheit hat Högn das \textit{Marienlied} seiner ehemaligen
Chorsängerin Mathilde Glasschröder gewidmet, da die Autographen dieser
letzten erhalten Komposition Högns sich im Wohnhaus der Sängerin
befanden. Högn komponierte das Lied zwar für dieses Jubiläum, bei den
Feiern wurde es aber wahrscheinlich nicht aufgeführt. Weder die
Zeitungen noch der Pfarrbote berichteten von einer Uraufführung. Das
ist nicht verwunderlich, waren doch weder die Sängerin noch Högn auf
den Initiator der Feierlichkeiten, Pfarrer Reicheneder, gut zu
sprechen.

Das Marienlied ist nicht die einzige Komposition Högns, die er im hohen
Alter verfasste. Für den Männerchor schrieb Högn noch nach seinem 80.
Geburtstag einige weltliche Kompositionen, die nicht erhalten sind.

Vier erhaltene umfangreiche Briefe, die Högn kurz vor Ende seines
Lebens einem ehemaligen Schüler bis nach Australien schickte,
beweisen zwar nicht, dass er schöpferisch tätig, jedoch dass er bis zu
letzt im Vollbesitz seiner geistigen Kräfte war.

Am 2. August 1958 konnte Högn seinen 80. Geburtstag feiern. Dieser wurde
den Verdiensten Högns entsprechend besonders feierlich begangen. Gleich
zwei Artikel erschienen im \textit{Viechtacher Bayerwaldboten}: Am
Geburtstag konnte sich jeder Leser anhand einer ausführlichen
Biographie über den bisherigen Lebensweg von Högn informieren. Drei
Tage später wurden in einem Artikel die Feierlichkeiten beschrieben.
Schon am Vorabend des Geburtstags sang der Männerchor unter der Leitung
von Franz Danziger Högn ein Ständchen. Der erste Tenor des Chors,
Richard Bartascheck, hielt anschließend auf den Jubilar eine
\textit{saubere Rede,} wie Augenzeugen berichteten. In seiner
\textit{großen Ansprache} bezeichnete er Högn als \textit{den Mozart
von Ruhmannsfelden.} Der Redner verstand etwas von Musik, hatte er doch
Gesang studiert. Am Geburtstag gratulierten der Vorstand der
Freiwilligen Feuerwehr, der Vorstand des Krieger- und
Veteranenvereins, die Bürgermeister von Zachenberg und Ruhmannsfelden
… Wahrscheinlich war es wirklich ein \textit{großer Teil der
Bevölkerung,} wie in dem Zeitungsbericht zu lesen ist. Högn ließ sich
von diesem Trubel anscheinend nicht aus der Ruhe bringen und erzählte
einige Episoden aus seinem Leben. Zum Höhepunkt der Feierlichkeiten
marschierte die Blaskapelle unter der Leitung von Ludwig Heinrich auf
und überraschte Högn mit seinem Lieblingslied.

Ende 1960 scheint sich Högns Gesundheitszustand zu verschlechtern. Er
beklagt sich, dass sein \zitat{gesundheitliches Befinden
nicht das beste} ist und das \zitat{Marschieren sehr
schlecht geht.} Um Besserung zu erfahren, sucht er sogar einen Arzt im
mehr als 50 Kilometer entfernten Straubing auf. Er ist immer mehr auf
Hilfe anderer angewiesen und seine Haushälterin Rosa Beischmied wird
zur \zitat{Krankenfürsorgerin.}

August Högn starb am 13. Dezember 1961 um 5 Uhr morgens. Sein Tod kam
nicht plötzlich, da er rechtzeitig mit den Sterbesakramenten versehen
worden war. Nach Aussagen von Högns Enkelin Gertraud von Molo war er
bis kurz vor seinem Tod rüstig und sein Tod dürfte eher auf eine
\zitat{\textup{kürzere Krankheit}} zurückzuführen gewesen
sein, als auf eine \zitat{lange schwere} Krankheit, wie es im
Nachruf im \textit{Viechtacher Bayerwaldboten} hieß. Im Sterberegister
der Pfarrei steht als Todesursache \textit{Herzinsuffizienz}, was auf
einen plötzlichen Tod schließen lässt. Waren bei der Überführung des
Leichnams am Todestag nur wenige enge Freunde anwesend, die dem
Leichenauto Richtung Deggendorf einige hundert Meter folgten, so dürfte
es beim Requiem am darauf folgenden Tag ein Großteil der Bevölkerung
gewesen sein, der von Högn Abschied nahm, denn an alle ehemaligen
Schüler der Volksschule ging die Einladung zum Besuch des
Trauergottesdienstes. Die Beerdigung fand am 15. Dezember in Deggendorf
statt.

Bei eisiger Kälte hielten die Lehrer Karl Schambeck, Franz Nemetz und
der Kreisschulrat Botschafter am Grab Trauerreden auf August Högn.
Besonders die Ansprache von Lehrer Nemetz blieb als eine sehr lange
Rede bei vielen Anwesenden in Erinnerung. Abordnungen der Vereine aus
Ruhmannsfelden waren mit Fahnen anwesend und eine Bläsergruppe der
freiwilligen Feuerwehr spielte für ihr Ehrenmitglied zum Abschied das
Lied \textit{Wir hatten einen Kameraden}.

August Högn fand seine letzte Ruhestätte auf dem Deggendorfer Friedhof
neben seiner 1926 verstorbenen Ehefrau Emma.