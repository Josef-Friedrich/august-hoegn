\documentclass{book}
\usepackage[utf8]{inputenc}
\usepackage[T1]{fontenc}
\usepackage[ngerman]{babel}

\newcommand{\zitat}[1]{\textit{#1}}

\title{%
  Der \textit{Mozart} von Ruhmannsfelden\newline
  \large 
  Leben und Werk des Schulrektors, Heimatforschers\newline
  und Komponisten August Högn (1878-1961)
}

\author{Josef Friedrich}

\begin{document}

\chapter{Vorwort}

Als \textit{Mozart von Ruhmannsfelden} wurde August Högn in einer
Ansprache zu seinem 80. Geburtstag bezeichnet. Der schmeichelhafte
Vergleich mit dem Salzburger Komponisten war eine Verbeugung vor seinem
jahrzehntelangen engagierten Mitwirken am musikalischen Leben in diesem
kleinen Ort des Bayerischen Waldes und besonders vor dem umfangreichen,
in Ruhmannsfelden noch nie da gewesenen, kompositorischen Schaffen, auf
das er zurückblicken konnte. Auch hatte er ein beachtliches
heimatkundliches Werk vorzuweisen und das rechtfertigte mehr als genug,
die am letzten runden Geburtstag erteilten Ehren. Was die Würdigung
seines Schaffens und die Anerkennung seiner Dienste für die
Allgemeinheit angeht, könnte der Unterschied zwischen seinen Lebzeiten
und der Gegenwart  kaum größer sein: Fast 50 Jahre nach seinem Tod
kennt kaum mehr ein Ruhmannsfeldener den Namen August Högn, geschweige
denn sein musikalisches und heimatkundliches Werk. Durch Zufall habe
ich im Notenschrank der Ruhmannsfeldener Pfarrkirche einige seiner
Handschriften entdeckt, die mein Interesse an seinem Leben und Wirken
weckten und zum Verfassen dieser Arbeit führten.

Ich möchte mit dieser Arbeit an das Leben und Werk des Rektors,
Hei\-matforschers und Komponisten August Högn erinnern, um seine Person
und sein Werk vor dem Vergessen zu bewahren. Da Högn sehr eng mit
meinem Heimatort Ruhmannsfelden verbunden war, wurde diese Arbeit fast
zwangs\-läufig zu einer „musikalischen Heimatkunde“. Insofern ist diese
Arbeit auch eine „Hommage an meine Heimat“.

Weitere Informationen über August Högn erhalten sie auf der
Internetseite August-Hoegn.de.

Die \textit{Josephi-Messe F-Dur op. 64} wird demnächst im Comes-Verlag,
Piding erscheinen. 

Ruhmannsfelden im März 2006\ \ Josef Friedrich

\chapter{Hauptteil}

\section{Kindheit in Deggendorf}

Die Högns kommen aus Söldenau. Sowohl August Högns Großvater Johann
Nepomuk Högn als auch sein Vater Andreas Högn wurden in Söldenau
gebo\-ren. Johann Nepomuk Högn war mit der Katharina Schwarzmüller aus
Weng bei Aldersbach verheiratet. Der genaue Zeitpunkt der Umsiedlung
nach Deg\-gendorf, dem Geburtsort von August Högn ist nicht bekannt.
Doch wahr\-scheinlich zog schon Johann Nepomuk mit seiner Familie
dorthin, denn er wird in einer Urkunde als Gastwirt aus Deggendorf
bezeichnet. Womöglich wechselte die Familie Högn den Wohnort schon als
Andreas Högn noch ein Kind war. Andreas Högn heiratete am 5. August
1867 die aus Geiselhöring stammende Helene Zöpfl. Diese war die Tochter
des Kaufmanns Josef Zöpfl und seiner Frau Anna Knott.

August Högn kam am 2. August 1878 zur Welt. August hatte zwei ältere
Geschwister, Theres und Ludwig. Seine jüngeren Brüder waren Joseph und
Otto.

Augusts Vater war von Beruf Buchbinder und eröffnete zusammen mit seiner
Ehefrau 1867 eine Buchbinderei und Buchhandlung im so genannten
„Kerndel’schen-Haus“ am Luitpoldplatz. Bereits nach sechs Jahren, also
1873, zog die Familie Högn in ein eigenes Haus in die Pfleggasse Nummer
1, wo sich bis zum heutigen Tag die Buchhandlung Högn befindet. Die
äußerst gün\-stige Lage der Buchhandlung im Zentrum war von bedeutendem
wirtschaftlichem Vorteil. Besonders an Markttagen, so beispielsweise
zum „Saumarkt“ in der Pfleggasse, strömten viele Menschen aus dem
Umland in die Stadt, wovon auch die Buchhandlung profitierte. Das
Sortiment wurde im Laufe der Zeit um Schreib-, Schul-, Spiel- und
Lederwaren erweitert. Ab 1890 komplettierte ein eigener
Postkartenverlag das Angebot.

Im Haus des späteren landgräflichen Magistratsrats, Landrats und
Land\-tagsabgeordneten Andreas Högn war eine gründliche Ausbildung und
somit eine grundlegende musikalische Schulung der Kinder
selbstverständlich. Es ist daher nicht verwunderlich, dass alle fünf
Kinder das Klavierspielen erlernten, selbst der von Geburt an fast
taube Joseph Högn.

August Högn besuchte von 1884 bis 1888 die Knabenschule in Deggen\-dorf
und war von 1888 bis 1890 Schüler der Unterstufe des Klosters Metten,
genannt Lateinschule, wo er auch im „Klosterseminar“, also im der
Schule angeschlossen Internat wohnte. Mit dem darauf folgenden
Übertritt in die Präparandenschule in Deggendorf war schon früh der
endgültige Berufsweg eingeschlagen. Die Übernahme des elterlichen
Geschäfts dürfte für den zweitgeborenen Sohn nie zur Debatte gestanden
haben. Dies war wahrscheinlich einer der Gründe, weshalb sich August
frühzeitig für den Beruf des Lehrers entschied. Ludwig Högn, der ältere
Bruder, erlernte ganz nach alter Tradition das Buchbinderhandwerk, um
einmal die Stellung seines Vaters einnehmen zu können. Er eröffnete
jedoch in Straubing eine Kunst-, Papier- und Galanteriewarenhandlung.
Somit konnte der jüngste Sohn Otto die Buchhandlung übernehmen.

\section{Ausbildung zum Lehrer und zum Musiker}

Die Entscheidung, eine Ausbildung zum Volksschullehrer anzutreten, wurde
für August Högn sicher dadurch erleichtert, dass sich in der
Deggendorfer Arachauergasse Nr. 94 (heute Bräugasse Nr. 14) wenige
Minuten Fußmarsch entfernt von Augusts Elternhaus eine
Präparandenschule befand. Wie zu Volksschulzeiten konnte er wieder bei
seinen Eltern wohnen und war nicht mehr auf das Mettener Internat
angewiesen. 

Die Lehrerausbildung dauerte damals fünf Jahre. An eine dreijährige
Vor\-bereitungsphase an einer Präparandenschule schloss sich die
eigentliche zwei\-jährige Ausbildung in der Lehrerbildungsanstalt in
Straubing an. Neben der Bezeichnung „Präparandenschule“, die so viel
bedeutet wie „Schule der Vor\-zubereitenden“, ist aus heutiger Sicht
vor allem ungewöhnlich, dass es damals gleich zwei Schularten gab, die
vom angehenden Lehrer durchlaufen werden mussten, nämlich die
dreijährige Vorbereitungsphase an einer Präparanden\-schule und die
zweijährige Ausbildung in der Lehrerbildungsanstalt. Dies lässt sich
aus der historischen Entwicklung der Lehrerausbildung erklären. Es war
Jahrhunderte lang Praxis, dass Handwerker zusätzlich zu ihrer
beruflichen Tätigkeit auch die Unterweisung der Schulkinder in den
Grundfertigkeiten wie Lesen, Schreiben und Rechnen übernahmen. Die
Verordnung vom 4. Sep\-tember 1823 schrieb erstmals verpflichtend eine
zweijährige Ausbildung für angehende Lehrer vor und hob so
gewissermaßen den Berufsstand des Volks\-schullehrers im Königreich
Bayern aus der Taufe. Zuvor sollten die Anwärter drei Jahre lang bei
einem \zitat{tüchtigen Schullehrer} oder einem
\zitat{vorzüglichen Geistlichen} eine Art Lehre oder
Praktikum absolvieren, ehe man sie ins Lehrerseminar aufnahm. Da sich
diese Art der Vorbildung als nicht sehr effektiv erwiesen hatte, wurde
mit dem Normativ vom 29. September 1866 die dreijährige
Vor\-bereitungszeit durch Einführung der Präparandenschulen straffer
organisiert.

Ebenso ungewöhnlich aus heutiger Sicht und kaum mit der Ausbildung der
Grund- und Hauptschullehrer vergleichbar, ist die starke Gewichtung des
Musikunterrichts in der gesamten damaligen Volksschullehrerausbildung,
besonders in den Anfangsjahren der Präparandenschule. Das Fach Musik –
unterteilt in die Teilbereiche \zitat{Gesang, Violine,
Klavier, Orgel }und\zitat{ Harmonielehre }– hatte innerhalb
des Fächerkanons einen so hohen Stellenwert, dass es neben
Religionslehre, Deutsch und Rechnen ebenfalls als Hauptfach bezeichnet
wurde. Mit sechs Wochenstunden machte der Musikunterricht mehr als ein
Fünftel der Gesamtstundenzahl aus. Das zeigt deutlich die Absicht, die
Ausbildung der angehenden Lehrer auf den Chorregenten- und
Organisten\-dienst auszurichten. Wie alle Fächer, so wurde auch der
Instrumental\-unterricht von Volksschullehrern erteilt, die an die
Präparandenschule berufen worden waren. Da die Schüler vor Eintritt in
die Präparandenschule keine Vorkenntnisse im Spiel der Musikinstrumente
mitzubringen brauchten, kam es oft vor, dass im Instrumentalunterricht
bei einer Gruppenstärke von durchschnittlich zehn Schülern sehr große
Leistungsunterschiede herrschten. So musste sich beispielsweise ein
fortgeschrittener Schüler, wie es August Högn im Klavierspiel war,
zusammen mit Anfängern eine Stunde teilen. Beim Klavierunterricht stand
die Hinführung auf das Orgelspiel im Vordergrund und somit die Pflege
des \zitat{gebundenen Spiels.} Die Schüler sollten in der
dreijährigen Ausbildung die Fähigkeit entwickeln, leichte Sonaten und
Sonatinen von Bertini, Czerny, Clementi, Dussek, und Kuhlau zu spielen.
Der Orgelunterricht begann ab dem II. Kurs, also dem zweiten Schuljahr,
nach Barners Schule \textit{Anfänge des Pedalspiels} und bediente sich
im III. Kurs der Orgelschule von Herzog. Praktische
Kirchenmusikerfahrung konnten die Schüler als Choristen werktags bei
der Gestaltung von Schulmessen und an Feiertagen bei Gottesdiensten an
der königlichen Kreisirrenanstalt mit Messkompositionen der Cäcilianer
Witt, Zangl, Haberl und Ett sammeln. Sowohl die Lehrer als auch die
Schüler waren Mitglieder des Bezirk-Cäcilien-Vereins Metten und des
Pfarr-Cäcilien-Vereins Deggendorf. Von 1890 bis 1895 war August Högn
laut Personalbogen Schüler an der Präparandenschule in Deggendorf, also
ganze fünf Jahre. Weshalb Högn die Präparandenschule zwei Jahre länger
als normal besuchte, konnte nicht geklärt werden. Vielleicht gab es an
der Präparanden\-schule Vorbereitungsklassen für sehr junge Schüler –
Högn war bei Antritt seiner Lehrerausbildung erst elf Jahre alt – oder
er hat eine oder mehrere „Ehrenrunden“ gedreht? 1895 setzte Högn seine
Ausbildung zum Lehrer an der Lehrerbildungsanstalt in Straubing fort.

So eng die Präparandenschule und die Lehrerbildungsanstalt inhaltlich
in\-einander griffen, so unterschiedlich dürfte August Högn die beiden
Schulen bezüglich der Gewährung von persönlichem Freiraum erlebt haben.
Der seit Bestehen der Lehrerbildungsanstalt geltende Internatszwang
verlieh dem Seminar in Straubing den Charakter einer geschlossenen
Anstalt. Ein von 5 Uhr früh bis 21 Uhr genau festgelegter Tagesablauf
forderte von den Schülern große Anpassung. Selbst Spaziergänge fanden
nicht ohne Aufsicht statt. Man legte großen Wert auf die Fähigkeit,
sich unterzuordnen, auch wenn dies nicht explizit im Lehrplan
aufgeführt war. 

Während die Stundenzahl und die Fächerverteilung in der Musik in etwa
mit dem Lehrplan der Präparandenschule zu vergleichen war,
unterrichteten in Straubinger Seminar nicht ehemalige Volksschullehrer
mit normaler Lehrerausbildung, sondern speziell ausgebildete Musiker.
Anton Schwarz prägte von 1892 bis 1923 das musikalische Leben an der
Lehrerbildungsanstalt maßgeblich. Er hatte nach zwölfjährigem
Volksschuldienst bei Joseph Rheinberger Komposition und Orgel an der
königlichen Musikschule in München studiert und konnte sein Wissen in
den Fächern Harmonielehre, Orgel und Gesang an die Schüler weitergeben.
Sein umfangreiches kompositorisches Schaffen, darunter vier Messen,
mehrere Offertorien und Motetten, lieferte für manche Schüler einen
Ansporn, sich selbst im Komponieren zu versuchen.

Die Seminaristen erhielten zwar in ihrer Ausbildung keinen
Komposi\-tionsunterricht, dafür vermittelte ihnen der
Harmonielehreunterricht die wich\-tigsten tonsetzerischen Regeln, mit
denen sie erste kompositorische Schritte wagen konnten. Wie die im Fach
Harmonielehre gestellten Aufgaben zeigen, gehörte der Umgang mit dem
vierstimmigen Chorsatz zum Handwerkszeug eines angehenden Lehrers. Man
kann deshalb August Högns Werke – sie haben meist den vierstimmigen
Satz als Grundgerüst – als Früchte des Musikunterrichts der
Lehrerausbildung betrachten. 

Ein Beleg dafür, dass Högn bereits in der Seminarzeit komponiert hat,
ist das \textit{Veni creator spiritus B-Dur} für vierstimmigen
Männerchor aus den Jahren 1897 und 1898, die älteste erhaltene
Komposition von Högn. Die reine Männerbesetzung lässt vermuten, dass
dieses Werk im Seminar unter Högns angehenden Lehrerkollegen gesungen
worden ist. Nicht zu überhören ist der Ein\-fluss des sogenannten
Cäcilianismus auf das kleine Chorstück. Diese historisierende
Kirchenmusikbewegung orientierte sich vor allem an der frühen
Vokalpolyphonie bis zur Renaissance. Högn verwendete im \textit{Veni
creator spiritus B-Dur} an mehreren Stellen imitatorische
Satztechniken, wie sie besonders in der Musik der Renaissance verwendet
wurden. So imitieren die einzelnen Singstimmen einem Kanon ähnlich den
Melodieverlauf anderer Singstimmen. Eine gewisse jugendliche
Experimentierfreude hatte Högn beim Komponieren dieses Stückes
unüberhörbar inspiriert. Sehr gewagt und auch ein wenig holprig klingt
eine Stelle, an der Högn einen lang gehaltenen verminderten Septakkord
– der Akkord mit der schärfsten Dissonanz – um eine kleine Sekunde –
dem kleinsten und schärfsten Intervall – plötzlich nach oben
verschiebt. Diese expressive Wendung hat Högn in seiner späteren
Fassung (op. 15 Nr. 2) getilgt. Ingesamt wirkt die Spätfassung viel
gemäßigter, ausgewo\-gener und deshalb gekonnter und reifer. Ein
Vergleich der beiden Fassungen zeigt deutlich den kompositorischen
Fortschritt Högns. 

Im Juli 1898 wurde August Högn das Reifezeugnis der
Lehrerbildungsan\-stalt Straubing mit der Note 3 erteilt.

\section{Wanderjahre}

Mit dem Eintritt in den so genannten Vorbereitungsdienst, am 1.
September 1898, begannen für Högn unruhige Jahre, wie sie für die erste
Zeit des Lehrer\-berufs typisch waren und auch heute noch sind. Im
September 1898 absol\-vierte er an seiner ehemaligen Knabenschule in
Deggendorf ein Praktikum bei den Lehrern Buchner und Edelmann. Danach
durfte er sich Aushilfslehrer nennen. In der Folgezeit übernahm er, wie
seine Berufsbezeichnung schon verrät, übergangsweise Vertretungen für
erkrankte Lehrer. Vier Monate unter\-richtete er aushilfsweise in
Neukirchen bei Haggn im Landkreis Straubing-Bogen (1. November 1898 bis
1. März 1899), nach einer zweimonatigen Pause zweieinhalb Monate in
Schaufling bei Deggendorf (1. Mai 1899 bis 15. Juli 1899) und nach den
Sommerferien bis Weihnachten drei Monate in Gerats\-kirchen in der Nähe
von Eggenfelden (15. September 1899 bis 25. Dezember 1899). Seine Zeit
als Aushilfslehrer endete mit dem so genannten „Tag der ersten
eidlichen Verpflichtung“ am 20. Dezember 1899. Von diesem Zeit\-punkt
an lautete seine Berufsbezeichnung Hilfslehrer. An seiner Verwendung
als Vertretung änderte sich anfangs nur wenig. Vom 26. Dezember 1899
bis 15. Januar 1902 unterrichtete er in Zeilarn bei Simbach am Inn und
vom 16. Januar 1902 bis 31. Dezember 1902 übernahm er seine erste
längerfristige Stelle in Wallersdorf. In diesem Jahr legte er die
Anstellungsprüfung für den Volksschuldienst mit der Note 3 ab und wurde
ab 1. Januar 1903 in Wallers\-dorf im Rang eines Schulverweser
weiterbeschäftigt. „Schulverweser“ ist eine 1896 eingeführte
Bezeichnung für den Vertreter des ersten Lehrers. In Wal\-lersdorf
lernte er auch seine spätere Ehefrau Emma kennen.

% \begin{flushleft}
% \tablefirsthead{}
% \tablehead{}
% \tabletail{}
% \tablelasttail{}
% \begin{supertabular}{m{11.925cm}}
% \centering\arraybslash   [Warning: Image ignored]
% % Unhandled or unsupported graphics:
% %\includegraphics[width=4.166cm,height=5.976cm]{a3-img/a3-img001.jpg}
%      [Warning: Image ignored] % Unhandled or unsupported graphics:
% %\includegraphics[width=4.05cm,height=5.953cm]{a3-img/a3-img002.jpg}
%  \\
% \end{supertabular}
% \end{flushleft}
% {\centering
% Emma und August Högn
% \par}

Am 20. Juli 1904 fand die Hochzeit von August Högn und der neun Jahre
jüngern, damals 16-jährigen Emma Gerstl mit üppigem Menü statt, wie der
Einladung entnommen werden kann. Emma Gerstl stammte aus einer
wohl\-habenden Bierbrauerfamilie, die in Gründobl, einem kleinen Ort in
der Nähe von Wallersdorf, ein großes Anwesen mit Wirtshaus besaß. Das
junge Paar musste heiraten, weil Emma Högn schwanger war. Sie gebar
Zwillinge, die kurz nach ihrer Geburt starben. Zum 1. Juni 1905 wurde
Högn nach Eber\-hardsreuth im Landkreis Grafenau versetzt. Hier kam die
Tochter Elfriede, genannt Frieda, am 14. August 1906 zur Welt. Der nach
dem Vater benannte Sohn August erblickte am 17. Januar 1912 schon in
Högns nächstem Einsatz\-ort das Licht der Welt: in Ruhmannsfelden.

Es liegt vielleicht auch an den vielen Ortwechseln, die Högn innerhalb
we\-niger Jahre vollziehen musste, dass gerade aus der Zeit seines
Eintritts ins Berufsleben nur wenige Kompositionen erhalten sind. Dass
er komponiert hat, beweisen zwei Werke: das \textit{Ave Maria F-Dur op.
4} und der \textit{Marsch „In} Treue \textit{fest!“ D-Dur}.

Für unsere Ohren klingt Högns \textit{Ave Maria }für zwei hohe Stimmen
mit Or\-gelbegleitung höchst seltsam. Der junge Komponist Högn lag um
die Wende vom 19. zum 20. Jahrhundert auf der Höhe seiner Zeit und hat
sein Werk im damals aktuellen Kirchenmusikstil des Cäcilianismus
geschrieben. Befremdlich an dem Stück klingt heutzutage vielleicht die
Kombination von Stilelementen sehr alter Musik mit spätromantischer
Stilistik. Die zwei Gesangsstimmen des \textit{Ave Maria} würden sich
ohne weiteres gut als ein Bicinium Orlando di Lassos (1530-1594)
verkaufen lassen. Die Orgelbegleitung erinnert dagegen mit ihrer
gewagten Harmonik und der diffusen Stimmführung eher an die Orgelmusik
des Spätromantikers Max Reger (1873-1916). 

Der \textit{Marsch „In Treue fest!“} ist August Högns einzige
Komposition, die ge\-druckt wurde. Er erschien 1905 im Selbstverlag des
Komponisten und wurde durch die \textit{Cl. Attenkofer’sche Buch- u.
Musikalienhandlung} in Straubing vertrieben. Högn brachte sein Werk,
das der niederbayerischen Lehrerschaft gewidmet ist, gleich in zwei
Fassungen auf den Markt, nämlich in einer Fassung für Klavier zu zwei
Händen und in einer für Klavier zu vier Händen. Das Stück konnte als
einzige Komposition Högns nicht in Ruhmannsfelden aufgefunden wer\-den.
Sie befindet sich im Bestand der Bayerischen Staatsbibliothek in
Mün\-chen. Vielleicht verkaufte sich der Marsch so gut, dass der Druck
1910, als Högn nach Ruhmannsfelden kam, bereits vergriffen war und
deshalb hier nicht vorhanden war.

% {\centering   [Warning: Image ignored]
% % Unhandled or unsupported graphics:
% %\includegraphics[width=5.477cm,height=7.825cm]{a3-img/a3-img003.png}
%  \par}
% {\centering
% Titelblatt des Marsches „In Treue fest“
% \par}

Verkaufsfördernd war vielleicht die abwechslungsreiche Kompositionsart
des Marsches. So treffen ständig Motive mit Achtel-, Triolen- und
Sechzehntel-Rhythmus aufeinander. Dieser Bemühungen um Abwechslung
bedurfte es möglicherweise auch deshalb, weil das Harmonieschema
mehrerer Formteile des Marsches identisch ist. Der zweite Teil klingt
nicht wie ein eigenständiger Abschnitt sondern wie eine Variation des
ersten Formteils. Wird der Marsch gemäß Högns Wiederholungsvorschriften
ausgeführt, so hört man dieses Har\-monieschema genau sechs Mal.
Besonders gelungen ist dagegen die Introduk\-tion und das Trio. Die
fanfarenartigen aufsteigenden Dreiklangsbrechungen im Bassregister
verleihen der Einleitung einen erhabenen und zugleich mit\-reißenden
Charakter, während das Trio in Moll den ruhenden Gegenpol zum
triumphalen Dur der anderen Abschnitte bildet.

Der Marsch hebt sich von Högns anderen Stücken auch deshalb ab, weil es
sich bei dem \textit{Marsch „In Treue fest!“} um eine der wenigen
weltlichen Komposi\-tionen und die einzige erhaltene rein instrumentale
Komposition von Högn handelt.

\section{Schnelle Integration in Ruhmannsfelden}

August Högn wurde zum 1. Januar 1910 an die Volksschule in
Ruhmannsfel\-den versetzt. Der Markt Ruhmannsfelden, in dem damals
ungefähr 1500 Men\-schen lebten, liegt im Bayerischen Wald und ist etwa
in der Mitte des Dreiecks zu finden, das die drei Städte Viechtach,
Regen und Deggendorf bilden. Das Einzugsgebiet der Volksschule umfasste
auch vor dem Ersten Weltkrieg nicht nur die Gemeinde Ruhmannsfelden,
sondern zusätzlich weite Teile der be\-nachbarten Gemeinde Zachenberg
und Randgebiete der Gemeinde Paters\-dorf, also in etwa das Gebiet der
Pfarrei St. Laurentius. Die Gemeinde Za\-chenberg, die geringfügig mehr
Einwohner als Ruhmannsfelden zählte, be\-stand aus 38 überwiegend
landwirtschaftlich geprägten Kleinstortschaften und hatte mit dem Dorf
Zachenberg kein wirkliches Zentrum, denn dort gab es weder ein Rathaus
noch ein Schulhaus. Nicht nur in Schulangelegenheiten war
Ruhmannsfelden damals wie auch heute Zentrum. Ansässige Ärzte und eine
seit 1910 bestehende Apotheke lieferten auch für die Bewohner der
Nach\-bargemeinden Achslach und Gotteszell einen Grund, nach
Ruhmannsfelden zu kommen. Man kann davon ausgehen, dass Högn
Ruhmannsfelden als einen fortschrittlichen Ort erlebt hat, da dort kurz
vor seiner Ankunft Investitionen getätigt und Reformen durchgeführt
wurden. So besaß Ruhmannsfelden seit dem Schulhausneubau im Jahr 1908
gleich drei Schulhäuser. Das älteste, 1834 erbaute wurde als
Lehrerwohnhaus genutzt. Auch die junge Familie Högn zog hier ein. In
institutioneller Hinsicht hatte sich an der Volksschule Ruhmannsfelden
kurz vor 1910 einiges geändert. Dem Anwachsen der Schülerzahl wurde
Rechnung getragen: statt früher vier unterrichteten nun sieben Lehrer,
also ein Lehrer pro Schülerjahrgang. Den jahrgangsübergreifenden und
nach Ge\-schlecht getrennten Klassen war somit ein Ende gesetzt. Die
einzelnen Klas\-sen umfassten aber immer noch fast 70 Schüler. 1908
hatten die Anträge der Gemeinden Zachenberg und Patersdorf auf
Einführung einer Sommerschule Erfolg. Damit Kinder vor allem zur
Erntezeit in der Landwirtschaft ihrer Eltern mithelfen konnten, wurde
vom 1. Mai bis 1. Oktober ein auf drei Stunden verkürzter Unterricht
eingeführt.

% {\centering   [Warning: Image ignored]
% % Unhandled or unsupported graphics:
% %\includegraphics[width=5.198cm,height=8.006cm]{a3-img/a3-img004.jpg}
%  \par}
% {\centering
% Das Rathaus in Ruhmannsfelden zu Högns Zeit. Es wurde vor einigen Jahren
% abgerissen. Seitdem ziert ein Bauzaun den Marktplatz von
% Ruhmannsfelden.
% \par}

Mit August Högn traten gleich drei neue Lehrer in Ruhmannsfelden ihren
Schuldienst an. In der Schulchronik wird Högn bald als zweiter. Lehrer
und somit Stellvertreter von Schulleiter Alois Auer aufgeführt. Das
hatte neben seiner schon zehn Jahre langen Berufserfahrung
wahrscheinlich auch den Grund, dass Högn die einzige männliche
Lehrkraft neben Auer war, die in den unruhigen Zeiten des Ersten
Weltkrieges an der Schule bleiben konnte. Högns Kriegseinsatz
beschränkte sich auf einen kurzen Heeresdienst in den Jahren 1915/1916.
Am 11. August 1915 wurde er zum \textit{10. Infanterie Regiment
Ersatzba\-tallion Straubing 4. Ring} einberufen und am 19. Oktober 1916
zur weiteren dienstlichen Verwendung entlassen. Für diesen
Kriegseinsatz verlieh ihm das NS-Regime 1936 – wie wahrscheinlich
vielen Teilnehmern am Ersten Welt\-krieg – das \textit{König Ludwig
Kreuz für Heimatverdienste während der Kriegszeit} und das
\textit{Ehrenkreuz für Kriegsteilnehmer}.

Eine weitere Neuerung in Ruhmannsfelden war, dass die Pfarrkirche St.
Laurentius im selben Jahr, als August Högn nach Ruhmannsfelden kam,
eine neue, pneumatische Orgel mit 22 Registern vom Orgelbaumeister
Ludwig Edenhofer aus Deggendorf bekam. Von Anfang an wirkte Högn an der
Kir\-chenmusik mit. Der versierte Orgelspieler war auch im
Kirchendienst mehr als ein würdiger Stellvertreter für Auer, der
zusammen mit seiner Frau Anna und Tochter Auguste den
Chorregentendienst versah.

Bereits ein halbes Jahr nach seiner Ankunft wurde Högn zum Vorstand
eines Vereins gewählt, was seine schnelle gesellschaftliche Integration
unter\-streicht. Der Turnverein Ruhmannsfelden suchte zu dem Zeitpunkt
händerin\-gend jemanden, der bereit war, den unbesetzten Posten des
Vorstands zu übernehmen. Vom 21. Mai 1910 bis 27. Dezember 1913 hatte
Högn dieses Amt inne. Auch wenn er nach dieser Zeit in keiner Funktion
der Vereinsspitze in Erscheinung tritt, blieb er über Jahrzehnte hinweg
vor allem als Leiter der Sänger- und Orchesterriege dem Verein treu.
1924 versuchte man noch ein\-mal Högn als Schriftführer in die
Vereinsspitze einzubinden, doch er lehnte ab, weil er schon bei der
Feuerwehr die Schriftführertätigkeit ausübte.

August Högn war bereits am 19. September 1902 der Wallersdorfer
Feuer\-wehr beigetreten und wechselte Anfang Januar 1910 in die
Feuerwehr Ruh\-mannsfelden. Am 26. Dezember 1910 wurde Högn dann zum
Schriftführer der Feuerwehr gewählt. Dass ein Lehrer diesen Posten
übernahm, hatte Tradition. Die Lehrer Raymund Schinagl und Max Weig
waren lange Zeit vor Högn als Schriftführer tätig gewesen. Der
Mitbegründer  der Feuerwehr und letzte Schriftführer Joseph Lukas starb
am 30. August 1910. Högn, als Lehrer im Schreiben souverän wie sonst
kein anderer bei der Feuerwehr, bot sich als Nachfolger für Lukas
geradezu an. Die Protokolle vor Högns Tätigkeit zeichnen sich
dementsprechend durch viele Rechtschreibfehler aus. August Högn blieb
40 Jahre lang Schriftführer der Feuerwehr. Neben dem Verfassen von
Protokollen übernahm er die Korrespondenz der Feuerwehr, hielt
zahlreiche Ansprachen und Vorträge und packte auch mal mit eigenen
Händen an, wie zum Beispiel 1933, als bei strömendem Regen die neue
Feuerwehrspritze vom Zug abgeladen werden musste.

Ein weiteres Aufgabenfeld, das die damalige dörfliche Gemeinschaft für
einen Lehrer bereithielt, übernahm Högn 1913: Von da an bis 1920 war er
Schreiber der Gemeinde Zachenberg. Nach dem Erlass der
Gemeindeverord\-nung wurden diese Arbeiten hauptsächlich den Lehrern
übertragen. So ver\-wundert es nicht, dass die angehenden Lehrer
speziell im Fach Gemeinde\-schreiberei unterrichtet wurden und dass vor
Högn die Lehrer Milter, Schinagl, Lechner und Hochstraßer für die
Gemeinde Zachenberg tätig waren. Högns Eifer, sich für das Gemeinwohl
zu engagieren, kommt auch dadurch besonders zum Ausdruck, dass er sogar
ein Zimmer seiner Dienstwohnung abtrat und in eine Art Gemeindekanzlei
umwandelte. Vorher wurden die Gemeindearbeiten, die für Lehrer eine
wichtige Nebeneinkunft darstellten, in einem Raum der Brauerei Rankl in
Ruhmannsfelden erledigt.

\section{Leiter des Turnverein-Orchesters}

Ähnlich wie andere Vereine am Ort, veranstaltete der Turnverein bunte
Abende oder führte Theaterstücke, Singspiele und sogar Operetten auf
und übernahm somit weit über die eigentliche Vereinsaufgabe hinaus eine
wichtige kulturelle Funktion. Für den Turnverein war die Veranstaltung
solcher Unter\-haltungsprogramme eine bedeutende Einnahmequelle für den
geplanten Turnhallenbau. Es ist daher keineswegs verwunderlich, dass
sich für August Högn in den Anfangsjahren in Ruhmannsfelden
insbesondere unter dem Dach des Turnvereins ein musikalisches
Betätigungsfeld auftat, obwohl die ur\-sprüngliche Aufgabe des Vereins
nichts mit Musik zu tun hatte. Der ehemali\-ge Vorstand unterstützte
den Verein nun als Leiter einer Sänger- und Orche\-sterriege. 

Am 27. September 1919 wurde die Sängerriege im Turnverein gegründet. Als
erster Leiter erscheint zwar Rudolf Schwannberger, Nachbar von Högn und
Kirchenchorsänger. Doch laut Turnverein-Protokoll fungierte Högn
be\-reits ab 29. Dezember 1919 als Dirigent. 1921 wird Schwannberger
als Ge\-sangswart und Högn als Gesangsdirigent der Sängerriege
bezeichnet, die sich im Saal der Brauerei Vornehm zum Proben traf. Die
Orchesterriege probte im Haus des Apothekers Voit. Dieses
Turnvereinsorchester dürfte mit dem Or\-chester, das auch den
Kirchenchor bei feierlichen Messen begleitete, iden\-tisch gewesen sein
und wurde einfach bei Darbietungen des Turnvereins als „Orchesterriege“
dem Sportclub „einverleibt“. Zwar müsste man das Orche\-ster von 1923
korrekterweise eher als erweitere Kammermusikbesetzung be\-stehend aus
sechs Streichinstrumenten und vier Blechblasinstrumenten be\-zeichnen,
doch immerhin stand ein kompletter Streichersatz mit zwei
Geigen\-stimmen, Bratsche, Cello und Kontrabass zur Verfügung. Das
Orchester\-material des Chorregenten Max Weig macht deutlich, dass
zumindest vor 1870 noch keine vollständige Streicherbesetzung vorhanden
war. Ein zwar kleines, aber komplett besetztes Streichorchester in
Ruh\-mannsfelden gab es erst in der ersten Hälfte des 20. Jahrhunderts.
Ein Grund für das erstarkende Streich\-orchester liegt möglicherweise
im überdurchschnittlich großen Bevölkerungs\-zuwachs zu Anfang des 20.
Jahrhunderts. Zwischen 1870 und 1920 wuchs die Bevölkerung um fast
fünfzig Prozent auf knapp 1500 Einwohner an. Die Lehrer waren aufgrund
der intensiven musikalischen Ausbildung im Streich\-orchester
einsetzbar. Parallel zur Bevölkerungszahl stieg auch die Zahl der Ärzte
und Apotheker, die meist während ihrer gymnasialen Ausbildung das
Spielen eines Streichinstruments erlernt hatten. So wirkten im
Orchester Högns namentlich bekannt der Tierarzt Dr. Haug (Violoncello),
der Apo\-theker Vitus Voit (Violine) und der Sohn des Arztes Dr.
Danziger, Franz Danziger (Violine), späterer Kirchenchorleiter und
Nachfolger Högns mit.

Leider ist nur wenig über die Theaterveranstaltungen des Turnvereins
überliefert. Es ist nicht bekannt, wie regelmäßig sie stattfanden und
über welchen Zeitraum sich der Verein derart kulturell engagierte.
Einen klei\-nen Einblick in das damalige kulturelle Leben gewährt uns
aber die nicht nur in fi\-nanzieller Hinsicht erfolgreichste
„Produktion“ des Turnvereins: die Auffüh\-rungen des Singspiels
\textit{Der Holledauer Fidel} von Erhard Kutschenreuter im Jahr 1923.
Dieses Singspiel erforderte eine große Anzahl an Mitwirkenden. Im
dritten Akt ist beispielsweise ein Trachtenfestzug verlangt. Große
Anfor\-derungen an die gesanglichen Fähigkeiten der Mitwirkenden
stellten die vielen Stücke für Sologesang, wie etwa das Liebeslied des
Fidel, das Duett des Sich\-bauern mit seiner Frau, das Lied der Reserl,
ein Kinderchor und die großen Chorszenen zu Beginn und zum Schluss des
Singspiels. Instrumentalstücke, wie zum Beispiel das polyphon angelegte
Vorspiel zum zweiten Akt der \textit{Holle\-dauer Marsch} und der
\textit{Waldler Marsch}, stellten eine Herausforderung für das
Turnverein-Orchester dar.

Die ungewöhnlich große Zahl der Mitwirkenden – auf dem Foto sind 60
Personen zu sehen – machte es notwendig, die Bühne im Vornehmsaal
aus\-nahmsweise an der Längsseite aufzustellen, so dass die Akteure nur
direkt vom Freien aus auf die Bühne gelangen konnten. Die große
Teilnehmerzahl ist auf die Unterstützung weiterer Vereine
zurückzuführen, insbesondere je\-doch auf die des Kirchenchores und der
Lehrerschaft. August Högn, der die gesamte musikalische Leitung
übernommen hatte und als Dirigent fungierte, war in seiner Eigenschaft
als Chorregent und Schulleiter der ideale Mann, weitere geeignete
Mitwirkende für das Singspiel zu gewinnen.

%   [Warning: Image ignored] % Unhandled or unsupported graphics:
% %\includegraphics[width=11.786cm,height=5.17cm]{a3-img/a3-img005.jpg}
 

Die Einkünfte der acht ausverkauften Aufführungen des Singspiels im Saal
der Brauerei Vornehm erbrachten einen Gewinn von 500 Mark, mit dem ein
Grundstück erworben werden konnte, das später als Turnplatz verwendet
wurde. Dieser durchschlagende Erfolg der Aufführungen des
\textit{Fidel} ist auch auf das Stück zurückzuführen. Mit dem
\textit{Fidel} hatte der im Rottal ansässige Lehrer Erhard
Kutschenreuter mit Abstand sein erfolgreichstes Stück geschrieben. Nach
der Uraufführung in Passau im Jahr 1920 erlebte das Singspiel schon
1938 die 3000. Aufführung. Die Ruhmannsfeldener Zuschauer strömten
viel\-leicht auch deswegen so zahlreich in die Vorstellungen, weil die
Handlung des Stückes zum Teil im Bayerischen Wald spielt: Der arme
Hopfenzupfer Fidel Waldhauser aus dem Bayerischen Wald verliebt sich in
Reserl, die Tochter des reichen Sichbauern aus der Hallertau. Trotz
auftretender Hindernisse, die un\-überwindbar zu sein scheinen, findet
das ungleiche Paar schließlich zusam\-men, und es kommt zur Hochzeit.
Högns Tochter Frieda spielte die Reserl. Den Zuschauern waren die
dargestellten sozialen Verhältnisse sicher gut bekannt, denn viele von
ihnen fuhren selbst, wie es bis in die fünfziger Jahre des letzten
Jahrhunderts üblich war, jährlich in die Hallertau zur Hopfenernte, um
Geld zu verdienen.

Am 21. Juli 1923 wurde August Högn von der Gemeinde das
Ehrenbür\-gerrecht \zitat{aus Anlass seines 25-jährigen
Dienstjubiläums }und \zitat{für die großen Verdienste, die er
sich um Schule und Gemeinde erwarb }verliehen. Dieser Titel stellte
damals eine besondere Auszeichnung dar und wurde nur an wenige
außerordentlich verdiente Bürger verliehen. Die Ehrenbürgerurkunde
bekamen vorher Pfarrer Mühlbauer (1906) sowie August Högns Vorgänger
als Schulleiter Alois Auer (1910). Nach Högn wurde sie 1933 an Adolf
Hitler verliehen.

Der Anlass für die Ehrenbürgerrechts-Verleihung an Högn waren die
Aufführungen des \textit{Holledauer Fidel} etwas länger als ein
Vierteljahr davor. Es ist offensichtlich, dass die Singspielabende, an
denen Högn in hervorragender Weise mitgearbeitet hatte und die dem
Turnverein zum Erwerb eines Turn\-platzes verhalfen, eher der Grund für
die Verleihung gewesen sind, als das 25. Dienstjubiläum, das wohl eher
einen zusätzlichen Grund darstellte.

%   [Warning: Image ignored] % Unhandled or unsupported graphics:
% %\includegraphics[width=11.779cm,height=6.512cm]{a3-img/a3-img006.jpg}
 

% {\centering
% Die Turnhalle
% \par}

Weit größerer Anstrengungen bedurfte es, um auf dem Turnplatz eine
eigene Turnhalle zu bauen. Die Einkünfte vieler weiterer bunter Abende
und Theateraufführungen, an denen Högn maßgeblich beteiligt war, wie
zum Beispiel der Operette \textit{Der Postillion} von Ludwig Eckl,
flossen direkt in den Turn\-hallenbau. Högn engagierte sich aber nicht
nur musikalisch für den Bau, son\-dern auch politisch. 1925 begrüßte er
eine Kommission des bayerischen Landtages in Ruhmannsfelden und äußerte
unter anderem die Bitte um Bezu\-schussung des Turnhallenbaus. Am 29.
Januar 1928 wurde die Halle schließ\-lich eingeweiht und ein
\zitat{großes Orchester aus lauter Ruhmannsfeldener Musikern
}spielte zur Feier des Tages.

% \subsection[]{  [Warning: Image ignored]
% % Unhandled or unsupported graphics:
% %\includegraphics[width=8.647cm,height=5.479cm]{a3-img/a3-img007.jpg}
%  }
% {\centering
% Schulhaus von 1834, seit 1908 Lehrerwohnhaus
% \par}

\subsection{Alte und neue Familie}

Völlig überraschend starb am 19. Juni 1926 August Högns Ehefrau Emma im
Alter von 39 Jahren an einem Gallendurchbruch. Mitten aus dem Leben
geris\-sen hinterließ sie, die kurze Zeit davor Großmutter geworden
war, einen erst 14 Jahre alten Sohn. Das Tragische an Emmas Tod war,
dass sie an einer Krankheit starb, die schon zur damaligen Zeit hätte
behandelt werden können, wären die Symptome frühzeitig erkannt worden.
Mit nur 47 Jahren war August Högn Witwer und musste sich nun allein um
die Erziehung seines Sohnes Gustl kümmern. Zur Verrichtung der
alltäglichen Arbeiten wurde die Haushälterin Rosa Beischmied
angestellt.

Über die Ehe Högns mit Emma ist kaum etwas bekannt. Gerüchten zufol\-ge
soll Högns Ehefrau eine Affäre mit dem Nachbarn und Kirchenchor\-sänger
Rudolf Schwannberger gehabt haben. Ein möglicher Grund, weshalb Högn
kein zweites Mal geheiratet hat, könnte seine große Liebe zu Emma
gewesen sein. Ein anderer dürfte auch seine partnerschaftliche
Beziehung zu Rosa Beischmied gewesen sein, die sich unweigerlich im
Lauf der Jahre ent\-wickelte. Beide wurden als eingespieltes Team
beschrieben. 35 Jahre lang, bis zu seinem Tod, begleitete Rosa
Beischmied Högn und lebte mit ihm in derselben Wohnung. Je nach
Lebenslage unterstützte ihn die „Högn Rosl“, wie sie von einigen
Zeitzeugen genannt wurde, etwa bei der Erziehung seines Sohnes, aber
auch beim Schuldienst, wenn sich bespielsweise Högns Schüler
Vitamintabletten auf seine Anordnung hin, bei Rosa abholen mussten.
Aber auch im Alter wurde Högn von Rosa gepflegt, besonders nach seinem
Schlag\-anfall. Nicht selbstverständlich und daher ebenso ein Indiz für
das doch über das rein Dienstliche hinausgehende Verhältnis zwischen
Högn und Beschmied war die Tatsache, dass Rosa Beischmieds
\textit{illegale} Tochter Mathilde, wie im Taufregister über die
uneheliche Tochter zu lesen ist, nach dem Tod ihrer Großeltern in Högns
Wohnung einziehen durfte. In der großen Wohnung im Schulhaus erhielt
sie ein eigenes Zimmer und fühlte sich von Högn soweit akzeptiert, dass
sie ihn heute als \textit{Ersatzvater} bezeichnet. Eine Anekdote, die
Mathilde Beischmied in einem Interview erzählte, mag das gute und sehr
freundschaftliche Verhältnis Högns zu seiner Ziehtochter beleuchten und
Einblick ins damalige „Familienleben“ geben: Als Belohnung dafür, dass
die kleine Mathilde für Högn das Bier holte, bestand Högn darauf, dass
auch sie etwas von dem Getränk bekam und fragte deshalb ihre Mutter
vorwurfsvoll: \zitat{\textup{„Kriegt sie heute kein Bier?“}
}Auch als er für mehrere Jahre seine noch schul\-pflichtige Enkelin
Inge bei sich aufnahm, zeigte sich Högn in der Funktion des
Ersatzvaters. Als Grund, weshalb sie zu ihrem Großvater kam, kann wohl
die Trennung ihrer Mutter Frieda von ihrem ersten Ehemann und die neue
Be\-kanntschaft mit ihrem späteren Ehemann Dr. Karl Schlumprecht
angesehen werden. Da auch Högns Sohn Gustl noch zu Hause wohnte, zählte
Högns „neue Familie“ zusammen mit der Enkelin Inge zeitweise fünf
Mitglieder.

Es gibt mehrere Anzeichen dafür, dass sich Högn nach dem Tod seiner Frau
allmählich ins rein Private zurückgezogen hat, noch mehr in einer
eigenen Welt lebte. Vielleicht war der frühe und plötzliche Tod seiner
Ehefrau auch ein Grund dafür, dass der gesellige \textit{Vereinsmeier}
der ersten Ruhmanns\-feldener Jahre sich in den \textit{Eigenbrötler}
der späteren Jahre verwandelte. Zeit\-zeugen kannten den älteren Högn
nur noch als einen \textit{nüchternen} Menschen, der sehr zurückgezogen
lebte, der mehr Zeit mit seiner Musik verbrachte, als mit seiner
Familie. Niemand konnte sich erinnern, dass Högn zum Ratsch außer Haus
gegangen wäre, nicht einmal mit seinen nächsten Nachbarn pflegte er
Umgang. Auch die Enkelkinder mussten auf die Gewohnheiten ihres
Groß\-vaters Rücksicht nehmen und durften ihn nicht stören, wenn er zum
Beispiel sein Mittagsschläfchen hielt. Die einzige gesellschaftliche
Aktivität, die er bis ins hohe Alter beibehielt, war die allmonatliche
Veranstaltung eines Gesell\-schaftstages, zu dem sich \textit{die
Alten}, ein Vereinigung der alten Bürger, Hand\-werker und Pensionisten
– die \textit{besseren} Bürger von Ruhmannsfelden – in der Brauerei
Amberger trafen. Neben dieser isolierten Lebensweise konnte beim
alternden Högn ein Sauberkeitsfimmel beobachtet werden, womöglich eine
Folge der sozialen Abkapselung. So wurde im Haushalt Högn streng darauf
geachtet, dass sein Besteck wirklich nur von ihm benutzt wurde, nicht
einmal seine Enkelkinder durften es benutzen. Zeitlebens aß er nur Brot
ohne Rinde, die er immer abschnitt, weil sie seiner Meinung nach schon
\textit{so viele Leute in der Hand gehabt hatten. }Machte er einen
Besuch, putzte er sich schon weit vor Betreten der Wohnung die Schuhe
mit seinem Taschentuch ab, das er dann aber wieder einschob und seiner
Haushälterin zum Waschen brachte.

\section{Chorregentendienst mit Unterbrechung}

Zum 1. September 1921 wurde, dem Protokoll der Kirchenverwaltungssitzung
zufolge, August Högn provisorisch der Chorregenten- und
Organistendienst übertragen. Provisorisch deshalb, weil ein Jahr zuvor
die gesetzlich geregelte Verbindung zwischen Schul- und Kirchendienst,
die mittels der Chorregen\-ten-, Organisten- und Mesnertätigkeiten der
Lehrer Jahrhunderte lang be\-stand, abgeschafft worden war. Besonders
am Vormittag stattfindende Beer\-digungen standen der Ausübung des
Organistendienst durch die Lehrer im Weg. Da sich viele Lehrer
freiwillig bereit erklärten, Chorregent zu bleiben, ließ sich diese
alte Tradition nicht von heute auf morgen abschaffen. Den Schulbehörden
blieb anscheinend nichts anderes übrig, als den Chorregenten\-dienst
der Lehrer, verbunden mit den Stundenausfällen, zumindest für eine
Übergangszeit zu dulden.

Eine andere Tradition, nämlich dass der Schulleiter den
Chorregenten\-dienst übernahm, wurde fortgeführt, obwohl der
Kirchendienst der Lehrer offiziell abgeschafft war. August Högn wurde
nach der Pensionierung des Bezirkshauptlehrers Alois Auer 1921 nicht
nur Schulleiter, sondern auch Chorregent. Max Weig war von 1879 bis zu
seinem Tod 1895 Schulleiter und somit auch Organist und Chorregent.
Erst in seinem Todesjahr kam vom kö\-niglichen Bezirksamt Viechtach die
Anweisung, dass ein Hilfslehrer den Chor\-regentendienst übernehmen
sollte. Von 1895 bis 1921 leitete Weigs Nach\-folger, Alois Auer,
zusammen mit seiner Frau Anna den Chor, ehe Högn ihn fortführte.

Vor allem aber dürften seine überdurchschnittlichen musikalischen
Fähig\-keiten Högn dazu bestimmt haben, die alte Tradition des Lehrers
und Chorre\-genten in Personalunion weiterzuführen. Bevor er die
Kirchenchorleitung übernahm, hatte er über 20 Jahre lang als guter
Tenor und versierter Organist – man sagte von ihm, dass er gleichzeitig
singen, spielen und dirigieren konnte – in verschiedenen Orten an der
Kirchenmusik mitgewirkt und Erfahrungen gesammelt. Nicht nur im
Manualspiel war er gewandt, wie die virtuos gesetzte Fassung seines
\textit{Marsch „In Treue fest!“} für Klavier zu zwei Händen beweist,
sondern auch im Pedaleinsatz sicher war. Unter den Stücken, die er auf
der Orgel spielte, war die berühmte und nicht gerade leicht zu
spielende \textit{Toccata und Fuge in d-moll} von Johann Sebastian
Bach. Angesichts einer derart gehobenen Orgelliteratur überrascht es
nicht, wenn manche Kirchenbesucher nach Ende des Gottesdienstes noch in
der Kirche blieben, um Högns Orgelspiel bis zum Schluss zu hören. Nicht
zu vergessen sind auch seine kompositorischen Fähigkeiten: Innerhalb
von wenigen Tagen konnte er für den Einsatz in der Kirchenmusik
passende Stücke schreiben. Ein befreundeter Chorleiter bat Högn in
einem Brief, ein Lied für seinen Männerchor zu schreiben. Laut einem
Vermerk auf der betreffenden Komposition war dieses schon zwei Tage
später fertig. Kein Wunder also, dass seine musikalischen Leistungen
auch im Urteil seiner Zeitgenossen lobende Anerkennung fanden. Bischof
Buchberger zum Beispiel hob bei einer Firmung hervor, er habe selten
einen so guten Organisten spielen gehört. Josef Brunner, der als
Organist an der Kirchenmusik unter Högn mitwirkte, meinte sogar:
\zitat{Der Högn war ein selten guter Musiker. So einer steht
nicht mehr auf.}

\zitat{\textup{Zur größten Zufriedenheit der ganzen
Kirchengemeinde} }leitete Högn laut Pfarrer Fahrmeier den Chor bis Ende
1924 und ein Ruhmannsfeldener Bür\-ger, der die Entwicklung des
Kirchenchors über einen längeren Zeitraum be\-urteilen konnte,
bestätigte, dass der Chor, der unter dem Lehrer Weig
\zitat{einen guten Namen hatte }und unter der Leitung von
Anna Auer \zitat{einen Niedergang }erlebte, bei Högn wieder
eine \zitat{Verbesserung erfuhr\textup{.}}

Dass Högn 1921 einen nicht sehr leistungsfähigen Chor übernommen hat,
erkennt man auch an seinen geistlichen Kompositionen, die bis 1924 für
den Ruhmannsfeldener Kirchenchor entstanden sind. Sowohl das
\textit{Tantum ergo} \textit{Nr.} \textit{1 Es-Dur op. 11}, das
\textit{Kommunionlied Es-Dur op. 12}, das \textit{Cäcilienlied E-Dur
op. 12 b}, das \textit{Marienlied Nr. 1 F-Dur op. 13 a}, die \textit{11
Veni creator Spiritus C-Dur op. 15} als auch die \textit{8 Adjuva nos
op. 15} stellen an den Chor nur geringe Anforderungen. Högn bezeichnete
sogar seine zu dieser Zeit entstandene Messe, die dem
Ruh\-mannsfeldener Pfarrpatron gewidmete \textit{„Laurentius“-Messe
C-Dur op. 14}, aus\-drücklich als \textit{leichte Messe zu Ehren des
heiligen Laurentius.} Besonders leicht ist in der gesamten Messe der
Tenor gesetzt, in unseren Breitengraden die am schwersten zu besetzende
Stimme, so dass diese Stimme notfalls auch von einer tiefen
Männerstimme übernommen werden kann. Ein vierstimmiger Satz, bestehend
aus einfacher Harmonik, und viele Stellen, an denen der Chor unisono
singt, ermöglichen es auch einem schwachen Chor, die Messe schnell
einzustudieren. Doch klingt die Messe deshalb keineswegs einfach. Die
ge\-samte Messe strahlt gerade wegen des sparsamen Einsatzes der
Ausdrucks\-möglichkeiten große Ruhe und tiefe Religiosität aus. Das
\textit{Hosanna} des Sanctus und Benedictus klingt durch seine
polyphone Satzweise sehr kunstvoll und ist trotzdem leicht zu singen. 

Högn wurde Ende 1924 von Max Rauscher als Chorregent abgelöst. Er wäre
sicher weit länger als drei Jahre Leiter der Kirchenmusik geblieben,
hätte sich nicht ein Nachfolger geradezu angeboten, der es ermöglichte,
auch in Ruhmannsfelden die Trennung von Schul- und Kirchendienst der
Lehrer zu vollziehen. Der 20-jährige Max Rauscher stammte aus einer
sehr musikali\-schen Familie, die nahe an der Pfarrkirche eine kleine
Konditorei mit Café besaß. Nach Abschluss der Kirchenmusikausbildung
war das am 14. Dezem\-ber 1924 in der Kirchenverwaltungssitzung
beschlossene Engagement am Heimatort für Rauscher sicher der bequemste
Weg, eine Stelle zu bekommen.

% \begin{flushleft}
% \tablefirsthead{}
% \tablehead{}
% \tabletail{}
% \tablelasttail{}
% \begin{supertabular}{m{11.724cm}}
% {\centering   [Warning: Image ignored]
% % Unhandled or unsupported graphics:
% %\includegraphics[width=4.512cm,height=6.246cm]{a3-img/a3-img008.jpg}
%      [Warning: Image ignored] % Unhandled or unsupported graphics:
% %\includegraphics[width=4.002cm,height=6.26cm]{a3-img/a3-img009.jpg}
%  \par}

% \centering\arraybslash Die Pfarrkirche St. Laurentius zur Zeit Högns,
% rechts der Altarraum\\
% \end{supertabular}
% \end{flushleft}

Doch Rauschers Beschäftigungsverhältnis an seinem Heimatort war nur von
kurzer Dauer. Nörgler, die schon zu Beginn von Rauschers Tätigkeit
we\-nig Vertrauen in ihn setzten, sahen sich sicher bestätigt, als
dieser kaum zwei Jahre nach seiner Anstellung eine in ihren Augen
überzogene Gehaltserhöhung von mehr als 150 Mark zusätzlich zu den 33
Mark, die ihm monatlich bezahlt wur\-den, bei Pfarrer Fahrmeier
einforderte. Fahrmeiers Reaktion war eindeutig: Er drohte mit
Kündigung. Von der Drohung unbeeindruckt, wandte sich Rauscher an das
Bischöfliche Ordinariat in Regensburg, um seinem Wunsch Nachdruck zu
verleihen. Als Fahrmeier von der Eingabe an das Bischöfliche Ordinariat
erfuhr, stellte er Rauscher \zitat{vor versammelter
Sängerschar} zur Rede und provozierte einen offenen Streit. Verärgert
über diese öffentliche Demütigung betonte Rauscher in seinem Brief an
Fahrmeier vom 15. November 1926, diese wäre vollkommen unangebracht
gewesen, und behauptete, dass eine Zurechtweisung \zitat{zur
Zeit von Lehrer Högn }passend gewesen wäre, als \zitat{nicht
bloß Lektüre während des Gottesdienstes gelesen, sondern von seiner
Tochter Liebeleien getrieben und Schokolade gegessen wurden. }Högn
konnte natürlich diese Vorwürfe nicht auf sich sitzen lassen. Die
\zitat{ungezogenen Anschuldigungen} verurteilte er in einem
Brief an Pfarrer Fahrmeier aufs Schärfste und ersuchte
\zitat{in der Wahrung der Autorität und im Ansehen unseres
hoch verdienten und beliebten H. H. Pfarrer Fahrmeier Max Rauscher zu
kündigen, damit endlich diesem unerhörten Treiben dieses jungen Mannes
Einhalt geboten ist und nicht derselbe und noch andere mit in der
Selbstüberhebung und Geringeinschätzung anderer gestärkt werden. }Wie
zu erwarten war, wurde dem Chorregenten Max Rauscher zum 1. Januar 1927
gekündigt, aber gleichzeitig ein neuer Dienstvertrag in Aussicht
gestellt, unter der Voraussetzung, dass er auf seine Gehaltsforderungen
verzichte und in der \zitat{Kirchenratssitzung Abbitte
leistet. }Da man beim Lesen der entsprechenden Korrespondenz deutlich
erkennen kann, dass es Rauscher von Anfang an ganz bewusst auf die
Kündigung angelegt hatte, verwundert es nicht, als er sich in der
Kirchenratssitzung, in der er Abbitte leisten sollte, nicht gemäß den
Vorstellungen der Pfarroberen verhielt und ihm deshalb endgültig
gekündigt wurde.

Als Rauscher durch seine Anschuldigung Högn in den Streit hineinzog und
dieser offen die Kündigung Rauschers verlangte, wurde eines deutlich:
Ihr Verhältnis zueinander war offenbar auch vorher nicht gut gewesen.
Gehörte Högn nicht auch zu den Nörglern, die schon zu Beginn von
Rauschers Tätig\-keit wenig Vertrauen in ihn setzten und Rauscher
während seiner zweijährigen Dienstzeit die Arbeit schwer machten, bis
er es schließlich mit einer übertrie\-benen Gehaltforderung bewusst auf
die Kündigung anlegte? Högn wäre mit Sicherheit gern weiterhin
Chorregent geblieben, wenn man, abgesehen vom finanziellen Aspekt, die
Abwechslung in der Tätigkeit betrachtet, die der Dienst bot, sowie die
Tatsache, dass er diesen mit Leib und Seele ausführte. Die Anstellung
Rauschers lieferte letztendlich den Grund dafür, dass Kirchen- und
Schuldienst nicht mehr von einer Person ausgeübt werden durften.

Eine Eintragung von Högn in einer Dirigierpartitur aus dem Notenbestand
der Ruhmannsfeldener Kirche ist beredtes Zeugnis dieser Rivalität
zwischen den beiden. Rauscher hatte einen vermeintlichen
Vorzeichenfehler in der Par\-titur einer Messe von Vinzenz Goller
korrigiert. In rechthaberischem Ton schrieb Högn, der von der
Richtigkeit des gedruckten Notentextes überzeugt war, später den
Kommentar \zitat{Grober Fehler! Nein! Muss „des“ heißen!}
dazu, anstatt das eingefügte Vorzeichen einfach auszuradieren – als
hätte er der Nachwelt seine Zweifel an Rauschers Kompetenz durch den
Eintrag in die Partitur, die eigentlich nur er selbst verwendete,
mitteilen wollen.

Pfarrer Fahrmeier schrieb nach der Kündigung Rauschers an die Regierung
von Niederbayern einen Brief, um eine Ausnahmegenehmigung für die
Über\-nahme des Chorregentendienstes durch Högn zu erhalten. Dies
zeigt, dass das Verhältnis Fahrmeier und Högn anscheinend ein sehr
gutes war. Nur Högn besitze, diesem Schreiben zufolge, die
\zitat{zur Übernahme des Kirchenchores notwen\-digen
musikalischen Fähigkeiten und Kenntnisse }und daher solle die
zuständige Be\-hörde den Chorregentendienst von Högn
\zitat{bis auf weiteres }genehmigen. Högn kannte Fahrmeier
schon vor seiner Ruhmannsfeldener Zeit. In Deggendorf war Fahrmeier als
Geistlicher zwischen 1896 und 1918 tätig. Er konnte als Freund und
Hauspfarrer der Familie Högn in Deggendorf bezeichnet werden. Wenn er
einen Ausflug nach Deggendorf unternahm, genehmigte er sich stets im
Hause Högn zuerst ein Bad, bevor er seinen Besorgungen in der Stadt
nachging.

August Högns zweites Engagement in der Kirche von Ruhmannsfelden sollte
von begrenzter Dauer sein. Nur bis zu den Sommerferien 1927
geneh\-migte die Regierung von Niederbayern Högns kirchenmusikalische
Tätigkeit, bis ein pensionierter Lehrer gefunden war, der die
Chorregentenstelle länger\-fristig übernehmen könnte. Doch unter der
Ruhmannsfeldener Bevölkerung erhob sich heftiger Widerstand gegen die
Anstellung eines Pensionisten, so dass die Stelle öffentlich
ausgeschrieben und von den fünf Kirchenmusikern, die sich beworben
hatten, entschied man sich für einen gewissen Georg Roßnagel. Sein
Dienstvertrag wurde für den 7. Juni 1927 zur Unterschrift auf\-gesetzt.
Anscheinend hat aber Roßnagel seinen Dienst nie angetreten, denn
derselbe Dienstvertrag wurde später, mit geändertem Namen, bei der
Anstel\-lung von Albert Schroll wieder verwendet, der am 31. Juli 1929
diesen Vertrag mit der Kirche in Ruhmannsfelden abschloss. Högns zweite
Amtszeit dauerte von Jahresbeginn 1927 bis Ende Juli 1929. Es vergingen
also nicht nur wenige Wochen, sondern über zweieinhalb Jahre, bis ein
Kirchenmusiker gefunden war, der Högn ablösen konnte.

Viele harmonisch anspruchsvolle Werke hat Högn in seiner zweiten
Chor\-regentenzeit komponiert. Kompositionen wie das \textit{Tantum
ergo Nr. 2 F-Dur} \textit{op. 32}, das \textit{Marienlied Nr. 3 F-Dur
op. 22} oder das \textit{Grablied Nr. 4 F-Dur op. 20} sind geradezu
durchsetzt von der Chromatik und erinnern stellenweise an komple\-xe
Wagner-Harmonik. Högn zieht in dieser zweiten Kompositionsphase im
Gegensatz zu seiner ersten weniger polyphone Satztechniken wie
Imitation oder Fugentechnik zum Lösen kompositorischer
Aufgabenstellungen heran, sondern vielmehr die Harmonik. Ideales
Betätigungsfeld zum Ausreizen der harmonischen Möglichkeiten waren die
Marienlieder, die er immer mehr zu einem Sololied mit geringer
werdender Chorbeteiligung umbaut. Fünf der dreizehn Marienlieder
entstanden in den Jahren 1927 bis 1929, darunter das \textit{Marienlied
Nr. 2 e-moll op. 19}, das \textit{Marienlied Nr. 3 F-Dur op. 22}, das
\textit{Marienlied Nr. 4 G-Dur op. 23}, das \textit{Marienlied Nr. 5
F-Dur op. 28} und das \textit{Marienlied Nr. 9 G-Dur op. 34}. In den
zwei Kommunionliedern aus dieser Zeit, nämlich
\textit{Kommu\-}\textit{nionlied G-Dur op. 21 b} und
\textit{Kommunionlied G-Dur op. 21 a}, sowie in den zwei Offertorien,
nämlich das \textit{Offertorium D-Dur op. 26} und das
\textit{Offertorium C-Dur op. 30}, lassen sich zwar nur wenige
erweiterte Akkorde finden, doch klingen diese Werke ebenfalls durchaus
farbig. Eine Besonderheit von Högns Werk aus seiner zweiten
Chorregentenzeit ist das häufig verwendete Moll. Vier der sechs Sätze
der \textit{Mater-Dei-Messe F-Dur op. 16} und das \textit{Marienlied
Nr. 2 e-moll op. 19} stehen in Moll. Im damaligen stark Dur-lastigen
Kirchenmusikrepertoire stellt das Moll eine fast exotisch wirkende
Klangfarbe dar. Aus den Kom\-positionen der zweiten Chorregentenzeit
Högns können auch die gestiegenen Fähigkeiten des Kirchenchores
abgelesen werden. Offenbar hat sich der Chor auch unter Max Rauscher
deutlich verbessert, so dass die Sänger die schwie\-rige Harmonik in
Högns spätromantischen Stücken meistern konnten.

\section{Ernennung zum Oberlehrer}

Am 1. November 1929 wurde August Högn vom Hauptlehrer zum Oberleh\-rer
befördert. Diese neue Amtsbezeichnung klingt weit weniger spektakulär
als der Titel „Rektor“, den Högn ab dem 1. April 1940 führte. Die
eigentliche Beförderung jedoch war die von 1929. 1940 fand eher eine
Umbenennung des Dienstgrades statt, ohne Auswirkung auf die
Gehaltsbezüge. Mit der Ernen\-nung zum Oberlehrer wurde Högn in die
Besoldungsgruppe A 4 a befördert und sein jährliches Grundgehalt stieg
auf 5800 Reichsmark. Sein Aufgaben\-feld als Schulleiter dürfte sich
seit 1. Oktober 1921, als er die Nachfolge des Bezirksoberlehrers Auer
antrat – Högn wurde dafür schon am 1. April 1920 zum Hauptlehrer
ernannt – also weder 1929 noch 1940 wesentlich verändert haben. Ein
weiteres Zeichen, dass die Beförderung zum Oberlehrer im Jahr 1929 eine
größere Tragweite hatte, als die anderen Beförderungen, war der
Be\-such des Bezirksschulrats Jungwirth in Högns Klasse am 5. Oktober
1929, also knapp einen Monat vor der Ernennung. Der Bericht über diese
Schulbe\-sichtigung des Schulrats gewährt uns einen kleinen Einblick in
Högns Unter\-richt. 

%   [Warning: Image ignored] % Unhandled or unsupported graphics:
% %\includegraphics[width=11.811cm,height=7.41cm]{a3-img/a3-img010.jpg}
 

% {\centering
% Klassenfoto von 1932 mit August Högn (links) und Pfarrer Fahrmeier
% (rechts)
% \par}

Im Sachunterricht stand Heimatgeschichte auf dem Programm. Zur
Vor\-bereitung auf den durchzunehmenden Pfingstritt in Kötzting ließ
Högn seine 37 anwesenden Schüler den schwarzen und weißen Regen bis zur
Mündung bei Kötzting gedanklich durchwandern. Die anschließende
Stoffdarbietung über den Pfingstritt befand der Schulrat als gut,
bemängelte jedoch, dass Högn die Selbsttätigkeit der Schüler durch
häufige Zwischenfragen hemmte. In der darauf folgenden Deutschstunde
lasen die Schüler nach dem Urteil des Sachverständigen mit wenigen
Ausnahmen das Kapitel Nr. 154 \textit{Der furchtsame Hase} aus dem
Lesebuch gut. Als Anschluss an das Lesestück wurde ein Aufsatz
vorbereitet. In der Rechenstunde erklärte Högn die dezimale
Schreibweise und den dezimalen Rest anschaulich und übte seine
praktische Umsetzung mit seinen Schülern.

So kam der Schulrat über Högns Unterrichtsstil zu folgendem Urteil:
\textit{Der Stand des Unterrichts entspricht, trotz der vielen sehr
schwachbegabten Schüler. Die Schul\-zucht ist sehr gut. Der Fleiß des
Hauptlehrers Högn verdient Lob. Högn bereitet sich gewis\-senhaft vor;
er ist stoffsicher. Er verfährt eingehend, anschaulich und gründlich
und sonst mit großem Fleiß, auch die Schwachen mitzubringen. Högn muss
jedoch noch mehr Gewicht auf die Selbsttätigkeit der Schüler legen.}

Er erteilte Högn die Note 2 für Fleiß, die Note 2 für die
Lehrbefähigung, die Note 3 für den Unterrichtserfolg und die Note 2 für
die erzieherische Wirksamkeit, was in der Summe die Note 2 ergab. Im
Protokoll wurde auch lo\-bend erwähnt, dass sich Högn an der
Erforschung der Heimatgeschichte be\-teiligt. Högn zeigte also bestimmt
keine Glanzleistung, doch zur Beförderung reichte es.

Ob nun als Hauptlehrer, Oberlehrer oder Rektor, es war sicher zu keiner
Zeit eine leichte Aufgabe für Högn, die Volksschule zu leiten. Zum
einen war die finanzielle Lage der beteiligten Gemeinden, die für die
Schulgebäude und die Lehrmittel zuständig waren, immer angespannt.
Nicht einmal die notwen\-digsten Lehrmittel konnten zu manchen Zeiten
im erforderlichen Maße be\-reitgestellt werden. Die finanziellen
Engpässe wirkten sich natürlich auch auf den Zustand der Schulgebäude
aus. So mussten beispielsweise im Jahr 1934 die Toiletten im so
genannten Mädchenschulhaus eine Zeit lang gesperrt wer\-den, weil sie,
laut Högn, derart baufällig waren, dass ihre Benutzung für die Kinder
lebensgefährlich gewesen wäre. 

Zum anderen schlug sich die schwierige Erwerbslage vieler Familien
nega\-tiv auf das Lernklima nieder. Besonders zur Erntezeit mussten
Kinder in der Landwirtschaft ihrer Eltern mithelfen, da sich kaum eine
Familie zusätzliche Arbeitskräfte leisten konnte. Aus diesem Grund
wurde in den Sommermo\-naten zeitweise der Unterricht gekürzt.

Eine dauerhafte Belastung für den Schulbetrieb stellten die langen
Schul\-wege dar, die die Schüler aus weit entfernten Ortschaften und
Höfen zu Fuß zurücklegen mussten. Über eine Stunde dauernde Märsche zur
Schule waren keine Seltenheit und wurden bei Regen oder Schnee für die
Schüler zur Tor\-tur.

Wahrscheinlich lag es an den damals geringeren Anforderungen, die man an
eine Landschule stellte – überspitzt formuliert, war man schon froh,
wenn die Schüler regelmäßig am Unterricht teilnahmen – und an der
Position Högns als Schulleiter, dass er sich gewisse Dinge erlauben
konnte, die nicht regelkonform waren und in der heutigen Zeit undenkbar
wären. So ließ er bei\-spielsweise die Schüler nicht nur für schulische
Zwecke, sondern auch für sei\-ne Privatangelegenheiten kleine Dienste
und Arbeiten während der Unter\-richtszeit erledigen. Zu diesen
Arbeiten gehörten unter anderem einen Zettel von Klassenzimmer zu
Klassenzimmer tragen, Lehrmittel aus dem entspre\-chenden Zimmer holen,
den Schulhof aufräumen, das Pflaster des der Schule nahe gelegenen
Kriegerdenkmals ausgrasen und Holzaufrichten. Diese Dienste hatten mit
der Schule einen gewissen Zusammenhang, aber die Teppiche aus Högns
Wohnung klopfen, von Högn geschossenes Wild zum Bahnhof brin\-gen, sein
Rad putzen oder es sogar reparieren waren Arbeiten, die rein ins
Private fielen und jeden Bezug zur Schule entbehrten. Diese Tätigkeiten
wur\-den aber nicht als Strafarbeit, sondern als Zeichen besonderen
Vertrauens ver\-standen, wie mehrere Zeitzeugen bestätigten.

% {\centering   [Warning: Image ignored]
% % Unhandled or unsupported graphics:
% %\includegraphics[width=11.765cm,height=4.951cm]{a3-img/a3-img011.jpg}
%  \par}
% {\centering
% Das Schulhaus von 1908
% \par}

Seitdem 1920 der Kirchendienst der Lehrer abgeschafft wurde, war der
Einsatz Högns als Organist bei Beerdigungen während der Unterrichtszeit
ein eindeutiger Verstoß gegen das damalige Schulrecht. Dass diese
Praxis auch noch zur NS-Zeit von den Behörden und den Eltern toleriert
wurde, lag an der besonders großen Macht der Institution Kirche auf dem
Land. Eine ge\-wöhnliche Beerdigung dauerte von 9 bis 11 Uhr, so
genannte \textit{levitierte} Be\-erdigungen für wohlhabende Verstorbene
dauerten wesentlich länger. Högns Schüler mussten während seiner
Abwesenheit in \textit{Stillarbeit} Aufgaben erledi\-gen, ein paar
Schüler wurden zu Aufpassern ernannt und die Lehrer der be\-nachbarten
Klassen statteten der verwaisten Klasse hin und wieder einen Be\-such
ab. Natürlich verhielten sich die Schüler nicht nach Högns Vorstellung.
Tumult im Klassenzimmer war noch das kleinere Übel. Manchmal kam es
so\-gar vor, dass sich einige Schüler bereits auf den Nachhauseweg
gemacht hatten, ehe Högn von der Kirche ins Klassenzimmer zurückkehrte.
Högn soll bei derartigen Vorkommnissen des Öfteren versucht haben,
diese Schüler mit dem Fahrrad einzuholen und zur Schule
zurückzubringen.

Die alltäglichen Beschwerlichkeiten, mit denen Lehrer damals überall in
Bayern zu kämpfen hatten – man bedenke die große Schüleranzahl in einer
Klasse – und die besonderen Schwierigkeiten in Ruhmannsfelden scheinen
sich unter anderem in der Art niedergeschlagen zu haben, wie Högn in
der Klasse für Disziplin sorgte. Viele der befragten Zeitzeugen
bezeichnen August Högn als strengen Lehrer. Dass er unfolgsame Schüler
mit Ohrfeigen, \textit{Tatzen}, Ziehen an den Haarspitzen und Schlägen
auf den Hintern bestrafte, ist daher kaum verwunderlich und außerdem
üblich für die damalige Zeit. Noch gewalt\-tätigere Maßnahmen sind von
ehemaligen Schülern als Zeichen einer gewissen Hilflosigkeit verstanden
worden, die sie beim alternden und überforderten Lehrer beobachten
konnten. Hierzu gehörten Strafen, wie zum Beispiel Schü\-lern die
Schiefertafel so stark über den Kopf zu schlagen, dass sie zerbrach,
einen Bund mit vielen Schlüsseln über den Kopf zu schlagen, den Kopf
der Schüler an die Tafel zu stoßen, oder sogar mit einem Blumentopf
nach einem Schüler zu werfen. Ebenso wenig zimperlich ging Högn mit
gewissen Schülern verbal um, als er sie \textit{ordinärer Hund},
\textit{dreckiger Hundschuft} oder \textit{blöder Hammel} nannte. Doch
nicht nur mit harten Strafen, die sich fester ins Gedächtnis ehemaliger
Schüler eingruben, als mancher pädagogische Kniff, versuchte er für
Ruhe im Klassenzimmer zu sorgen. Als geschickter Lehrer zeigte er sich,
als er die Schüler am Ende jeder Unterrichtsstunde zur Entspannung und
Auflockerung aufstehen ließ, um ihre überschüssigen Energien vor und
nicht während der Stunde loszuwerden. Das negative Bild vom strafenden
Lehrer hellt sich auf, wenn Zeitzeugen ihren ehemaligen Lehrer trotz
all seiner Wutausbrüche als sehr gerecht bezeichnen.

Als Entschädigung für manche Schwierigkeiten im Schulalltag galt für den
leidenschaftlichen Musiker sicher der Unterricht in diesem Fach. Dieser
hatte den größten Stellenwert unter allen Fächern, die Högn
unterrichtete. Zu man\-chen Zeiten wurde jeden Tag eine Stunde
gesungen. Selbst als kurz vor Ende des Zweiten Weltkriegs ein Lehrer
zwei Klassen unterrichten musste – eine Klasse hatte täglich nur drei
Stunden Unterricht – wurde jeden Tag vor Be\-ginn des Unterrichts statt
des Gebets ein Lied gesungen. Da die Schule kein Klavier besaß,
begleitete Högn die Volkslieder normalerweise auf der Violine. Es kam
aber auch vor, dass die Lieder unbegleitet gesungen wurden. Nach
Aussagen mehrerer ehemaliger Schüler sang man unter August Högn
durch\-aus mehrstimmig. Ein anfangs dreistimmiges, zum Schluss hin
sogar teilweise fünfstimmiges Arrangement Högns von dem Lied
\textit{Tauet Himmel}, das mit \textit{aus meinen Kinderliedern}
überschrieben ist, bestätigt dies. Eine ehemalige Schülerin kann sich
sogar an den Fall erinnern, als Högn speziell für ihre Banknachbarin
eine zweite Stimme arrangierte: Die Schülerin sang einmal spontan zu
einem einstimmigen Volkslied eine zweite Stimme. Schon am nächsten
Schultag legte ihr Högn eine ausgeschriebene zweite Stimme vor. Beim
Singen legte August Högn durchaus Wert auf Stimmbildung. Er verlangte
zum Beispiel, dass mit Bruststimme und im Stehen gesungen wurde, da
sich so die \zitat{Lungen besser weiten }könnten. Den
Schülern standen Liederbücher zur Verfügung, aus denen sie sich selbst
Lieder zum Singen aussuchen durften. Es handelte sich wahr\-scheinlich
um ein damals gängiges Repertoire an allgemein bekannten
Volks\-liedern, das gemeinsam musiziert wurde. Aber auch politische
Lieder kamen im Singunterricht vor. Eine Zeitzeugin berichtete, dass um
1930, also zur Zeit der Rheinlandbefreiung und dem Aufkommen der
nationalsozialistischen Bewegung, häufiger patriotische Lieder wie zum
Beispiel das Lied \textit{Kräftiger Zweig} \textit{der deutschen Eiche}
gesungen wurden. Ab 28. Dezember 1938 war der \textit{Sing\-kamerad}
das einzige an den bayerischen Volksschulen zugelassene Liederbuch und
von diesem Zeitpunkt an konnte man vor allem Lieder mit
NS-Gedan\-kengut aus den Klassenzimmern hören. Högns Klassen bildeten
hier keine Ausnahme.

% {\centering   [Warning: Image ignored]
% % Unhandled or unsupported graphics:
% %\includegraphics[width=10.795cm,height=7.123cm]{a3-img/a3-img012.png}
%  \par}
% {\centering
% Arrangement von August Högn für den Schulgebrauch
% \par}

\section{Chorregent in schwieriger Zeit}

Am 25. Januar 1940 starb der erst 36 Jahre alte Chorregent und
Gemeindese\-kretär Albert Schroll, der zehn Jahre lang in
Ruhmannsfelden den Kirchen\-chor geleitetet hatte. Er erkrankte schon
1939, so dass der Lehrer Ertl für Schroll als Klavierbegleiter bei
einem Singspiel einspringen musste. Högn ver\-trat Schroll als
Chorregent und Organist während seiner Krankheit und nach dessen Tod
übernahm er seine Stelle aushilfsweise, wie den Chorabrechnun\-gen zu
entnehmen ist. Der 61-jährige und daher nicht mehr wehrfähige Högn
blieb aber Chorregent und Organist während des gesamten Zweiten
Welt\-kriegs, da aufgrund der Kriegssituation kein hauptamtlicher
Kirchenmusiker als Nachfolger für Schroll verfügbar war. Zu Beginn von
Högns dritter Chor\-regentenzeit kehrte man zur gewohnten Praxis
zurück, bei Beerdigungen den Unterricht ausfallen zu lassen. Diese in
der Weimarer Republik geduldete Re\-gelung wurde von den Behörden des
religionsfeindlichen NS-Regimes schließ\-lich ab 1944 unterbunden.
Josef Brunner – er war zu jung für den Kriegsein\-satz – und eine
Mallersdorfer Schwester, die an der Kinderbewahranstalt ar\-beitete,
übernahmen deshalb den Organistendienst bei Beerdigungen an Stelle von
Högn.

Der Beginn des Krieges stellte auch für die Streichorchestertradition in
Ruhmannsfelden einen scharfen Einschnitt dar. Schon 1939 wurden die
Leh\-rer und Orchestermitglieder Gruber, Schultz, Friedrich und
Kestlmeier einge\-zogen. Einheimische Musiker des Orchesters, wie
Lorenz Schlagintweit, blie\-ben von einem Kriegseinsatz ebenso wenig
verschont. Ein weiterer Grund, der zum Niedergang des Streichorchesters
beitrug, waren die radikalen Ein\-schränkungen in der Lehrerausbildung
durch die Reform im Jahr 1935. Große Aufführungen mit Orchester an
Festtagen, wie von mehreren Zeitzeugen be\-richtet, dürften daher eher
vor Kriegsbeginn, also zur Zeit von Chorregent Schroll, stattgefunden
haben. Als Ersatz für das Streichorchester etablierte sich eine
Blechbläserformation, die sich aus Musikern der umliegenden
Blas\-kapellen zusammensetzte. 

Die Männerstimmen des Kirchenchores waren den Umständen entspre\-chend
mit nur zwei Sängern, nämlich Schwannberger und Holzfurtner, dünn
besetzt. Auch von der Ruhmannsfeldener Orgel forderte der Krieg seinen
Tribut. Angesichts des geringen Metallgewinns, den die am 12. Juli 1944
zum Ausbau bestimmten dünnwandigen Pfeifen und Leitungen erbrachten,
wird einerseits die aussichtslose Lage der Kriegswirtschaft,
andererseits die bewuss\-te Schikane der Nazis gegenüber der Kirche
deutlich. Die Orgel – ihr fehlten die Register Quintatön, Vox coelestis
des ersten und alle metallischen Pfeifen einschließlich der
Windleitungen des zweiten Manuals – erlitt in der Nach\-kriegszeit
durch eine lange Trockenheit zusätzlich großen Schaden, ehe die Orgel
1947 vom Orgelbaumeister Kratochwill aus Plattling restauriert werden
konnte.

Angesichts der vielen Toten, die der Zweite Weltkrieg forderte,
verwun\-dert es nicht, dass Högn in dieser Zeit vor allem
Kompositionen, die bei Beerdigungen eingesetzten werden konnten,
schrieb. So entstanden in dieser Zeit drei der insgesamt vier
Grablieder Högns, nämlich das \textit{Grablied Nr. 1 Es-Dur op. 35},
das \textit{Grablied Nr. 2 Es-Dur} und das \textit{Grablied Nr. 3
Es-Dur op. 44}. Das erste Grablied komponierte Högn auf einen Text, der
explizit auf den Tod eines Soldaten Bezug nimmt. Erst in einer späteren
Version erhielt das \textit{Grab\-lied Nr. 1} einen Text, der bei jedem
Todesfall passte. Weitere Stücke aus dieser Zeit, die bei Trauerfeiern
aufgeführt werden konnten, waren das \textit{Libera e-moll op. 50} und
der \textit{Weihegesang Es-Dur} mit nationalsozialistischem Text.
Ebenso entstanden zu der Zeit zwei Marienlieder, nämlich das
\textit{Marienlied Nr. 6 F-Dur op. 41} und das \textit{Marienlied Nr. 7
G-Dur op. 45}, zwei Tantum ergo, nämlich das \textit{Tantum ergo Nr. 4
A-Dur op. 47} und das \textit{Tantum ergo Nr. 3 Es-Dur} \textit{op. 49}
und mit dem \textit{Pange lingua F-Dur op. 43} und dem \textit{Pange
lingua Es-Dur op. 46} zwei Pange lingua. Auch das \textit{Kommunionlied
C-Dur op. 37 b}, das \textit{Lied von Gotteszell G-Dur op}.
\textit{42}, das \textit{Offertorium G-Dur op. 48} und das
\textit{Benedictus G-Dur op. 50} sind Stücke, die während des Kriegs
komponiert wurden.

Ob Högns Probenpraxis – die Proben fanden in seiner Wohnung statt – eine
Anpassung an den durch den Krieg geschmälerten Chor war oder ob der
Probenort schon in seiner ersten und zweiten Chorregentenzeit die
Wohnung war, kann aus Mangel an Zeitzeugen der früheren
Chorregentenzeiten nicht geklärt werden. Die Probenarbeit in seiner
Dienstwohnung hatte für Högn einige praktische Vorteile: Hätte der Chor
in der Kirche geprobt, wäre es im Winter sehr kalt gewesen, da auf der
Orgelempore ein Fenster undicht war. Einen beheizten Proberaum konnte
die Pfarrgemeinde damals nicht zu Ver\-fügung stellen, so dass Högns
große Lehrerwohnung eine willkommene Al\-ternative zur Kirche war.
Außerdem besaß Högn keinen Kirchenschlüssel, er hätte also dem Mesner
Bescheid geben müssen, damit die Kirche abends of\-fen blieb. Nach der
Probe hätte Högn den Schlüssel im Pfarrhof zurückge\-ben müssen, was
relativ umständlich gewesen wäre. Einige Zeitzeugen be\-haupten, dass
nicht nur eine einzige Stimmgruppe oder der komplette Chor in der
Wohnung probte, sondern auch das Blechbläserquartett. Nur die
General\-probe fand in der Kirche statt. Die Proben wurden auch nicht
regelmäßig abgehalten, beispielsweise jede Woche an einem bestimmten
Wochentag, son\-dern fanden nur dann statt, wenn ein neues Repertoire
für große Festtage er\-arbeitet werden musste. Eine regelmäßige
Probenarbeit machte anscheinend wenig Sinn, wenn man sich den täglichen
Einsatz der Hauptsänger und -sän\-gerinnen in den Gottesdiensten
vergegenwärtigt. Durch die vielen Aufführun\-gen hatte der Chor
außerdem Routine.

Vielleicht ist es aber auch auf die Proben in der privaten Wohnung
zu\-rückzuführen, dass die Kirchenmusik in Ruhmannsfelden dem
deutschland\-weiten Kulturaufschwung nach Ende des Krieges nicht
folgte, sondern eher einen Niedergang erlebte. Die in Högns Wohnung
abgehaltenen Proben schienen zu einer Privatveranstaltung verkommen zu
sein. Nur für die Haupt\-sängerinnen hielt Högn Proben. Die Sängerin
Maria Freisinger hat jahrelang sonntags an den Gottesdiensten
mitgewirkt, doch an Proben konnte sie sich nicht erinnern. Der
öffentliche Chor wurde zu einem nach außen abgeschlos\-senen Zirkel von
eng befreundeten Hauptsängerinnen, zu dem nur schwer neue Mitglieder
stoßen konnten.

Dass der Zustrom von neuen Sängerinnen und Sängern ausblieb, lag sicher
auch am Verhalten der Hauptsängerinnen. Sie hatten aus einem bestimmten
Grund kein Interesse an einem größeren Chor: Geld. Denn je mehr
Personen im Chor mitgesungen hätten, desto öfter wäre die finanzielle
Entschä\-digung, die dem Hauptchor zustand, aufzuteilen gewesen. Diese
war etwa ge\-nau so hoch wie die des Chorregenten. Das „Gehalt“ der
einzelnen Sängerin\-nen wäre somit immer kleiner ausgefallen. Aus
dieser Sichtweise ist das Ver\-halten der Sängerin Theres Raster
gegenüber neuen Chormitgliedern durch\-aus nachvollziehbar. Raster war
für den Notenschrank zuständig und be\-stimmte sogar öfter als Högn,
welches Stück gesungen werden sollte. Waren zu wenige Singstimmen
vorhanden, und das war bei den meist handgeschrie\-benen
Notenmaterialien oft der Fall, gab sie insbesondere den jungen
Sän\-gerinnen keine Noten. Neue Chormitglieder fühlten sich
logischerweise wenig akzeptiert, wenn sie wiederholt kein Notenblatt
bekamen, noch dazu, wenn ihnen nicht klargemacht wurde, warum sie keine
Noten bekommen hatten.

Der Hauptgrund, weshalb der Chor der Nachkriegszeit so wenige
Mitglie\-der zählte, ist aber in der Person von August Högn zu sehen.
Mit 70 Jahren hatte er nicht mehr den Elan, neue Sänger zu integrieren
oder sich dem in\-triganten Verhalten der Hauptsängerinnen entgegen zu
stellen. Vielleicht hat sich nicht nur sein hohes Alter, sondern auch
die Tatsache, dass er zeitlebens den Chorregentendienst nur
provisorisch oder aushilfsweise, paradoxerweise aber fast zwanzig Jahre
lang ausübte, wie in jedem Protokoll ausdrücklich erwähnt wird, und
immer nur als \textit{Notnagel} angesehen wurde, auf seine
Ein\-satzbereitschaft negativ ausgewirkt. Seinen drei großen
heimatkundlichen Ab\-handlungen, die alle in der Nachkriegszeit
entstanden sind, ist zu entnehmen, dass Högn ab 1945 sein
Hauptbetätigungsfeld nicht mehr in der Kirchen\-musik sah.

War der Einsatz des Blechbläserquartetts zu Beginn des Weltkrieges noch
aus der Not geboren, so machen Högns Kompositionen, die in der
Nach\-kriegszeit entstanden sind, und die vielen Arrangements, die er
für Bläser\-quartett gesetzt hat, den Eindruck, dass Högn Gefallen an
dem Quartett ge\-funden hatte und gar nicht mehr daran dachte, an die
Streichorchester\-tradition anzuknüpfen. Die \textit{Blaskapelle
Ruhmannsfelden} war ein Ensemble, in dem sich kurz nach dem Krieg
ehemalige Militärmusiker, vertriebene Profi\-musiker und gute Amateure
zu einer hervorragenden und durch zahlreiche Rundfunkaufnahmen
ausgezeichneten Bläserformation zusammenfanden. Aus diesem Ensemble
ließ sich für den Einsatz in der Kirchenmusik ein höchst
leistungsfähiges Quartett ausgliedern, dessen Qualität das
Streichorchester auch zu seinen besten Zeiten bei weitem nicht erreicht
hätte.

Darüber hinaus war der Aufwand, Stimmen für ein Bläserquartett zu
schreiben, wesentlich geringer, als Notenmaterial für ein
Streichquintett anzu\-fertigen. Im Gegensatz zum Streichorchester, das
nur bei „a-capella“-Stellen pausierte, beschränkte sich die Beteiligung
des Bläserquartetts auf kürzere und sehr effektvolle Einwürfe. Das
„colla-parte“-Spiel der Blechblasinstrumente deutete Högn in gedruckten
Partituren, die er arrangierte, lediglich durch eine Einrahmung der
entsprechenden Stellen mit Balken an. Da Högn gewöhnlich die Stimmen
für Streicher im Gegensatz zu den Blechbläsern eigenständig und nicht
„colla-parte“ mit den Singstimmen führte, wäre bei einem Stück mit
Streichorchesterbegleitung das Anfertigen einer neuen Partitur
unvermeidbar gewesen.

Anlässlich einer Firmung nach dem Zweiten Weltkrieg sind Högns
Kom\-positionen \textit{Ecce sacerdos F-Dur op. 57} und \textit{Juravit
Dominus B-Dur op.} \textit{58} ent\-standen. Zu jeder der zwei seltenen
geistlichen Kompositionen ließ sich im Noten-Archiv in Ruhmannsfelden
genau ein gedrucktes Werk mit derselben liturgischen Bestimmung finden,
nämlich das \textit{Ecce sacerdos As-Dur op. 12} von Richard Kempf und
das \textit{Juravit Dominus op. 19} von G. M. Alt.

Allein schon der Vergleich des Bassverlaufs zu Beginn des \textit{Ecce
sacerdos} bei\-der Komponisten macht deutlich, dass Högn offenbar
stellenweise von Kempf „abgekupfert“ hat. 

Auch beim \textit{Juravit Dominus B-Dur op. 58} zog Högn ein
gleichnamiges Stück eines anderen Komponisten als Quelle heran, nämlich
die Komposition von G. M. Alt. Er hat das Titelblatt wortwörtlich bei
Alt abgeschrieben und nur seinen Namen anstelle des Namens von Alt
gesetzt. Musikalisch geht Högn im Gegensatz zum \textit{Ecce sacerdos}
in seinem \textit{Juravit Dominus} einen ganz eigen\-ständigen Weg.

Da die gedruckten Bezugskompositionen, nach dem Zustand des
Noten\-papiers zu urteilen, kein einziges Mal gesungen wurden, liegt
der Grund, wes\-halb Högn eigene Werke zur Firmung schrieb, auf der
Hand: Die gekauften Werke waren für die Ruhmannsfeldener Verhältnisse
anscheinend nicht ge\-eignet.

Mit der anspruchsvollen Komposition von Kempf wäre der Kirchenchor
sicher an seine Grenzen gestoßen. Dies verhinderte Högn, indem er eine
eigene, leichtere Version des \textit{Ecce sacerdos} anfertigte. Alts
Stück ist im Gegen\-satz dazu eine sehr einfache, aber auch
einfallslose Komposition, an der Högn offenbar wenig Gefallen gefunden
hat. Arrangieren hätte er die Kompo\-sitionen von Kempf und Alt sowieso
müssen, da das Blechbläserquartett bei so großen Anlässen nicht fehlen
durfte. Also lag die Entscheidung gleich „maßgeschneiderte“ eigene
Werke für die Firmung zu schreiben, eigentlich auf der Hand. 

In gewisser zeitlicher Nähe zu den zwei oben erwähnten Werken dürften
auch das \textit{Pange lingua Es-Dur op. 51}, die
\textit{Fronleichnams-Prozessionsgesänge} \textit{Es-Dur op. 52}, das
\textit{Marienlied Nr. 8 G-Dur op. 54}, das \textit{Marienlied Nr. 10
F-Dur op. 56}, das \textit{Marienlied Nr. 11 F-Dur op. 59}, das
\textit{Pange lingua G-Dur} mit deutschem Text und das \textit{Ehre sei
Gott C-Dur} entstanden sein. Eine wichtige Komposition aus dieser Zeit
ist leider verloren gegangen: Die \textit{Herz-Jesu-Litanei}. Högn
versuchte 1947 die Litanei drucken zu lassen. Sowohl der
Sebaldus-Verlag als auch der Gregorius-Verlag lehnten aber mit Verweis
auf die \textit{gegenwärtige schwierige Lage auf dem Papiermarkt }ab.

\subsection{Der „Mitläufer“}
In Folge des Entnazifizierungs-Prozesses wurde August Högn am 20.
Februar 1947 vom Spruchkammergericht Viechtach zu einer Geldstrafe von
300 Reichsmark verurteilt und in die Gruppe der „Mitläufer“ eingestuft.
Die Mili\-tärregierung suspendierte Högn am 11. September 1945 aus dem
Schuldienst. Bis sein Fall im Februar 1947 verhandelt wurde, verfügte
Högn über keinerlei Einkünfte. Er lebte von den Bezügen des
Chorregentendienstes und den Nahrungsmitteln, die er als Gegenleistung
für seine Arbeit als Erntehelfer bekam. Seine große Dienstwohnung im
Schulhaus musste er 1946 räumen. Er zog in eine nahe gelegene Wohnung
im Haus des Gemeindesekretärs Hertl.

Obwohl der fünfzigseitige Akt seines Prozesses fast ausschließlich
Entla\-stendes enthält, genügte allein die Mitgliedschaft in
zugegebenermaßen mehre\-ren NS-Organisationen, dass Högn nach dem
Prozess nicht der Gruppe der „Entlasteten“ angehörte, wie er sich
selbst gerne eingeordnet gesehen hätte, sondern als „Mitläufer“
bezeichnet wurde. Högn war am 1. Mai 1933 in die NSDAP eingetreten,
trug seitdem die Mitglieds-Nummer 2663243 und zahlte monatlich drei
Reichsmark Beitrag. 1934 trat er in den NS-Lehrerbund und den
Reichsluftschutzbund ein, 1935 in die Deutschen Jägerschaft, 1936 in
die NS-Volkswohlfahrt und schließlich 1942 in die Reichsmusikkammer.

% {\centering   [Warning: Image ignored]
% % Unhandled or unsupported graphics:
% %\includegraphics[width=5.821cm,height=7.851cm]{a3-img/a3-img013.jpg}
%  \par}
% {\centering
% August Högn mit Enkel Werner Schlumprecht (1939)
% \par}

Högn rechtfertigte sich im Prozess für seine NSDAP-Mitgliedschaft und
führte aus, dass \textit{der Beitritt zur Partei keine persönliche
freie Willensäußerung oder ein offenes Bekenntnis zum Nazi-Programm
oder gar Sympathie für Hitler war, sondern die Folge des Zwanges, der
sich von allen Seiten geltend machte und die Furcht vor einer
berufli\-chen Benachteiligung. }Diese Aussage wurde auch vom
Spruchkammergericht ge\-stützt, das dem Betroffenen die
\textit{formelle Mitgliedschaft zur NSDAP} bescheinigte und darlegte,
dass er dem\textit{ Nationalsozialismus passiv gegenüberstand und sich
politisch nicht aktiv betätigt} hatte.

Högn gab zu seiner Verteidigung, laut Prozess-Akten, an, dass er
bei\-spielsweise nie die Parteiuniform getragen, vielfach mit
\textit{Grüß Gott} gegrüßt habe, den Religions-Unterricht durch das
Einüben religiöser Gesänge unter\-stützt und in den Klassenzimmern
persönlich die entfernten Kruzifixe wieder an ihren alten Platz gehängt
habe. In einem Brief an das Spruchkammergericht Viechtach nannte er
sich sogar einen \textit{Vertrauensmann aller hiesigen Antifaschisten.}

Tatsächlich gibt es Belege für Högns oppositionelles Verhalten. Ende
1943 erging vom Abschnittsleiter der NSDAP in Cham eine Anweisung an
das Landratsamt Viechtach, dass Högn die Erlaubnis zur Ausübung des
Chorre\-gentendienstes entzogen werden sollte. Bucher beschuldigte
Högn, eigen\-mächtig in gewissen Klassen zwei, statt wie in mehreren
Rundschreiben vor\-geschrieben, nur eine Religionsstunde pro Woche
angesetzt zu haben. Als Be\-strafung sollte die enge kirchliche Bindung
Högns in Form seines Chorregen\-tendienstes gelöst werden, den er dann
ab 6. Januar 1944 abgeben musste, ihn aber tatsächlich nur werktags
aufgab.

Die Auseinandersetzungen Högns mit dem örtlichen Jägerverband in den
Jahren 1935 und 1936, der wie alle Jägerverbände im „Reichsbund
Deutsche Jägerschaft“ eingegliedert war, zeigen Högn ebenfalls von
seiner aufmüpfigen Seite. In der Jägerei setzten die Nazis sehr früh
ihre radikalen Reformen um. Kein geringerer als der
\textit{Reichsjägermeister} Hermann Göring verkündete das neue
Jagdgesetz, das ab 18. Januar 1934 galt. Die Jäger mussten demnach den
Wild\-bestand in ihrem Revier genau schätzen und durften nur mehr einen
gewissen Prozentsatz des geschätzten Bestandes abschießen. Diese
Einschränkungen brachten für viele Jäger finanzielle Einbußen mit sich,
die manche sehr schmerzten. Viele Jäger mussten das Wildfleisch
verkaufen, um erst einmal die für das Jagdrevier zu entrichtende Pacht
– so war es auch in Högns Fall – zu erwirtschaften. Diese
Abschuss-Einschränkung versuchte August Högn zu umgehen, indem er den
Bestand in seinem Revier einfach viel zu hoch angab. Die stolze Zahl
von zwanzig starken Böcken übermittelte Högn 1935 den zu\-ständigen
Behörden, obwohl im Vorjahr nur ein einziger starker Bock in sei\-nem
Revier geschossen werden konnte. Prompt wurde Högn vom zustän\-digen
Hegeringführer Max Forster auf diese Übertreibung hingewiesen. Högn gab
bei dieser Zurechtweisung keineswegs klein bei, vielmehr muss er in
einem Brief an Forster, der leider nicht erhalten ist, noch ordentlich
nachge\-legt haben. Forster fühlte sich von diesem Brief, der
\textit{im ersten Zorn geschrieben }worden sei, so angegriffen, dass er
monierte, Högn wolle ihn \textit{durch persönliche Angriffe im
Vertrauen auf seine gewandte Feder niederkämpfen.} Forster folgerte aus
dem Brief, dass Högn \textit{eine Zusammenarbeit nicht wünsche und die
Tätigkeit der Vollzugsorgane nicht anerkennen wolle. }Auch 1936 ging
Högn auf Konfrontationskurs mit dem Jägerverband. Beim Pflichtschießen
am 30. Mai 1936 verwendeten zwar die Jäger Paukner und Völkl Högns
Gewehre, doch Högn selbst blieb zum anberaumten Termin ohne
Entschuldigung fern. Auch zum Wiederholungstermin am 15. Juni 1936
erschien er nicht.

% {\centering   [Warning: Image ignored]
% % Unhandled or unsupported graphics:
% %\includegraphics[width=5.004cm,height=6.78cm]{a3-img/a3-img014.jpg}
%  \par}
% {\centering
% August Högn 1944:
% \par}

% {\centering
% Das Bild eines eher im 
% \par}

% {\centering
% inneren Konflikt lebenden, gebrochenen Mannes
% \par}

% {\centering
% als das eines stolzen, überzeugten Nazis.
% \par}

Es ist deshalb nicht verwundlich, dass viele Bürger bereit waren, Högn
eine oppositionelle Einstellung zu den Nazis für das
Spruchkammergericht zu bescheinigen, darunter waren angesehene
Persönlichkeiten, wie der Pfarrer und die Bürgermeister von
Ruhmannsfelden, Zachenberg und Patersdorf. Auch drei zufällig befragte
Personen, die wahrscheinlich von dem Ermittler des Spruchkammergerichts
bei einer Visite auf der Straße vernommen wur\-den, schilderten Högn
als Gegner der Nazis. Als treuer Parteianhänger ist Au\-gust Högn auch
keinem der sechs zu dieser Thematik fast 60 Jahre nach Kriegsende
befragten Zeitzeugen in Erinnerung geblieben. Als fanatische
An\-hängerin der NS-Ideologie prägten sich dagegen einige der Befragten
die seit 1942 an der Volksschule unterrichtende Lehrerin Charlotte
Werner ein. In ihrem Unterricht mussten Schüler aus einem Kalender
Vorträge über heraus\-ragende Persönlichkeiten des NS-Regimes halten.
Vergleichbares ist aus Högns Unterricht nicht bekannt.

Rückschlüsse auf Högns negative Einstellung zu Hitlers Politik lassen
sich nicht zuletzt aus seinem in der Nachkriegszeit entstandenen
Geschichtswerk ziehen. Folgende Textpassage aus der \textit{Geschichte
von Zachenberg} lässt sogar einen Interpretationsspielraum in Bezug auf
seine anfangs positive Einstellung zur NS-Ideologie, die sich im Lauf
der Zeit zur Ablehnung verändert haben könn\-te: \textit{Die 1932/33
hereinbrechende Hitlerzeit hat zwar auf der einen Seite dieser
schlimmen Arbeitslosigkeit ein Ende gesetzt, aber auf der anderen Seite
den unheilvollen zweiten Weltkrieg heraufbeschworen, der das größte
Unglück, das je über ein Land und seine Bevölkerung kommen konnte, in
übervollem Maße über Deutschland ausgeschüttet hat. }Als ob er sich
selbst für seine anfängliche Gefolgschaft rechtfertigen wollte, führte
er die Bekämpfung der Arbeitslosigkeit durch Hitler als eine zu
würdigende Leistung an, geht aber sehr deutlich auf Distanz zur
späteren NS-Politik, die die \textit{Kriegsfurie}, – so nennt er den
Weltkrieg – verursacht hat, und dessen Folgen aufzählt. Wie ein
Läuterungsversuch erscheinen die ausgedehnten Passagen in seiner
\textit{Geschichte von Ruhmannsfelden}, in denen er den beiden
Widerständlern, Bürgermeister Sturm und Studienrat Leonhard Donauer,
die nur mit viel Glück der Exekution durch die SS entronnen waren, ein
Denkmal setzt.

Außer Frage steht, dass Högn kein Rassist war. Franz Danziger,
Kirchen\-chorsänger, Violinist und schließlich Högns Nachfolger als
Chorregent, war ein \textit{Halbjude}, wie es im Jargon der
Nationalsozialisten geheißen hätte. Er ver\-liert in seinen Memoiren
kein schlechtes Wort über Högn. Vielmehr ehrte er seinen ehemaligen
Lehrer, indem er einige seiner Kompositionen aufführte.

Aufgrund dieser Tatsachen könnte man meinen, Högn wäre ein
Wider\-ständler gewesen, er sei vom Spruchkammergericht zu Unrecht als
„Mitläufer“ eingestuft worden und in Wirklichkeit ein „Entlasteter“
gewesen. Einige De\-tails zu seiner Vergangenheit, die Högn von einer
ganz anderen Seite zeigen, hat er dem Spruchkammergericht jedoch
verschwiegen, etwa dass sein Schwiegersohn ein hoher Nazi-Funktionär
war.

Die Bewunderung, die Högn für seinen Schwiegersohn Dr. Karl
Schlum\-precht, besonders für dessen politische Karriere zeigte und das
gute Verhältnis zu ihm, sind Indizien dafür, dass Högn die Ideologie
des Nationalsozialismus nicht von Grund auf abgelehnt hat. Schlumprecht
wurde am 1. Oktober 1929 als zweiter Staatsanwalt nach Deggendorf
versetzt. Hier lernte er Högns Tochter Elfriede Kroiss kennen, die zu
diesem Zeitpunkt noch mit dem Deggendorfer Bierbrauer und Gastwirt
Johann Kroiss verheiratet war. Diese Ehe wurde geschieden und am 4.
Juni 1932 heirateten Karl Schlumprecht und Elfriede Högn. Bereits am 1.
Dezember 1930 trat Schlumprecht – er war seit 1920 Mitglied der
deutschnationalen Volkspartei – in die NSDAP ein. Ab 1931 setzte er
sich aktiv für die propagandistischen Zwecke der Partei als
so\-genannter „Gauredner“ und ab 1933 als „Reichsredner“ ein. Nach der
Macht\-übernahme der NSDAP war er zunächst ab 11. März 1933
Personalreferent des bayerischen Innenministers Wagner und von 26.
April 1933 bis April 1937 erster nationalsozialistischer
Oberbürgermeister von Bayreuth. Anschließend arbeitete er als
Ministerialdirektor im Finanzministerium in München und wurde zum
Leiter der Wirtschaftsabteilung beim Chef der Zivilverwaltung der 10.
Armee mit Dienstsitz in Radom/Polen (1941) und daraufhin zum Leiter der
Wirtschaftsabteilung beim Militärbefehlshaber in Belgien und
Nordfrank\-reich (1941-1943) ernannt. Zurückgekehrt nach München
übernahm er vom 16. Juli 1943 bis 21. April 1944 die Leitung des
bayerischen Wirtschaftsmini\-steriums. Im Laufe seiner
Bilderbuch-Karriere in NS-Staat war er zeitweise Mitglied des
bayerischen Landtags, Mitglied des Reichstags und SS-Brigade\-führer.
Vor allem die berufliche Karriere Schlumprechts war der Grund,
wes\-halb Högn auf seinen Schwiegersohn stolz war. Er wäre es
sicherlich nicht gewesen, hätte er die NS-Ideologie in den Grundzügen
abgelehnt.

% {\centering   [Warning: Image ignored]
% % Unhandled or unsupported graphics:
% %\includegraphics[width=7.124cm,height=5.292cm]{a3-img/a3-img015.jpg}
%  \par}
% {\centering
% Dr. Karl Schlumprecht mit Frau Frieda, geborene Högn
% \par}

Högn wusste den guten Draht zu einem hohen NS-Funktionär zu nutzen –
nicht zum eigenen Vorteil, sondern zu dem der Allgemeinheit. Der
damalige Ruhmannsfeldener Bürgermeister Sturm stellte am 1. September
1938 an den Ministerialdirektor Schlumprecht im Finanzministerium einen
Antrag auf Unterstützung bei der Finanzierung des Außenputzes der
Turnhalle. Der Ver\-merk auf dem Brief \textit{In Abdruck an H.
Oberlehrer Högn, hier zur Kenntnis} zeigt ein\-deutig, dass diese
Petition durch Högn, den ehemaligen Turnvereinsvorstand und
Veranstalter von Benefizkonzerten zugunsten der Errichtung der
Turn\-halle, eingefädelt worden war.

Im Blick auf Högns NS-Vergangenheit gibt uns sein \textit{Weihegesang
Es-Dur} für Chor und eine Blechbläserformation, eine Komposition mit
stark NS-propa\-gandistischen Zügen, ein Rätsel auf. Nur schwer lässt
sich aus heutiger Sicht der \textit{Vertrauensmann aller hiesigen
Antifaschisten,} wie sich Högn selbst bezeichnet hat, mit dem –
überspitzt formuliert – „NS-Komponisten Högn“ vereinbaren. Wie kann ein
Mann, der seine oppositionelle Einstellung zum NS-Staat durch\-aus
glaubhaft und manche Aspekte auch beweisbar darlegte, eine solche
Pro\-paganda-Musik schreiben? Wäre nicht deshalb eine Einteilung Högns
in die Gruppe der „Aktivisten“ durch das Spruchkammergericht Viechtach
eher ge\-rechtfertig gewesen, als in die Gruppe der „Mitläufer“? Das
Urteil des Spruch\-kammergerichts Viechtach, dass Högn dem
\textit{Nationalsozialismus passiv gegenüber\-stand und sich politisch
nicht aktiv betätigt }habe, ist demnach schlichtweg falsch.

% {\centering   [Warning: Image ignored]
% % Unhandled or unsupported graphics:
% %\includegraphics[width=8.472cm,height=5.68cm]{a3-img/a3-img016.png}
%  \par}
% {\centering
% Sopran-Stimme des Weihegesang Es-Dur
% \par}

Historisch gesehen ist der \textit{Weihegesang} ein kleiner
Sensationsfund. Der sparsame Högn hat nach Kriegsende und dem Untergang
der Nazis auf den Rückseiten der Notenblätter des \textit{Weihegesangs}
pikanterweise das Grablied \textit{Näher mein Gott zu dir!} notiert und
den \textit{Weihegesang} lediglich durchgestrichen. Wahr\-scheinlich
war nach Kriegsende das Notenpapier so knapp, dass Högn es sich nicht
erlauben konnte, diese brisante Komposition mit noch freien Rückseiten
zu vernichten. Der \textit{Weihegesang} überdauerte so in der Rubrik
„Trauergesänge“ des Notenschranks als \textit{Näher mein Gott zu Dir}
die NS-Zeit um fast 60 Jahre.

Die durch Regentropfen verwischte Tinte des \textit{Weihegesangs}
verrät, dass die Komposition im Freien aufgeführt wurde. Der Einsatzort
war wahrscheinlich die von der Wehrmacht veranstalteten weltlichen
Trauerfeiern für Soldaten, die vor den so genannten christlichen
„Heldengottesdiensten“ stattfanden. Da in den seltensten Fällen der
Leichnam der gefallenen Soldaten überführt werden konnte, zog bei
dieser Trauerfeier eine Abordnung von bis zu zehn Soldaten auf
Fronturlaub nicht zum Grab sondern zum Kriegerdenkmal, wo
wahr\-scheinlich auch der \textit{Weihegesang} zu hören war. Diese
weltliche Trauerfeier ist ein Beispiel von vielen, wie sich der
Nationalsozialismus christlicher Riten be\-diente und in seine neu
kreierten Feiern einbaute. Ähnlich eklektizistisch ging Högn bei seiner
Komposition vor: Er lehnte den \textit{Weihegesang} stark an das
christliche Grablied an.

Der \textit{Weihegesang} klingt genauso farbig, lieblich und an manchen
Stellen kitschig wie die vier \textit{Grablieder Nr. 1-4}.

Auch wenn Högn im Unterschied zu den Grabliedern einzelne Phrasen im
Chor im Unisono und die Blechbläser rhythmisierter setzte, lässt sich
die musikalische Ähnlichkeit nicht leugnen. Es scheint, als hätte Högn
seine NS-Musik nicht anders wie in seinem lang eingeübten
Kirchenmusikstil kompo\-nieren können. War der \textit{Weihegesang}
deshalb schlechte Propaganda-Musik? Das Gegenteil ist der Fall: Der
belanglose und liebliche Tonfall verleiht dem Gesang eine positive
Grundstimmung, verbreitet Optimismus und suggeriert dem Hörer eine
„heile Welt“. Die rhythmisierte Bläserbegleitung lässt den
\textit{Weihegesang} von fern an einen Marsch erinnern, das für die
propagandistischen Zwecke zuverlässigste und deshalb von den Nazis am
meisten eingesetzte Musikgenre. Ein „propagandistischer Kunstgriff“ ist
Högn bei der Vertonung der vorletzten Textzeile jeder Strophe gelungen.


Högn verzichtet in diesen Takten auf den vierstimmigen Satz und führt
alle Chorstimmen im Unisono. Der Effekt, dass alle Sänger eine Melodie
sin\-gen, suggeriert einerseits ein Gemeinschaftsgefühl, denn jeder
Zuhörer könnte sich an diesen Stellen dem Chor anschließen und wie alle
anderen Sänger \textit{mit einer Stimme} singen. Andererseits bewirkt
der plötzliche Übergang vom Uni\-sono in den vierstimmigen Satz von der
vorletzten zur darauf folgenden letz\-ten Textzeile der jeweiligen
Strophen eine große klangliche Steigerung und damit eine besondere
Hervorhebung der letzten Textzeile. So bleiben dem Zuhörer vor allem
die letzte und im Fall des \textit{Weihegesangs} auch die für die zu
übermittelnde Botschaft wichtigsten Zeile einer Strophe, vor allem aber
der Schluss des Textes in Erinnerung: \zitat{Ein Geist
\textup{[}}NS]\zitat{ lässt Deutschland neu erstehn.}

% Die Ihr dereinst fürs Vaterland gezogen in die Schlacht\newline
% und dort das teure Leben uns zum Opfer habt gebracht.\newline
% Die Ihr mit Eurem Herzensblut die Wohlstatt habt getränkt.\newline
% An Euch, die selge Heldenschar, die Heimat treu gedenkt.

% \newline
% Wohl droht aufs neu ob unserm Haupt die schwierge Wetternacht.\newline
% Verrat und Meineid haben freudlos, leidlos uns gemacht.\newline
% Es ist, als ob umsonst vermachet ihr den Lebenshauch.\newline
% Geborsten ist der Ehrenschild und tot der Väter Brauch.\newline
% \newline
% Soll das der schnöde Dank für Euer Lebensopfer sein,\newline
% dass zaghaft, wehrlos wir dem feigen Untergang uns weihn?\newline
% Steigt ein, der aus Walhallas Höhe, hoch ist es Zeit zur Tat,\newline
% Ihr deutschen Recken! Ein Geist lässt Deutschland neu erstehn!

Högns \textit{Weihegesang} transportierte anhand seines Texts eines
unbekannten Au\-tors folgende Botschaft an die Trauergäste: Angesichts
der schwierigen Kriegslage (\textit{Wohl droht aufs neu … die schwierge
Wetternacht }– Stalingrad, Z. 5) und aufgrund der noch immer zu
geringen Anstrengungen für den Krieg von Seiten der Bevölkerung,
(\textit{Verrat und Meineid haben freudlos, leidlos uns gemacht. }Z. 6;
\textit{zaghaft, wehrlos} Z. 10) besteht die Gefahr, dass Deutschland
den Krieg ver\-liert. (\textit{feiger Untergang }Z. 10). Der Tod des
Angehörigen wäre in diesem Fall um\-sonst gewesen (\textit{Es ist, als
ob umsonst vermachet ihr den Lebenshauch} Z. 7;\textit{ Soll das der
schnöde Dank für Euer Lebensopfer sein} Z. 9). Als Rettung werden
Hitler (\textit{der aus Walhallas Höhe }Z. 11) und die
nationalsozialistischen Bewegung (\textit{ein Geist }Z. 12)
angepriesen, die den Menschen neuen Mut geben (\textit{hoch ist es Zeit
zur Tat} Z. 11) und die wie „Götter“ Deutschland \textit{neu erstehn}
(Z. 12) lassen. Der hochgra\-dig propagandistische Text des
Weihegesangs benützt also den Tod eines Sol\-daten, um einen
Durchhalteappell an die versammelten Trauergäste zu rich\-ten.

Nach dem Inhalt dieses Textes zu urteilen, kann der \textit{Weihegesang}
erst kurz vor Ende des Kriegs geschrieben worden sein. Diese
Propaganda-Musik ent\-stand deshalb möglicherweise um den Zeitpunkt
herum, als Högn die Aus\-übung des Chorregentendienstes ab dem 6.
Januar 1944 durch die NS-Behör\-den untersagt wurde. Vielleicht
komponierte Högn den \textit{Weihegesang} nur, um weiterhin den
Chorregentendienst ausüben zu können. Dass seit Högns Ent\-lassung nur
noch stille Messen bei den Heldengottesdiensten gelesen wurden, fiel
auch dem Wehrführer der Feuerlöschpolizei und Stellvertreter des
Orts\-gruppenleiters Josef Hengkofer auf, der in seinem Brief vom 12.
Januar 1944 den Regierungspräsidenten von Regensburg bat, Högn die
Ausübung des Chordiensts wieder zu erlauben. Festliche Musik für die
NS-Heldenzeremo\-nie war also durchaus ein wirksames Druckmittel, die
Nazis kooperativ zu machen. Vielleicht versuchte Högn tatsächlich, mit
Hilfe des \textit{Weihegesangs} den Veranwortlichen seinen guten Willen
zu beweisen, damit die Sanktionen rück\-gängig gemacht wurden oder
härtere Sanktionen, wie etwa eine Entlassung aus dem Schuldienst,
ausblieben.

\section{Heimatforschung als Betätigungsfeld im Ruhestand}

Die Heimatkunde war nicht nur ein Hobby des Pensionisten Högn. Nach der
Pensionierung verfasste Högn drei Abhandlungen zu heimatkundlichen
The\-men. Heimatkundliche Forschung gehörte traditionsgemäß zum
Aufgabenfeld der Lehrer, wie eine Aktion der Regierung von Niederbayern
aus der Zeit der Weimarer Republik mit dem Motto \textit{Pflege des
Heimatgedankens} zeigt. Sie hatte das Ziel, durch Heimatkunde, den
Patriotismus und das Nationalbewusstsein der Bürger zu stärken, um
damit einen Beitrag zur Lösung der nationalen Pro\-bleme zu leisten.
Ein Rundschreiben vom 12. November 1925, das auch Högn vorlag, mit dem
Betreff \zitat{Anlegung gemeindlicher Ortsgeschichten} macht
die Intention dieser Aktion in wenigen Sätzen deutlich:
\zitat{Die Voraussetzung für den Aufstieg unseres Volkes aus
dem jetzigen Tiefstande ist die Rückkehr zu deutscher Ein\-fachheit,
Zucht und Sitte. Dazu muss der Sinn für die Heimaterde geweckt} und
\zitat{die Liebe zum Vaterlande entzündet [...] werden. Auf
diesen Grundlagen baut sich neben der kör\-perlichen und geistigen
Ertüchtigung der Jugend die Zukunft des deutschen Reiches auf.} Vor
allem \zitat{die Herren Lehrer [...] sollten“
}daher\zitat{ „ohne weiteres Säumen und mit allem Nach\-druck
[...] die Vorbereitung für die Herstellung der Ortsgeschichten}
aufnehmen. Dass die angetragenen Vorhaben eine gewisse Verbindlichkeit
hatten, zeigt schon die Aufforderung, dass \zitat{bis zum 1.
April 1928 berichtet werden wolle, [...] in welchen Gemeinden die
Vorarbeiten in Angriff genommen worden sind (Anrede des Bearbeiters).
}Högn hat sich dieser Aufgabe angenommen, wie die vielen
heimatkundlichen Zeitungsartikel zeigen, die in der Zeit von 1926 bis
1928 erschienen sind. Doch das waren nur einzelne heimatkundliche
Beiträge. Die eigentlich gefor\-derte komplette Ortsgeschichte blieb
aus. Im Ruhestand hatte Högn schließ\-lich Zeit, die geforderte
umfassende Ortschronik zu schreiben. Bei der Ge\-schichte von
Zachenberg etwa, gibt es ganz konkrete Anhaltspunkte, wonach die
Initiative zur Inangriffnahme der Arbeit nicht von Högn selbst ausging,
sondern eine an ihn herangetragene Bitte den Grund für die Entstehung
der \textit{Geschichte} lieferte.

Zwei Jahre nach Högns endgültiger Pensionierung, erschien 1949 die
\textit{Ge\-schichte von Ruhmannsfelden.} Dem rüstigen Rentner muss es
eher leicht gefallen sein, Material für seine Arbeit zu finden, doch
hatte er doch bereits in den vergangenen Jahrzehnten viele Fakten
gesammelt und brauchte die vielen bereits existierenden Einzelbeiträge
zur Heimatgeschichte nur noch in einem Werk zusammenfassen. Die
geschichtliche Entwicklung zweier wichtiger In\-stitutionen am Ort
hatte Högn schon vor dem Zweiten Weltkrieg bearbeitet:

Die Entwicklung der Schule war in einer Beilage zum \textit{Deggendorfer
Donau\-boten} 1927 erschienen und anlässlich der 100-jährigen
Einweihungsfeier der Pfarrkirche wurde 1928 bis 1929 in mehreren Teilen
die Geschichte der Pfarr\-kirche St. Laurentius im \textit{Viechtacher
Tagblatt} abgedruckt. Information über die Entstehung des Namens
„Ruhmannsfelden“ oder den Zeitpunkt der Markt\-rechtsverleihung waren
Thema zweier Zeitungsartikel aus dem Jahre 1926 und fanden an
exponierter Stelle Eingang in die \textit{Geschichte von
Ruhmannsfelden}. Högn hatte sich schon 1922 zur Frage des Ursprungs des
Ortsnamens an den Strau\-binger Historiker Dr. Joseph Keim gewandt, der
die noch heute gültige Er\-klärung gab, wonach der Ort nach dem
Rodungsarbeiter „Rumar“ benannt wurde. Davor wurde der Name etwas
trivial mit „Ruht der Mann in Felde“ er\-klärt. Auch versuchte Högn
sowohl in dem erwähnten Artikel als auch in der \textit{Geschichte von
Ruhmannsfelden} dem Mythos, dass einst ein Schloss am Leitenberg
bestand, entgegenzuwirken. Trotz einiger Recherchearbeiten ist es
unverkenn\-bar, dass Högn besonders zur frühen Geschichte auf sehr
wenige Infor\-mationen zurückgreifen konnte. Das einleitende Kapitel
\textit{Wie hat es vor seiner Entstehung ausgesehen?} ähnelt in seinem
Aufbau mehr einem Märchenanfang als dem Beginn einer historischen
Abhandlung und entbehrt sicherlich aller wis\-senschaftlichen
Grundlagen. Wegen fehlender Dokumente musste von Na\-men wie etwa
\textit{Ruhmannsfelden }oder \textit{Laurentius-Kirche} möglichst auf
die geschicht\-liche Entwicklung geschlossen werden. Manchmal folgerte
Högn aus diesen Anhaltspunkten etwas zuviel. Es stimmt sicher, dass
alle dem Hl. Laurentius geweihten Kirchen außerhalb der Ortschaften
standen, dass aber die Lauren\-tius Kirche in Ruhmannsfelden zuerst aus
Holz war, unter den Aldersbachern dann aus Stein und dass dann öfter
Gottesdienste stattfanden, wie er schreibt, ist wohl reine Spekulation.
Als Lückenfüller dürften die Kapitel über die Na\-men der
Ruhmannsfeldener Fluren und Gassen und das Kapitel \textit{Höhenlagen
in unserem Heimatgau und Barometerstand} gedient haben.

% \begin{flushleft}
% \tablefirsthead{}
% \tablehead{}
% \tabletail{}
% \tablelasttail{}
% \begin{supertabular}{c}
%   [Warning: Image ignored] % Unhandled or unsupported graphics:
% %\includegraphics[width=5.271cm,height=7.209cm]{a3-img/a3-img017.png}
%    [Warning: Image ignored] % Unhandled or unsupported graphics:
% %\includegraphics[width=4.868cm,height=7.216cm]{a3-img/a3-img018.png}
%  \\
% \end{supertabular}
% \end{flushleft}
% {\centering
% Die Titelblätter der Heimatgeschichten
% \par}

Die Entstehung der \textit{Geschichte und Chronik der freiwilligen
Feuerwehr Ruhmanns\-felden} ist eng mit dem Ende von Högns
Schriftführertätigkeit bei der Feu\-erwehr verbunden. Am Stefani-Tag
1950 wurde Johann Freisinger zum Schriftführer der Feuerwehr gewählt
und löste August Högn nach 40-jähriger Dienstzeit ab. Die
\zitat{in dankbarer Erinnerung} der Feuerwehr gewidmete
Ge\-schichte hat Högn geschrieben, um seine Tätigkeit als Schriftführer
abzu\-run\-den und sein gesammeltes Wissen in diese Arbeit einfließen
zu lassen. Ein passender Übergabetermin der Chronik an die Feuerwehr
wäre der zur Verab\-schiedung von Högn eigens veranstaltete Ehrenabend
am 11. März 1951 ge\-wesen, doch die Chronik wurde dem letzten Eintrag
zufolge erst nach dem 1. April 1951 fertig. Die Arbeit war schon
angekündigt und fand lobende Er\-wähnung in der Abschiedsrede des
Feuerwehrkommandanten. Die fertige Chronik überreichte Högn dann kurze
Zeit später persönlich seinem Nach\-folger als Schriftführer.

Die Abhandlung ist nach Jahren gegliedert und lässt daher oft
Einzelfakten unverbunden nebeneinander stehen, was das Verständnis beim
Lesen er\-schwert. Sie ist wohl eher als eine Stoffsammlung zur
weiteren Bearbeitung anzusehen, denn als ein unterhaltsamer Lesestoff
für die breite Bevölkerung und war sicher nie für den Druck bestimmt.
Lediglich die länger ausfor\-mulierte Gründung der Feuerwehr und der
Bericht über den lange dauernden und deshalb in mehreren Jahren
auftauchenden Streit der Gemeinden Ruh\-mannsfelden und Zachenberg um
eine gemeinsam angeschaffte Feuerwehr\-spritze bleiben dem Leser in
Erinnerung und gehen nicht in der Flut von Einzelinformationen unter.

Ein auswärtiger, nicht in der Gemeinde Zachenberg ansässiger Bürger
lieferte die Initiative zur Entstehung von Högns
\textit{Heimat-Geschichte der Gemeinde Zachenberg}. Der Finanzzollrat
Anton Trellinger fertigte einen 63-seitigen Ak\-ten-Auszug über die
Gemeinde Zachenberg an. Da Trellinger die Gemeinde Zachenberg nur
flüchtig kannte und gesundheitlich angeschlagen war, bat er in einem
Brief die Gemeinde Zachenberg um einen ortskundigen Fachmann, der seine
Arbeit ergänzen und fortführen könnte. Den Sachbearbeitern bei der
Gemeinde dürfte es nicht schwer gefallen sein, Trellinger die gesuchte
Person zu nennen: August Högn. Lange bevor Anton Trellinger der
Gemeinde seine Arbeit übersandte, nämlich am 25. Februar 1952, und am
selben Tag Högn persönlich anschrieb, hatte dieser schon mit der
Recherche begonnen und den Mettener Benediktiner-Pater Wilhelm Fink von
der beabsichtigten Geschichte von Zachenberg in Kenntnis gesetzt und um
wissenschaftliche Betreuung gebeten.

Trellinger lieferte mit seinen Archivauszügen nicht nur einen wichtigen
Grundstock für die \textit{Heimat-Geschichte der Gemeinde Zachenberg},
sondern er gab ihm auch Tipps und Literaturhinweise. Nach fast
zweijähriger Arbeitszeit wur\-de die \textit{Geschichte von Zachenberg}
an Weihnachten 1953 fertig und an Wilhelm Fink zum Korrekturlesen
übersandt. Fink hatte besonders zum ersten Teil der Abhandlung einige
Änderungsvorschläge. Es ist deshalb nicht verwunderlich, dass sich die
ersten Teile der \textit{Geschichte von Ruhmannsfelden} und der
\textit{Geschichte von Zachenberg} in Aufbau und Darstellung ziemlich
stark unterscheiden, obwohl beide Teile die geschichtliche Entwicklung
von den Anfängen der beiden be\-nachbarten und sich unter den gleichen
Herrschaftsverhältnissen entwickeln\-den Gemeinden zum Thema haben.
Fink hat offensichtlich die \textit{Geschichte von Ruhmannsfelden}
nicht Korrektur gelesen. Im von Fink als druckreif befundenen zweiten
Teil geht Högn auf die 15 Dörfer, 13 Weiler und 10 Einöden, also auf
die insgesamt 38 Ortschaften der Gemeinde Zachenberg im Einzelnen ein.

Die Kapitel über die einzelnen Ortschaften ähneln mehr genealogischen
Studien als historischen Abhandlungen. Manchmal lassen sich aus diesen
Ka\-piteln Stammbäume einzelner Bauernfamilien über Jahrhunderte hinweg
able\-sen. Meist jedoch konnten die Angaben aus den sehr alten
Aktenauszügen Trellingers nicht mit den neueren Informationen aus dem
Gemeindearchiv oder von den befragten Bewohnern der jeweiligen
Ortschaft bruchlos anein\-ander gereiht werden. Die sehr schematische
Beschreibung der kleinsten Ort\-schaften selbst – sie beginnt immer mit
der Herkunfts-Erklärung des Ortsna\-mens und endet nach Vorstellung der
einzelnen Bauernfamilien mit Angabe der damals aktuellen Einwohner- und
Häuserzahl – wird durch Sagen und Erzählungen aufgelockert, die Högn
von älteren Bewohnern der einzelnen Orte erzählt bekam und in Form von
Nacherzählungen wiedergibt. Viele der von Wilhelm Fink vorgeschlagenen
Ergänzungen zum dritten, allgemeinen Teil – hier geht es in erster
Linie um die damals aktuelle Organisation der Gemeinde – hat Högn nicht
angenommen, weil beiden Gemeinden, Zachen\-berg und Ruhmannsfelden,
doch sehr ähnliche Strukturen aufwiesen und er bereits Einiges in der
Geschichte von Ruhmannsfelden erwähnt hatte, wie zum Beispiel die aus
der Pfarrei hervorgegangenen Priester, die Lehrer oder die Polizei.
Deshalb erörtert er hier hauptsächlich die aktuellen wirtschaftli\-chen
und infrastrukturellen Aspekte der Gemeinde Zachenberg.

Als er die überarbeitete Fassung der Geschichte kurz vor Ostern 1954 der
Gemeinde Zachenberg übergab, rechnete er sicher mit einer schnellen
Druck\-legung seiner Arbeit, ähnlich wie bei der \textit{Geschichte von
Ruhmannsfelden}, sonst hätte er sich nicht schon von der Druckerei
Michael Laßleben in Kallmünz ein Angebot machen lassen und sich sogar
schon einen Werbetext überlegt.

Erst am 7. Mai 1956 versicherte der Zachenberger Bürgermeister Bielmeier
Högn auf Anfrage, dass der Gemeinderat über die Veröffentlichung der
\textit{Ge\-schichte von Zachenberg} in der nächsten Sitzung beraten
würde. Die Gemeinde\-vertreter hatten wohl von Anfang an kein Interesse
an einer Drucklegung. Kein Wort über den Druck des Buches ist im
Protokoll der Gemeinderats\-sitzung vom 17. Mai 1956 zu finden.
Stattdessen wurde beschlossen, Högn zum Ehrenbürger der Gemeinde
Zachenberg zu ernennen und ihm als \zitat{Ent\-schädigung
}für seine Bemühungen ein \zitat{Geldgeschenk im Ermessen des
1. Bürgermei\-sters} zukommen zu lassen. Trotz dieser Ehren dürfte die
immer noch nicht eingeleitete Veröffentlichung der \textit{Geschichte
von Zachenberg} für Högn eine Ent\-täuschung gewesen sein. Noch an
seinem 80. Geburtstag, also über vier Jahre nach Fertigstellung, machte
er sich Hoffnung auf eine baldige Drucklegung, die bis zum heutigen Tag
auf sich warten lässt.

August Högn war nicht nur auf heimatkundlichem Gebiet schriftstellerisch
tätig. Es gibt Anzeichen dafür, dass er sich auch mit musikhistorischen
The\-men beschäftigt haben könnte. Ein Briefwechsel aus dem Jahr 1947
mit Fer\-dinand Haberl, Direktor der Kirchenmusikschule in Regensburg,
in dem es um die Entstehung und Herkunft des Weihnachtsliedes
\textit{Stille Nacht} geht, stützt diese Annahme.

\section{Die \textit{Josephi}{}-Messe}

Die \textit{Josephi}{}-Messe ist ein Spätwerk Högns. Der genaue
Zeitraum, wann das Werk begonnen und wann es vollendet wurde, ist nicht
bekannt. Doch wissen wir, dass das nach der Messe geschriebene
\textit{Marienlied Nr. 12 F-Dur op. 63} 1953 fertiggestellt wurde.
Wahrscheinlich arbeitete er beim Komponieren der Messe gleichzeitig
auch an einem oder mehreren Geschichtswerken.

Namensgeber der Messe war Högns Hausarzt Dr. Joseph Stern. Als Högn die
Messe fertig hatte, bot er Stern mit Verweis auf ihre gute Freundschaft
an, der neuen Messe einen Namen zu geben, da er zwar eine Messe
komponiert hatte, aber keinen Namen für sie hatte. Stern machte den
Vorschlag, die Messe nach seinem Namenspatron zu nennen. Möglicherweise
beeinflusste der Liebhaber alter Sprachen Högn auch darin, den Titel
der Josephi-Messe lateinisch zu formulieren, sie also \textit{Missa in
honorem Sancti Josephi} zu nennen.

Eine Reihe von Dokumenten belegt eine rege Aufführungspraxis über
Jahrzehnte hinweg. Die erste uns bekannte Aufführung fand am 14. Juni
1953 zur Installation des neuen Pfarrers Franz Seraph Reicheneder
statt. Ein Zei\-tungsartikel bescheinigte dem
\zitat{klangvollen Chor} unter der Leitung des Kompo\-nisten
den \zitat{ergreifenden} Vortrag der \zitat{Missa
St. Josephi.}

Wie es zu der Aufführung der Messe in der Kirche St. Magdalena in
Platt\-ling am 19. März 1957 kam, ist nicht bekannt. In einer
Ankündigung am Tag der Aufführung in der Plattlinger Zeitung wird Högns
\zitat{romantisch, liebenswerter Kompositionsstil }gelobt.
Weiter heißt es: \zitat{Das Benediktus für Sopransolo und
Chor wird sicher allen gefallen, die die Musik nicht erst über den
Verstand, sondern gleich ins Herz fließen lassen wollen. }Auf einer
Postkarte bedankte sich der Chorleiter Gustl Gudmer bei Högn für das
Ausleihen des Notenmaterials und versicherte ihm: \zitat{Sie
hat am Josephsfest gut gefallen. Ich habe verschiedene Stimmen darüber
gehört. Unser }\zitat{geistlicher Rat hier hat sich sehr
gefreut über die Aufführung.}

Dem Engagement von Högns ehemaliger Chorsängerin Maria Schröck ist es zu
verdanken, dass die Messe am 19. März 1959 in Deggendorf vom Chor an
St. Martin aufgeführt wurde. Maria Schröck, die 1950 nach Deggendorf
geheiratet hatte und seitdem Mitglied im Kirchenchor von St. Martin
war, machte ihrem Chorregenten Fritz Goller, dem Neffen von Vinzenz
Goller, den Vorschlag, eine Messe von August Högn einzustudieren.
Goller ging auf diesen Vorschlag anfangs nicht ein. Erst als Maria
Schröck ihm drohte, eine Aufführung der Messe in der Pfarrkirche Mariä
Himmelfahrt anzustreben, ging Goller auf ihr Anliegen ein. 

Zur „Josephi“-Messe äußerte sich Fritz Goller in einem Zeitungsartikel,
der einen Tag vor der Aufführung erschien:\zitat{ }Die
Messe\zitat{ verrät gediegene Hand\-werkskunst, die sich in
einer sauberen satztechnischen Handschrift äußert, Sinn für
harmo\-nische Farbigkeit hat, ohne in spätromantische Chromatik
abzugleiten, und sich der Mittel des Kontrastes in der
Gegenüberstellung kraftvoller Unisoni und motivisch aufgelockerter
Chorsätze bedient. }August Högn war bei der Aufführung in Deggendorf
anwe\-send. Man gratulierte ihm zu seiner Komposition.

Beim Pfarrerwechsel in Ruhmannsfelden 1974 kam die
\textit{„Josephi“-Messe} zweimal zum Einsatz. Sie erklang unter der
Leitung von Karl Geiger zur Ver\-abschiedung des alten Pfarrers
Reicheneder am 4. August 1974 und zur Installation des neuen Pfarrers
Otto Krottenthaler am 29. Sepember 1974.

In dieser Messe ist Högn ein besonderes Kunststück gelungen: Die
ge\-samte Komposition baut auf ein einziges Motiv aus vier Tönen auf.
Dieses sogenannte Soggetto ist in allen Sätzen der Messe außer im
Benedictus zu finden. Die Messe klingt deshalb keinesfalls
gleichförmig, denn es gelang Högn, jedem Satz einen unverwechselbaren
musikalischen Ausdruck zu ver\-leihen, obwohl er immer auf dieses eine
Soggetto zurückgriff. So klingt es im Kyrie ruhig und fließend, im
Sanctus feierlich und erhaben. Fast nebenbei und sich nicht in den
Vordergrund spielend wird es im Gloria und Credo in eine Vielzahl von
anderen musikalischen Einfällen ungezwungen eingebaut. Melan\-cholisch
und beinahe düster lässt es dagegen die Moll-Eintrübung zu Beginn des
Agnus Dei erscheinen.

Viele verschiedene stilistische Ebenen lassen sich in der
\textit{„Josephi“-Messe} beobachten. Man gewinnt fast den Eindruck,
Högn wollte mit dieser Messe am Ende sein bisheriges Werk noch einmal
in einer Komposition zusammen\-fassen. Die umfangreiche imitatorische
Arbeit im Kyrie verweist auf sein Frühwerk im Stil des strengen
Cäcilianismus. Die harmonisch farbige und chromatisch angehauchte
mittlere Kompositionsphase schimmert an vielen chromatischen
Stimmführungen besonders im Kyrie und Agnus Dei durch. Hier lässt sich
Högns Vorbild Peter Griesbacher, der ein großer Wagner-Ver\-ehrer war,
erkennen. Manche Passage des Credo, der Anfang des \textit{Pleni sunt
coeli} im Sanctus und das gesamte Benedictus sind stark von der
Volksmusik be\-einflusst, die Högn sein ganz Leben lang beschäftigt
hat. Die Messe ist Högns reifste Komposition, die einen Vergleich mit
Werken bedeutender Kompo\-nisten nicht zu scheuen braucht.

\subsection{Unwürdiger Abschied als Chorregent}

Ähnlich große Wellen wie die Entlassung des Chorregenten Max Rauscher im
Jahr 1927 schlug 1953 August Högns Ausscheiden als Kirchenchorleiter.
Dem Chorleiterwechsel ging ein Pfarrerwechsel voraus. Am 17. Februar
1953 ver\-starb Pfarrer Jakob Bauer. Der neue Pfarrherr Franz Seraph
Reicheneder wur\-de am 26. Mai 1953 in Ruhmannsfelden empfangen und am
14. Juni schließ\-lich feierlich mit einer Aufführung der Josephi-Messe
installiert. August Högn wollte eigentlich bei Antritt des neuen
Pfarrers seinen Rücktritt als Chorregent und Organist bekannt geben,
doch der neue Pfarrer lehnte sein Ersuchen anstandsgemäß ebenso ab wie
einige Zeit vorher Pfarrer Jakob Bauer und in der Übergangszeit
Pfarrprovisor Georg Huber. Högns Rücktrittsabsichten sprachen sich
sogar herum, so dass Josef Brunner – er war während der Kriegszeit
Aushilfsorganist in Ruhmannsfelden – eine Bewerbung für die
möglicherweise frei werdende Chorregentenstelle bei der
Kirchenverwaltung einreichte.

Umso verwunderlicher ist die Dramatik mit der sich Högns Ausscheiden als
Chorregent dann tatsächlich vollzog: Nachdem seine Bitte um Rücktritt
von Pfarrer Reicheneder abgelehnt worden war, scheint Högn mit einer
län\-gerfristigen Dienstzeit als Chorregent und Organist gerechnet zu
haben, sonst hätte er nicht schon 1953 zwei Marienlieder zum
marianischen Jahr 1954 komponiert. Reicheneder aber, von Högns
Amtsmüdigkeit überzeugt, war wahrscheinlich davon ausgegangen, dass der
75-jährige bald abdanken würde. Vielleicht hatte er auch deswegen der
von vorneherein als Nachfolgerin favo\-risierten Maria Reisinger eine
baldige Anstellung in Aussicht gestellt. Maria Reisinger, ein
Waisenkind, war Reicheneders Ziehtochter, um deren Ausbil\-dung er sich
gekümmert hatte. 

Mit Sicherheit hat Reicheneder die Ablehnung des Rücktrittsgesuchs aus
folgenden zwei Gründen bereut: Zum einen war für seine Ziehtochter
keine Anstellung in absehbarer Zeit greifbar, zum anderen leisteten zum
damaligen Zeitpunkt Högn und sein Chor äußerst schlechte Darbietungen.
Die Auffüh\-rung der \textit{Josephi}{}-Messe durch den
\zitat{klangvollen Chor} zu Reicheneders Installation war
nicht repräsentativ für den alltäglichen Kirchenmusikbetrieb, da hier
viele Aushilfen mitwirkten und hatte möglicherweise bei Reicheneder
einen fal\-schen Eindruck hinterlassen.

Der Chor, der während der Woche sang, bestand nur aus vier Personen: Den
drei Sängerinnen Mathilde Glasschröder, Barbara Essigmann, Theres
Raster, und August Högn, der selbst eine Männerstimme übernahm. Vor
al\-lem die Heterogenität der einzelnen Stimmen, die durch die
Kleinstbesetzung hervortrat, hinterließ bei so manchem Kirchenbesucher
einen schlechten Eindruck. Wurde Mathilde Glasschröder als sängerische
Naturbegabung mit großer Stimme, vergleichbar einer Opernsängerin,
beschrieben, blieb Theres Raster als außerordentlich schlechte Sängerin
mit \textit{fürchterlichem} Stimmklang in Erinnerung. Die alternde
Stimme Högns scheint sich ebenso wenig in einen Gesamtklang eingebunden
zu haben. Diese Besetzung lässt sich deutlich am \textit{Marienlied Nr.
12 op. 63}, Högns letztem Werk seiner Chorregentenzeit ablesen. Das
Stück ist größtenteils zweistimmig solistisch und nur in den letzten
vier Takten dreistimmig besetzt, so dass neben den guten Sopranistinnen
Mathilde Glasschröder und Barbara Essigmann die schlechte Alt-Sängerin
Theres Raster kurz zum Einsatz kam. 

Neben dem schlechtem Gesang stellte für Reicheneder möglicherweise auch
das kirchenmusikalische Repertoire einen Stein des Anstoßes dar. Ein
und dasselbe Lied, das zum Schluss des Gottesdienstes gesungen wurde,
hatte sich als Standardstück eingebürgert und nicht nur werktags wurden
Teile aus dem Gloria und Credo übersprungen.

Obwohl in Ruhmannsfelden die Bereitschaft zum Singen recht groß war – es
gab neben dem Kirchenchor einen Männerchor und eine weltliche
Lieder\-tafel – gab es lediglich ein Gesangsquartett als Kirchenchor.
Reicheneder muss als Hauptgrund für diesen Missstand die laxe
Probenpraxis in Högns Woh\-nung angesehen haben, den er durch Ansetzung
von öffentlichen Proben zu beseitigen versuchte – wohlgemerkt mit Högn
als Chorleiter. Es ist höchst ungewöhnlich, dass nicht der Chorleiter
über die Anberaumung der Proben entscheidet. Einerseits wollte
Reicheneder durch diese Maßnahme, das Niveau der musikalischen
Darbietungen erhöhen. Andererseits hoffte er wohl insge\-heim, dass
Högn durch diese Bevormundung abdanken und die Stelle für seine
Vertrauensperson Maria Reisinger frei machen würde. Die mit Högn
befreundeten Chorsängerinnen Mathilde Glasschröder und Barbara
Essig\-mann erschienen nicht zu den festgesetzten Proben und lieferten
somit Reicheneder einen Grund zum Einschreiten. 

In seinen Briefen an die zwei Chorsängerinnen und an Högn teilte der
Pfarrer ihnen kurz vor Weihnachten 1953 ihre „Kündigung“ mit. Über den
genauen Wortlaut der „Kündigungsschreiben“ ließe sich natürlich viel
spekulieren. Bekannt ist nur, dass die Briefe große Verärgerung bei den
Betroffenen auslösten. Fraglich ist allerdings, ob diese
„Kündigungsschreiben“ wirklich in Zusammenhang mit Högns schwerem
Schlaganfall stehen, wie von der Sängerin Barbara Essigmann behauptet.
Tatsache ist, dass Högn seitdem nicht mehr in der Lage war, Orgel zu
spielen.

Deshalb wirkt der Brief, den Högn nach seinem Krankenhausaufenthalt am
25. Januar 1954 an Reicheneder schrieb, schon etwas befremdlich. Mit
der \zitat{berechtigten Forderung auf Ruhestand auch im
Kirchenchordienst} verkündete Högn in diesem Schreiben seinen Rücktritt
als Chorregent und Organist und bezog sich dabei nicht auf die Folgen
seines Schlaganfalls oder auf das „Kündigungs\-schreiben.“ In seinem
Antwortbrief vom 6. Februar 1954 nahm Reicheneder zu Högns Brief
Stellung, in dem dieser von sich aus abdankte, und unterstrich, dass er
Högns \zitat{Standpunkt voll und ganz verstehen} könne,
seinen Rücktritt aber \zitat{sehr bedauere.}

Wären nur diese zwei Briefe und nicht zusätzlich die genauen
Erinnerun\-gen von mehreren Zeitzeugen, könnte man eine ganz andere
Schlussfolgerung daraus ziehen. Sowohl Reicheneder als auch Högn waren
daran interessiert, der Nachwelt eine andere Version des Ausscheidens
Högns als die der Kün\-digung durch Reicheneder zu überliefern. Nur
zwischen den Zeilen, kann man in beiden Briefen die vorhergehenden
Ereignisse erahnen, wie zum Beispiel an der Stelle, an der Högn die
\zitat{Kündigung seitens des Hochwürdigen Herrn Pfarrers und
der gleichen} als \zitat{blödes Weibergeschwätz} bewertet und
darauf hinweist, dass \zitat{eine vertrag\-liche Abmachung
über den Kirchenchordienst zwischen Pfarramt Ruhmannsfelden }und ihm
\textit{niemals bestanden} habe.\zitat{\textup{ }}Und seine
angebliche \zitat{Stimmbandlähmung, }die Rei\-cheneder als
Entschuldigung anführte, weshalb er Högn keinen Kranken\-besuch
abstattete, erscheint in diesem Zusammenhang mehr als eine faule
Ausrede.

Über den wahren Wortlaut der „Kündigungsbriefe“ ließe sich natürlich
viel spekulieren. Allein schon ihr Verschwinden ist ein Beweis für ihre
Brisanz. Reicheneder war ein passionierter Historiker und Archivar. Es
gibt wohl keinen auf die Gegend bezogenen Zeitungsartikel aus
Reicheneders Ruh\-mannsfeldener Zeit, der nicht in seine über dreißig
prall gefüllte Ordner um\-fassende \textit{Chronik Ruhmannsfelden}
angefügt wurde. Auch seine Korrespondenz dokumentierte Reicheneder sehr
genau. Viele von ihm verfasste Briefe lassen sich im Pfarrarchiv im
Kohlepapierabdruck nachlesen, wie etwa das Schreiben an Högn vom 6.
Februar 1954. Die „Kündigungsschreiben“ hat Reicheneder ganz bewusst
nicht archiviert, damit kein schlechtes Licht auf ihn fällt. Einen
unwürdigeren Abschied hätte Reicheneder Högn nach 43 Jahren Dienstzeit
an der Kirchenmusik in Ruhmannsfelden kaum bieten können.

Diese Vorgehensweise Reicheneders gegenüber Högn ist kein Einzelfall.
Ein Ereignis wenige Jahre später ist bezeichnend für seine
Problemlösungs\-strategie mit der Brechstange: Nachdem Reicheneder bei
einer Sonntagspre\-digt unter anderem über die
\zitat{Leistungsabzeichen für nächtliche Liebesfahrten}
wetterte, die beim Fußballerball 1957 verliehen wurden, ließen sich die
Beschuldigten nicht zurechtweisen und veranstalten, nach Angaben des
Geistlichen selbst, aus einer Trotzreaktion heraus ein
Faschingsbegräbnis am Aschermittwoch mit Musik und Saufgelage.
\zitat{Bis die Haupträdelsführer beim Pfarramt vorstellig
gewor\-den sind,} wie es in einer Pressemitteilung des Pfarramts hieß,
sollten nun die Glocken in Ruhmannsfelden schweigen. Diese Aktion
machte natürlich nicht nur in der regionalen Presse die Runde. Die
Glocken läuteten erst wieder, als sich der Bürgermeister in den Fall
einschaltete und für eine Lösung des Pro\-blems sorgte, die schriftlich
festgehalten wurde.

In der Kirchenverwaltungssitzung vom 21. Februar 1954 wurde Högns
Nachfolge endgültig geregelt. Maria Reisinger erhielt die
Organistenstelle und Franz Danziger übernahm die Chorleitung.

\section{Die letzten Lebensjahre}

Um seine Pensionierung musste Högn kämpfen. Ehe sein Fall im
Entnazifi\-zierungsprozess nicht vom Spruchkammergericht verhandelt
wurde, konnte er weder auf Wiedereinstellung nach auf Pensionierung
hoffen. Als er am 20. Februar 1947 endlich „entnazifiziert“ war, musste
erst ein „Pensionierungs\-gesuch“ gestellt werden, bevor er in den
Ruhestand treten konnte, was erfah\-rungsgemäß lange dauerte. In dieser
Übergangszeit sollte er wieder als Lehrer arbeiten, möglichst an einer
Schule außerhalb Ruhmannsfeldens. Eine Wieder\-einstellung an einer
anderen als der Volksschule Ruhmannsfelden wollte Högn – zumal er im
69. Lebensjahr stand – unbedingt vermeiden. Anträge von Högn und dem
Bürgermeister Muhr auf Wiedereinstellung an der Volksschule
Ruhmannsfelden hatten schließlich Erfolg und Högn übernahm vom 19. März
bis zum 2. April 1947 dreißig Wochenstunden des erkrankten Lehrers
Hans-Georg Dutsfeld. Zum 1. September 1947 wurde Högn endgültig in den
Ruhe\-stand versetzt. Sein Ruhestand konnte vor 1953 wohl kaum als
solcher be\-zeichnet werden, wenn man seine Aktivitäten für die Kirche
und die Heimat\-kunde betrachtet. Einen tiefen Einschnitt in Högns
unruhiges Rentnerleben stellte sein Schlaganfall Ende 1953 dar.

Vieles spricht dafür, dass Högn trotz seines Schlaganfalls auch nach
1953 ein aktives Leben führte. Er war sogar noch schöpferisch tätig,
wie sein \textit{Ma\-rienlied Nr. 13} beweist, das zur 300-Jahrfeier
der Wallfahrtskapelle Oster\-brünnl im September 1960 entstanden ist.
Den Text hat Högn ursprünglich für ein Lied zum Marianischen Jahr 1954
geschrieben. Die Kündigung durch Reicheneder verhinderte aber die
Fertigstellung des Stücks. Das Jubiläum der Wallfahrtskirche bot einen
neuen Anlass den selbstverfassten Text zu ver\-tonen.

Högn verwendete nur bei den Marienliedern Nr. 12 und Nr. 13 eigene
Texte. Beide weisen Högn als eher mäßigen Dichter aus. Im Text zum
\textit{Marien\-lied Nr. 12 F-Dur op. 63} kann in drei Verszeilen eine
Unterbrechung des jam\-bischen Versmaßes beobachtet werden. Insgesamt
sind die Zeilen viel zu lang. Einerseits verblasst dadurch der Effekt
des Reimes, andererseits konnten die langen Zeilen nicht immer mit
einer sinnvollen Aussage gefüllt werden. Beim Text des
\textit{Marienlieds Nr. 13} können zwar keine Fehler im Metrikfluss
oder im Reimschema festgestellt werden, dafür verstärkt der sehr
einfache Aufbau aus Paarreim und trochäischem Versmaß den trivialen
Gesamteindruck des Gedichtes.

% {\itshape
% Kommt herbei ihr Christen all}

% {\itshape
% von dem Berg und aus dem Tal!}

% {\itshape
% Kommt zu ihr, der Königin –}

% {\itshape
% Aller Welt Beherrscherin!}

% {\itshape
% Kommet all und rufet laut –}

% {\itshape
% bis herab die Mutter schaut –}

% {\itshape
% und dann Euch den Segen gibt –}

% {\itshape
% weil Euch ja die Mutter liebt!}

% \newline
% \textit{Kommt herzu aus nah u. fern!}

% {\itshape
% Singt das Lob dem Gnadenstern!}

% {\itshape
% Ruft um Hilf zur Mittlerin!}

% {\itshape
% Sagt den Dank der Helferin!}

% {\itshape
% Allezeit und immerfort}

% {\itshape
% stehet ihr in sicherem Hort!}

% \subparagraph{Himmelsglück als ewgen Lohn}
% {\itshape
% schenket Euch der Gottessohn!}

Der Umstand, dass Högn bei der Entstehung des Marienliedes kein Chor zur
Verfügung stand, schlug sich in der Besetzung nieder. Waren die meisten
früheren Marienlieder für Chor mit großen solistischen Passagen
gesetzt, so ist dieses ein reines Sololied für eine hohe Stimme. Die
Begleitung kann von einem Klavier, Harmonium oder einer Orgel
übernommen werden. Nach dem Aufbau zur urteilen, ist die Begleitstimme
mit seiner permanenten Achtelbe\-wegung, mehr für Klavier als für Orgel
gedacht und ähnelt manchen klassi\-schen Kunstliedern. Zu groß ist der
Unterschied zu den choralartigen Begleit\-sätzen für Orgel der
früheren, so dass Högn wohl kaum an eine Ausführung mit Orgel gedacht
hatte. Möglicherweise spiegelt die Klavierbegleitung auch die
krankheitsbedingt eingeschränkte Spielfähigkeit Högns wieder. Die durch
die linksseitige Lähmung beeinträchtige Hand ist in diesem Lied
wesentlich ein\-facher gesetzt als die rechte.

% {\centering   [Warning: Image ignored]
% % Unhandled or unsupported graphics:
% %\includegraphics[width=5.159cm,height=7.606cm]{a3-img/a3-img019.jpg}
%  \par}
% {\centering
% Das letzte Foto von August Högn
% \par}

Mit Sicherheit hat Högn das \textit{Marienlied }seiner ehemaligen
Chorsängerin Mathilde Glasschröder gewidmet, da die Autographen dieser
letzten erhalten Komposition Högns sich im Wohnhaus der Sängerin
befanden. Högn kom\-ponierte das Lied zwar für dieses Jubiläum, bei den
Feiern wurde es aber wahrscheinlich nicht aufgeführt. Weder die
Zeitungen noch der Pfarrbote be\-richteten von einer Uraufführung. Das
ist nicht verwunderlich, waren doch weder die Sängerin noch Högn auf
den Initiator der Feierlichkeiten, Pfarrer Reicheneder, gut zu
sprechen.

Das Marienlied ist nicht die einzige Komposition Högns, die er im hohen
Alter verfasste. Für den Männerchor schrieb Högn noch nach seinem 80.
Ge\-burtstag einige weltliche Kompositionen, die nicht erhalten sind.

Vier erhaltene umfangreiche Briefe, die Högn kurz vor Ende seines
Le\-bens einem ehemaligen Schüler bis nach Australien schickte,
beweisen zwar nicht, dass er schöpferisch tätig, jedoch dass er bis zu
letzt im Vollbesitz seiner geistigen Kräfte war.

Am 2. August 1958 konnte Högn seinen 80. Geburtstag feiern. Dieser wurde
den Verdiensten Högns entsprechend besonders feierlich begangen. Gleich
zwei Artikel erschienen im \textit{Viechtacher Bayerwaldboten}: Am
Geburtstag konnte sich jeder Leser anhand einer ausführlichen
Biographie über den bisherigen Lebensweg von Högn informieren. Drei
Tage später wurden in einem Artikel die Feierlichkeiten beschrieben.
Schon am Vorabend des Geburtstags sang der Männerchor unter der Leitung
von Franz Danziger Högn ein Ständ\-chen. Der erste Tenor des Chors,
Richard Bartascheck, hielt anschließend auf den Jubilar eine
\textit{saubere Rede, }wie Augenzeugen berichteten. In seiner
\textit{großen Ansprache} bezeichnete er Högn als \textit{den Mozart
von Ruhmannsfelden.} Der Redner verstand etwas von Musik, hatte er doch
Gesang studiert. Am Geburtstag gratulierten der Vorstand der
Freiwilligen Feuerwehr, der Vorstand des Krie\-ger- und
Veteranenvereins, die Bürgermeister von Zachenberg und Ruh\-mannsfelden
… Wahrscheinlich war es wirklich ein \textit{großer Teil der
Bevölkerung,} wie in dem Zeitungsbericht zu lesen ist. Högn ließ sich
von diesem Trubel an\-scheinend nicht aus der Ruhe bringen und erzählte
einige Episoden aus sei\-nem Leben. Zum Höhepunkt der Feierlichkeiten
marschierte die Blaskapelle unter der Leitung von Ludwig Heinrich auf
und überraschte Högn mit seinem Lieblingslied.

Ende 1960 scheint sich Högns Gesundheitszustand zu verschlechtern. Er
beklagt sich, dass sein \zitat{gesundheitliches Befinden
nicht das beste} ist und das \zitat{Marschie\-ren sehr
schlecht geht. }Um Besserung zu erfahren, sucht er sogar einen Arzt im
mehr als 50 Kilometer entfernten Straubing auf. Er ist immer mehr auf
Hilfe anderer angewiesen und seine Haushälterin Rosa Beischmied wird
zur \zitat{Kran\-kenfürsorgerin.}

August Högn starb am 13. Dezember 1961 um 5 Uhr morgens. Sein Tod kam
nicht plötzlich, da er rechtzeitig mit den Sterbesakramenten versehen
worden war. Nach Aussagen von Högns Enkelin Gertraud von Molo war er
bis kurz vor seinem Tod rüstig und sein Tod dürfte eher auf eine
\zitat{\textup{kürzere Krankheit} }zurückzuführen gewesen
sein, als auf eine \zitat{lange schwere} Krankheit, wie es im
Nachruf im \textit{Viechtacher Bayerwaldboten} hieß. Im Sterberegister
der Pfarrei steht als Todesursache \textit{Herzinsuffizienz}, was auf
einen plötzlichen Tod schließen lässt. Waren bei der Überführung des
Leichnams am Todestag nur wenige enge Freunde anwesend, die dem
Leichenauto Richtung Deggendorf einige hundert Meter folgten, so dürfte
es beim Requiem am darauf folgenden Tag ein Großteil der Bevölkerung
gewesen sein, der von Högn Abschied nahm, denn an alle ehemaligen
Schüler der Volksschule ging die Einladung zum Besuch des
Trauergottesdienstes. Die Beerdigung fand am 15. Dezember in Deggendorf
statt.

Bei eisiger Kälte hielten die Lehrer Karl Schambeck, Franz Nemetz und
der Kreisschulrat Botschafter am Grab Trauerreden auf August Högn.
Beson\-ders die Ansprache von Lehrer Nemetz blieb als eine sehr lange
Rede bei vie\-len Anwesenden in Erinnerung. Abordnungen der Vereine aus
Ruhmannsfel\-den waren mit Fahnen anwesend und eine Bläsergruppe der
freiwilligen Feu\-erwehr spielte für ihr Ehrenmitglied zum Abschied das
Lied \textit{Wir hatten einen Kameraden}.

August Högn fand seine letzte Ruhestätte auf dem Deggendorfer Friedhof
neben seiner 1926 verstorbenen Ehefrau Emma.

\subsection{Nach Högns Tod}
Nach Högns Tod wurde die Haushälterin Rosa Beischmied mit der Auflösung
seiner Wohnung beauftragt. Die Angehörigen ließen wahrscheinlich den
Großteil von Högns Werken zurück. Es lag daher an der Haushälterin,
ihren weiteren Verwendungszweck zu bestimmen. Nach den Fundorten der
Kom\-positionen zu urteilen, hatte Beischmied alle geistlichen Werke
der Pfarrei übergeben. An vier Orten, die in Verbindung zur Kirche
stehen, wurden Wer\-ke von Högn gefunden, im Notenschrank und auf dem
Dachboden des linken Seitenschiffs der Pfarrkirche St. Laurentius, im
Pfarrhof und in der Wohnung des ehemaligen Kirchenchorleiters Franz
Danziger. Im Haus der Sängerin Mathilde Glasschröder waren
Kompositionen von Högn zu entdecken, die möglicherweise schon vor Högns
Tod in den Besitz der Sängerin übergingen.

Angesicht der veschiedenen Fundorte ist es nicht verwunderlich, dass
manche Kompositionen nicht mehr zugänglich sind. Vor allem Högns
weltliches Werk ging verloren. Da Franz Danziger nicht nur
Kirchenchorleiter, sondern zur Zeit der Auflösung von Högns Wohnung
auch Leiter des Männerchores war und im Turnverein-Orchester unter Högn
mitwirkte, müsste seine Wohnung ein ergiebiger Fundort auch von
weltlichen Kompositionen gewesen sein. Hier war aber kein einziges Werk
außer den entdeckten geist\-lichen Kompositionen zum Vorschein
gekommen. Der Sohn des ehemaligen Kirchenchorleiters räumte in einem
Brief vom 25. Oktober 2002 ein, dass seine Mutter bei Aufräumarbeiten
eventuell auch Werke von Högn vernichtet haben könnte. Centa
Schwannberger, die Ehefrau von Rudolf Schwannberger, der zusammen mit
Högn die Sängerriege des Turnvereins leitete, hat den No\-tenbestand
ihres Mannes den Geißkopfsängern übergeben. Hier verlaufen sich die
Spuren.

Högns Geschichtswerk ist komplett erhalten. Högn hat sich noch zu
Leb\-zeiten um eine sichere Aufbewahrung seiner heimatkundlichen Werke
ge\-kümmert und die \textit{Geschichte von Zachenberg} und die
\textit{Geschichte der Feuerwehr Ruh\-mannsfelden} seinem Nachfolger
bei der Feuerwehr und Gemeindesekretär von Zachenberg Johann Freisinger
übergeben. Dass das Geschichtswerk nicht ab\-handen gekommen ist, muss
auch dem Pfarrer Reicheneder gedankt werden, der außer der
\textit{Geschichte von Ruhmannsfelden} alle heimatkundlichen
Abhandlun\-gen Högns abgetippt und seiner \textit{Chronik
Ruhmannsfelden} angefügt hat.

Högns Kompositionen wurden nach dessen Tod wie schon nach Högns
Ausscheiden aus dem Chorregentenamt regelmäßig von seinem Nachfolger
Franz Danziger weiter aufgeführt. Danziger fertigte sogar deutsche
Versionen der lateinischen \textit{Laurentius-Messe C-Dur op. 14} und
der lateinischen \textit{Mater-Dei-Messe F-Dur op. 16} an. Auch die
späteren Kirchenchorleiter Karl Geiger und August Lankes ließen Högns
Kompositionen singen.

Das \textit{Marienlied Nr. 8 G-Dur op. 54} arrangierte Danziger für den
Achslacher Männergesangsverein, der dieses Arrangement bis heute in
seinem Repertoire hat. Der am weitesten von Ruhmannsfelden entfernte
Aufführungsort einer Komposition Högns war Altötting. Högns Tochter
Elfriede Schlumprecht hat die \textit{Josephi-Messe} durch den
Altöttinger Chor aus Anlass eines Familienjubilä\-ums aufführen lassen.
Einzelheiten über diese Aufführung sind nicht bekannt.

Von Högns Geschichtswerk blieb als einziges Werk die \textit{Geschichte
von Ruh\-mannsfelden} im Bewusstsein der Bevölkerung. Diese hat sich zu
einem Stan\-dardwerk der hiesigen Heimatgeschichte entwickelt:
Vollkommen unbekannt blieben dagegen – weil nie gedruckt – die
\textit{Geschichte von Zachenberg} und die \textit{Geschichte der
freiwilligen Feuerwehr Ruhmannsfelden.} Högns heimatkundliche
Zei\-tungsartikel waren nur über das Pfarrarchiv und das Privatarchiv
des ehema\-ligen Gemeindesekretärs Johann Freisinger zugänglich.

Insgesamt aber nahm die Bekanntheit des Namens Högn im Laufe der Jahre
stark ab. Wenn jemanden der Name Högn noch was sagt, dann als Au\-tor
der \textit{Geschichte von Ruhmannsfelden}. Högn als Komponist geriet
vollkommen in Vergessenheit, obwohl sein kompositorisches Werk
eindeutig höher als sein Geschichtswerk einzuschätzen ist.

% {\centering   [Warning: Image ignored]
% % Unhandled or unsupported graphics:
% %\includegraphics[width=4.046cm,height=9.186cm]{a3-img/a3-img020.jpg}
%  \par}
% {\centering
% Totenbrett zum Andenken von August Högn
% \par}

% {\centering
% an der Wallfahrtskirche Osterbrünnl
% \par}

% {\centering
% in Ruhmannsfelden
% \par}


\chapter{Anhang}
\section{Werkverzeichnis}

\subsection{Geistliche Musik}
% \begin{flushleft}
% \tablefirsthead{}
% \tablehead{}
% \tabletail{}
% \tablelasttail{}
% \begin{supertabular}{m{5.4360003cm}m{0.156cm}m{5.794cm}}
% \multicolumn{3}{m{11.786cm}}{Messen}\\
% \multicolumn{2}{m{5.7920003cm}}{„Laurentius“-Messe C-Dur op. 14} &
% 4-st. gem. Chor, Blechbläserquartett, Orgel\\
% \multicolumn{2}{m{5.7920003cm}}{„Mater-Dei“-Messe F-Dur op. 16} &
% 4-st. gem. Chor, Streichquintett, Orgel\\
% \multicolumn{2}{m{5.7920003cm}}{„Josephi“-Messe F-Dur op. 62} &
% 4-st. gem. Chor, Soli, 2 Vl., Blechbläser\-quartett, Orgel\\
% \multicolumn{3}{m{11.786cm}}{Tantum ergo}\\
% \multicolumn{2}{m{5.7920003cm}}{4 Tantum ergo op. 32} &
% \\
% \multicolumn{2}{m{5.7920003cm}}{Tantum ergo Nr. 1 Es-Dur op. 11} &
% 4-st. gem. Chor, Streichquintett, Orgel\\
% \multicolumn{2}{m{5.7920003cm}}{Tantum ergo Nr. 2 F-Dur op. 32} &
% 4-st. gem. Chor, Streichquintett, Orgel\\
% \multicolumn{2}{m{5.7920003cm}}{Tantum ergo Nr. 3 Es-Dur op. 49} &
% 4-st. gem. Chor , Streichquintett, Orgel\\
% \multicolumn{2}{m{5.7920003cm}}{Tantum ergo Nr. 4 A-Dur op. 47} &
% 4-st. gem. Chor, Streichquintett, Orgel\\
% \multicolumn{3}{m{11.786cm}}{Pange lingua}\\
% \multicolumn{2}{m{5.7920003cm}}{Pange lingua G-Dur (deutsch)} &
% 4-st. gem. Chor, Blechbläserquartett\\
% \multicolumn{2}{m{5.7920003cm}}{Pange lingua F-Dur op. 43 } &
% 4-st. gem. Chor, Orgel\\
% \multicolumn{2}{m{5.7920003cm}}{Pange lingua Es-Dur op. 46 } &
% 4-st. gem. Chor, Orgel\\
% \multicolumn{2}{m{5.7920003cm}}{Pange lingua Es-Dur op. 51 } &
% 4-st. gem. Chor, Orgel\\
% \multicolumn{3}{m{11.786cm}}{Marienlieder}\\
% \multicolumn{2}{m{5.7920003cm}}{Ave Maria F-Dur op. 4} &
% Unter- und Oberstimme, Orgel\\
% \multicolumn{2}{m{5.7920003cm}}{Lieder zum Lobe und Preise Mariens} &
% \\
% \multicolumn{2}{m{5.7920003cm}}{Marienlied Nr. 1 F-Dur op. 13 a } &
% 4-st. gem. Chor, Orgel\\
% \multicolumn{2}{m{5.7920003cm}}{Marienlied Nr. 2 e-moll op. 19 } &
% Sopran-Solo, 4-st. gem. Chor, Orgel\\
% \multicolumn{2}{m{5.7920003cm}}{Marienlied Nr. 3 F-Dur op. 22 } &
% Sopran-Solo, 4-st. gem. Chor, Orgel\\
% \multicolumn{2}{m{5.7920003cm}}{Marienlied Nr. 4 G-Dur op. 23 } &
% Sopran-Solo, 4-st. gem. Chor, Orgel\\
% \multicolumn{2}{m{5.7920003cm}}{Marienlied Nr. 5 F-Dur op. 28 } &
% 4-st. gem. Chor, Orgel\\
% \multicolumn{2}{m{5.7920003cm}}{Marienlied Nr. 6 F-Dur op. 41 } &
% 4-st. Frauenchor, Orgel\\
% \multicolumn{2}{m{5.7920003cm}}{Marienlied Nr. 7 G-Dur op. 45
% (fragmentarisch)} &
% Sopran-Solo, 4-st. gem. Chor, Orgel\\
% \multicolumn{2}{m{5.7920003cm}}{Marienlied Nr. 8 G-Dur op. 54} &
% Sopran-Solo, 4-st. gem. Chor, Orgel\\
% \multicolumn{2}{m{5.7920003cm}}{Marienlied Nr. 9 G-Dur op. 34} &
% 4-st. Männerchor\\
% \multicolumn{2}{m{5.7920003cm}}{Marienlied Nr. 10 F-Dur op. 56 } &
% 2 Sopran- und Alt-Solo, 4-st. gem. Chor, Orgel\\
% \multicolumn{2}{m{5.7920003cm}}{Marienlied Nr. 11 F-Dur op. 59} &
% Bariton-Solo, 4-st. gem. Chor, Orgel\\
% \multicolumn{2}{m{5.7920003cm}}{Marienlied Nr. 12 F-Dur op. 63} &
% Sopran- und Alt-Solo, Orgel\\
% \multicolumn{2}{m{5.7920003cm}}{Marienlied (Nr. 13) C-Dur} &
% Sopran-Solo, Klavier o. Harmonium\\
% \multicolumn{3}{m{11.786cm}}{Grablieder}\\
% \multicolumn{2}{m{5.7920003cm}}{Grablied für gefallene Soldaten Es-Dur
% op. 35} &
% 4-st. gem. Chor, Blechbläserquartett\\
% \multicolumn{2}{m{5.7920003cm}}{{\bfseries \textmd{Grablieder op. 35}}}
% &
% \\
% \multicolumn{2}{m{5.7920003cm}}{Grablied Nr. 1 Es-Dur op. 35 } &
% 4-st. gem. Chor, Blechbläserquartett\\
% \multicolumn{2}{m{5.7920003cm}}{Grablied Nr. 2 Es-Dur} &
% 4-st. gem. Chor, Blechbläserquartett\\
% \multicolumn{2}{m{5.7920003cm}}{Grablied Nr. 3 Es-Dur op. 44} &
% 4-st. gem. Chor, Blechbläserquartett\\
% \multicolumn{2}{m{5.7920003cm}}{Grablied Nr. 4 F-Dur op. 20 } &
% Sopran-Solo, 4-st. gem. Chor, Orgel\\
% \multicolumn{3}{m{11.786cm}}{Offertorien}\\
% \multicolumn{2}{m{5.7920003cm}}{Offertorium D-Dur op. 26 } &
% 4-st. gem. Chor, Orgel\\
% \multicolumn{2}{m{5.7920003cm}}{Offertorium C-Dur op. 30 } &
% 4-st. gem. Chor, Orgel\\
% \multicolumn{3}{m{11.786cm}}{Kommunionlieder}\\
% \multicolumn{2}{m{5.7920003cm}}{Kommunionlied Es-Dur op. 12} &
% 4-st. gem. Chor, Orgel\\
% \multicolumn{2}{m{5.7920003cm}}{Kommunionlied G-Dur op. 21 a} &
% 4-st. gem. Chor, Orgel\\
% \multicolumn{2}{m{5.7920003cm}}{Kommunionlied G-Dur op. 21 b} &
% 4-st. gem. Chor, Orgel\\
% \multicolumn{2}{m{5.7920003cm}}{Kommunionlied C-Dur op. 37 b} &
% 4-st. gem. Chor, Orgel\\
% \multicolumn{3}{m{11.786cm}}{Veni creator Spiritius}\\
% \multicolumn{2}{m{5.7920003cm}}{Veni creator Spiritus B-Dur} &
% 4-st. Männerchor\\
% \multicolumn{2}{m{5.7920003cm}}{11 Veni creator Spiritus op. 15} &
% \\
% \multicolumn{2}{m{5.7920003cm}}{Veni creator Spiritus C-Dur Nr. 1} &
% 4-st. gem. Chor\\
% \multicolumn{2}{m{5.7920003cm}}{Veni creator Spiritus D-Dur Nr. 2} &
% 4-st. gem. Chor\\
% \multicolumn{2}{m{5.7920003cm}}{Veni creator Spiritus Es-Dur Nr. 3} &
% 4-st. gem. Chor\\
% \multicolumn{2}{m{5.7920003cm}}{Veni creator Spiritus E-Dur Nr. 4} &
% 4-st. gem. Chor\\
% \multicolumn{2}{m{5.7920003cm}}{Veni creator Spiritus F-Dur Nr. 5} &
% 4-st. gem. Chor\\
% \multicolumn{2}{m{5.7920003cm}}{Veni creator Spiritus Fis-Dur Nr. 6} &
% 4-st. gem. Chor\\
% \multicolumn{2}{m{5.7920003cm}}{Veni creator Spiritus G-Dur Nr. 7} &
% 4-st. gem. Chor\\
% \multicolumn{2}{m{5.7920003cm}}{Veni creator Spiritus As-Dur Nr. 8} &
% 4-st. gem. Chor\\
% \multicolumn{2}{m{5.7920003cm}}{Veni creator Spiritus A-Dur Nr. 9} &
% 4-st. gem. Chor\\
% \multicolumn{2}{m{5.7920003cm}}{Veni creator Spiritus B-Dur Nr. 10} &
% 4-st. gem. Chor\\
% \multicolumn{2}{m{5.7920003cm}}{Veni creator Spiritus H-Dur Nr. 11} &
% 4-st. gem. Chor\\
% \multicolumn{3}{m{11.786cm}}{Adjuva nos}\\
% Adjuva nos Es-Dur op. 8  &
% \multicolumn{2}{m{6.15cm}}{4-st. gem. Chor, Streichquintett, Orgel}\\
% 8 Adjuva nos op. 15 &
% \multicolumn{2}{m{6.15cm}}{}\\
% Adjuva nos C-Dur Nr. 1 &
% \multicolumn{2}{m{6.15cm}}{4-st. gem. Chor}\\
% Adjuva nos D-Dur Nr. 2 &
% \multicolumn{2}{m{6.15cm}}{4-st. gem. Chor}\\
% Adjuva nos Es-Dur Nr. 3 &
% \multicolumn{2}{m{6.15cm}}{4-st. gem. Chor}\\
% Adjuva nos E-Dur Nr. 4 &
% \multicolumn{2}{m{6.15cm}}{4-st. gem. Chor}\\
% Adjuva nos F-Dur Nr. 5 &
% \multicolumn{2}{m{6.15cm}}{4-st. gem. Chor}\\
% Adjuva nos G-Dur Nr. 6 &
% \multicolumn{2}{m{6.15cm}}{4-st. gem. Chor}\\
% Adjuva nos A-Dur Nr. 7 &
% \multicolumn{2}{m{6.15cm}}{4-st. gem. Chor}\\
% Adjuva nos B-Dur Nr. 8 &
% \multicolumn{2}{m{6.15cm}}{4-st. gem. Chor}\\
% \multicolumn{3}{m{11.786cm}}{verschiedene Genre}\\
% \multicolumn{2}{m{5.7920003cm}}{Cäcilienlied E-Dur op. 12 b} &
% 3-st. Frauenchor, Orgel\\
% \multicolumn{2}{m{5.7920003cm}}{Libera e-moll op. 50} &
% 4-st. gem. Chor\\
% \multicolumn{2}{m{5.7920003cm}}{Benedictus G-Dur op. 50} &
% 4-st. gem. Chor\\
% \multicolumn{2}{m{5.7920003cm}}{Fronleichnams-Prozessionsgesänge Es-Dur
% op. 52} &
% 4-st. gem. Chor, Blechbläserquartett\\
% \multicolumn{2}{m{5.7920003cm}}{Ecce sacerdos F-Dur op. 57} &
% 4-st. gem. Chor, Blechbläserquartett, Orgel\\
% \multicolumn{2}{m{5.7920003cm}}{Juravit Dominus B-Dur op. 58} &
% 4-st. gem. Chor, Blechbläserquartett, Orgel\\
% \multicolumn{2}{m{5.7920003cm}}{„Ehre sei Gott“ C-Dur} &
% 4-st. gem. Chor\\
% \multicolumn{2}{m{5.7920003cm}}{{\itshape Herz-Jesu-Litanei (verloren)}}
% &
% \\
% \end{supertabular}
% \end{flushleft}
\subsection{Weltliche Musik}

% \begin{flushleft}
% \tablefirsthead{}
% \tablehead{}
% \tabletail{}
% \tablelasttail{}
% \begin{supertabular}{m{5.7920003cm}m{5.794cm}}
% Marsch „In Treue fest!“ D-Dur  &
% Klavier\\
% Weihegesang Es-Dur (fragmenta\-risch) &
% 4-st. gem. Chor, Blechbläserquartett\\
% Lied von Gotteszell G-Dur op. 42 (Arrangement) &
% 4-st. Männerchor\\
% \end{supertabular}
% \end{flushleft}

\subsection{Geschichtswerk}

\textit{Geschichte von Ruhmannsfelden,} Michael Laßleben, Kallmünz, 1949


\textit{Geschichte der Gemeinde Zachenberg,} Manuskript, 1954
(Högn/Trellinger)

\textit{Geschichte und Chronik der freiwilligen Feuerwehr
Ruhmannsfelden,} Manuskript, 1951 

Zeitungsartikel:

\textit{Geschichtliches vom Markt Ruhmannsfelden: Der Name
„Ruhmannsfelden“ – Die Bezeichnung „Markt“, }Durch Gäu und Wald –
Beilage zum Deggendorfer Donauboten, 6.11.1926

\textit{Geschichtliches vom Markt Ruhmannsfelden: Das Wappen von
Ruhmannsfelden – Schloss und Schlossberg Ruhmannsfelden, }Durch Gäu und
Wald – Beilage zum Deggendorfer Donauboten, 15.12.1926

\textit{Geschichtliches vom Markt Ruhmannsfelden: Von der Schule in
Ruhmannsfelden in früherer Zeit bis 1835,} Durch Gäu und Wald – Beilage
zum Deggendorfer Donauboten, 20.8.1927

\textit{Geschichtliches vom Markt Ruhmannsfelden: Von der Schule in
Ruhmannsfelden ab 1835}, Erscheinungsort und -datum ungekannt, gefunden
in der Chronik der Volksschule Ruhmannsfelden

\textit{Das Wallfahrtskirchlein Osterbrünnl bei Ruhmannsfelden,} Durch
Gäu und Wald - Beilage zum Deggendorfer Donauboten, 1927/Nr. 23  und 
1928/Nr. 2

\textit{Was Ruhmannsfelden für Jubiläen feiern könnte?,} Viechtacher
Tagblatt, 9.9.1928

\textit{Wie hat es um Ruhmannsfelden herum ausgesehen vor seiner
Entstehung?,} Viechtacher Tagblatt, 25.10.1928

\textit{Pfarrkirche St. Laurentius Ruhmannsfelden,} Viechtacher
Tagblatt, 1928/29 (in drei Artikeln erschienen)

\section{Interviews mit Zeitzeugen von Högn}

Interview mit Wilhelm Ederer, Aug. 2002

Interview mit Barbara Essigmann, 27.12.2002

Interview mit Ida Högn, 29.12.2002

Interview mit Maria Schröck, 30.12.2002

Interview mit Barbara Essigmann, 2.1.2003

Interview mit Wilhelm Ederer, 2.1.2003

Interview mit Ida Högn, 3.1.2003

Interview mit Josef Brunner, 3.1.2003

Interview mit Dr. Doraliesa Wiegmann, 19.1.2003

Interview mit Mathilde Beischmied, 21.1.2003

Interview mit Dr. Josef Stern, 21.2.2003

Interview mit Emilie Seidl, 23.4.2003

Interview mit Lorenz Schlagintweit, 29.11.2003

Interview mit Johann Freisinger, 29.12.2003

Interview mit Stephan Leitner, 19.2.2004

Interview mit Maria Freisinger, 25.8.2004

Interview mit Max Holler, 26.8.2004

Interview mit Centa Schwannberger, 14.9.2004

Interview mit Mathilde Beischmied, 14.9.2004

Interview mit Gertraud von Molo, 23.11.2004

Interview mit Lilo Leuze, 2.12.2004

Interview mit Wilhelm Ederer, 28.12.2004

Interview mit Josef Raster, 28.12.2004

Interview mit Johann Glasschröder, 28.12.2004

Interview mit Lilo Leuze, 14.1.2005

Interview mit Eva Ertl, 9.2.2005

\section{Quellenverzeichnis}

\subsection{Literatur}

Dantl, Georg\textbf{,} \textit{Vom Schullehrling zum Schulmeister –
Geschichte der Lehrerbildung im 19. Jahrhundert,} in: Oberpfälzer
Raritäten, Band 5, Verlag der Buchhandlung Taubald, Weiden, 1989

Gärtner, Helmut\textbf{,} \textit{Deggendorfer Originale – Originelles
Deggendorf,} Morsak Verlag, Grafenau, 2. Auflage, 1995

Geyer, Otto\textbf{,} \textit{Schule und Lehrer in Niederbayern,} Hrsg.
Otto Glaser, Niederbaye\-rischer Bezirkslehrerverein im BLLV,
Neue-Presse-Verlag-GmbH, Passau, 1964, 235 Seiten

Goller, Martina\textbf{,} \textit{Die Musik in der Lehrerbildung
Niederbayerns und ihre Ausstrah\-lung am Beispiel niederbayerischer
Lehrerkomponisten dargestellt an der Präparanden\-schule Deggendorf und
am Lehrerbildungsseminar Straubing,} Zulassungsarbeit zur ersten
Staatsprüfung für das Lehramt an Hauptschulen in Bayern einge\-reicht
bei Prof. Dr. Eckard Nolte im Fach Musikpädagogik, Deggendorf, April
1988

Högn, August\textbf{,} \textit{Geschichte und Chronik der freiwilligen
Feuerwehr Ruhmannsfelden,} Manuskript, 1951 (Feuerwehr)

Högn, August\textbf{,} \textit{Geschichte von Ruhmannsfelden,} Michael
Laßleben, Kallmünz, 1949 (Ruhmannsfelden)

Högn, August\textbf{,} \textit{Heimat-Geschichte der Gemeinde
Zachenberg,} Manuskript, 1954 (Zachenberg)

Lippert, Heinrich\textbf{,} \textit{Die Präparandenschule Deggendorf
(1866-1924) – Zur Geschichte einer niederbayerischen
Lehrerbildungsanstalt,} in: Deggendorfer Geschichtsblät\-ter 17,
Deggendorf, 1996, S. 153 – 192

Proft, Hans\textbf{\textit{,}}\textit{ „Immer froh und heiter bleibt der
Kutschenreuter“ – Leben und Werk des niederbayerischen Komponisten
Erhard Kutschenreuter,} Verlag Karl Stutz, Passau, 2004

Stengel, Georg Josef\textbf{,} \textit{Geschichte der
Lehrerbildungsanstalt Straubing von 1824-1924, }Manz, Straubing, 1925

\subsection{Quellen aus Archiven}

Reicheneder-Chronik:

Die Seelsorger der Pfarrei (Seelsorger)

Auszüge a. d. Protokollbüchern: Ehrenbürger von Ruhmannsfelden
(Ehren\-bürger)

Schul- und Bildungswesen in Ruhmannsfelden (Schulwesen)

Der Pfarrmesner von Ruhmannsfelden (Pfarrmesner)

Religiöse Feiern in der Pfarrei (Feiern)

Die Pfarrkirche, C. Nach 1820, II. Einrichtung – Die Orgel (Orgel)

Chronik der Volksschule Ruhmannsfelden

Protokollbuch der Feuerwehr Ruhmannsfelden

\section{Dank}

Vielen Dank an die vielen Mithelfer, durch die diese Arbeit erst möglich
wurde:

Franz Aichinger (Auskunft), Mathilde Beischmied (Interview), Hennes
Berger (Turnvereins-Chronik), Gerhard Bielmeier (Zugang zum
Gemeindearchiv), Josef Brunner (Interview), Franz jun. Danziger
(Kompositionen, Dokumente, Auskünfte), Wilfried Deisser (juristische
Beratung), Wilhelm Ederer (3 Inter\-views, Fotos), Eva Ertl (Interview,
Fotos), Barbara Essigmann (2 Interviews), Johann Freisinger (Interview,
Ausleihen der Manuskripte des Geschichtswerk, Fotos), Lotte Freisinger
(Hilfe bei Recherchen im Pfarrarchiv), Maria Frei\-singer (Interview),
Martin Friedrich (Beratung), Helmuth Gärtner (Auskunft), Gemeinde
Baiersdorf (Auskunft), Gemeinde Oberaudorf (Auskunft), Johann
Glasschröder (Interview, Fotos, Komposition), Martina Goller (Suchen
nach Kompositionen, Ausleihen ihrer Zulassungsarbeit), Martha Grotz
(Entziffern von Dokumenten), Ida Högn (2 Interviews), Peter Högn
(Auskünfte, Buchpräsentation Deggendorf), Michael Hüttinger
(Auskünfte), Max Holler (Interview), Alfons, HV-Straubing Huber
(Literatur-Tipps), Michael, P. Dr. Kaufmann (Auskunft), Kirchenchor
Ruh\-mannsfelden (Aufführungen von Kompositionen), Kloster Metten
(Aus\-kunft), August Lankes (Aufführungen von Kompositionen), Stefan
Leitner (Interview, Dokumente), Lilo Leuze (Interview, Fotos), Michael
Maimer (technische Beratung (Computer)), Martina Masarwa (Foto),
Helmuth, Pfarrer Meier (Zugang zum Pfarrarchiv, Überlassen des
Kopierers, Beratung), Gertraud von Molo (Interview, Dokumente, Fotos),
Pfarramt Kollnburg (Auskunft), Pfarramt Mariä Himmelfahrt, Deggendorf
(Auskunft), Pfarramt St. Martin, Deggendorf (Auskunft), Thanos Prapas
(Lektorat), Hans Proft (Dokument), Rudolf Radlbeck (telefonische
Auskunft, Buchpräsentation Ruhmannsfelden), Josef Raster (Interview,
Fotos), Irmgard Rebhahn (Vermittlung eines Interviews, Fotos,
Auskünfte), Rektor Roßmeißl (Ausleihen der Schulchronik), Lorenz
Schlagintweit (Inter\-view, Foto), Maria Schröck (Interview, Fotos,
Dokumente), Centa Schwann\-berger (Interview, Fotos), Johannes Schwarz
(technische Beratung (Com\-puter)), Emilie Seidl (Interview), Karin
Stadler (Auskunft), Josef Steinbauer (Jäger-Chronik), Josef Stern
(Interview), Karl Stutz (Verlag), Peter Voit (Nachforschungen um
Kompositionen), Doraliesa Wiegmann (Interview), und allen die vergessen
wurden …
\end{document}