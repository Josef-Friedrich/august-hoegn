\section{Chorregentendienst mit Unterbrechung}

Zum 1. September 1921 wurde, dem Protokoll der Kirchenverwaltungssitzung
zufolge, August Högn provisorisch der Chorregenten- und
Organistendienst ü\-ber\-tra\-gen. Provisorisch deshalb, weil ein Jahr zuvor
die gesetzlich geregelte Verbindung zwischen Schul- und Kirchendienst,
die mittels der Chorregenten-, Organisten- und Mesnertätigkeiten der
Lehrer Jahrhunderte lang bestand, abgeschafft worden war. Besonders
am Vormittag stattfindende Beerdigungen standen der Ausübung des
Organistendienst durch die Lehrer im Weg. Da sich viele Lehrer
freiwillig bereit erklärten, Chorregent zu bleiben, ließ sich diese
alte Tradition nicht von heute auf morgen abschaffen. Den Schulbehörden
blieb anscheinend nichts anderes übrig, als den Chorregentendienst
der Lehrer, verbunden mit den Stundenausfällen, zumindest für eine
Übergangszeit zu dulden.

Eine andere Tradition, nämlich dass der Schulleiter den
Chorregentendienst übernahm, wurde fortgeführt, obwohl der
Kirchendienst der Lehrer offiziell abgeschafft war. August Högn wurde
nach der Pensionierung des Bezirkshauptlehrers Alois Auer 1921 nicht
nur Schulleiter, sondern auch Chorregent. Max Weig war von 1879 bis zu
seinem Tod 1895 Schulleiter und somit auch Organist und Chorregent.
Erst in seinem Todesjahr kam vom königlichen Bezirksamt Viechtach die
Anweisung, dass ein Hilfslehrer den Chorregentendienst übernehmen
sollte. Von 1895 bis 1921 leitete Weigs Nachfolger, Alois Auer,
zusammen mit seiner Frau Anna den Chor, ehe Högn ihn fortführte.

Vor allem aber dürften seine überdurchschnittlichen musikalischen
Fähigkeiten Högn dazu bestimmt haben, die alte Tradition des Lehrers
und Chorregenten in Personalunion weiterzuführen. Bevor er die
Kirchenchorleitung übernahm, hatte er über 20 Jahre lang als guter
Tenor und versierter Organist – man sagte von ihm, dass er gleichzeitig
singen, spielen und dirigieren konnte – in verschiedenen Orten an der
Kirchenmusik mitgewirkt und Erfahrungen gesammelt. Nicht nur im
Manualspiel war er gewandt, wie die virtuos gesetzte Fassung seines
\textit{Marsch „In Treue fest!“} für Klavier zu zwei Händen beweist,
sondern auch im Pedaleinsatz sicher war. Unter den Stücken, die er auf
der Orgel spielte, war die berühmte und nicht gerade leicht zu
spielende \textit{Toccata und Fuge in d-moll} von Johann Sebastian
Bach. Angesichts einer derart gehobenen Orgelliteratur überrascht es
nicht, wenn manche Kirchenbesucher nach Ende des Gottesdienstes noch in
der Kirche blieben, um Högns Orgelspiel bis zum Schluss zu hören. Nicht
zu vergessen sind auch seine kompositorischen Fähigkeiten: Innerhalb
von wenigen Tagen konnte er für den Einsatz in der Kirchenmusik
passende Stücke schreiben. Ein befreundeter Chorleiter bat Högn in
einem Brief, ein Lied für seinen Männerchor zu schreiben. Laut einem
Vermerk auf der betreffenden Komposition war dieses schon zwei Tage
später fertig. Kein Wunder also, dass seine musikalischen Leistungen
auch im Urteil seiner Zeitgenossen lobende Anerkennung fanden. Bischof
Buchberger zum Beispiel hob bei einer Firmung hervor, er habe selten
einen so guten Organisten spielen gehört. Josef Brunner, der als
Organist an der Kirchenmusik unter Högn mitwirkte, meinte sogar:
\zitat{Der Högn war ein selten guter Musiker. So einer steht
nicht mehr auf.}

\zitat{\textup{Zur größten Zufriedenheit der ganzen
Kirchengemeinde}} leitete Högn laut Pfarrer Fahrmeier den Chor bis Ende
1924 und ein Ruhmannsfeldener Bürger, der die Entwicklung des
Kirchenchors über einen längeren Zeitraum beurteilen konnte,
bestätigte, dass der Chor, der unter dem Lehrer Weig
\zitat{einen guten Namen hatte} und unter der Leitung von
Anna Auer \zitat{einen Niedergang} erlebte, bei Högn wieder
eine \zitat{Verbesserung erfuhr\textup{.}}

Dass Högn 1921 einen nicht sehr leistungsfähigen Chor übernommen hat,
erkennt man auch an seinen geistlichen Kompositionen, die bis 1924 für
den Ruhmannsfeldener Kirchenchor entstanden sind. Sowohl das
\textit{Tantum ergo} \textit{Nr.} \textit{1 Es-Dur op. 11}, das
\textit{Kommunionlied Es-Dur op. 12}, das \textit{Cäcilienlied E-Dur
op. 12 b}, das \textit{Marienlied Nr. 1 F-Dur op. 13 a}, die \textit{11
Veni creator Spiritus C-Dur op. 15} als auch die \textit{8 Adjuva nos
op. 15} stellen an den Chor nur geringe Anforderungen. Högn bezeichnete
sogar seine zu dieser Zeit entstandene Messe, die dem
Ruhmannsfeldener Pfarrpatron gewidmete \textit{„Laurentius“-Messe
C-Dur op. 14}, ausdrücklich als \textit{leichte Messe zu Ehren des
heiligen Laurentius.} Besonders leicht ist in der gesamten Messe der
Tenor gesetzt, in unseren Breitengraden die am schwersten zu besetzende
Stimme, so dass diese Stimme notfalls auch von einer tiefen
Männerstimme übernommen werden kann. Ein vierstimmiger Satz, bestehend
aus einfacher Harmonik, und viele Stellen, an denen der Chor unisono
singt, ermöglichen es auch einem schwachen Chor, die Messe schnell
einzustudieren. Doch klingt die Messe deshalb keineswegs einfach. Die
gesamte Messe strahlt gerade wegen des sparsamen Einsatzes der
Ausdrucksmöglichkeiten große Ruhe und tiefe Religiosität aus. Das
\textit{Hosanna} des Sanctus und Benedictus klingt durch seine
polyphone Satzweise sehr kunstvoll und ist trotzdem leicht zu singen.

Högn wurde Ende 1924 von Max Rauscher als Chorregent abgelöst. Er wäre
sicher weit länger als drei Jahre Leiter der Kirchenmusik geblieben,
hätte sich nicht ein Nachfolger geradezu angeboten, der es ermöglichte,
auch in Ruhmannsfelden die Trennung von Schul- und Kirchendienst der
Lehrer zu vollziehen. Der 20-jährige Max Rauscher stammte aus einer
sehr musikalischen Familie, die nahe an der Pfarrkirche eine kleine
Konditorei mit Café besaß. Nach Abschluss der Kirchenmusikausbildung
war das am 14. Dezember 1924 in der Kirchenverwaltungssitzung
beschlossene Engagement am Heimatort für Rauscher sicher der bequemste
Weg, eine Stelle zu bekommen.

\begin{figure}
%
\begin{subfigure}[b]{0.5\linewidth}
\centering
\img[height=7cm]{Pfarrkirche}
\caption{Außenaufnahme}
\end{subfigure}
%
\begin{subfigure}[b]{0.5\linewidth}
\centering
\img[height=7cm]{Pfarrkirche-Altarraum}
\caption{Der Altarraum der Pfarrkirche}
\end{subfigure}
%
\caption{Die Pfarrkirche St. Laurentius zur Zeit Högns}
\end{figure}

Doch Rauschers Beschäftigungsverhältnis an seinem Heimatort war nur von
kurzer Dauer. Nörgler, die schon zu Beginn von Rauschers Tätigkeit
wenig Vertrauen in ihn setzten, sahen sich sicher bestätigt, als
dieser kaum zwei Jahre nach seiner Anstellung eine in ihren Augen
überzogene Gehaltserhöhung von mehr als 150 Mark zusätzlich zu den 33
Mark, die ihm monatlich bezahlt wurden, bei Pfarrer Fahrmeier
einforderte. Fahrmeiers Reaktion war eindeutig: Er drohte mit
Kündigung. Von der Drohung unbeeindruckt, wandte sich Rauscher an das
Bischöfliche Ordinariat in Regensburg, um seinem Wunsch Nachdruck zu
verleihen. Als Fahrmeier von der Eingabe an das Bischöfliche Ordinariat
erfuhr, stellte er Rauscher \zitat{vor versammelter
Sängerschar} zur Rede und provozierte einen offenen Streit. Verärgert
über diese öffentliche Demütigung betonte Rauscher in seinem Brief an
Fahrmeier vom 15. November 1926, diese wäre vollkommen unangebracht
gewesen, und behauptete, dass eine Zurechtweisung \zitat{zur
Zeit von Lehrer Högn} passend gewesen wäre, als \zitat{nicht
bloß Lektüre während des Gottesdienstes gelesen, sondern von seiner
Tochter Liebeleien getrieben und Schokolade gegessen wurden.} Högn
konnte natürlich diese Vorwürfe nicht auf sich sitzen lassen. Die
\zitat{ungezogenen Anschuldigungen} verurteilte er in einem
Brief an Pfarrer Fahrmeier aufs Schärfste und ersuchte
\zitat{in der Wahrung der Autorität und im Ansehen unseres
hoch verdienten und beliebten H. H. Pfarrer Fahrmeier Max Rauscher zu
kündigen, damit endlich diesem unerhörten Treiben dieses jungen Mannes
Einhalt geboten ist und nicht derselbe und noch andere mit in der
Selbstüberhebung und Geringeinschätzung anderer gestärkt werden.} Wie
zu erwarten war, wurde dem Chorregenten Max Rauscher zum 1. Januar 1927
gekündigt, aber gleichzeitig ein neuer Dienstvertrag in Aussicht
gestellt, unter der Voraussetzung, dass er auf seine Gehaltsforderungen
verzichte und in der \zitat{Kirchenratssitzung Abbitte
leistet.} Da man beim Lesen der entsprechenden Korrespondenz deutlich
erkennen kann, dass es Rauscher von Anfang an ganz bewusst auf die
Kündigung angelegt hatte, verwundert es nicht, als er sich in der
Kirchenratssitzung, in der er Abbitte leisten sollte, nicht gemäß den
Vorstellungen der Pfarroberen verhielt und ihm deshalb endgültig
gekündigt wurde.

Als Rauscher durch seine Anschuldigung Högn in den Streit hineinzog und
dieser offen die Kündigung Rauschers verlangte, wurde eines deutlich:
Ihr Verhältnis zueinander war offenbar auch vorher nicht gut gewesen.
Gehörte Högn nicht auch zu den Nörglern, die schon zu Beginn von
Rauschers Tätigkeit wenig Vertrauen in ihn setzten und Rauscher
während seiner zweijährigen Dienstzeit die Arbeit schwer machten, bis
er es schließlich mit einer übertriebenen Gehaltforderung bewusst auf
die Kündigung anlegte? Högn wäre mit Sicherheit gern weiterhin
Chorregent geblieben, wenn man, abgesehen vom finanziellen Aspekt, die
Abwechslung in der Tätigkeit betrachtet, die der Dienst bot, sowie die
Tatsache, dass er diesen mit Leib und Seele ausführte. Die Anstellung
Rauschers lieferte letztendlich den Grund dafür, dass Kirchen- und
Schuldienst nicht mehr von einer Person ausgeübt werden durften.

Eine Eintragung von Högn in einer Dirigierpartitur aus dem Notenbestand
der Ruhmannsfeldener Kirche ist beredtes Zeugnis dieser Rivalität
zwischen den beiden. Rauscher hatte einen vermeintlichen
Vorzeichenfehler in der Partitur einer Messe von Vinzenz Goller
korrigiert. In rechthaberischem Ton schrieb Högn, der von der
Richtigkeit des gedruckten Notentextes überzeugt war, später den
Kommentar \zitat{Grober Fehler! Nein! Muss „des“ heißen!}
dazu, anstatt das eingefügte Vorzeichen einfach auszuradieren – als
hätte er der Nachwelt seine Zweifel an Rauschers Kompetenz durch den
Eintrag in die Partitur, die eigentlich nur er selbst verwendete,
mitteilen wollen.

Pfarrer Fahrmeier schrieb nach der Kündigung Rauschers an die Regierung
von Niederbayern einen Brief, um eine Ausnahmegenehmigung für die
Übernahme des Chorregentendienstes durch Högn zu erhalten. Dies
zeigt, dass das Verhältnis Fahrmeier und Högn anscheinend ein sehr
gutes war. Nur Högn besitze, diesem Schreiben zufolge, die
\zitat{zur Übernahme des Kirchenchores notwendigen
musikalischen Fähigkeiten und Kenntnisse} und daher solle die
zuständige Behörde den Chorregentendienst von Högn
\zitat{bis auf weiteres} genehmigen. Högn kannte Fahrmeier
schon vor seiner Ruhmannsfeldener Zeit. In Deggendorf war Fahrmeier als
Geistlicher zwischen 1896 und 1918 tätig. Er konnte als Freund und
Hauspfarrer der Familie Högn in Deggendorf bezeichnet werden. Wenn er
einen Ausflug nach Deggendorf unternahm, genehmigte er sich stets im
Hause Högn zuerst ein Bad, bevor er seinen Besorgungen in der Stadt
nachging.

August Högns zweites Engagement in der Kirche von Ruhmannsfelden sollte
von begrenzter Dauer sein. Nur bis zu den Sommerferien 1927
genehmigte die Regierung von Niederbayern Högns kirchenmusikalische
Tätigkeit, bis ein pensionierter Lehrer gefunden war, der die
Chorregentenstelle längerfristig übernehmen könnte. Doch unter der
Ruhmannsfeldener Bevölkerung erhob sich heftiger Widerstand gegen die
Anstellung eines Pensionisten, so dass die Stelle öffentlich
ausgeschrieben und von den fünf Kirchenmusikern, die sich beworben
hatten, entschied man sich für einen gewissen Georg Roßnagel. Sein
Dienstvertrag wurde für den 7. Juni 1927 zur Unterschrift aufgesetzt.
Anscheinend hat aber Roßnagel seinen Dienst nie angetreten, denn
derselbe Dienstvertrag wurde später, mit geändertem Namen, bei der
Anstellung von Albert Schroll wieder verwendet, der am 31. Juli 1929
diesen Vertrag mit der Kirche in Ruhmannsfelden abschloss. Högns zweite
Amtszeit dauerte von Jahresbeginn 1927 bis Ende Juli 1929. Es vergingen
also nicht nur wenige Wochen, sondern über zweieinhalb Jahre, bis ein
Kirchenmusiker gefunden war, der Högn ablösen konnte.

Viele harmonisch anspruchsvolle Werke hat Högn in seiner zweiten
Chorregentenzeit komponiert. Kompositionen wie das \textit{Tantum
ergo Nr. 2 F-Dur} \textit{op. 32}, das \textit{Marienlied Nr. 3 F-Dur
op. 22} oder das \textit{Grablied Nr. 4 F-Dur op. 20} sind geradezu
durchsetzt von der Chromatik und erinnern stellenweise an komplexe
Wagner-Harmonik. Högn zieht in dieser zweiten Kompositionsphase im
Gegensatz zu seiner ersten weniger polyphone Satztechniken wie
Imitation oder Fugentechnik zum Lösen kompositorischer
Aufgabenstellungen heran, sondern vielmehr die Harmonik. Ideales
Betätigungsfeld zum Ausreizen der harmonischen Möglichkeiten waren die
Marienlieder, die er immer mehr zu einem Sololied mit geringer
werdender Chorbeteiligung umbaut. Fünf der dreizehn Marienlieder
entstanden in den Jahren 1927 bis 1929, darunter das \textit{Marienlied
Nr. 2 e-moll op. 19}, das \textit{Marienlied Nr. 3 F-Dur op. 22}, das
\textit{Marienlied Nr. 4 G-Dur op. 23}, das \textit{Marienlied Nr. 5
F-Dur op. 28} und das \textit{Marienlied Nr. 9 G-Dur op. 34}. In den
zwei Kommunionliedern aus dieser Zeit, nämlich
\textit{Kommu}\textit{nionlied G-Dur op. 21 b} und
\textit{Kommunionlied G-Dur op. 21 a}, sowie in den zwei Offertorien,
nämlich das \textit{Offertorium D-Dur op. 26} und das
\textit{Offertorium C-Dur op. 30}, lassen sich zwar nur wenige
erweiterte Akkorde finden, doch klingen diese Werke ebenfalls durchaus
farbig. Eine Besonderheit von Högns Werk aus seiner zweiten
Chorregentenzeit ist das häufig verwendete Moll. Vier der sechs Sätze
der \textit{Mater-Dei-Messe F-Dur op. 16} und das \textit{Marienlied
Nr. 2 e-moll op. 19} stehen in Moll. Im damaligen stark Dur-lastigen
Kirchenmusikrepertoire stellt das Moll eine fast exotisch wirkende
Klangfarbe dar. Aus den Kompositionen der zweiten Chorregentenzeit
Högns können auch die gestiegenen Fähigkeiten des Kirchenchores
abgelesen werden. Offenbar hat sich der Chor auch unter Max Rauscher
deutlich verbessert, so dass die Sänger die schwierige Harmonik in
Högns spätromantischen Stücken meistern konnten.