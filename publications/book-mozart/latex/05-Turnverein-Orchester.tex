\section{Leiter des Turnverein-Orchesters}

Ähnlich wie andere Vereine am Ort, veranstaltete der Turnverein bunte
Abende oder führte Theaterstücke, Singspiele und sogar Operetten auf
und übernahm somit weit über die eigentliche Vereinsaufgabe hinaus eine
wichtige kulturelle Funktion. Für den Turnverein war die Veranstaltung
solcher Unterhaltungsprogramme eine bedeutende Einnahmequelle für den
geplanten Turnhallenbau. Es ist daher keineswegs verwunderlich, dass
sich für August Högn in den Anfangsjahren in Ruhmannsfelden
insbesondere unter dem Dach des Turnvereins ein musikalisches
Betätigungsfeld auftat, obwohl die ursprüngliche Aufgabe des Vereins
nichts mit Musik zu tun hatte. Der ehemalige Vorstand unterstützte
den Verein nun als Leiter einer Sänger- und Orchesterriege.

Am 27. September 1919 wurde die Sängerriege im Turnverein gegründet. Als
erster Leiter erscheint zwar Rudolf Schwannberger, Nachbar von Högn und
Kirchenchorsänger. Doch laut Turnverein-Protokoll fungierte Högn
bereits ab 29. Dezember 1919 als Dirigent. 1921 wird Schwannberger
als Gesangswart und Högn als Gesangsdirigent der Sängerriege
bezeichnet, die sich im Saal der Brauerei Vornehm zum Proben traf. Die
Orchesterriege probte im Haus des Apothekers Voit. Dieses
Turnvereinsorchester dürfte mit dem Orchester, das auch den
Kirchenchor bei feierlichen Messen begleitete, identisch gewesen sein
und wurde einfach bei Darbietungen des Turnvereins als „Orchesterriege“
dem Sportclub „einverleibt“. Zwar müsste man das Orchester von 1923
korrekterweise eher als erweitere Kammermusikbesetzung bestehend aus
sechs Streichinstrumenten und vier Blechblasinstrumenten bezeichnen,
doch immerhin stand ein kompletter Streichersatz mit zwei
Geigenstimmen, Bratsche, Cello und Kontrabass zur Verfügung. Das
Orchestermaterial des Chorregenten Max Weig macht deutlich, dass
zumindest vor 1870 noch keine vollständige Streicherbesetzung vorhanden
war. Ein zwar kleines, aber komplett besetztes Streichorchester in
Ruhmannsfelden gab es erst in der ersten Hälfte des 20. Jahrhunderts.
Ein Grund für das erstarkende Streichorchester liegt möglicherweise
im überdurchschnittlich großen Bevölkerungszuwachs zu Anfang des 20.
Jahrhunderts. Zwischen 1870 und 1920 wuchs die Bevölkerung um fast
fünfzig Prozent auf knapp 1500 Einwohner an. Die Lehrer waren aufgrund
der intensiven musikalischen Ausbildung im Streichorchester
einsetzbar. Parallel zur Bevölkerungszahl stieg auch die Zahl der Ärzte
und Apotheker, die meist während ihrer gymnasialen Ausbildung das
Spielen eines Streichinstruments erlernt hatten. So wirkten im
Orchester Högns namentlich bekannt der Tierarzt Dr. Haug (Violoncello),
der Apotheker Vitus Voit (Violine) und der Sohn des Arztes Dr.
Danziger, Franz Danziger (Violine), späterer Kirchenchorleiter und
Nachfolger Högns mit.

Leider ist nur wenig über die Theaterveranstaltungen des Turnvereins
überliefert. Es ist nicht bekannt, wie regelmäßig sie stattfanden und
über welchen Zeitraum sich der Verein derart kulturell engagierte.
Einen kleinen Einblick in das damalige kulturelle Leben gewährt uns
aber die nicht nur in finanzieller Hinsicht erfolgreichste
„Produktion“ des Turnvereins: die Aufführungen des Singspiels
\textit{Der Holledauer Fidel} von Erhard Kutschenreuter im Jahr 1923.
Dieses Singspiel erforderte eine große Anzahl an Mitwirkenden. Im
dritten Akt ist beispielsweise ein Trachtenfestzug verlangt. Große
Anforderungen an die gesanglichen Fähigkeiten der Mitwirkenden
stellten die vielen Stücke für Sologesang, wie etwa das Liebeslied des
Fidel, das Duett des Sichbauern mit seiner Frau, das Lied der Reserl,
ein Kinderchor und die großen Chorszenen zu Beginn und zum Schluss des
Singspiels. Instrumentalstücke, wie zum Beispiel das polyphon angelegte
Vorspiel zum zweiten Akt der \textit{Holledauer Marsch} und der
\textit{Waldler Marsch}, stellten eine Herausforderung für das
Turnverein-Orchester dar.

Die ungewöhnlich große Zahl der Mitwirkenden – auf dem Foto sind 60
Personen zu sehen – machte es notwendig, die Bühne im Vornehmsaal
ausnahmsweise an der Längsseite aufzustellen, so dass die Akteure nur
direkt vom Freien aus auf die Bühne gelangen konnten. Die große
Teilnehmerzahl ist auf die Unterstützung weiterer Vereine
zurückzuführen, insbesondere jedoch auf die des Kirchenchores und der
Lehrerschaft. August Högn, der die gesamte musikalische Leitung
übernommen hatte und als Dirigent fungierte, war in seiner Eigenschaft
als Chorregent und Schulleiter der ideale Mann, weitere geeignete
Mitwirkende für das Singspiel zu gewinnen.

\begin{figure}
\img{Holledauer-Fidel}
\caption{Holledauer Fidel}
\end{figure}

Die Einkünfte der acht ausverkauften Aufführungen des Singspiels im Saal
der Brauerei Vornehm erbrachten einen Gewinn von 500 Mark, mit dem ein
Grundstück erworben werden konnte, das später als Turnplatz verwendet
wurde. Dieser durchschlagende Erfolg der Aufführungen des
\textit{Fidel} ist auch auf das Stück zurückzuführen. Mit dem
\textit{Fidel} hatte der im Rottal ansässige Lehrer Erhard
Kutschenreuter mit Abstand sein erfolgreichstes Stück geschrieben. Nach
der Uraufführung in Passau im Jahr 1920 erlebte das Singspiel schon
1938 die 3000. Aufführung. Die Ruhmannsfeldener Zuschauer strömten
vielleicht auch deswegen so zahlreich in die Vorstellungen, weil die
Handlung des Stückes zum Teil im Bayerischen Wald spielt: Der arme
Hopfenzupfer Fidel Waldhauser aus dem Bayerischen Wald verliebt sich in
Reserl, die Tochter des reichen Sichbauern aus der Hallertau. Trotz
auftretender Hindernisse, die unüberwindbar zu sein scheinen, findet
das ungleiche Paar schließlich zusammen, und es kommt zur Hochzeit.
Högns Tochter Frieda spielte die Reserl. Den Zuschauern waren die
dargestellten sozialen Verhältnisse sicher gut bekannt, denn viele von
ihnen fuhren selbst, wie es bis in die fünfziger Jahre des letzten
Jahrhunderts üblich war, jährlich in die Hallertau zur Hopfenernte, um
Geld zu verdienen.

Am 21. Juli 1923 wurde August Högn von der Gemeinde das
Ehrenbürgerrecht \zitat{aus Anlass seines 25-jährigen
Dienstjubiläums} und \zitat{für die großen Verdienste, die er
sich um Schule und Gemeinde erwarb} verliehen. Dieser Titel stellte
damals eine besondere Auszeichnung dar und wurde nur an wenige
außerordentlich verdiente Bürger verliehen. Die Ehrenbürgerurkunde
bekamen vorher Pfarrer Mühlbauer (1906) sowie August Högns Vorgänger
als Schulleiter Alois Auer (1910). Nach Högn wurde sie 1933 an Adolf
Hitler verliehen.

Der Anlass für die Ehrenbürgerrechts-Verleihung an Högn waren die
Aufführungen des \textit{Holledauer Fidel} etwas länger als ein
Vierteljahr davor. Es ist offensichtlich, dass die Singspielabende, an
denen Högn in hervorragender Weise mitgearbeitet hatte und die dem
Turnverein zum Erwerb eines Turnplatzes verhalfen, eher der Grund für
die Verleihung gewesen sind, als das 25. Dienstjubiläum, das wohl eher
einen zusätzlichen Grund darstellte.

\begin{figure}
\img{Turnhalle}
\caption{Die Turnhalle}
\end{figure}

Weit größerer Anstrengungen bedurfte es, um auf dem Turnplatz eine
eigene Turnhalle zu bauen. Die Einkünfte vieler weiterer bunter Abende
und Theateraufführungen, an denen Högn maßgeblich beteiligt war, wie
zum Beispiel der Operette \textit{Der Postillion} von Ludwig Eckl,
flossen direkt in den Turnhallenbau. Högn engagierte sich aber nicht
nur musikalisch für den Bau, sondern auch politisch. 1925 begrüßte er
eine Kommission des bayerischen Landtages in Ruhmannsfelden und äußerte
unter anderem die Bitte um Bezuschussung des Turnhallenbaus. Am 29.
Januar 1928 wurde die Halle schließlich eingeweiht und ein
\zitat{großes Orchester aus lauter Ruhmannsfeldener Musikern} spielte zur Feier des Tages.

\begin{figure}
\img{Schulhaus-1834}
\caption{Schulhaus von 1834, seit 1908 Lehrerwohnhaus}
\end{figure}