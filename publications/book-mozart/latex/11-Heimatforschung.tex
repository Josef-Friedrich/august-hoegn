\section{Heimatforschung als Betätigungsfeld im Ruhestand}

Die Heimatkunde war nicht nur ein Hobby des Pensionisten Högn. Nach der
Pensionierung verfasste Högn drei Abhandlungen zu heimatkundlichen
Themen. Heimatkundliche Forschung gehörte traditionsgemäß zum
Aufgabenfeld der Lehrer, wie eine Aktion der Regierung von Niederbayern
aus der Zeit der Weimarer Republik mit dem Motto \textit{Pflege des
Heimatgedankens} zeigt. Sie hatte das Ziel, durch Heimatkunde, den
Patriotismus und das Nationalbewusstsein der Bürger zu stärken, um
damit einen Beitrag zur Lösung der nationalen Probleme zu leisten.
Ein Rundschreiben vom 12. November 1925, das auch Högn vorlag, mit dem
Betreff \zitat{Anlegung gemeindlicher Ortsgeschichten} macht
die Intention dieser Aktion in wenigen Sätzen deutlich:
\zitat{Die Voraussetzung für den Aufstieg unseres Volkes aus
dem jetzigen Tiefstande ist die Rückkehr zu deutscher Einfachheit,
Zucht und Sitte. Dazu muss der Sinn für die Heimaterde geweckt} und
\zitat{die Liebe zum Vaterlande entzündet [...] werden. Auf
diesen Grundlagen baut sich neben der körperlichen und geistigen
Ertüchtigung der Jugend die Zukunft des deutschen Reiches auf.} Vor
allem \zitat{die Herren Lehrer [...] sollten“
}daher \zitat{„ohne weiteres Säumen und mit allem Nachdruck
[...] die Vorbereitung für die Herstellung der Ortsgeschichten}
aufnehmen. Dass die angetragenen Vorhaben eine gewisse Verbindlichkeit
hatten, zeigt schon die Aufforderung, dass \zitat{bis zum 1.
April 1928 berichtet werden wolle, [...] in welchen Gemeinden die
Vorarbeiten in Angriff genommen worden sind (Anrede des Bearbeiters).
}Högn hat sich dieser Aufgabe angenommen, wie die vielen
heimatkundlichen Zeitungsartikel zeigen, die in der Zeit von 1926 bis
1928 erschienen sind. Doch das waren nur einzelne heimatkundliche
Beiträge. Die eigentlich geforderte komplette Ortsgeschichte blieb
aus. Im Ruhestand hatte Högn schließlich Zeit, die geforderte
umfassende Ortschronik zu schreiben. Bei der Geschichte von
Zachenberg etwa, gibt es ganz konkrete Anhaltspunkte, wonach die
Initiative zur Inangriffnahme der Arbeit nicht von Högn selbst ausging,
sondern eine an ihn herangetragene Bitte den Grund für die Entstehung
der \textit{Geschichte} lieferte.

Zwei Jahre nach Högns endgültiger Pensionierung, erschien 1949 die
\textit{Geschichte von Ruhmannsfelden.} Dem rüstigen Rentner muss es
eher leicht gefallen sein, Material für seine Arbeit zu finden, doch
hatte er doch bereits in den vergangenen Jahrzehnten viele Fakten
gesammelt und brauchte die vielen bereits existierenden Einzelbeiträge
zur Heimatgeschichte nur noch in einem Werk zusammenfassen. Die
geschichtliche Entwicklung zweier wichtiger Institutionen am Ort
hatte Högn schon vor dem Zweiten Weltkrieg bearbeitet:

Die Entwicklung der Schule war in einer Beilage zum \textit{Deggendorfer
Donauboten} 1927 erschienen und anlässlich der 100-jährigen
Einweihungsfeier der Pfarrkirche wurde 1928 bis 1929 in mehreren Teilen
die Geschichte der Pfarrkirche St. Laurentius im \textit{Viechtacher
Tagblatt} abgedruckt. Information über die Entstehung des Namens
„Ruhmannsfelden“ oder den Zeitpunkt der Marktrechtsverleihung waren
Thema zweier Zeitungsartikel aus dem Jahre 1926 und fanden an
exponierter Stelle Eingang in die \textit{Geschichte von
Ruhmannsfelden}. Högn hatte sich schon 1922 zur Frage des Ursprungs des
Ortsnamens an den Straubinger Historiker Dr. Joseph Keim gewandt, der
die noch heute gültige Erklärung gab, wonach der Ort nach dem
Rodungsarbeiter „Rumar“ benannt wurde. Davor wurde der Name etwas
trivial mit „Ruht der Mann in Felde“ erklärt. Auch versuchte Högn
sowohl in dem erwähnten Artikel als auch in der \textit{Geschichte von
Ruhmannsfelden} dem Mythos, dass einst ein Schloss am Leitenberg
bestand, entgegenzuwirken. Trotz einiger Recherchearbeiten ist es
unverkennbar, dass Högn besonders zur frühen Geschichte auf sehr
wenige Informationen zurückgreifen konnte. Das einleitende Kapitel
\textit{Wie hat es vor seiner Entstehung ausgesehen?} ähnelt in seinem
Aufbau mehr einem Märchenanfang als dem Beginn einer historischen
Abhandlung und entbehrt sicherlich aller wissenschaftlichen
Grundlagen. Wegen fehlender Dokumente musste von Namen wie etwa
\textit{Ruhmannsfelden} oder \textit{Laurentius-Kirche} möglichst auf
die geschichtliche Entwicklung geschlossen werden. Manchmal folgerte
Högn aus diesen Anhaltspunkten etwas zuviel. Es stimmt sicher, dass
alle dem Hl. Laurentius geweihten Kirchen außerhalb der Ortschaften
standen, dass aber die Laurentius Kirche in Ruhmannsfelden zuerst aus
Holz war, unter den Aldersbachern dann aus Stein und dass dann öfter
Gottesdienste stattfanden, wie er schreibt, ist wohl reine Spekulation.
Als Lückenfüller dürften die Kapitel über die Namen der
Ruhmannsfeldener Fluren und Gassen und das Kapitel \textit{Höhenlagen
in unserem Heimatgau und Barometerstand} gedient haben.

\begin{figure}
\begin{subfigure}[b]{0.5\linewidth}
\img{Geschichte-von-Ruhmannsfelden}
\caption{Titelblatt der Geschichte von Ruhmannsfelden}
\end{subfigure}
\begin{subfigure}[b]{0.5\linewidth}
\img{Geschichte-von-Zachenberg}
\caption{Titelblatt der Geschichte von Zachenberg}
\end{subfigure}
\end{figure}

Die Entstehung der \textit{Geschichte und Chronik der freiwilligen
Feuerwehr Ruhmannsfelden} ist eng mit dem Ende von Högns
Schriftführertätigkeit bei der Feuerwehr verbunden. Am Stefani-Tag
1950 wurde Johann Freisinger zum Schriftführer der Feuerwehr gewählt
und löste August Högn nach 40-jähriger Dienstzeit ab. Die
\zitat{in dankbarer Erinnerung} der Feuerwehr gewidmete
Geschichte hat Högn geschrieben, um seine Tätigkeit als Schriftführer
abzurunden und sein gesammeltes Wissen in diese Arbeit einfließen
zu lassen. Ein passender Übergabetermin der Chronik an die Feuerwehr
wäre der zur Verabschiedung von Högn eigens veranstaltete Ehrenabend
am 11. März 1951 gewesen, doch die Chronik wurde dem letzten Eintrag
zufolge erst nach dem 1. April 1951 fertig. Die Arbeit war schon
angekündigt und fand lobende Erwähnung in der Abschiedsrede des
Feuerwehrkommandanten. Die fertige Chronik überreichte Högn dann kurze
Zeit später persönlich seinem Nachfolger als Schriftführer.

Die Abhandlung ist nach Jahren gegliedert und lässt daher oft
Einzelfakten unverbunden nebeneinander stehen, was das Verständnis beim
Lesen erschwert. Sie ist wohl eher als eine Stoffsammlung zur
weiteren Bearbeitung anzusehen, denn als ein unterhaltsamer Lesestoff
für die breite Bevölkerung und war sicher nie für den Druck bestimmt.
Lediglich die länger ausformulierte Gründung der Feuerwehr und der
Bericht über den lange dauernden und deshalb in mehreren Jahren
auftauchenden Streit der Gemeinden Ruhmannsfelden und Zachenberg um
eine gemeinsam angeschaffte Feuerwehrspritze bleiben dem Leser in
Erinnerung und gehen nicht in der Flut von Einzelinformationen unter.

Ein auswärtiger, nicht in der Gemeinde Zachenberg ansässiger Bürger
lieferte die Initiative zur Entstehung von Högns
\textit{Heimat-Geschichte der Gemeinde Zachenberg}. Der Finanzzollrat
Anton Trellinger fertigte einen 63-seitigen Akten-Auszug über die
Gemeinde Zachenberg an. Da Trellinger die Gemeinde Zachenberg nur
flüchtig kannte und gesundheitlich angeschlagen war, bat er in einem
Brief die Gemeinde Zachenberg um einen ortskundigen Fachmann, der seine
Arbeit ergänzen und fortführen könnte. Den Sachbearbeitern bei der
Gemeinde dürfte es nicht schwer gefallen sein, Trellinger die gesuchte
Person zu nennen: August Högn. Lange bevor Anton Trellinger der
Gemeinde seine Arbeit übersandte, nämlich am 25. Februar 1952, und am
selben Tag Högn persönlich anschrieb, hatte dieser schon mit der
Recherche begonnen und den Mettener Benediktiner-Pater Wilhelm Fink von
der beabsichtigten Geschichte von Zachenberg in Kenntnis gesetzt und um
wissenschaftliche Betreuung gebeten.

Trellinger lieferte mit seinen Archivauszügen nicht nur einen wichtigen
Grundstock für die \textit{Heimat-Geschichte der Gemeinde Zachenberg},
sondern er gab ihm auch Tipps und Literaturhinweise. Nach fast
zweijähriger Arbeitszeit wurde die \textit{Geschichte von Zachenberg}
an Weihnachten 1953 fertig und an Wilhelm Fink zum Korrekturlesen
übersandt. Fink hatte besonders zum ersten Teil der Abhandlung einige
Änderungsvorschläge. Es ist deshalb nicht verwunderlich, dass sich die
ersten Teile der \textit{Geschichte von Ruhmannsfelden} und der
\textit{Geschichte von Zachenberg} in Aufbau und Darstellung ziemlich
stark unterscheiden, obwohl beide Teile die geschichtliche Entwicklung
von den Anfängen der beiden benachbarten und sich unter den gleichen
Herrschaftsverhältnissen entwickelnden Gemeinden zum Thema haben.
Fink hat offensichtlich die \textit{Geschichte von Ruhmannsfelden}
nicht Korrektur gelesen. Im von Fink als druckreif befundenen zweiten
Teil geht Högn auf die 15 Dörfer, 13 Weiler und 10 Einöden, also auf
die insgesamt 38 Ortschaften der Gemeinde Zachenberg im Einzelnen ein.

Die Kapitel über die einzelnen Ortschaften ähneln mehr genealogischen
Studien als historischen Abhandlungen. Manchmal lassen sich aus diesen
Kapiteln Stammbäume einzelner Bauernfamilien über Jahrhunderte hinweg
ablesen. Meist jedoch konnten die Angaben aus den sehr alten
Aktenauszügen Trellingers nicht mit den neueren Informationen aus dem
Gemeindearchiv oder von den befragten Bewohnern der jeweiligen
Ortschaft bruchlos aneinander gereiht werden. Die sehr schematische
Beschreibung der kleinsten Ortschaften selbst – sie beginnt immer mit
der Herkunfts-Erklärung des Ortsnamens und endet nach Vorstellung der
einzelnen Bauernfamilien mit Angabe der damals aktuellen Einwohner- und
Häuserzahl – wird durch Sagen und Erzählungen aufgelockert, die Högn
von älteren Bewohnern der einzelnen Orte erzählt bekam und in Form von
Nacherzählungen wiedergibt. Viele der von Wilhelm Fink vorgeschlagenen
Ergänzungen zum dritten, allgemeinen Teil – hier geht es in erster
Linie um die damals aktuelle Organisation der Gemeinde – hat Högn nicht
angenommen, weil beiden Gemeinden, Zachenberg und Ruhmannsfelden,
doch sehr ähnliche Strukturen aufwiesen und er bereits Einiges in der
Geschichte von Ruhmannsfelden erwähnt hatte, wie zum Beispiel die aus
der Pfarrei hervorgegangenen Priester, die Lehrer oder die Polizei.
Deshalb erörtert er hier hauptsächlich die aktuellen wirtschaftlichen
und infrastrukturellen Aspekte der Gemeinde Zachenberg.

Als er die überarbeitete Fassung der Geschichte kurz vor Ostern 1954 der
Gemeinde Zachenberg übergab, rechnete er sicher mit einer schnellen
Drucklegung seiner Arbeit, ähnlich wie bei der \textit{Geschichte von
Ruhmannsfelden}, sonst hätte er sich nicht schon von der Druckerei
Michael Laßleben in Kallmünz ein Angebot machen lassen und sich sogar
schon einen Werbetext überlegt.

Erst am 7. Mai 1956 versicherte der Zachenberger Bürgermeister Bielmeier
Högn auf Anfrage, dass der Gemeinderat über die Veröffentlichung der
\textit{Geschichte von Zachenberg} in der nächsten Sitzung beraten
würde. Die Gemeindevertreter hatten wohl von Anfang an kein Interesse
an einer Drucklegung. Kein Wort über den Druck des Buches ist im
Protokoll der Gemeinderatssitzung vom 17. Mai 1956 zu finden.
Stattdessen wurde beschlossen, Högn zum Ehrenbürger der Gemeinde
Zachenberg zu ernennen und ihm als \zitat{Entschädigung
}für seine Bemühungen ein \zitat{Geldgeschenk im Ermessen des
1. Bürgermeisters} zukommen zu lassen. Trotz dieser Ehren dürfte die
immer noch nicht eingeleitete Veröffentlichung der \textit{Geschichte
von Zachenberg} für Högn eine Enttäuschung gewesen sein. Noch an
seinem 80. Geburtstag, also über vier Jahre nach Fertigstellung, machte
er sich Hoffnung auf eine baldige Drucklegung, die bis zum heutigen Tag
auf sich warten lässt.

August Högn war nicht nur auf heimatkundlichem Gebiet schriftstellerisch
tätig. Es gibt Anzeichen dafür, dass er sich auch mit musikhistorischen
Themen beschäftigt haben könnte. Ein Briefwechsel aus dem Jahr 1947
mit Ferdinand Haberl, Direktor der Kirchenmusikschule in Regensburg,
in dem es um die Entstehung und Herkunft des Weihnachtsliedes
\textit{Stille Nacht} geht, stützt diese Annahme.