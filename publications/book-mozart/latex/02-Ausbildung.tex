\section{Ausbildung zum Lehrer und zum Musiker}

Die Entscheidung, eine Ausbildung zum Volksschullehrer anzutreten, wurde
für August Högn sicher dadurch erleichtert, dass sich in der
Deggendorfer Arachauergasse Nr. 94 (heute Bräugasse Nr. 14) wenige
Minuten Fußmarsch entfernt von Augusts Elternhaus eine
Präparandenschule befand. Wie zu Volksschulzeiten konnte er wieder bei
seinen Eltern wohnen und war nicht mehr auf das Mettener Internat
angewiesen.

Die Lehrerausbildung dauerte damals fünf Jahre. An eine dreijährige
Vorbereitungsphase an einer Präparandenschule schloss sich die
eigentliche zweijährige Ausbildung in der Lehrerbildungsanstalt in
Straubing an. Neben der Bezeichnung „Präparandenschule“, die so viel
bedeutet wie „Schule der Vorzubereitenden“, ist aus heutiger Sicht
vor allem ungewöhnlich, dass es damals gleich zwei Schularten gab, die
vom angehenden Lehrer durchlaufen werden mussten, nämlich die
dreijährige Vorbereitungsphase an einer Präparandenschule und die
zweijährige Ausbildung in der Lehrerbildungsanstalt. Dies lässt sich
aus der historischen Entwicklung der Lehrerausbildung erklären. Es war
Jahrhunderte lang Praxis, dass Handwerker zusätzlich zu ihrer
beruflichen Tätigkeit auch die Unterweisung der Schulkinder in den
Grundfertigkeiten wie Lesen, Schreiben und Rechnen übernahmen. Die
Verordnung vom 4. September 1823 schrieb erstmals verpflichtend eine
zweijährige Ausbildung für angehende Lehrer vor und hob so
gewissermaßen den Berufsstand des Volksschullehrers im Königreich
Bayern aus der Taufe. Zuvor sollten die Anwärter drei Jahre lang bei
einem \zitat{tüchtigen Schullehrer} oder einem
\zitat{vorzüglichen Geistlichen} eine Art Lehre oder
Praktikum absolvieren, ehe man sie ins Lehrerseminar aufnahm. Da sich
diese Art der Vorbildung als nicht sehr effektiv erwiesen hatte, wurde
mit dem Normativ vom 29. September 1866 die dreijährige
Vorbereitungszeit durch Einführung der Präparandenschulen straffer
organisiert.

\begin{figure}
\begin{subfigure}[b]{0.5\linewidth}
\img{Andreas-Hoegn}
\caption{Andreas Högn}
\end{subfigure}
\begin{subfigure}[b]{0.5\linewidth}
\img{Ludwig-Hoegn}
\caption{Ludwig Högn}
\end{subfigure}
\end{figure}

\begin{figure}
\begin{subfigure}[b]{0.5\linewidth}
\img{Joseph-Hoegn}
\caption{Joseph Högn}
\end{subfigure}
\begin{subfigure}[b]{0.5\linewidth}
\img{Otto-Hoegn}
\caption{Otto Högn (oder Andreas Högn?)}
\end{subfigure}
\end{figure}

Ebenso ungewöhnlich aus heutiger Sicht und kaum mit der Ausbildung der
Grund- und Hauptschullehrer vergleichbar, ist die starke Gewichtung des
Musikunterrichts in der gesamten damaligen Volksschullehrerausbildung,
besonders in den Anfangsjahren der Präparandenschule. Das Fach Musik –
unterteilt in die Teilbereiche \zitat{Gesang, Violine,
Klavier, Orgel} und \zitat{Harmonielehre} – hatte innerhalb
des Fächerkanons einen so hohen Stellenwert, dass es neben
Religionslehre, Deutsch und Rechnen ebenfalls als Hauptfach bezeichnet
wurde. Mit sechs Wochenstunden machte der Musikunterricht mehr als ein
Fünftel der Gesamtstundenzahl aus. Das zeigt deutlich die Absicht, die
Ausbildung der angehenden Lehrer auf den Chorregenten- und
Organistendienst auszurichten. Wie alle Fächer, so wurde auch der
Instrumentalunterricht von Volksschullehrern erteilt, die an die
Präparandenschule berufen worden waren. Da die Schüler vor Eintritt in
die Präparandenschule keine Vorkenntnisse im Spiel der Musikinstrumente
mitzubringen brauchten, kam es oft vor, dass im Instrumentalunterricht
bei einer Gruppenstärke von durchschnittlich zehn Schülern sehr große
Leistungsunterschiede herrschten. So musste sich beispielsweise ein
fortgeschrittener Schüler, wie es August Högn im Klavierspiel war,
zusammen mit Anfängern eine Stunde teilen. Beim Klavierunterricht stand
die Hinführung auf das Orgelspiel im Vordergrund und somit die Pflege
des \zitat{gebundenen Spiels.} Die Schüler sollten in der
dreijährigen Ausbildung die Fähigkeit entwickeln, leichte Sonaten und
Sonatinen von Bertini, Czerny, Clementi, Dussek, und Kuhlau zu spielen.
Der Orgelunterricht begann ab dem II. Kurs, also dem zweiten Schuljahr,
nach Barners Schule \textit{Anfänge des Pedalspiels} und bediente sich
im III. Kurs der Orgelschule von Herzog. Praktische
Kirchenmusikerfahrung konnten die Schüler als Choristen werktags bei
der Gestaltung von Schulmessen und an Feiertagen bei Gottesdiensten an
der königlichen Kreisirrenanstalt mit Messkompositionen der Cäcilianer
Witt, Zangl, Haberl und Ett sammeln. Sowohl die Lehrer als auch die
Schüler waren Mitglieder des Bezirk-Cäcilien-Vereins Metten und des
Pfarr-Cäcilien-Vereins Deggendorf. Von 1890 bis 1895 war August Högn
laut Personalbogen Schüler an der Präparandenschule in Deggendorf, also
ganze fünf Jahre. Weshalb Högn die Präparandenschule zwei Jahre länger
als normal besuchte, konnte nicht geklärt werden. Vielleicht gab es an
der Präparandenschule Vorbereitungsklassen für sehr junge Schüler –
Högn war bei Antritt seiner Lehrerausbildung erst elf Jahre alt – oder
er hat eine oder mehrere „Ehrenrunden“ gedreht? 1895 setzte Högn seine
Ausbildung zum Lehrer an der Lehrerbildungsanstalt in Straubing fort.

So eng die Präparandenschule und die Lehrerbildungsanstalt inhaltlich
ineinander griffen, so unterschiedlich dürfte August Högn die beiden
Schulen bezüglich der Gewährung von persönlichem Freiraum erlebt haben.
Der seit Bestehen der Lehrerbildungsanstalt geltende Internatszwang
verlieh dem Seminar in Straubing den Charakter einer geschlossenen
Anstalt. Ein von 5 Uhr früh bis 21 Uhr genau festgelegter Tagesablauf
forderte von den Schülern große Anpassung. Selbst Spaziergänge fanden
nicht ohne Aufsicht statt. Man legte großen Wert auf die Fähigkeit,
sich unterzuordnen, auch wenn dies nicht explizit im Lehrplan
aufgeführt war.

Während die Stundenzahl und die Fächerverteilung in der Musik in etwa
mit dem Lehrplan der Präparandenschule zu vergleichen war,
unterrichteten in Straubinger Seminar nicht ehemalige Volksschullehrer
mit normaler Lehrerausbildung, sondern speziell ausgebildete Musiker.
Anton Schwarz prägte von 1892 bis 1923 das musikalische Leben an der
Lehrerbildungsanstalt maßgeblich. Er hatte nach zwölfjährigem
Volksschuldienst bei Joseph Rheinberger Komposition und Orgel an der
königlichen Musikschule in München studiert und konnte sein Wissen in
den Fächern Harmonielehre, Orgel und Gesang an die Schüler weitergeben.
Sein umfangreiches kompositorisches Schaffen, darunter vier Messen,
mehrere Offertorien und Motetten, lieferte für manche Schüler einen
Ansporn, sich selbst im Komponieren zu versuchen.

Die Seminaristen erhielten zwar in ihrer Ausbildung keinen
Kompositionsunterricht, dafür vermittelte ihnen der
Harmonielehreunterricht die wichtigsten tonsetzerischen Regeln, mit
denen sie erste kompositorische Schritte wagen konnten. Wie die im Fach
Harmonielehre gestellten Aufgaben zeigen, gehörte der Umgang mit dem
vierstimmigen Chorsatz zum Handwerkszeug eines angehenden Lehrers. Man
kann deshalb August Högns Werke – sie haben meist den vierstimmigen
Satz als Grundgerüst – als Früchte des Musikunterrichts der
Lehrerausbildung betrachten.

Ein Beleg dafür, dass Högn bereits in der Seminarzeit komponiert hat,
ist das \textit{Veni creator spiritus B-Dur} für vierstimmigen
Männerchor aus den Jahren 1897 und 1898, die älteste erhaltene
Komposition von Högn. Die reine Männerbesetzung lässt vermuten, dass
dieses Werk im Seminar unter Högns angehenden Lehrerkollegen gesungen
worden ist. Nicht zu überhören ist der Einfluss des sogenannten
Cäcilianismus auf das kleine Chorstück. Diese historisierende
Kirchenmusikbewegung orientierte sich vor allem an der frühen
Vokalpolyphonie bis zur Renaissance. Högn verwendete im \textit{Veni
creator spiritus B-Dur} an mehreren Stellen imitatorische
Satztechniken, wie sie besonders in der Musik der Renaissance verwendet
wurden. So imitieren die einzelnen Singstimmen einem Kanon ähnlich den
Melodieverlauf anderer Singstimmen. Eine gewisse jugendliche
Experimentierfreude hatte Högn beim Komponieren dieses Stückes
unüberhörbar inspiriert. Sehr gewagt und auch ein wenig holprig klingt
eine Stelle, an der Högn einen lang gehaltenen verminderten Septakkord
– der Akkord mit der schärfsten Dissonanz – um eine kleine Sekunde –
dem kleinsten und schärfsten Intervall – plötzlich nach oben
verschiebt. Diese expressive Wendung hat Högn in seiner späteren
Fassung (op. 15 Nr. 2) getilgt. Ingesamt wirkt die Spätfassung viel
gemäßigter, ausgewogener und deshalb gekonnter und reifer. Ein
Vergleich der beiden Fassungen zeigt deutlich den kompositorischen
Fortschritt Högns.

Im Juli 1898 wurde August Högn das Reifezeugnis der
Lehrerbildungsanstalt Straubing mit der Note 3 erteilt.