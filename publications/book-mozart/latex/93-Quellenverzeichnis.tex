\section{Quellenverzeichnis}

\subsection{Literatur}

Dantl, Georg\textbf{,} \textit{Vom Schullehrling zum Schulmeister –
Geschichte der Lehrerbildung im 19. Jahrhundert,} in: Oberpfälzer
Raritäten, Band 5, Verlag der Buchhandlung Taubald, Weiden, 1989

Gärtner, Helmut\textbf{,} \textit{Deggendorfer Originale – Originelles
Deggendorf,} Morsak Verlag, Grafenau, 2. Auflage, 1995

Geyer, Otto\textbf{,} \textit{Schule und Lehrer in Niederbayern,} Hrsg.
Otto Glaser, Niederbayerischer Bezirkslehrerverein im BLLV,
Neue-Presse-Verlag-GmbH, Passau, 1964, 235 Seiten

Goller, Martina\textbf{,} \textit{Die Musik in der Lehrerbildung
Niederbayerns und ihre Ausstrahlung am Beispiel niederbayerischer
Lehrerkomponisten dargestellt an der Präparandenschule Deggendorf und
am Lehrerbildungsseminar Straubing,} Zulassungsarbeit zur ersten
Staatsprüfung für das Lehramt an Hauptschulen in Bayern eingereicht
bei Prof. Dr. Eckard Nolte im Fach Musikpädagogik, Deggendorf, April
1988

Högn, August\textbf{,} \textit{Geschichte und Chronik der freiwilligen
Feuerwehr Ruhmannsfelden,} Manuskript, 1951 (Feuerwehr)

Högn, August\textbf{,} \textit{Geschichte von Ruhmannsfelden,} Michael
Laßleben, Kallmünz, 1949 (Ruhmannsfelden)

Högn, August\textbf{,} \textit{Heimat-Geschichte der Gemeinde
Zachenberg,} Manuskript, 1954 (Zachenberg)

Lippert, Heinrich\textbf{,} \textit{Die Präparandenschule Deggendorf
(1866-1924) – Zur Geschichte einer niederbayerischen
Lehrerbildungsanstalt,} in: Deggendorfer Geschichtsblätter 17,
Deggendorf, 1996, S. 153 – 192

Proft, Hans \textbf{\textit{,}}\textit{„Immer froh und heiter bleibt der
Kutschenreuter“ – Leben und Werk des niederbayerischen Komponisten
Erhard Kutschenreuter,} Verlag Karl Stutz, Passau, 2004

Stengel, Georg Josef\textbf{,} \textit{Geschichte der
Lehrerbildungsanstalt Straubing von 1824-1924,} Manz, Straubing, 1925

\subsection{Quellen aus Archiven}

\noindent Reicheneder-Chronik:

\begin{compactitem}
\item Die Seelsorger der Pfarrei (Seelsorger)
\item Auszüge a. d. Protokollbüchern: Ehrenbürger von Ruhmannsfelden
(Ehrenbürger)
\item Schul- und Bildungswesen in Ruhmannsfelden (Schulwesen)
\item Der Pfarrmesner von Ruhmannsfelden (Pfarrmesner)
\item Religiöse Feiern in der Pfarrei (Feiern)
\item Die Pfarrkirche, C. Nach 1820, II. Einrichtung – Die Orgel (Orgel)
\item Chronik der Volksschule Ruhmannsfelden
\item Protokollbuch der Feuerwehr Ruhmannsfelden
\end{compactitem}