\section{Wanderjahre}

Mit dem Eintritt in den so genannten Vorbereitungsdienst, am 1.
September 1898, begannen für Högn unruhige Jahre, wie sie für die erste
Zeit des Lehrerberufs typisch waren und auch heute noch sind. Im
September 1898 absolvierte er an seiner ehemaligen Knabenschule in
Deggendorf ein Praktikum bei den Lehrern Buchner und Edelmann. Danach
durfte er sich Aushilfslehrer nennen. In der Folgezeit übernahm er, wie
seine Berufsbezeichnung schon verrät, übergangsweise Vertretungen für
erkrankte Lehrer. Vier Monate unterrichtete er aushilfsweise in
Neukirchen bei Haggn im Landkreis Straubing-Bogen (1. November 1898 bis
1. März 1899), nach einer zweimonatigen Pause zweieinhalb Monate in
Schaufling bei Deggendorf (1. Mai 1899 bis 15. Juli 1899) und nach den
Sommerferien bis Weihnachten drei Monate in Geratskirchen in der Nähe
von Eggenfelden (15. September 1899 bis 25. Dezember 1899). Seine Zeit
als Aushilfslehrer endete mit dem so genannten „Tag der ersten
eidlichen Verpflichtung“ am 20. Dezember 1899. Von diesem Zeitpunkt
an lautete seine Berufsbezeichnung Hilfslehrer. An seiner Verwendung
als Vertretung änderte sich anfangs nur wenig. Vom 26. Dezember 1899
bis 15. Januar 1902 unterrichtete er in Zeilarn bei Simbach am Inn und
vom 16. Januar 1902 bis 31. Dezember 1902 übernahm er seine erste
längerfristige Stelle in Wallersdorf. In diesem Jahr legte er die
Anstellungsprüfung für den Volksschuldienst mit der Note 3 ab und wurde
ab 1. Januar 1903 in Wallersdorf im Rang eines Schulverweser
weiterbeschäftigt. „Schulverweser“ ist eine 1896 eingeführte
Bezeichnung für den Vertreter des ersten Lehrers. In Wallersdorf
lernte er auch seine spätere Ehefrau Emma kennen.

\begin{figure}

\begin{subfigure}[b]{0.5\linewidth}
\centering
\img[height=7cm]{Emma-Hoegn}
\caption{Emma Högn}
\end{subfigure}%
%
\begin{subfigure}[b]{0.5\linewidth}
\centering
\img[height=7cm]{August-Hoegn}
\caption{August Högn}
\end{subfigure}
\caption{Das Ehepaar Högn}
\end{figure}

Am 20. Juli 1904 fand die Hochzeit von August Högn und der neun Jahre
jüngern, damals 16-jährigen Emma Gerstl mit üppigem Menü statt, wie der
Einladung entnommen werden kann. Emma Gerstl stammte aus einer
wohlhabenden Bierbrauerfamilie, die in Gründobl, einem kleinen Ort in
der Nähe von Wallersdorf, ein großes Anwesen mit Wirtshaus besaß. Das
junge Paar musste heiraten, weil Emma Högn schwanger war. Sie gebar
Zwillinge, die kurz nach ihrer Geburt starben. Zum 1. Juni 1905 wurde
Högn nach Eberhardsreuth im Landkreis Grafenau versetzt. Hier kam die
Tochter Elfriede, genannt Frieda, am 14. August 1906 zur Welt. Der nach
dem Vater benannte Sohn August erblickte am 17. Januar 1912 schon in
Högns nächstem Einsatzort das Licht der Welt: in Ruhmannsfelden.

Es liegt vielleicht auch an den vielen Ortwechseln, die Högn innerhalb
weniger Jahre vollziehen musste, dass gerade aus der Zeit seines
Eintritts ins Berufsleben nur wenige Kompositionen erhalten sind. Dass
er komponiert hat, beweisen zwei Werke: das \textit{Ave Maria F-Dur op.
4} und der \textit{Marsch „In} Treue \textit{fest!“ D-Dur}.

Für unsere Ohren klingt Högns \textit{Ave Maria} für zwei hohe Stimmen
mit Orgelbegleitung höchst seltsam. Der junge Komponist Högn lag um
die Wende vom 19. zum 20. Jahrhundert auf der Höhe seiner Zeit und hat
sein Werk im damals aktuellen Kirchenmusikstil des Cäcilianismus
geschrieben. Befremdlich an dem Stück klingt heutzutage vielleicht die
Kombination von Stilelementen sehr alter Musik mit spätromantischer
Stilistik. Die zwei Gesangsstimmen des \textit{Ave Maria} würden sich
ohne weiteres gut als ein Bicinium Orlando di Lassos (1530-1594)
verkaufen lassen. Die Orgelbegleitung erinnert dagegen mit ihrer
gewagten Harmonik und der diffusen Stimmführung eher an die Orgelmusik
des Spätromantikers Max Reger (1873-1916).

Der \textit{Marsch „In Treue fest!“} ist August Högns einzige
Komposition, die gedruckt wurde. Er erschien 1905 im Selbstverlag des
Komponisten und wurde durch die \textit{Cl. Attenkofer’sche Buch- u.
Musikalienhandlung} in Straubing vertrieben. Högn brachte sein Werk,
das der niederbayerischen Lehrerschaft gewidmet ist, gleich in zwei
Fassungen auf den Markt, nämlich in einer Fassung für Klavier zu zwei
Händen und in einer für Klavier zu vier Händen. Das Stück konnte als
einzige Komposition Högns nicht in Ruhmannsfelden aufgefunden werden.
Sie befindet sich im Bestand der Bayerischen Staatsbibliothek in
München. Vielleicht verkaufte sich der Marsch so gut, dass der Druck
1910, als Högn nach Ruhmannsfelden kam, bereits vergriffen war und
deshalb hier nicht vorhanden war.

\begin{figure}
\centering
\img[width=6cm]{Marsch-In-Treue-fest}
\caption{Titelblatt des Marsches „In Treue fest“}
\end{figure}

Verkaufsfördernd war vielleicht die abwechslungsreiche Kompositionsart
des Marsches. So treffen ständig Motive mit Achtel-, Triolen- und
Sechzehntel-Rhythmus aufeinander. Dieser Bemühungen um Abwechslung
bedurfte es möglicherweise auch deshalb, weil das Harmonieschema
mehrerer Formteile des Marsches identisch ist. Der zweite Teil klingt
nicht wie ein eigenständiger Abschnitt sondern wie eine Variation des
ersten Formteils. Wird der Marsch gemäß Högns Wiederholungsvorschriften
ausgeführt, so hört man dieses Harmonieschema genau sechs Mal.
Besonders gelungen ist dagegen die Introduktion und das Trio. Die
fanfarenartigen aufsteigenden Dreiklangsbrechungen im Bassregister
verleihen der Einleitung einen erhabenen und zugleich mitreißenden
Charakter, während das Trio in Moll den ruhenden Gegenpol zum
triumphalen Dur der anderen Abschnitte bildet.

Der Marsch hebt sich von Högns anderen Stücken auch deshalb ab, weil es
sich bei dem \textit{Marsch „In Treue fest!“} um eine der wenigen
weltlichen Kompositionen und die einzige erhaltene rein instrumentale
Komposition von Högn handelt.