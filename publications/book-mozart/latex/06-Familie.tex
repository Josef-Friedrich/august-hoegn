\section{Alte und neue Familie}

Völlig überraschend starb am 19. Juni 1926 August Högns Ehefrau Emma im
Alter von 39 Jahren an einem Gallendurchbruch. Mitten aus dem Leben
gerissen hinterließ sie, die kurze Zeit davor Großmutter geworden
war, einen erst 14 Jahre alten Sohn. Das Tragische an Emmas Tod war,
dass sie an einer Krankheit starb, die schon zur damaligen Zeit hätte
behandelt werden können, wären die Symptome frühzeitig erkannt worden.
Mit nur 47 Jahren war August Högn Witwer und musste sich nun allein um
die Erziehung seines Sohnes Gustl kümmern. Zur Verrichtung der
alltäglichen Arbeiten wurde die Haushälterin Rosa Beischmied
angestellt.

Über die Ehe Högns mit Emma ist kaum etwas bekannt. Gerüchten zufolge
soll Högns Ehefrau eine Affäre mit dem Nachbarn und Kirchenchorsänger
Rudolf Schwannberger gehabt haben. Ein möglicher Grund, weshalb Högn
kein zweites Mal geheiratet hat, könnte seine große Liebe zu Emma
gewesen sein. Ein anderer dürfte auch seine partnerschaftliche
Beziehung zu Rosa Beischmied gewesen sein, die sich unweigerlich im
Lauf der Jahre entwickelte. Beide wurden als eingespieltes Team
beschrieben. 35 Jahre lang, bis zu seinem Tod, begleitete Rosa
Beischmied Högn und lebte mit ihm in derselben Wohnung. Je nach
Lebenslage unterstützte ihn die „Högn Rosl“, wie sie von einigen
Zeitzeugen genannt wurde, etwa bei der Erziehung seines Sohnes, aber
auch beim Schuldienst, wenn sich bespielsweise Högns Schüler
Vitamintabletten auf seine Anordnung hin, bei Rosa abholen mussten.
Aber auch im Alter wurde Högn von Rosa gepflegt, besonders nach seinem
Schlaganfall. Nicht selbstverständlich und daher ebenso ein Indiz für
das doch über das rein Dienstliche hinausgehende Verhältnis zwischen
Högn und Beschmied war die Tatsache, dass Rosa Beischmieds
\textit{illegale} Tochter Mathilde, wie im Taufregister über die
uneheliche Tochter zu lesen ist, nach dem Tod ihrer Großeltern in Högns
Wohnung einziehen durfte. In der großen Wohnung im Schulhaus erhielt
sie ein eigenes Zimmer und fühlte sich von Högn soweit akzeptiert, dass
sie ihn heute als \textit{Ersatzvater} bezeichnet. Eine Anekdote, die
Mathilde Beischmied in einem Interview erzählte, mag das gute und sehr
freundschaftliche Verhältnis Högns zu seiner Ziehtochter beleuchten und
Einblick ins damalige „Familienleben“ geben: Als Belohnung dafür, dass
die kleine Mathilde für Högn das Bier holte, bestand Högn darauf, dass
auch sie etwas von dem Getränk bekam und fragte deshalb ihre Mutter
vorwurfsvoll: \zitat{\textup{„Kriegt sie heute kein Bier?“}
}Auch als er für mehrere Jahre seine noch schulpflichtige Enkelin
Inge bei sich aufnahm, zeigte sich Högn in der Funktion des
Ersatzvaters. Als Grund, weshalb sie zu ihrem Großvater kam, kann wohl
die Trennung ihrer Mutter Frieda von ihrem ersten Ehemann und die neue
Bekanntschaft mit ihrem späteren Ehemann Dr. Karl Schlumprecht
angesehen werden. Da auch Högns Sohn Gustl noch zu Hause wohnte, zählte
Högns „neue Familie“ zusammen mit der Enkelin Inge zeitweise fünf
Mitglieder.

Es gibt mehrere Anzeichen dafür, dass sich Högn nach dem Tod seiner Frau
allmählich ins rein Private zurückgezogen hat, noch mehr in einer
eigenen Welt lebte. Vielleicht war der frühe und plötzliche Tod seiner
Ehefrau auch ein Grund dafür, dass der gesellige \textit{Vereinsmeier}
der ersten Ruhmannsfeldener Jahre sich in den \textit{Eigenbrötler}
der späteren Jahre verwandelte. Zeitzeugen kannten den älteren Högn
nur noch als einen \textit{nüchternen} Menschen, der sehr zurückgezogen
lebte, der mehr Zeit mit seiner Musik verbrachte, als mit seiner
Familie. Niemand konnte sich erinnern, dass Högn zum Ratsch außer Haus
gegangen wäre, nicht einmal mit seinen nächsten Nachbarn pflegte er
Umgang. Auch die Enkelkinder mussten auf die Gewohnheiten ihres
Großvaters Rücksicht nehmen und durften ihn nicht stören, wenn er zum
Beispiel sein Mittagsschläfchen hielt. Die einzige gesellschaftliche
Aktivität, die er bis ins hohe Alter beibehielt, war die allmonatliche
Veranstaltung eines Gesellschaftstages, zu dem sich \textit{die
Alten}, ein Vereinigung der alten Bürger, Handwerker und Pensionisten
– die \textit{besseren} Bürger von Ruhmannsfelden – in der Brauerei
Amberger trafen. Neben dieser isolierten Lebensweise konnte beim
alternden Högn ein Sauberkeitsfimmel beobachtet werden, womöglich eine
Folge der sozialen Abkapselung. So wurde im Haushalt Högn streng darauf
geachtet, dass sein Besteck wirklich nur von ihm benutzt wurde, nicht
einmal seine Enkelkinder durften es benutzen. Zeitlebens aß er nur Brot
ohne Rinde, die er immer abschnitt, weil sie seiner Meinung nach schon
\textit{so viele Leute in der Hand gehabt hatten.} Machte er einen
Besuch, putzte er sich schon weit vor Betreten der Wohnung die Schuhe
mit seinem Taschentuch ab, das er dann aber wieder einschob und seiner
Haushälterin zum Waschen brachte.