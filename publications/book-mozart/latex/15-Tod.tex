\section{Nach Högns Tod}

Nach Högns Tod wurde die Haushälterin Rosa Beischmied mit der Auflösung
seiner Wohnung beauftragt. Die Angehörigen ließen wahrscheinlich den
Großteil von Högns Werken zurück. Es lag daher an der Haushälterin,
ihren weiteren Verwendungszweck zu bestimmen. Nach den Fundorten der
Kompositionen zu urteilen, hatte Beischmied alle geistlichen Werke
der Pfarrei übergeben. An vier Orten, die in Verbindung zur Kirche
stehen, wurden Werke von Högn gefunden, im Notenschrank und auf dem
Dachboden des linken Seitenschiffs der Pfarrkirche St. Laurentius, im
Pfarrhof und in der Wohnung des ehemaligen Kirchenchorleiters Franz
Danziger. Im Haus der Sängerin Mathilde Glasschröder waren
Kompositionen von Högn zu entdecken, die möglicherweise schon vor Högns
Tod in den Besitz der Sängerin übergingen.

Angesicht der veschiedenen Fundorte ist es nicht verwunderlich, dass
manche Kompositionen nicht mehr zugänglich sind. Vor allem Högns
weltliches Werk ging verloren. Da Franz Danziger nicht nur
Kirchenchorleiter, sondern zur Zeit der Auflösung von Högns Wohnung
auch Leiter des Männerchores war und im Turnverein-Orchester unter Högn
mitwirkte, müsste seine Wohnung ein ergiebiger Fundort auch von
weltlichen Kompositionen gewesen sein. Hier war aber kein einziges Werk
außer den entdeckten geistlichen Kompositionen zum Vorschein
gekommen. Der Sohn des ehemaligen Kirchenchorleiters räumte in einem
Brief vom 25. Oktober 2002 ein, dass seine Mutter bei Aufräumarbeiten
eventuell auch Werke von Högn vernichtet haben könnte. Centa
Schwannberger, die Ehefrau von Rudolf Schwannberger, der zusammen mit
Högn die Sängerriege des Turnvereins leitete, hat den Notenbestand
ihres Mannes den Geißkopfsängern übergeben. Hier verlaufen sich die
Spuren.

Högns Geschichtswerk ist komplett erhalten. Högn hat sich noch zu
Lebzeiten um eine sichere Aufbewahrung seiner heimatkundlichen Werke
gekümmert und die \textit{Geschichte von Zachenberg} und die
\textit{Geschichte der Feuerwehr Ruhmannsfelden} seinem Nachfolger
bei der Feuerwehr und Gemeindesekretär von Zachenberg Johann Freisinger
übergeben. Dass das Geschichtswerk nicht abhanden gekommen ist, muss
auch dem Pfarrer Reicheneder gedankt werden, der außer der
\textit{Geschichte von Ruhmannsfelden} alle heimatkundlichen
Abhandlungen Högns abgetippt und seiner \textit{Chronik
Ruhmannsfelden} angefügt hat.

Högns Kompositionen wurden nach dessen Tod wie schon nach Högns
Ausscheiden aus dem Chorregentenamt regelmäßig von seinem Nachfolger
Franz Danziger weiter aufgeführt. Danziger fertigte sogar deutsche
Versionen der lateinischen \textit{Laurentius-Messe C-Dur op. 14} und
der lateinischen \textit{Mater-Dei-Messe F-Dur op. 16} an. Auch die
späteren Kirchenchorleiter Karl Geiger und August Lankes ließen Högns
Kompositionen singen.

Das \textit{Marienlied Nr. 8 G-Dur op. 54} arrangierte Danziger für den
Achslacher Männergesangsverein, der dieses Arrangement bis heute in
seinem Repertoire hat. Der am weitesten von Ruhmannsfelden entfernte
Aufführungsort einer Komposition Högns war Altötting. Högns Tochter
Elfriede Schlumprecht hat die \textit{Josephi-Messe} durch den
Altöttinger Chor aus Anlass eines Familienjubiläums aufführen lassen.
Einzelheiten über diese Aufführung sind nicht bekannt.

Von Högns Geschichtswerk blieb als einziges Werk die \textit{Geschichte
von Ruhmannsfelden} im Bewusstsein der Bevölkerung. Diese hat sich zu
einem Standardwerk der hiesigen Heimatgeschichte entwickelt:
Vollkommen unbekannt blieben dagegen – weil nie gedruckt – die
\textit{Geschichte von Zachenberg} und die \textit{Geschichte der
freiwilligen Feuerwehr Ruhmannsfelden.} Högns heimatkundliche
Zeitungsartikel waren nur über das Pfarrarchiv und das Privatarchiv
des ehemaligen Gemeindesekretärs Johann Freisinger zugänglich.

Insgesamt aber nahm die Bekanntheit des Namens Högn im Laufe der Jahre
stark ab. Wenn jemanden der Name Högn noch was sagt, dann als Autor
der \textit{Geschichte von Ruhmannsfelden}. Högn als Komponist geriet
vollkommen in Vergessenheit, obwohl sein kompositorisches Werk
eindeutig höher als sein Geschichtswerk einzuschätzen ist.

\begin{figure}
\centering
\img[width=6cm]{Totenbrett}
\caption{Totenbrett zum Andenken von August Högn an der Wallfahrtskirche Osterbrünnl in Ruhmannsfelden}
\end{figure}