\section{Kindheit in Deggendorf}

\begin{figure}
\img{Das-Elternhaus}
\caption{Das Elternhaus}
\end{figure}

Die Högns kommen aus Söldenau. Sowohl August Högns Großvater Johann
Nepomuk Högn als auch sein Vater Andreas Högn wurden in Söldenau
geboren. Johann Nepomuk Högn war mit der Katharina Schwarzmüller aus
Weng bei Aldersbach verheiratet. Der genaue Zeitpunkt der Umsiedlung
nach Deggendorf, dem Geburtsort von August Högn ist nicht bekannt.
Doch wahrscheinlich zog schon Johann Nepomuk mit seiner Familie
dorthin, denn er wird in einer Urkunde als Gastwirt aus Deggendorf
bezeichnet. Womöglich wechselte die Familie Högn den Wohnort schon als
Andreas Högn noch ein Kind war. Andreas Högn heiratete am 5. August
1867 die aus Geiselhöring stammende Helene Zöpfl. Diese war die Tochter
des Kaufmanns Josef Zöpfl und seiner Frau Anna Knott.

August Högn kam am 2. August 1878 zur Welt. August hatte zwei ältere
Geschwister, Theres und Ludwig. Seine jüngeren Brüder waren Joseph und
Otto.

Augusts Vater war von Beruf Buchbinder und eröffnete zusammen mit seiner
Ehefrau 1867 eine Buchbinderei und Buchhandlung im so genannten
„Kerndel\-’schen-Haus“ am Luitpoldplatz. Bereits nach sechs Jahren, also
1873, zog die Familie Högn in ein eigenes Haus in die Pfleggasse Nummer
1, wo sich bis zum heutigen Tag die Buchhandlung Högn befindet. Die
äußerst günstige Lage der Buchhandlung im Zentrum war von bedeutendem
wirtschaftlichem Vorteil. Besonders an Markttagen, so beispielsweise
zum „Saumarkt“ in der Pfleggasse, strömten viele Menschen aus dem
Umland in die Stadt, wovon auch die Buchhandlung profitierte. Das
Sortiment wurde im Laufe der Zeit um Schreib-, Schul-, Spiel- und
Lederwaren erweitert. Ab 1890 komplettierte ein eigener
Postkartenverlag das Angebot.

Im Haus des späteren landgräflichen Magistratsrats, Landrats und
Landtagsabgeordneten Andreas Högn war eine gründliche Ausbildung und
somit eine grundlegende musikalische Schulung der Kinder
selbstverständlich. Es ist daher nicht verwunderlich, dass alle fünf
Kinder das Klavierspielen erlernten, selbst der von Geburt an fast
taube Joseph Högn.

August Högn besuchte von 1884 bis 1888 die Knabenschule in Deggendorf
und war von 1888 bis 1890 Schüler der Unterstufe des Klosters Metten,
genannt Lateinschule, wo er auch im „Klosterseminar“, also im der
Schule angeschlossen Internat wohnte. Mit dem darauf folgenden
Übertritt in die Präparandenschule in Deggendorf war schon früh der
endgültige Berufsweg eingeschlagen. Die Übernahme des elterlichen
Geschäfts dürfte für den zweitgeborenen Sohn nie zur Debatte gestanden
haben. Dies war wahrscheinlich einer der Gründe, weshalb sich August
frühzeitig für den Beruf des Lehrers entschied. Ludwig Högn, der ältere
Bruder, erlernte ganz nach alter Tradition das Buchbinderhandwerk, um
einmal die Stellung seines Vaters einnehmen zu können. Er eröffnete
jedoch in Straubing eine Kunst-, Papier- und Galanteriewarenhandlung.
Somit konnte der jüngste Sohn Otto die Buchhandlung übernehmen.