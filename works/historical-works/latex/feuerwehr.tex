%!TEX program = lualatex
\documentclass[12pt,a4paper]{book}
\usepackage{hoegn}

\author{August Högn}
\title{Geschichte von Chronik der Freiwilligen Feuerwehr Ruhmannsfelden}
\date{1951}

\begin{document}

\maketitle



Meinen Eltern Josef und Anita Friedrich gewidmet

Mein Dank gilt:

Herrn Pfarrer Meier, Lotte Freisinger,

Textgrundlage:

Abschrift von Pfarrer Reicheneder aus der Reicheneder-Chronik

unter der Rubrik „Weltliche Vereine: Feuerwehr- und Löschwesen“

Projekt August Högn Geschichtswerk

Ruhmannsfelden, 2003

1. Auflage zu 10 Stück

AUGUST HÖGN

1878 1961

GESCHICHTE DER

FREIWILLIGEN

FEUERWEHR

RUHMANNSFELDEN

EDITIERT VON

JOSEF FRIEDRICH 2003

INHALTSVERZEICHNIS

\tableofcontents

\part{Vorwort}

Die vorliegende Chronik der freiwilligen Feuerwehr Ruhmannsfelden macht nicht
den Anspruch auf Vollständigkeit. Es wurde nur zusammengetragen, was erreichbar
war.

Es wächst daraus die Aufgabe Altes zu ergründen und nachzutragen, Neues sofort
einzutragen. Dann wird die Chronik einmal etwas Wertvolles für die freiwillige
Feuerwehr Ruhmannsfelden.

August Högn, Oberlehrer

\part[Geschichte von Chronik der Feuerwehr]{Geschichte von Chronik der
Freiwilligen Feuerwehr Ruhmannsfelden}

\chapter{Vor der Gründung der freiwilligen Feuerwehr Ruhmannsfelden}

Der Markt Ruhmannsfelden wurde des Öfteren von Schicksalsschweren Bränden
heimgesucht. 1522 wurde der ganze Markt durch eine Feuersbrunst vernichtet. 1574
brannte das Pfarrgotteshaus vollständig nieder. 1633 wurden das Pfarrvikarhaus
und die Klostertaverne in Asche gelegt. Es sei erinnert an die größeren Brände
in den Jahren 1820, 1889 und 1894. Durch alle Jahrhunderte und Jahrzehnte
hindurch hat sich in Ruhmannsfelden die gegenseitige Hilfsbereitschaft bei
Feuersbränden erwiesen und durch alle Zeiten hindurch galt den durch viele
Brände schwer heimgesuchten Bewohnern des Marktes Ruhmannsfelden der Wahlspruch:
„Einer für alle, alle für Einen.“

Früher gab es weder Pflichtfeuerwehren, noch freiwillige Feuerwehren. Es galt
bei Ausbruch eines Brandes für Mann und Frau, für Alt und Jung als
Selbstverständlichkeit zu rennen, zu retten, zu löschen mitzuhelfen, wo es ging.
Da wurde unaufgefordert gepumpt an den Spritzen und wenn die Männer anderweitig
sich am Rettungswerke beteiligen mussten, so traten die Frauen an die Pumpstange
und Feuereimer flogen von Hand zu Hand. In den Städten wurden eigene Feuerwachen
gebildet aus den Reihen der Bürger und Bürgerssöhne. Diese Feuerwachen, die
zuerst nur zeitweise aufgestellt wurden, die wurden später zu einer dauernden
Einrichtung. Aus diesen haben sich zunächst in den Städten die freiwilligen
Feuerwehren gebildet. Da man deren Zweckmäßigkeit und Nützlichkeit alsbald
erkannte, haben sich in Märkten und dann in Dörfern solche freiwilligen
Feuerwehren gebildet. Der Kreis, die Bezirke und Gemeinden zeigten stets ihr
wohlwollendes Interesse für die Bedürfnisse einer hilfsbereiten und
schlagfertigen Feuerwehr und förderten und unterstützten die Feuerwehren und
damit das Feuerlöschwesen in hervorragendem Maße.

Bitten, Gesuche, Wünsche, Anträge für Förderung des Feuerlöschwesens wurden
sowohl von Bezirks- und Kreisvertretungen als auch von den Bezirksämtern und
Regierungen wärmstens befürwortete und genehmigt. Wenn wir in den Akten vor
Jahrzehnten oder Jahrhunderten nachlesen, immer wieder lesen wir, dass für
„Sicherheit“, in diesem Falle also für Feuerlöschwesen und Feuerwehr keine
Mittel gescheut wurden und keine Ausgaben zu hoch erschienen.

\chapter{Auszüge aus den ältesten Ruhmannsfeldener Akten}

Aus den diesbezüglichen noch vorgefundenen ältesten Ruhmannsfeldener Akten lesen
wir:

\section*{1814}

\begin{itemize}
\item Die Gemeinderechnung vom Jahre 1814 weist unter Rubrik „Inventarium“ aus:
1. Alte hölzerne Feuerspritzen, 2. alte meßingere Feuerspritzen. (Die unter 1.
aufgeführten hölzernen Feuerspritzen sich 1820 verbrannt.)
\end{itemize}

\section*{1817}

\begin{itemize}
\item Dem Joseph Stoiber, Schlossermeister ist für Beschlagung der großen,
meßernen Feuerspritzen 7 fl., für Beschlagung der hölzernen 5 fl. 42 kr. bezahlt
worden.

\item Von der Schmiedearbeit hiezu ist dem Andrä Baumann bezahlt 36 kr., dem
Lorenz Tafner, Zimmermann ist für das Zusammenrichten der drei Feuerspritzen mit
drei neuen Röhren, dann für Leinöl, Farb bezahlt lt. Schein 5 fl. 50 kr.
\end{itemize}

\section*{1820}

\begin{itemize}
\item Da am 1. Juli 1820 durch eine unglücklichen entstandenen Brand 12
bürgerliche Hausungen samt der Kirche in Asche gelegt wurden, dabey selbst die
kleinen zwey hölzernen Feuerspritzen verbrannten und die eine meßingere
verdorben wurde, so ist letztere durch den Schlossermeister Joseph Stoiber
wieder in brauchbaren Zustand hergestellt worden.
\end{itemize}

\section*{1821}

\begin{itemize}
\item den beiden Zimmerleuten Tafner und Zitzelsberger wurde an Herstellung der
2 Feuerwehrleitern 3 fl. Dem Schied Joseph Baumann für Beschlagung dieser zwei
Feuerwehrleitern 48 kr. bezahlt.
\end{itemize}

\section*{1826}

\begin{itemize}
\item Anton Priglmeier Kupferschmied erhielt für eine neue Feuerlöschspritzen
mit Ausnahme des Rohrs, welches schon vorhanden war 56 fl. 9 kr.
\end{itemize}

\section*{1835}

\begin{itemize}
\item Laut Rechnung musste bei einer damaligen Bürgeraufnahme „ein bestimmter
Beytrag“ zu den Feuerlösch Requisiten geleistet werden (meistens 1 fl.).
\end{itemize}

\section*{1842}

\begin{itemize}
\item Schmied Joseph Baumann erhielt für Reparatur der kleinen Feuerspritze laut
Schein 36. kr.

\item Vorstehender erhielt für Reisen:

\begin{itemize}
\item a.) nach Deggendorf
\item b.) nach Passau
\end{itemize}

betr. Anschaffung einer neuen Feuerspritzen 2. fl. 48 kr.

\item  Zeugschmied Joseph Stoiber erhielt für Mitreise nach Deggendorf, um die
neue Feuerspritze zu probieren 48 kr.
\end{itemize}

\section*{1843}

\begin{itemize}
\item Der Pferdeknecht Michael Baumgartner erhielt zur Belohnung seiner
fleißigen Wasserzufuhr bei dem Brande des Mossmüller Zeugweberhause in der Nacht
vom 7. Mai 1843 laut Schein 36 kr.

\item Für das vom Glockengießer Samasa in Passau überschickte messingenes
Schöpfwerk zur neuen Feuerspritze an Porto bezahlt 1 fl. 27 kr.

\item Dem Rotgerber Gerhard Lukas für ein abgegebenes Stück Leder zur
Feuerspritze 42 kr.

\item Dem Glockengießer Samasa in Passau für das abgelieferte neue messingene
Druckwerk zur Feuerspritze den Gesamtbetrag laut Quittung übersandt 225 fl.
Porto 6 kr.

\item Dem Schmied Joseph Baumann von hier für Herstellung des Wagens samt Kasten
zur Feuerspritze mit vollständigen Eisenbeschlägen und gefertigten Schrauben
hieran laut Schein 55 fl. Rest 177 fl.
\end{itemize}

\section*{1844}

\begin{itemize}
\item Zur Bezahlung der neuen Feuerspritze an den Schmied Joseph Baumann von hier
musste eine neue Umlage nach dem Steuerfuß erhoben werden.
\end{itemize}

\section*{1850}

\begin{itemize}
\item erhielt Schmied Wolfgang Stegmeier für 2 Steften und Anhängschließen zu
den Feuerleiten 50 kr.
\end{itemize}

\section*{1856}

\begin{itemize}
\item Laut Gemeinderechnung wurde für Reparatur an Feuerhaken an Schmied
Wolfgang Stegmeier verausgabt 1 fl. 39 kr.

\item An Fuhrmann Peter Oischinger von Gotteszell für den Transport der neu
reparierten Feuerspritze von Viechtach nach Ruhmannsfelden laut Schein 1 fl. 15
kr.
\end{itemize}

\section*{1857}

\begin{itemize}
\item erhielt Kupferschmied Prüglmeier von Viechtach für die Reparatur der
großen Feuerspritze 30 fl. 12 kr.

\item Schmied Friedrich Rauch für Reparatur an der Feuerspritze 24 kr.
\end{itemize}

\section*{1860}

\begin{itemize}
\item Die Unterbringung der Feuerlöschspritzen betreffend: „Die zwei Gemeinde
Löschspritzen befanden sich seit einiger Zeit in der Wagenschupfe des
Lukas´schen Bräuanwesens. Da der jetzige Anwesensbesitzer Peter Schrötter die
Unterkunft in eine andere Stelle wünscht, da ihm selbst wegen vielen
Wagengeschirr der Platz mangelt und man die Überzeugung machte, daß sich
derartige Unterkunft ohnedies nicht gut eignet, da diese oft mit Rauzen
dergestalt verrammelt waren, daß man selbe oft nur mit zuviel Zeitverlust
hervorholen konnte, hat man den Ökonom Johann Pfeffer, der einen geräumigen
Platz in seinem Stadel durch einen Anbau gewonnen, contrahiert, denselben die
Unterbringung der Spritzen gegen eine jährl. Vergütung von 2 Gulden verständigt,
aus Gemeindemitteln zu berichten.ii“ (Später dann zu Göstl, bis das Feuerhaus
gebaut wurde.)
\end{itemize}

\section*{1863}

\begin{itemize}
\item am 4. Oktober wurde eine Feuerlöschordnung produziert, die in ihren 10
enthaltenen Artikeln als durchwegs gut und ohne alle Abänderung anzunehmen sei.

\item bekam der Markt 14 Stück Feuereimer von den Magirus-Werken in Ulm unter
Nachnahme und durch Vermittlung des Herrn Kaufmann Schwaighofer in Deggendorf
laut Schein 17 fl. 55 kr.
\end{itemize}

\section*{1867}

\begin{itemize}
\item Ein Beschluss der Gesamtgemeinde Ruhmannsfelden vom 14.7.1867 lautet: „Es
wurde aus Communalmitteln 12 Stück zwilchene Wassereimer angeschafft. Jedes neu
aufgenommene Gemeindemitglied (auch bei Insassen) hat einen desgleichen Eimer
anzukaufen und an den Spritzenmeister abzuliefern.iii“
\end{itemize}

\section*{1866}

\begin{itemize}
\item Laut Rechnung von diesem Jahriv wurde verausgabt an den Wagnermeister
Hölzl für neue Deichsel zur Feuerspritze 2. fl. 26 kr., dem Joseph Baumann,
Schmied, für geleistete Arbeit zur Feuerspritze 12 fl. 6 kr., dem Anton
Prieglmayer von Viechtach für Reparaturarbeiten an der Feuerspritze 7 fl. 24 kr.
\end{itemize}

\chapter{Die Gründung der freiwilligen Feuerwehr Ruhmannsfelden}

Nachdem bereits im Bezirke Viechtach um jene Zeit eine freiwillige Feuerwehr
bestand und zwar die freiwillige Feuerwehr Viechtach (gegründet 15.8.1863) und
in Ruhmannsfelden lebhaft von der Gründung einer Freiwilligen Feuerwehr
debattiert wurde, ging am 13. August 1867 im ganzen Markt von Haus zu Haus
folgendes Schreiben herum:

„Cirkular.

Wir haben beschlossen im Markte Ruhmannsfelden eine Feuerwehr zu organisieren
und laden hiemit alle ledigen der Feiertagsschule entwachsenen Mannspersonen
ein, sich als Mitglieder zu betheiligen. Alle Jene, welche beitreten, wollen auf
der Kehrseite ihren Namen verzeichen und sich am künftigen Maria
Himmelfahrtstage Mittags 12 Uhr zu einer Besprechung in der Behausung des
unterzeichneten Marktvorstandes einfinden.

Am 13.8.1867. Die Marktverwaltung Ruhmannsfelden.

Lukas Marktvorstand.“

Auf der Kehrseite des vorstehenden „Cirkulars“ stehenden Unterschriften:

\begin{compactitem}
\item „Plötz, Maurermeister, Montur selbst
\item Georg Brunner, Montur selbst
\item Anton Fritz, Montur selbst
\item Ludwig Hohenwarter, Montur selbst,
\item Joseph Leitner, one
\item Xaver Schreiner, mit Montur,
\item Joseph Frell, Montur
\item Joseph Mösl, one
\item Alois Stadler, one
\item Jakob Maier, one
\item Johann Leitner, one
\item Xaver Zadler, mit Montur,
\item Alois Wurzer, one
\item Joseph Friedl, one
\item Franz Rauch, one
\item Alois Sagstetter, mit
\item Johann Futscher, one
\item Weinzierl Michael, one
\item Karl Warzer, one
\item Georg Bomkratz, mit Montur
\item Alois Kleebauer, one
\item Johann Vull, one
\item Xaver Hell, one
\item Alois Blüml, one
\item Georg Oberberger, one
\item Jakob Achatz, one
\item Michael Baumgartner, one
\item Joseph Hopfner, mit Mondur,
\item Johann, one,
\item Wurzer Michl, one,
\item Joseph Zadler, one
\item Wächtlinger, Muntur,
\item Georg Seiderer,
\item Joseph Meindl.
\end{compactitem}

Darlehen machen:

\begin{compactitem}
\item Schinagl 1 fl.
\item Bieland 1 fl.
\item Probst 1 fl. Banknote
\item Göstl 1 fl.
\item Hollmeier 1 fl.
\item Lukas 1 fl.
\item Hillinger 1 fl.
\item Stadler 1 fl.
\item Wimmer Aufschl. 1 fl.
\item Dr. Rötzer 1 fl.
\item Moosmiller 1 fl.
\item Schreiner Fragner 1 fl.“
\end{compactitem}

Am 15. August 1867 wurde nun die Gründung der feiwilligen Feuerwehr
Ruhmannsfelden vollzogen und unter dem damaligen Marktsvorstand Joseph Lukas,
Lederermeister, der als Gründer der freiwilligen Feuerwehr Ruhmannsfelden gilt
bis 1883 Kommandant derselben war und dann zum Bezirksfeuerwehrvertreter für den
Bezirk Viechtach ernannt wurde. Die freiwillige Feuerwehr Ruhmannsfelden war nun
die zweite im Bezirke Viechtach. Ihr folgten:

\begin{tabular}{ll}
1869 & Arnbruck\\
1873 & Wiesing\\
1874 & Schönau\\
& Blossersberg\\
& Kollnburg\\
& Kirchaitnach\\
1875 & Gotteszell\\
& Allersdorf I\\
& Schlatzendorf\\
& Drachselsried\\
1876 & Prackenbach\\
& Patersdorf\\
& Achslach\\
& Moosbach\\
& Geiersthal\\
1877 & Zachenberg\\
1878 & Ruhmannsdorf\\
& Wettzell\\
1879 & Teisnach I\\
& Böbrach\\
1891 & Teisnach II\\
1895 & Allersdorf II\\
1898 & Teisnach III\\
1899 & Altnussberg\\
1905 & Thalersdorf\\
1909 & Triefenried\\
\end{tabular}

\chapter[Verzeichnis der Mitglieder]{Verzeichnis der Mitglieder der freiwilligen
Feuerwehr Ruhmannsfelden nach der Gründung}

Der freiwilligen Feuerwehr Ruhmannsfelden sind nach der Gründung 64 Mann
beigetreten:

\subsection*{I. Verwaltungsrath}

\begin{tabular}{rlll}
& Name & Stand & Bemerkung\\
\hline
1 & Lukas Joseph & Lederer & Hauptmann\\
2 & Göstl Johann & HandeIsmann & Zeugwart\\
3 & Probst Alois & dto & Vorstand\\
4 & Dr. Rötzer & prakt.Arzt & Chirurg\\
5 & Moosmüller Joseph & Hutmacher & Kassier\\
6 & Schreiner Johann & Fragner & Auditor\\
7 & Schinagl Raymund & Schullehrer & Schriftführer\\
8 & Seiderer Georg & Hutmachergeselle & Adjutant\\
9 & Förstl Georg & Schulgehilfe & Vertretter d. Schriftführers\\
10 & Virigl Max & Schullehrer & II. Adjutant\\
\end{tabular}

\subsection*{II. Mannschaft}

\subsection*{A. Steigerrotte}

\begin{tabular}{rlll}
& Name & Stand & Bemerkung\\
\hline
1 & Fritz Anton & Kaminkehrer & Rottenführer\\
2 & Leitner Joseph & Zimmergeselle\\
3 & Hell Xaver & Färbermeister\\
4 & Kleebauer Alois & Weißgerberssohn & Obmann\\
5 & Meindl Joseph & Lederergeselle\\
6 & Pomgratz Georg & Hafnerssohn\\
7 & Stadler Alois & Schuhmacherssohn\\
8 & Wiesgrill Joseph & Metzgergeselle\\
9 & Zattler Joseph jun. & Schreinerssohn\\
10 & Josephs Sizl & Binder\\
11 & Franz Rauch\\
12 & Schmaus Max & Gürtler\\
13 & Leitner Christoph\\
14 & Johann Leitner\\
15 & Martin Hartl\\
16 & Anton Bielmeier & Spritze\\
17 & Johann Bauer\\
\end{tabular}

\subsection*{B. Retterrotte I}

\begin{tabular}{rlll}
& Name & Stand & Bemerkung\\
\hline
1 & Mösl Alois & Schuhmachermeister & Rottenführer\\
2 & Achatz Jakob & Weberssohn & Obmann\\
3 & Hopfner Joseph & Schuhmachergeselle\\
4 & Rauscher Ludwig & Buchbinder\\
5 & Sagstetter Alois & Postexpeditor\\
6 & Schreiner Andrä & Lebzelter\\
7 & Schreiner Lorenz & dto\\
8 & Zitzelsberger Alois & Weberssohn\\
9 & Maier Jakob & Schneider\\
\end{tabular}

\subsection*{Retterrotte II}

\begin{tabular}{rlll}
& Name & Stand & Bemerkung\\
\hline
1 & Wimmer August & kgl. Aufschläger & Rottenführer\\
2 & Maier Joseph & Schneidermeister & Obmann\\
3 & Bayerer Karl & Postbote\\
4 & Futscher Johann & Sattler\\
5 & Hillinger Joseph & Schneidermeister\\
6 & Sagstetter Joseph & Posthalter\\
7 & Schroll Michl & Schneidermeister\\
8 & Stadler Xaver & Schumachermeister\\
\end{tabular}

\subsection*{C. Werkleute}

\begin{tabular}{rlll}
& Name & Stand & Bemerkung\\
\hline
1 & Bieland Georg & Maurermeister & Rottenführer\\
2 & Weiß Michl & Hausbesitzer & Obmann\\
3 & Almer Wilhelm & Hausbesitzer\\
4 & Brunner Georg & Weber\\
5 & Vull Alois & Zimmermann\\
6 & Pritzl Jakob & Binder\\
7 & Wurzer Michl & Mauerer\\
8 & Zattler Joseph sen. & Schreinermeister\\
9 & Jakob Bielmeier & Hausbesitzer\\
10 & Joseph Stegmeier & Schmiedsohn\\
\end{tabular}

\subsection*{D. Spritzenmannschaft}

\begin{tabular}{rlll}
& Name & Stand & Bemerkung\\
\hline
1 & Hollmeier Joseph & Schlosser & 1. Spritzenmeister\\
2 & Moosmüller Joseph & Hutmacher (Kassier) & 1. Obmann\\
3 & Frohnhofer Joseph & Bindermeister & 2. Spritzenmeister\\
4 & Sixl Johann jun. & Bindermeister & 2. Obmann\\
5 & Englmeier Joseph & Postbote\\
6 & Hanninger Joseph & Müllerssohn & Junker (NB. gestorben)\\
\end{tabular}

\subsection*{E. Signalist}

\begin{tabular}{rlll}
& Name & Stand & Bemerkung\\
\hline
1 & Mösl Joseph & Hausbesitzer und Musiker\\
\end{tabular}

64 Mann eingetreten

\chapter[Ereignisse seit Gründung]{Ereignisse seit Gründung der freiwilligen
Feuerwehr Ruhmannsfelden}

\section*{1867}

\begin{itemize}
\item Die freiwillige Feuerwehr Ruhmannsfelden will eine Steigleiter nach dem
Muster der Viechtacher Steigleiter.

\item erbaute die freiwillige Feuerwehr Ruhmannsfelden ein Steighaus (cirka 15 m
hoch, mit 3 Etagen, Einsteigtüren, Steigbaum).
\end{itemize}

\section*{1868}

\begin{itemize}
\item Im Juli war hier Feuerwehrprobe, bei der die zwei Feuerwehrmänner Blessing
und Baumer von Viechtach anwesend waren.

\item Im August feierte die freiwillige Feuerwehr Ruhmannsfelden Gründungsfest.

\item erhielt laut Gemeinderechnung Schlosser Holtmeier für Reparatur an der
Feuerspritze laut Schein 3 fl. 27 kr.

\item für Reparatur an der kleinen Feuerspritze Karl Prieglmeier laut Schein 17
fl. 24 kr.

\item ausgezahlt dem Wagnermeister Hölzl und Schmiedmeister Rauch Arbeitslohn an
der Feuerspritze laut Schein 22 fl. 3 kr.

\item dem Leopold Baumann Maler für Reparaturarbeit an der Feuerspritze laut
Schein 42 kr.

\item dem Sebastian Rauch, Schmiedmeister für gefertigte Arbeit an der
Feuerspritze laut Schein 42 kr.

\item dem Anton Schmid für Überbringung der Feuerspritze von Viechtach nach
Ruhmannsfelden laut Schein 1 fl. 36 kr.

\item an Magirus in Ulm für gelieferte Feuereimer laut Schein 25 fl. 39 kr.

\item erfolgte eine Einladung der freiwilligen Feuerwehr Viechtach an die
freiwillige Feuerwehr Ruhmannsfelden zwecks Prüfung des neuen Requisitenwagens
anlässlich einer Fahrt von Viechtach nach Ruhmannsfelden.
\end{itemize}

\section*{1869}

\begin{itemize}
\item Am 20. Juni erhielt die freiwillige Feuerwehr in Ruhmannsfelden eine Einladung
zur Fahnenweihe der freiwilligen Feuerwehr Cham.

\item Im Dezember selbigen Jahres traf von Deggendorf ein Brief folgendes Inhalts ein:

\begin{quote}
„Sehr verehrtester Herr Hauptmann! Anliegend übersende ich Ihnen das allen
Feuerwehren so beliebte Feuerwehrlied nebst Melodram, das wirket so sehr auf die
Mitglieder und spornt den Geist, das Gemüth jedes Feuerwehrmannes an, es ist
recht hübsch und leicht zum aufführen. Einer spricht das Melodram, dann vier
Männerstimmen singen die Lieder und 1 oder 2 Signalisten blasen die angegebenen
bei Ihnen üblichen Signale. 1 schlägt auf der Glocke 12 Uhr und trommelt – und
das ist das ganze Personal. Ich lasse Ihnen Melodram und Singstimmen um den
billigen Preis von lf. 24 kr. ab und Können Sie selben gerade dem Postillon
mitgeben, haben Sie noch keine Signale und Ordonanzmärsche für 2 Signalisten, so
schicke ich selbe um ein ganz geringes Honorar. Ich hoffe, daß ich eine Ehre
aufhebe Ihnen wie bei dem Corps und grüße Sie Hochachtungsvollst! Ihr ergebener
Eduard Grill Musiker“.
\end{quote}
\end{itemize}

\section*{1870}

\begin{itemize}
\item ergeht eine Einladung an die freiwillige Feuerwehr Ruhmannsfelden zur
Teilnahme an der 2. Landesversammlung der bayerischen Feuerwehren in Regensburg
am 29. bis 31. Mai 1870, an welcher sich laut Unterschrift auf der Einladung Jakob
Bielmeier beteiligte.

\item erging vom Feuerwehrkorps Deggendorf an das Feuerwehrkorps in
Ruhmannsfelden eine Anfrage, ab noch in diesem Jahre ein niederbayerischer
Feuerwehrtag abgehalten oder bis zum Friedensschluss gewartet werden sollte und
ob das freiwillige Feuerwehrkorps zum bayerischen Kreisverband beitreten wolle.

\item fand die Fahnenweihe der freiwilligen Feuerwehr Viechtach statt. Die
Lokalbahn existierte damals noch nicht. Der Weg von Ruhmannsfelden nach
Viechtach musste zu Fuß zurückgelegt werden. Für die Teilnehmer an dieser
Fahnenweihe fand am Sonntag, den 3. Juli ein Reisemarsch statt von
Ruhmannsfelden nach Gotteszell auf den Keller bei Hr. Kilger Bräu.
\end{itemize}

\section*{1871}

\begin{itemize}
\item erhält das Feuerwehrkommando Ruhmannsfelden aus der Gemeindekasse für
Anschaffung von 10 Feuereimern 1 fl.

\item Laut Zirkular von 8.3.1871 wurden sämtliche Mitglieder der hiesigen
Feuerwehr aufgefordert zu einer Abend 7 Uhr im Gastlokale des Bürgermeisters
Lukas anberaumten Feuerwehrversammlung in Uniform respektive Mütze zu erscheinen
zu einer Besprechung bezüglich der Friedensfeier.

\item lief beim königlichen Bezirksamt Viechtach eine Klage ein bezüglich der
Feuereimer. Infolge dessen erging von der Marktverwaltung Ruhmannsfelden an das
Feuerwehrkorps Ruhmannsfelden folgendes Schreiben:

\item „Nach Auftrag des kgl. Bezirksamtes Viechtach wird in rubr. Betreff das
Feuerwehrkorps Ruhmannsfelden beauftragt am 17. Juni Samstag laufendes Jahres
sämtliche Feuereimer vorzuweisen und zu diesem Behufe dieselben in die Behausung
des Bürgermeisters bringen zu lassen. Das Weitere wird sich schon herausstellen.
Achtungsvoll Lukas Bgst.“

\item Am 27 August 1871 fand die Fahnenweihe des freiwilligen Feuerwehrkorps in
Zwiesel statt, an der sich 16 Feuerwehrmänner von Ruhmannsfelden beteiligten.
\end{itemize}

\section*{1873}

\begin{itemize}
\item Am 7. Juni erließ das königliche Bezirksamt Viechtach eine
distriktspolizeiliche Feuerlöschordnung.
\end{itemize}

\section*{1874}

\begin{itemize}
\item In einem Gemeindebeschluss vom 28.1. betreffend Anschaffung einer
Feuerspritze im Verein der Marktsgemeinde Ruhmannsfelden und der Gemeinde
Zachenberg heißt es, dass eine neue Feuerspritze auf gemeinschaftlich zu
bezahlende Weise anzuschaffen sei und wird die Beschaffung derselben dem
Bürgermeister Lukas von Ruhmannsfelden beauftragt und bevollmächtigt. Es sei
schon „eine solche Spritze anzuschaffen, daß dieselbe auf die Dörfer hinaus
leicht zu transportieren ist und in Erwägung, daß die Gemeinde Zachenberg in
ihrer Lage sich zunächst neben Ruhmannsfelden hinzieht und von da aus dieselbe
am leichtest fortgeschafft werden kann, auch auf die entferntesten Punkte auch
bezüglich der Zugpferde sichere und schleunigere Hilfe allenthalben überall hin
geboten ist, so hat diese neue anzuschaffende Feuerspritze im Markte
Ruhmannsfelden seine Aufbewahrungsstelle, wird auch die Aufsicht dem
Bürgermeister von Ruhmannsfelden übertragen.“ Der anwesende gegenwärtige
Bürgermeister der Gemeinde Patersdorf lehnte den Beitritt zur gemeinschaftlichen
Anschaffung obigen Spritze im Namen seiner Gemeinde ab. (Siehe Gemeindebeschluss
vom 9.11.05 und vom 6.10.06)

\item Am 4. Juni bei der Fronleichnamsprozession brannte der Bruckhof ab. Im
Gemeindebeschluss vom 5.6.1874 heißt es: „Nachdem sich beim Brande im Hause des
Joseph Bauer in Bruckhof gezeigt hat, daß die vorhandenen Feuereimer sich als
sehr praktisch und viel zu wenig erwiesen haben, so wird beschlossen, daß deren
mehrere Feuereimer anzuschaffen sind und zur leichteren Bezweckung der
Geldmittel hiezu hat ein jeder sich zu Verehelichende, der Marktgemeinde
Ruhmannsfelden je für einen das Geld im Betrage von 1 fl. 30 kr. pro Stück bei
der Gemeinde zu erlegen und ist in der Taxnote der Heimat- und
Bürgeraufnahmegebühr zu bezeichnen und zu berechnen.“

\item Am 27.Mai erging eine Aufruf lauf Zirkular an alle Mitglieder der
freiwilligen Feuerwehr Ruhmannsfelden, da sich bei der letzten Versammlung
wenige Mitglieder betätigt haben, unausbleiblich am Sonntag den 30. Mai Mittags
12 Uhr im Lokal (Post) mit sämtlichen Ausrüstungsgegenständen zu erscheinen.

\item Der Verwaltungsrat des freiwilligen Feuerwehrkorps Vilshofen dankt in
einem Schreiben vom 20. August 1874 für die freundliche Einladung zur
Fahnenweihe der freiwilligen Feuerwehr Ruhmannsfelden mit dem Beifügen, dass
„der Einladung keine Folge geleistet werden kann, da zu gleicher Zeit in
Vilshofen das Landwirtschaftsfest abgehalten wird, zu welchen das Festkomite
unsere Tätigkeit in Anspruch nimmt. Der Commandant des freiwilligen
Feuerwehrkorps der Stadt Vilshofen.“

\item Am 13. September fand die Fahnenweihe des freiwilligen Feuerwehrkorps
Ruhmannsfelden statt. Fahnenmutter war die Metzgermeistersgattin Hollervi (+
18.7.1931), die auch beim 40-jährigen Gründungsfest der freiwilligen Feuerwehr
Ruhmannsfelden im Jahre 1907 der Einladung folgend sich mit ihrer Gegenwart und
Beteiligung zu unserer aller Freude beteiligte. Nach den üblichen Festesfeiern
am Vor- und Haupttage war an diesem Nachmittag 2 Uhr Signal zur Zugaufstellung,
denn Auszug in die Romantische Schloss Leuthen, wo selbst gesellige Unterhaltung
mit Musik, Feuerwerk etc. stattfindet. Abends war Rückzug ins Feuerwehrlokal und
Festball.
\end{itemize}

\section*{1875}

\begin{itemize}
\item wurde eine Gemeindeumlage in Höhe von 450 fl. erhoben mit der Bemerkung:
„Ist eine Gemeindeumlage in diesem Maßstabe bezüglich Anschaffung einer neuen
Feuerlöschmaschine nothwendig und zwar nach bisheriger Festsetzung durch
Gemeindebeschluß vom Grundsteuergulden 24 kr. vom Gewerbesteuergulden 6 kr.“

\item ergeht vom Vororte der niederbayerischen Feuerwehren Passau an das
Feuerwehrkorps Ruhmannsfelden die Anfrage, ob der niederbayerische Feuerwehrtag
vor oder nach dem bayerischen Feuerwehrtag stattfinden solle, was dahin
entschieden wurde, dass der niederbayerische Feuerwehrtag in Passau am 29.
August, also vor dem bayerischen Feuerwehrtag in Kempten abgehalten werden
solle.
\end{itemize}

\section*{1876}

\begin{itemize}
\item Der Bezirksvertreter Hr. Hauptmann Schmid von Viechtach beruft Feuerwehren
im Bezirk zu einer Versammlung nach Viechtach behufs Gründung eines
Bezirksverbandes der Feuerwehren des Bezirkes Viechtach. Diesem schlossen sich
an: Viechtach, Ruhmannsfelden (vertreten durch Bielmeier und Mayer), Arnbruck,
Wiesing, Kollnburg, Schönau, Kirchaitnach, Blossersberg, Allersdorf, Wiesing,
Prackenbach. Bezirksvertreter wurde Hr. Hauptmann Schmid von Viechtach,
Ersatzvertreter wurde Hr. Anton Kasparbauer von Viechtach.

\item Die Anschaffung der neuen Feuerspritze für Ruhmannsfelden und Zachenberg
gemeinschaftlich zieht sich sehr in die Länge, denn erst ein Gemeindebeschluss
vom 14.9.76, also nach mehr als 2 ½ Jahren, befasst sich wieder mit der
Anschaffung dieser Feuerspritze und lautet:

\begin{quote}
„In Anbetracht des kgl. bay. Amtes Viechtach, Auftrag vom 9.9.74 wurde Beschluß
gefasst bezüglich den zu handhabenden zwei Projekten zur Anschaffung einer
Feuerlöschmaschine für die Gemeinden Ruhmannsfelden und Zachenberg entschließt
sich der Ausschluß dahin, für die Maschine des Braun in Nürnberg im Betrage von
730 fl,. jedoch aber mit der Bedingung, daß auf zwei Jahre Garantie geleistet
werde, dann in Hinsicht der Sicherheit und Ausdauer derselben in Benützung in
den Hohlwegen und schlechter Wegen, voll großer Steine, der Maschine wie auch
der Wagen an derselben aushalten und nicht brechen und brauchbar sind beim
etwaigen Gebrauche, ferners daß dieselbe, da mit demselben im Markte besonders
die Bäche in ihrem Wassergehalt wegen der Entfernung u. der Höhe der Steigung
nicht genützt werden können, das Wasser aus den im Markte reichlich mit Wasser
versehenen Brunnen in einer Tiefe von 30 Fuß = 9 Meter herausgehoben werden
könne mit dieser Maschine, da eine solche Maschine, die das nicht im Stande ist,
hierorts für nichts ist, da wir ohnehin eine gute Spritze ohne Sauger haben.“
\end{quote}

\item wurde an Johann Nepomuk Baumann für Reparatur einer Spritze verausgabt 1
fl. 45 kr.

\item an das königliche Bezirksamt Viechtach den treffenden Lastenanteil für
Anschaffung einer Löschmaschine 292 fl.

\item an Johann Sagstetter für Beiführung der Löschmaschine von Viechtach nach
Ruhmannsfelden 7 fl. 30 kr.
\end{itemize}

\section*{1877}

\begin{itemize}
\item Die neue Feuerspritze sollte nun auch einen geeigneten Unterkunftsplatz
bekommen, Zu diesem Zwecke sollte ein eigenes Spritzenhaus gebaut werden. Am
11.1.1877 wurde der Gemeindebeschluss gefasst betreffendvii „Erbauung eines
Feuerrequisitenhauses“, dass „mit Eintritt der günstigen Jahreszeit mit dem Bau
des Requisitenhauses begonnen werden muß und daß die Kosten teils durch Umlagen,
teils durch den bis dahin anfallenden Bierpfenning gedeckt werden soll“.
Außerdem wurde am 23.2.1877 beschossen, dass „zur Erbauung des neuen Feuerhauses
von sämtlichen Fehrwerksbesitzern die benötigten Bausteine unentgeltlich
beigefahren werden sollen.“ Am 24.4.1877 wurde die Maurer- und
Zimmermannsarbeiten zu dem neuen Feuerhause verakkordiert. Den Zuschlag für die
Mauererarbeiten erhielt Johann Plötz, Maurermeister mit 390.- M und für die
Zimmermannsarbeiten Hr. Anton Bielmeier, Schreinermeister mit 325.- M.

\item Am 19.3. war die Beerdigung des verstorbenenviii Gründungsmitgliedes und
Kassiers der freiwilligen Feuerwehr Ruhmannsfelden Hr. Joseph Moosmüller.

\item Am Dienstag, den 22. Mai, feierte die freiwillige Feuerwehr Ruhmannsfelden
ihr zehnjähriges Gründungsfest. Tags vorher war schon Empfang auswärtiger
Vereine und Zapfenstreich. Am Festtage selbst war Kirchenzug, Feldmesse und
nachmittags Ausmarsch sämtlicher Vereine zur Schäffer Markendentenhütte beim
Zachenberger Eisenbahneinschnitt, dann Rückmarsch zuix Hr. Gastgeber Münch,
Gartenmusik und Festball.
\end{itemize}

\section*{1879}

\begin{itemize}
\item wurde in Ruhmannsfelden auf Grund der distriktpolizeilichenx
Feuerlöschordung eine Pflichtfeuerwehr gebildet.

\item fand hier das Bezirksgaufest statt. Zu diesem Behufe wurde eine eigene
Exerzierübung abgehalten.
\end{itemize}

\section*{1881}

\begin{itemize}
\item Durch Gemeindebeschluss vom 29.9.1881 wurde beschlossen: „Es sei der
Notwendigkeit halber bei Brandfällen im hiesigem Markte energisch eingreifen zu
können, dafür zu sorgen, daß entsprechende Wasserreserven angelegt werden. 1.
Eine Wasserreserve in der Bachgasse, wozu von dem Hausbesitzer Stadler ein
entsprechender Teil seines neben dem Hause und der Straße liegenden
Wiesenkomplexes auf Gemeindekosten anzuschaffen. 2. Eine Wasserreserve neben dem
Sagmeister`schen Brunnen zu leiten.“

\item Hierzuxi wurde am 6. Oktober 1881 ergänzend beschlossen: „Es sei der
innere Raum der Wasserreserve mit Schwartlingen auszuschlagen, diese mit einem
sogenannten „Gurta“ zu verbinden und zu größeren Haltbarkeit die vier
Seitenwände mit Kreuzweisen Bäumen zu verbinden. Bei Fertigstellung der Reserve
ist dieselbe mit einem Schwartlingzaun von ziemlicher Höhe zu umgeben und muß
derselben in gutem Zustand auf Kosten der Gemeinde erhalten werden.“
\end{itemize}

\section*{1883}

\begin{itemize}
\item wurde Hr. Joseph Lukas, Lederermeister von hier, der voll aufopfernder
Hingabe sich der Feuerwehrsache widmete, sehr viele Kenntnisse und Erfahrungen
in Feuerwehrangelegenheiten besaß und auch dementsprechend bei den Behörden und
Feuerwehrkameraden eingeschätzt wurde, zum Bezirksfeuerwehrvertreter des
Bezirkes Viechtach gewählt und bestätigt, was auch der freiwilligen Feuerwehr
Ruhmannsfelden zur größten Ehre gereichte.

\item Durch Gemeindebeschluss vom 12. Mai 1883 wurde beschlossen, „es sei von
den Entleihern der Feuerleitern zu Bauten etc. per Stück tägl. 25 Pfg. zu
bezahlen. Verursachte Schäden müssen von den Entlehnern vergütet werden. Bei
Entlehnung von Hanfschläuchen sei pro Tag für 2 M 5 Pfg. zu entrichten.“

\item Bei Neueintritt eines Mitgliedes in die freiwillige Feuerwehr wurden
vorgedruckte Erklärungen ausgefüllt.

\item Laut Verzeichnis der Mitglieder der freiwilligen Feuerwehr Ruhmannsfelden
war der Mitgliederstand am 10.6.1883 73 Mann, darunter „Verwaltungsrath 7 Mann,
Steigerrotte 15 Mann, I. Retterrotte 8 Mann, II. Retterrotte 9 Mann, Werkleute
12 Mann, Spritzenmannschaft 11 Mann, Schlauchlegerrotte 9 Mann, Signalisten 2
Mann. Summe: 73 Mann. Lukas“
\end{itemize}

\section*{1884}

\begin{itemize}
\item wurden neu angeschafft: ein Wagenheber ein neuer Schlauchhaspel, 8 Paar
Normalgewinde, ein Anhänghaken an die Löschmaschine.
\end{itemize}

\section*{1885}

\begin{itemize}
\item Zur raschen Orientierung der Feuerwehrleute bei Ausbruch eines Brandes in
Ruhmannsfelden und Umgebung wurde lautxii Gemeindebeschluss vom 25.6.1885
nachstehendes Signalement beschlossen:

\begin{quote}
\begin{enumerate}
\item „Brand in Ruhmannsfelden, Stegmühle und Bruckmühle: wird zuerst mit der
großen Glocke das Zeichen gegeben, dann mit sämtlichen Glocken absatzweise
geläutet.

\item Brand in der Pfarrgemeinde: wird zuerst mit der mittleren Glocke das
Zeichen gegeben, dann die kleinere Glocke absatzweise geläutet.

\item Brand außer der Pfarrei und nächster Umgebung: wird zuerst mit der kleinen
Glocke das Zeichen gegeben, dann die mittlere Glocke absatzweise geläutet.

\item Bei weiteren Entfernungen wird nur durch die freiwillige Feuerwehr durch
Signale das Zeichen gegeben.“
\end{enumerate}
\end{quote}
\end{itemize}

\section*{1887}

\begin{itemize}
\item Am 27. August erhielt das freiwillige Feuerwehrkorps Ruhmannsfelden neue
Statuten.

\item Am 27. Oktober 1887 beschloss die Gemeinde Verwaltung, "es sei von einer
Neuanlage einer Wasserreserve Umgang zu nehmen, da das jetzige Marktwasser durch
Abzapfung an zwei Stellen in den Feuerwehrschläuchen den Spritzen zugeleitet
werden kann. Dafür seien im Bedürfnisfalle 50 m neue Hanfschläuche mit 4
Normgewinden anzuschaffen, wodurch dem Wassermangel in ausgiebiger Abhilfe
entgegen gesetzt wird.“
\end{itemize}

\section*{1888}

\begin{itemize}
\item Am 7. April 1888 wurde durch Gemeindebeschluss dem Verwaltungsrat der
freiwilligen Feuerwehr die Vollmacht erteilt, auf Kosten der Gemeindekasse 15
Ehrendiplome anzuschaffen.

\item Durch Beschluss vom 7. April 1888 wurden um 10 bis 12 M Karabinerhaken und
Schlauchhalter angeschafft. Der Gemeindebeschlussxiii vom 12. Mai 1883 wurde
aufgehoben, nachdem durch Beschluss vom 29. April 1888 die Feuerleitern und
Hanfschläuche an Private nicht mehr zur Benützung hinaus gegeben werden durften.

\item Am 6. und 7. Oktober 1888 feierte die freiwillige Feuerwehr Ruhmannsfelden
das 20-jährige Gründungsfest. Auf dem Programm stand auch die Verteilung der
Ehrendiplome für 15-jährige Dienstzeit (Siehe Beschluss vom 7.4.1888).
Ehrendiplome für 15-jährige Dienstzeit wurden bei dieser Gelegenheit auf der auf
dem Marktplatz errichteten Festtribüne überreicht den Herren:

\begin{itemize}
\item Lukas Joseph, Kommandantxiv
\item Meier Joseph, Kassier
\item Bielmeier Jakob, Zeugwart
\item Pritzl Jakob, Zeugwart
\item Meindl Joseph, Rottenführer
\item Stadler Alois, Spritzenmann
\item Wurzer Michael sen., Obmann
\item Wurzer Michael jun., II. Spritzenmeister
\item Frohnhofer Joseph; Spritzenmann
\item Almer Wilhelm, Rottenführer
\item Achatz Jakob, Werkmann
\item Mösl Alois, Spritzenmann
\item Schmaus Max, Ordnungsmann
\end{itemize}
\end{itemize}

\section*{1889}

\begin{itemize}
\item Am 2.2.1889 fand die Wahl der Chargierten der freiwilligen Feuerwehr
Ruhmannsfelden statt für die Wahlperiode 1889/91. Vorstand wurde Herr
Hocheitinger, Kommandant Hr. Lukas, Schriftführer Hr. Lehrer Weig und Kassier
Hr. Joseph Meier.

\item Am 30. April 1889 war ein großer Brand. Dieser Brand wäre unserer
Pfarrkirche bald wieder zum Verhängnis geworden. Um die Mitternachtsstunde des
genannten Tages brannten 7 Anwesen im oberen Markte, Dietrich, Sixl, Weinzierl,
Meindl, Hirtreiter, Reisinger und Baumann, die ihre Anwesen um die Pfarrkirche
herum hatten, vollständig nieder. Zum größten Glück hatte die Pfarrkirche um
diese Zeit schon harte Bedachung (Platten). Trotzdem fing der Dachstuhl des
Presbyteriums schon zu brennen an. Das Feuer konnte aber glücklicher Weise noch
bekämpft werden, sodass er nicht weiter greifen konnte, sonst wäre die
Laurentius-Kirche sicherlich zum 3. Male ein Raub der Flammen geworden. (1574,
1820)

\item Am 19. Mai 1889 fand hier die Bezirksversammlung der Feuerwehren des
Bezirks Viechtach statt.

\item Am 22. Mai 1889 kam H. Hr. Bischof Ignatius von Regensburg nach
Ruhmannsfelden zur Firmung. Auf Einladung des H. Hr. Pfarrers Englhirt beteiligt
sich die freiwillige Feuerwehr Ruhmannsfelden beim Empfang des H. Hr. Bischofs
zahlreich.

\item Am 7.6.1889 wurde das Ehrendiplom für 15-jährige Dienstzeit verliehen
(siehe 7.10.1888) an die Herren:

\begin{itemize}
\item Köppl Sebastian, Steiger
\item Hell Xaver, Signalist
\item Donauer Leonhard
\item (Bader Edererxv
\item Gottfried Bielmeier, Bleistiftbemerkung: hats auch erhalten).
\end{itemize}

\item Durch Gemeindebeschluss vom 18.6.1889 sind die bereits vorhandenen 80
Feuereimer auf 100 zu erhöhen, die vorhandenen Feuerhaken auf 20 in
verschiedener Stärke und Größe zu ergänzen, eine Laterne mit brennendem Licht
vor jedem Hause bei ausgebrochenem Brande anzubringen.

\item Als Feuerboten wurden aufgestellt: Johann Bayerer, eventuell sein Sohn
nach Gotteszell, Georg Ernst nach Achslach, Heinrich Linsmeier nach Prünst,
Patersdorf, Linden und Viechtach. Zur Abholung der zur Verfügung stehenden
Abgrotz-Spritze von der Pulverfabrik wird der Ökonom Joseph Hell bestimmt. Den
Fuhrwerksbesitzern von Stegmühle, Bruckmühle und Leithenmühle wird die Beifuhr
von Wasser eingeschärft.

\item 100 m Schlauch mit 5 Normalgewinden werden angeschafft. An dem Teilständer
bei dem Bierbrauer Michl Weiß ist eine Öffnung anzubringen, in welche ein
Hydrant einzuschrauben ist.

\item Im Jahre 1889 wurden von der Gemeinde Ruhmannsfelden insgesamt für
Feuerlöschzwecke 203 M verausgabt.

\item Nach dem Brand am 30.4.1889 wurde nach Räumung des Brandplatzes der
Bauplatz des abgebrannten Michl Dietrich, 9 dzm groß, um 3 000 M von der
Gemeinde Ruhmannsfelden angekauft und damit der Friedhof vergrößert.
\end{itemize}

\section*{1891}

\begin{itemize}
\item Am 2.3. fand hier die Feier des 70. Geburtsfestes Seiner königlichenxvi
Hoheit des Prinzregenten Luitpold von Bayern, dem Protektor aller Feuerwehren
Bayerns statt, bei der sich auch die Mitglieder der freiwilligen Feuerwehr
Ruhmannsfelden zahlreich beteiligten.
\end{itemize}

\section*{1892}

\begin{itemize}
\item Bei der am 4.2.1892 stattgefundenen Generalversammlung wird an Stelle des
früheren Vorstandes der freiwilligen Feuerwehr Ruhmannsfelden Hr. Alois Probst
der Kürschnermeister Alois Hocheitinger Vorstand. Kassier wird für den
verstorbenen Herrn Moosmüller der Schneidermeister Joseph Meier.

\item Am Montag, den 9. Mai 1892 hielt H. Hr. Primiziant Peter Fenzl seinen
feierlichen Einzug bei dem sich die freiwillige Feuerwehr Ruhmannsfelden
zahlreich beteiligte.

\item Am Sonntag, den 15. Mai 1892 war im Gasthaus des Hr. Bierbrauers Ben.
Schaffer in Ruhmannsfelden Bezirksfeuerwehrversammlung.

\item Am Sonntag, den 3. Juli 1892 beteiligte sich die freiwillige Feuerwehr
Ruhmannsfelden beim feierlichen Einzug Seiner Bischöflichen Gnaden Ignatius
Senestrey zur hl. Firmung.
\end{itemize}

\section*{1893}

\begin{itemize}
\item Am 6.4. starb der Schneidermeister Joseph Meier. Als Kassier der
freiwilligen Feuerwehr Ruhmannsfelden wurde nunmehr der Schneider Alois Meier
gewählt.

\item Am 1. Mai 1893 zählte die freiwillige Feuerwehr Ruhmannsfelden 73 Mann.

\item Am Sonntag, den 14. Mai 1893 war die 25-jährige Gründungsfeier der
freiwilligen Feuerwehr Ruhmannsfelden, bei welcher auch Ehrenzeichen für
25-jährige Dienstzeit verteilt wurden an 12 Mitglieder der freiwilligen
Feuerwehr Ruhmannsfelden durch den königlichen Bezirksamtmann Hr. Heerwagenxvii.

\item Am 15. Juni war hier Bezirksfeuerwehrversammlung.
\end{itemize}

\section*{1894}

\begin{itemize}
\item Am 15. Juni werden die Mitglieder Hr. Michael Wurzer und Hr. Alois Stadler
zum Empfange eines Ehrenzeichens in Vorschlag gebracht.

\item Am Ludwigstage (25. August) nachmittags 3 Uhr brach ein großer Brand in
Ruhmannsfelden bei Alois Metzger, Wagnermeister in der Bachgasse aus. Es war
gerade die Zeit der Getreideernte. Die Leute waren großenteils auf den Feldern
beschäftigt. Bis sie in ihre Behausung kamen, mussten sie auf Rettung des
eigenen Habes, soweit noch möglich war, sehen und bis die auswärtigen
Feuerwehren kamen, breitete sich das Feuer blitzschnell aus, übersprang bei
Brauerei Wilhelm die Straße und äscherte dann von Lukas (neue Welt) bis Zadler
herauf insgesamt 18 Wohnhäuser und 57 Nebengebäude ein. Mit der Ortsfeuerwehr
bekämpften 21 Feuerwehren den Brandherd, darunter die Feuerwehren Gotteszell,
Patersdorf, Achslach, Pulverfabrik, Teisnach II, Teisnach I, Geierstahl,
Böbrach, Viechtach, March, Regen, Zwiesel, Theresienthal, Bischofsmais,
Grafling, Eisenstein, Deggendorf, Schaching, Zachenberg und Allersdorf. Für die
Abgebrannten wurde eine Landessammlung veranstaltet. Beim Brand 1894 ging der
Feuerwehr Ruhmannsfelden ein Bild zugrunde.
\end{itemize}

\section*{1895}

\begin{itemize}
\item Bei der am 7. Januar stattgefundenen Generalversammlung legte Hr. Joseph
Lukas die Stelle als Kommandant der freiwilligen Feuerwehr Ruhmannsfelden
nieder, da er durch die Arbeiten als Bezirksfeuerwehrvertreter ohnehin sehr in
Anspruch genommen war. Bezirksfeuerwehrvertreter war von 1876 bis 1882 Hr. Anton
Schmid, Hauptmann in Viechtach und von 1883 bis 1909 Hr. Joseph Lukas,
Kommandant der freiwilligen Feuerwehr Ruhmannsfelden wurde Hr. Joseph Rauch,
Kaufmann.

\item Am 17.2.1895 beschloss der Gemeindeausschuss Ruhmannsfelden die Ausführung
einer märktischen Wasserleitung nach den vom Wasserversorgungsbüro in München
ausgearbeiteten Detailprojekten. Mit der Herstellung dieser Wasserleitung und
der Aufstellung von Oberflurhydranten war für die Feuersicherheit und die rasche
Bekämpfung eines Brandes im Markte Ruhmannsfelden ungemein Wertvolles geleistet,
gleichzeitig aber auch für die freiwillige Feuerwehr Ruhmannsfelden ein ganz
neue Einstellung bei Ausbruch eines Brandes im Markte geboten (Siehe Beschluss
vom 19.6.1903).
\end{itemize}

\section*{1896}

\begin{itemize}
\item Durch Gemeindebeschluss vom 12.4.1896 sei die Kommunespritze, von welcher
die Gemeinde Zachenberg Miteigentümerin ist, und auf cirka 500.- M gewertet
wird, als Entschädigung für Einlegung der Wasserleitungsrohre auf Gemeindewege
von Zachenberg an diese Gemeinde abzulassen.

\item 1896 wurde von der Firma J. Chr. Braun in Nürnberg eine fahrbare
Schubleiter neuster Konstruktion im Preise von 600,- M von der Gemeinde
Ruhmannsfelden gekauft, ebenso 200,- M Hanfschläuche.
\end{itemize}

\section*{1897}

\begin{itemize}
\item wurden 200 m Hanfschläuche, 12 Schlauchbüchsen gekauft und 213,- M als
Teilzahlung für die Schubleiter bezahlt.

\item Die freiwillige Feuerwehr zählte am 1.1.1897 64 Mitglieder.

\item Am 20.6. nachmittags 1 Uhr fand die Übergabe der Wasserleitung an die
Gemeindeverwaltung in feierlicher Weise statt, wozu auch die freiwillige
Feuerwehr Ruhmannsfelden eingeladen war.

Da auch bald Klagen einliefen bezüglich des Wasserstandes in den
Wasserkammern und man besorgt war, es könnte bei einem Brande im Markte das
Wasser nicht ausreichen, stellte die freiwillige Feuerwehr die Forderung bei Hr.
Bürgermeister Fromholzer, dass von seitens der freiwilligen Feuerwehr öfters
Kontrolle in den Wasserkammern geübt werden dürfte. Zu diesem Zwecke wurden an
die freiwillige Feuerwehr die hierzu notwendigen Schlüssel ausgehändigt.

\item Am 2.8.1897 beschließt die Gemeindeverwaltung, es seinen in Folge
schnellen Eingreifens bei ausbrechender Feuersgefahr 8 Feuerwächter aufzustellen
und zwar für

\begin{itemize}
\item Posten 1, oberer Markt: Johann Hell
\item Posten 2, Holler-Eck: Josef Klein
\item Posten 3, Obere Gasse: Friedrich Rausch
\item Posten 4, Kaltes Eck: Andreas Hobelsberger
\item Posten 5, mittlerer Markt: Georg Rankl
\item Posten 6, unterer Markt: Alois Fromholzer
\item Posten 7, untere Bachgasse: Joseph Lukas
\item Posten 8, Marktplatz: Joseph Schrötter
\end{itemize}

Die aufgestellten Posten werden je mit cirka 30 m Schläuchen, 1 Strahlrohr
und 1 Hydrantenschlüssel ausgerüstet. Über Haftung und Behandlung wird mit jedem
einzelnen Posten ein eigenes Protokoll aufgenommen:

\item Am 16.4.1897 wurde der Bierbrauereibesitzer Hr. Joseph Schrötter als
Vorstand der freiwilligen Feuerwehr, am 23. Mai gleichen Jahres Hr. Alois
Hobelsberger als Kommandant und am 2.2.1898 Hr. Joseph Klein, Schuhmacherssohn,
als Kommandant gewählt.
\end{itemize}

\section*{1898}

\begin{itemize}
\item Durch Gemeindebeschluss vom 4.11.1898 wurde die im unteren Markte
befindliche verschlemmte Wasserreserve nicht mehr in Stand gesetzt, „da dies
eine zwecklose Ausgabe und ohnehin 4 Hydranten dort.“

\item Bezüglich der Überlassung der Kommunespritze an die Gemeinde Zachenberg
(siehe Beschluss vom 12.4.96) will die Gemeinde Zachenberg die seinerzeit
gemeinschaftlich angeschaffte Spritze. Aber nach dem Beschluss vom 10.11.1898
soll diesem Ansuchen nur stattgegeben werden, wenn seitens Zachenberg 200.- M
Entschädigung gezahlt werden. Die Schläuche verbleiben in Ruhmannsfelden.
\end{itemize}

\section*{1899}

\begin{itemize}
\item wurden für Feuerlöschwesen von der Gemeinde Ruhmannsfelden 112.- M
verausgabt.

\item Am 14.5.1899 war in Ruhmannsfelden Bezirksfeuerwehrversammlung, bei der an
5 Mitglieder der freiwilligen Feuerwehr Ruhmannsfelden Ehrenzeichen verteilt
wurden.

\item Am 1.1.1899 wurde die freiwillige Feuerwehr Ruhmannsfelden in den
bayerischen Landesfeuerwehrverband aufgenommen.
\end{itemize}

\section*{1901}

\begin{itemize}
\item wurde unter zahlreicher Beteiligung der freiwilligen Feuerwehr
Ruhmannsfelden das 80. Geburtsfest Seiner königlichen Hoheit des Prinzregenten
Luitpold gefeiert.

\item Durch Generalversammlungsbeschluss vom 14.7.1901 wurde Hr. Friedrich Rauch
Vorstand und Hr. Joseph Beßendorfer, Gerbereibesitzer, Kommandant der
freiwilligen Feuerwehr Ruhmannsfelden.

\item Durch Gemeindebeschluss vom 15.9.1901 wurden die Chargen der
Pflichtfeuerwehr ergänzt, bzw. neu besetzt und zwar durch den Bürgermeister.
\end{itemize}

\section*{1902}

\begin{itemize}
\item Am 12.1.1902 wurde als Vorstand der freiwilligen Feuerwehr Ruhmannsfelden
Hr. Alois Meier, Schneidermeister gewählt und als Kassier Hr. Wilhelm Ederer,
appr. Bader, der diese Stelle bis zu seinem Ableben bekleidete (+16.8.1923).
Nachdem aber der Vorstand Alois Meier unterm 17.5.1903 mit Tod abgegangen ist,
wurde an seine Seite der Brauereibesitzer, Hr. Benediktxviii Schaffer gewählt.

\item Am 1.1.1902 hatte die freiwillige Feuerwehr Ruhmannsfelden 55 Mitglieder.

\item 1902 wurde das Hochreservoir erweitert.
\end{itemize}

\section*{1903}

\begin{itemize}
\item Unterm 19.6.1903 wünscht die Marktsgemeindeverwaltung Ruhmannsfelden, dass
bei der freiwilligen Feuerwehr Ruhmannsfelden eine spezielle Truppe, nämlich
eine Hydrantenkompanie gebildet werde. Der Gemeindeversammlungsbeschlussxix
lautet:

\begin{quote}
„Gemäß Zuschrift der Gemeindeverwaltung Ruhmannsfelden vom 19.6.1903 behufs
Bildung einer Hydrantenrotte, welche den versammelten Feuerwehrmitgliedern
bekannt gegeben wurde, hat hierüber die anwesende Versammlung beschlossen, es
sei dem Wunsche der Markts-Gemeinde Verwaltung zu willfahren, dass nämlich eine
Rotte aus hiesiger Wehr gebildet wird, welche zur Handhabung der Hydranten der
Wasserleitung dahier, sowie mit dem gesamten Rohrnetz, Absperrschieber,
Hausleitungen, Quellenleitung etc. einüben werden, so zwar gegeben falls der
Brunnenwart verhindert sei, im Notfall sämtliche Hantierungen etc. ausführen
werden. Zugleich wird noch konstatiert und das Ansuchen gestellt an die Gemeinde
Verwaltung, es möchte die nötige Anleitung hierzu von Seite des Brunnenwart der
Mannschaft dieser Rotte genügend Aufklärung beigebracht werden.“
\end{quote}

\item Im Juli 1903 erhielt der Brunnenwart die Anweisung, den Übungen der
Hydrantenabteilung beizuwohnen und die Mannschaft in allem, was die
Wasserleitung betrifft, anzuhalten.

\item Zum Bäckermeister Wiesinger kommt eine Schlauchstation.

\item Durch Gemeindebeschluss vom 24.12.1903 wurde betreffs Schläuche zu den
Hydranten beschlossen: „es seien 450 m Schläuche anzuschaffen und dieselben in
je drei Längen a 15 m an die Schlauchstationen hinauszugeben.“
\end{itemize}

\section*{1904}

\begin{itemize}
\item Am 17. Januar 1904 wurde Hr. Wenz. Kiesbauer als Vorstand, Hr. Gottfried
Bielmeier als Kommandant der freiwilligen Feuerwehr gewählt.

\item Der Generalversammlungsbeschluss der freiwilligen Feuerwehr vom 25.1.1904
befasst sich mit dem derzeitigen Bestande des Fußbodens im Feuerhause und
lautet:

\begin{quote}
„Es ist bekannte Tatsache die Maschinen, Schubleiter, entweder zur
Einfahrt oder Ausfahrt ist es große Mühe und erfordert Kräfte, selbe zu
befördern. Ferner ist es an und für sich für die im Feuerhauses untergebrachten
Geräte von sehr großem Nachteil und Schaden zumal durch die offenen Zugfenster
der Staub gehoben wird selber auf die lagerten Löschgeräte zu liegen kommt.

Es liegt im eigenen Interesse der Gemeinde in Anbetracht des hohen Wertes der
sämtlichen Geräte diesem Übelstande bald möglichst bei Seite zu schaffen und den
Boden entweder durch Beton oder durch Pflaster geeignet reperieren zu lassen um
mehr Reinlichkeit erhalten werden kann.“
\end{quote}

Dieser Boden wurde dann gepflastert, was er heute noch ist. Mit diesen Auslagen
(Pflasterung) wurden 1904 insgesamt 753.- ausgegeben.

\item Da die Hydranten bei allen möglichen Anlässen von Privaten eigenmächtig
geöffnet wurden, nahm die freiwillige Feuerwehr durch Beschluss vom 5. April
1904 dagegen Stellung und beschloss: „Es wurde zu öfteren die Wahrnehmung
gemacht, dass Hydranten nutzlos geöffnet werden. Jeder Bürger hier hat seinen
Wasserzins zu entrichten, leider auf solche Art geht ein großes Quantum nutzlos
verloren. In Anbetracht dieser Unordnung, welche sich allmählich einschleicht,
sieht sich der Verwaltungsrat der freiwillige Feuerwehr von hier veranlasst um
diesen Übelstand entgegen zu treten, an die löblichen Verwaltung der Marktes
Ruhmannsfelden ein Gesuch zu richten, dahin lautend, daß die Öffnung streng
überwacht wird.“

\item Am 10. Februar 1904 fand im Schaffer´schen Bräuhaus eine gesellige
Abendunterhaltung statt anlässlich der Vermählung der Mitgliedes Joseph Hell.

\item Am Donnerstag, 31. März 1904, fand die Beerdigung des verstorbenen
Mitgliedes Hr. Michael Brem unter zahlreicher Anteilnahme statt.

\item Im März 1904 wurde hier das 83. Geburtstagsfest Seiner königlichen Hoheit
des Prinzregenten Luitpold gefeiert, wobei Hr. Bezirksvertreter Lukas die
Festrede hielt.

\item Am 15. Mai selben Jahres fand hier Bezirksfeuerwehrversammlung statt.

\item Am Sonntag, den 24. Juli war Wanderung zum Mitgliede der freiwilligen
Feuerwehr Ruhmannsfelden Hr. Leopold Kilger in Gotteszell.

\item Herr Bürgermeister Fromholzer wird vom Vorstand der freiwilligen Feuerwehr
Ruhmannsfelden ersucht, sich von den schadhaften Schlauchverdichtungen zu
überzeugen.

\item Bei der am 21. Juli 1904 stattgefunden Versammlung des Vereins der
„Wilden“ wurde der Antrag gestellt, dass, nachdem der Gesellschaftstag
(Donnerstag) durch Zugang zweier Gasthäuser im Turnus zu sehr in die Länge
zieht, es wäre wünschenswert, wenn ein zweiter Gesellschaftstag festgesetzt
würde, wozu auch die freiwillige Feuerwehr hierzu eingeladen wurde. Bezugnehmend
auf vorstehenden Antrag hat sich der versammelte Verwaltungsrat dahin geeinigt,
dem vorstehenden Antrag stattzugeben, nämlich einen zweiten Gesellschaftstag,
welcher wöchentlich am Montag stattfinden sollte bereitwilligst genehmigt, mit
dem Bemerken, dass der hiesige Turnverein zu den beiden Gesellschaftstag
eingeladen werde.

\item Am 26.2.1904 brach im Kaufhaus Ponschab ein Brand aus. Ein
Verwaltungsbeschluss vom 5.4.1904 sagt hierüber:

\begin{quote}
„Da bei dem am 26. Februar 1. Jahres ausgebrochenen Brande Kaufhaus Ponschab
dahier Schlauchstationinhaber Hr. Friedrich Rauch die Abgabe von Schleichen
verweigerte, wurde Klage gestellt und beschloß der Verwaltungsrat auf Grund
dessen, die Sache der zuständigen Gemeinde Verwaltung Ruhmannsfelden anzuzeigen
und beantragte die z. Z. stehende Schlauchstation bei Hr. Friedrich Rauch an den
Bäckermeister Hr. Xaver Obermeier, welcher bei Nachtzeit immer wach ist, dorthin
verlegen, da diese Stelle geeignet erschient.“
\end{quote}

\item Schließlich wurde der Antrag gestellt, es wollte vom Verwaltungsrat ein
Gesuch an die Marktgemeinde dahier eingereicht werden, es wolle die
Marktsverwaltung in Bälde 2 Anstellleitern von leichterer Form 6 bis 7 cm hoch
gütigst angeschafft werden, sowie die zur Zeit vorhandenen defekten
Anstellleitern einer gründlichen Reparatur unterzogen werden.

\item Laut Gemeindebeschluss vom 24.7.1904 wurde beschlossen, die gemeinsame
Löschmaschine nicht an die Gemeinde Zachenberg abzugeben.
\end{itemize}

\section*{1905}

\begin{itemize}
\item Im Juli 1905 erschienen die Satzungen der Sterbekasse des bayerischen
Landesfeuerwehrverbandes.

\item Vom 7. bis 10. September 1905 war der 10. bayerische Landesfeuerwehrtag in
Passau, an dem sich 25 Mann der freiwilligen Feuerwehr Ruhmannsfelden
beteiligten. Heute wird noch von den damals Beteiligten von den schönen Tagen in
Passau und von der herrlichen Dampfschiffsfahrt nach Linz erzählt.

\item 9.11.1905: Aufstellung einer Löschmaschine in Zachenberg. Auf Anregung des
Kreisvertreters der freiwilligen Feuerwehren soll die Gemeinde Ruhmannsfelden an
die Gemeinde Zachenberg eine Abfindungssumme für die im Jahre 1875 von beiden
Gemeinden gemeinsam angeschaffte Feuerlöschmaschine leisten. In der Sache ist in
Betracht zu ziehen, dass die Gemeinde Ruhmannsfelden an die Gemeinde Zachenberg
eine Abfindungssumme für die im Jahre 1875 von beiden Gemeinden gemeinsam
angeschaffte Feuerlöschmaschine leisten. In der Sache ist in Betracht zu ziehen,
dass die Gemeinde Ruhmannsfelden seit 30 Jahren die zur bezeichneten Maschine
nötigen Schläuche, sowie Reparaturen fast ganz alleinig bezahlte und bei Bränden
das nötige Gespann unentgeltlich stellte. Ebenso besorgte Ruhmannsfelden die
Reinigung der Schläuche, der Maschine und die Instandsetzung derselben. Die
Gemeinde Zachenberg leistete bei der Anschaffung einen höheren Betrag als
Ruhmannsfelden und das mit Recht, denn bei den schwierig zu befahrenden Wegen
dieser Gemeinde musste die Maschine mehr ausgenützt respektive zu Schaden
gebracht werden als im Markte Ruhmannsfelden. Aus diesem geht hervor, dass die
Gemeinde Ruhmannsfelden jene der Gemeinde Zachenberg weit übersteigen und
letztere Gemeinde einen rechtlichen und zu rechtfertigenden Anspruch auf eine
Abfindungssumme nicht zusteht. Deshalb beschließt die Gemeinde Ruhmannsfelden
eine einmalige Abfindungssumme von 150.- M zu geben und der Gemeinde Zachenberg
außerdem eine alte Löschmaschine (ohne Sauger) zu überlassen, wogegen die
bisherige gemeinsame Maschine alleiniges Eigentum der Gemeinde Ruhmannsfelden
wird.
\end{itemize}

\section*{1906}

\begin{itemize}
\item Ein Gemeindebeschluss vom 6.4.1906 lautet, dass die betreffende
Löschmaschine ausgehändigt wird, wenn die Gemeinde Ruhmannsfelden mit 250 M von
der Gemeinde Zachenberg entschädigen wird. 1906 bekam Zachenberg diese
Feuerspritze. 1906 wurden auch angeschafft 300 m Hanfschläuche mit Normalgewinde
und ein Schlauchhaspel.

\item Im April 1906 erging an das Kommando der freiwilligen Feuerwehr
Ruhmannsfelden der Auftrag die Schlauchstation zu visitieren, außerdem die
Mannschaft mit der Handhabung der Hydranten bekannt zu machen.

\item Am 20. Mai 1906 fand eine Schulübung für die Hydranten- und
Steigermannschaft statt.

\item Bei der am 31. Juli 1906 stattgefundenen Primizfeier des Primizianten Hr.
Alois Auer von Ruhmannsfelden beteiligte sich die freiwillige Feuerwehr
Ruhmannsfelden sehr zahlreich.

\item Der Mitgliederstand der freiwilligen Feuerwehr Ruhmannsfelden war im Jahre
1906 76 Mann.
\end{itemize}

\section*{1907}

\begin{itemize}
\item Am 9.1. war die Beerdigung des langjährigen außerordentlichen Mitgliedes
Johann Ramsauer.

\item Am 9.6. gleichen Jahres beteiligte sich die freiwillige Feuerwehr
Ruhmannsfelden beim Einzug des H. Hr. Bischofs von Regensburg.

\item Bei der Fahnenweihe des Schützenvereins „Deutsche Eiche“ Ruhmannsfelden am
Sonntag, den 14.7.1907, beteiligte sich auch die freiwillige Feuerwehr
Ruhmannsfelden.

\item Von der Verwaltung der Gemeinde des Marktes Ruhmannsfelden wurde an den
Verwaltungsrat respektive an die freiwillige Feuerwehr dahier nachstehendes
mitgeteilt:


\begin{quote}
„Schutz des Feuerhauses durch Beschluß der Gemeindeverwaltung vom 14.3.1907
dahier verpflichtet sich die Gemeinde für die nötige Ordnung und Reinlichkeit,
die vorhandenen Löschgeräte in reinlichem Zustand zu bewahren, ferneres die
vorhandenen Saug- und Druckspritzen, Schuleiter, Schlauchhaspel durch Scheuern
etc. fortwährend zu jeder Jahreszeit und Gelegenheit dieselben bereit zu halten,
die Ein- und Ausgangstore des Feuerhauses, insbesonders zur Winterzeit von Eis
und Schnee zugängig erhalten.“
\end{quote}

Für diese vorstehenden aufgeführten Punkte respektive Arbeit entrichtet
alljährlich, sowie vom 1.1.1907 bis 1. Januar 1908 eine Aversumme von 30.- M an
die freiwillige Feuerwehr Ruhmannsfelden aus der Kasse des Marktgemeinde
Ruhmannsfelden.

Durch dieses Anerbieten der Marktgemeinde Ruhmannsfelden erklärt sich der
Verwaltungsrat der freiwilligen Feuerwehr einstimmig dahin, Vorstehendes
anerkennen, mit der Versicherung, dass die freiwillige Feuerwehr die übernommene
Aufgabe nach jeder Richtung hin, voll ganz pflichtgemäß nachkommen werde.

\item Laut Beschluss der Verwaltungsratssitzung der freiwilligen Feuerwehr
Ruhmannsfelden vom 14.4.1907 wurde dem derzeitigen Requisitenmeister Hr. Michael
Sixl die obige ordnungsgemäße Aufgabe übergeben und werden hierfür aus der Kasse
der freiwilligen Feuerwehr Ruhmannsfelden jährlich 15 M an Hr. Sixl ausbezahlt,
mit dem Bemerken, dass gegebenenfalls mehr Arbeit, wie Schläuche waschen,
trocknen, selbstverständlich von Seite des Corps mehrere Feuerwehrmänner hierzu
beordert werden.

\item An Stelle des Kommandant Hr. Georg Bielmeier wurde Hr. Alois Bielmeier
gewählt.

\item Weiters wurde im gleichem Beschluss vom 10.6.1907 der Antrag gestellt, da
die freiwillige Feuerwehr Ruhmannsfelden am 16.7.1867 gegründet wurde, sohin
volle 40 Jahre besteht, das 40-jährige Gründungsfest zu feiern. Um dieses
seltene Fest in würdiger Weise zu feiern, wurde einstimmig beschlossen selbes am
18.8.1907 zu begehen. Sofort wurde das Programm folgender Weise festgesetzt:

\begin{quote}
„Samstag den 17.8. abends 5 Uhr Zusammenkunft aller Corpsmitglieder im Bräuhaus
unseres werten Vereinsmitgliedes Hr. Josef Zitzelsberger. Sonntag, den 18.
August morgens 4 Uhr Tagrewelle, Zusammenkunft 7 Uhr früh im Lokal. Im Laufe der
Zeit Empfang auswärtiger Vereine, um ½ 10 Uhr Aufstellung des Zuges zum
Festgottesdienste, nach Beendigung Rückzug zur Festtribüne, Festrede sowie
Verteilung und Schmückung der Fahnen mit Gedänkbändern anwesender Vereine,
sodann Rückzug ins Local. Mittags 12 Uhr Mittagstisch, a 1.- M. Um ½ 3 Uhr
Aufstellung der Vereine, dann Zug durch die Straßen des Marktes zum
Schafferkeller. Bei ungünstiger Witterung unterbleibt der Festzug, Versammlung
im Local.“
\end{quote}

Um dieses Fest nach Möglichkeit zu verherrlichen, die auswärtigen geladenen
Vereine in würdiger Weise in unserem Kreise, die Stunden angenehm unter uns
verleben, wurde sofort ein Comide gewählt und zwar die Herren Joseph Hell, Sixl
Michl, Pritzl Joseph, Bielmeier Xaver, Brem Xaver und Haas Joseph.

Außerdem wurde behufs des 40-jährigen Gründungsfestes beschlossen, dass an alle
jene Männer, welche bei der Gründung damals tätig waren, Vereinsgedenkzeichen,
sowie auch an allen Männern, welche an diesem Feste teilnehmen,
Vereinsgedenkzeichen feierlich überreicht werden.

Ferners wurde einstimmig beschlossen, dass nur allein Vereine des
Feuerlöschwesens zum Gründungsfest geladen werden. Für die Festmusik wurden vom
Corps der Betrag von 40.- M eingestellt, sollte mit der Wiesinger Kapelle
unterhandelt werden, für 60.- M soll gutgestanden werden. Fahnenmutter, sowie
Festjungfrauen werden mit Equibasche abgeholt ins Lokal und Josef Friedrich. Als
Meldereiter wurden bestimmt Hr. Xaver Obermeier, sowie die Herren Johann Zellner
und Hr. Hacker.

Schließlich wurde beantragt, es wolle auch die Lokal Akt. Versammlung der
Eisenbahn Viechtach das Ansuchen gestellt werden, wenn möglich am Festtag ein
Zug eingestellt werde, welcher um 7 und 8 Uhr hier eintrifft

\begin{quote}
„Notiz

Ein seltenes schönes Fest feiert am 18. August 1907 die freiwillige Feuerwehr
Ruhmannsfelden, nämlich das

Vierzigjähriges Gründungsfest,

welches in schönster erhabenster Weise zur Freude aller Mitglieder, sowie aller
Ortsbewohner stattgefunden hat.

In Kürze nur einiges: Wie ähnliche Feste wurde auch dieses eingeleitet. Um 10
Uhr Vormittags Aufstellung zum Festgottesdienste, wobei H. Hr. Kammerer
Mühlbauer einen von Herzen zu Herzen gehenden zum Zwecke des Festes erhabenen
Vortrag an die zahlreichen Wehrmänner hielt. Nach dem Festgottesdienste bewegte
sich der imposante Zug zur Festtribüne, wo der kleine Feuerwehrmann Donauer
einen sinnreichen Prolog mit kerniger meisterhafter Stimme zum Vortrag brachte.
Hierauf reihte sich die Festrede an, welche Hr. Ersatzfeuerwehrvertreter
Schedlbauer hielt.

In seiner ¾ stündigen Rede betonte der Rednerxx, die nach vierzigjährigen
Bestehen der hiesigen Wehr, derer misslichen und harten Verhältnisse der vielen
Jahre her, welche der Wehr gegenüberstanden, betonte aber auch derer Opfer und
Verdienste einzig und allein den bedrohten Nachbar nicht bloß hierorts bei den
großen vielen Bränden, sondern auch in hiesigem Bezirke, desgleichen auch
äußerem Bezirke, jederzeit, wo Gefahr droht, hilfreich und rettend sich gezeigt,
den verheerenden Element Grenze zu ziehen.

Auch sprach Herr Redner dem hiesigen Corps für seine bisherige Tätigkeit und der
großen Opfer im Feuerlöschwesen seinen wärmsten Dank aus, fügte aber schließlich
den Wunsch bei es wolle auch für die Zukunft das Feuerwehrcorps Ruhmannsfelden
seine bisherige Aufgabe und Tätigkeit förderhin bewahren.

Schließlich wird noch bemerkt, dass an jene Mitglieder, welche bei derer
Gründung der Wehr tätig waren, der sieben an der Zahl ein Gedenkzeichen an die
Brust geheftet. Die Namen derjenigen sind: Lukas J., Bielmeier A., Pritzl J.,
Meindl J., Pongratz J., Hopfner J., Stadler A.

Nicht übersehen dürfen wir auch zu erwähnen sei, dass 1874 bei der dortigen
Fahnenweihe die Stelle das Fahnenmutter Frau Kathie Holler vertreten hat, welche
auch bei dem 40-jährigen Gründungsfest unserer Einladung folgend mit ihrer
Gegenwart und Beteiligung zu unserer aller Freude sich beteiligt.

Unserer Einladung folgend haben sich 25 Feuerwehren mit Fahnen an diesem unserem
Jubiläumsfeste beteiligt, wobei zum Danke jedermann ein Gedenkzeichen als
Erinnerung an die Brust geheftet wurde.

Um 3 Uhr Nachmittags nahm der Zug Aufstellung und bewegte sich durch Vorantritt
der vortrefflichen Musikkapelle Wiesinger durch die Straßen der Marktes zum
Festplatze (Schafferkeller), wo die geselligste Unterhaltung sich entwickelte,
verschiedene Toaste folgten und die Musikkapelle Wiesinger conzertierte die
herrlichsten Weisen. Abend war der Festplatz imposant festlich beleuchtet.

Zu schnell flogen die Stunden und allmählich trennten sich die Kameraden zur
Heimreise mit dem Bewußtsein hierorts ein in jeder Beziehung schönes Fest
gefeiert zu haben.“
\end{quote}

\item Hr. Alois Bielmeier, Tischlermeister, wurde auch für weiterhin als
Kommandant gewählt.

\item In einem sehr netten Schreiben an das Commando der freiwilligen Feuerwehr
Ruhmannsfelden entschuldigt Hr. königlicher Bezirksamtmann von Viechtach sein
Fernbleiben von der 40-jährigen Gründungsfeier.
\end{itemize}

\section*{1908}

\begin{itemize}
\item Am Montag, den 13. Januar 1908 war der herkömmliche Festball mit
Fackelzug.

\item Am Sonntag, den 10. Mai 1908 war hier Bezirksfeuerwehrversammlung. Mit
dieser war zugleich ein 25-jähriges Jubelfest verbunden worden für den
Bezirksvertreter Hr. J. Lukas, der 25 Jahre lang an der Spitze des
Bezirksfeuerwehrverbandes Viechtach stand. Es wurde ihm als Ehrengeschenk ein
hochfeines silbernes Feuerwehr-Tintenzeug überreicht. Auch wurden ihm die besten
Glückwünsche, der Dank und Anerkennung der hohen Kreisregierung von Niederbayern
für seine langjährigen und verdienstvollen Leistungen auf dem Gebiete des
Feuerlöschwesens ausgesprochen, nachdem am 4. März 1902 „Im Namen Sr. Majestät
des Königs, Seiner kgl. Hoheit Prinz Luitpold des Königreiches Baiern Verweser,
das Feuerwehr-Verdienstkreuz huldvollst verliehen wurde.“ laut Urkunde des
Staatsministeriums des Innern.

\item Auf Vorschlag des Kreisfeuerwehrausschusses von Niederbayern erhielt die
freiwillige Feuerwehr Ruhmannsfelden vom Landwirtschaftlichen Kredit-Verein
Augsburg als Viertelsanteil anlässlich des 40-jährigen Geschäftjubiläums einen
Zuschuss von 250.- M.
\end{itemize}

\section*{1909}

\begin{itemize}
\item Am 25.4.1909 wurde an den Bezirksfeuerwehrausschluss Viechtach ein Gesuch
gerichtet um einen Zuschuss betreffend Anschaffung von Schlittensohlen an die
Saug- und Druckspritze für den Winter.

\item Bei der am Sonntag, den 20.6.1909 stattgefundenen Fahnenweihe des
katholischen Gesellenvereins Ruhmannsfelden und bei der am Sonntag, den 4.7.1909
stattgefundenen Standartenweihe der Kavallerie-Vereinigung Ruhmannsfelden und
Umgebung beteiligenten sich die Mitglieder der freiwilligen Feuerwehr
Ruhmannsfelden sehr zahlreich.

\item Bei der Generalversammlung am 26.12.1909 wurde Hr. Alois Bielmeier wieder
als Kommandant gewählt.
\end{itemize}

\section*{1910}

\begin{itemize}
\item Am 30.8.1910 starb nach längerem Krankenlager der Gründer und
Schriftführer der freiwilligen Feuerwehr Ruhmannsfelden und
Bezirksfeuerwehrvertreter für den Bezirk Viechtach, Hr. Joseph Lukas. Die
Beerdigung fand unter großer Anteilnahme statt. Hr. Kreisfeuerwehrvertreter,
königlicher Kommerzienrat J. Kanzler von Passau ließ einen prächtigen Kranz
niederlegen. Vertreten waren auch das königliche Bezirksamt Viechtach, Hr.
Bezirksfeuerwehrersatzvertreter des Bezirkes. Beim Leichenbegängnis gingen
sämtliche hiesige Vereine voraus, dann kam die freiwillige Feuerwehr
Ruhmannsfelden, dann die auswärtigen Feuerwehren, dann kamen die Kränze
tragenden Knaben, dann ein Feuerwehrmann, der das Kreuz trug. Der Sarg wurde
begleitet von 6 Fackelträgern, die die Ehrenwache am Katafalk übernahmen. Am 4.
September 1910 war eine Gedächtnisfeier für die verstorbenen Mitglieder der
freiwilligen Feuerwehr Ruhmannsfelden mit Kirchenzug, Trauergottesdienst und
Konzert am Nachmittag.

\item Für den verstorbenen Schriftführer Hr. Joseph Lukas wurde am 26.12.1910
Hr. Oberlehrer August Högn als Schriftführer gewählt.

\item Die freiwillige Feuerwehr Ruhmannsfelden nahm zahlreich an der in Teisnach
am 29.12.1910 stattgefundenen Beerdigung des dort verstorbenen Fabrikdirektors
Fritsche teil.
\end{itemize}

\section*{1911}

\begin{itemize}
\item Im März 1911 beteiligte sich die freiwillige Feuerwehr Ruhmannsfelden an
der 90. Geburtstagsfeier Seiner königlichen Hoheit des Prinzregenten, die hier
feierlich begangen wurde.

\item Bei der Bezirksfeuerwehrversammlung am 14.5.1911 wurde der
Bezirksfeuerwehrvertreter für den Bezirk Viechtach gewählt.
\end{itemize}

\section*{1912}

\begin{itemize}
\item Am Donnerstag, den 19.12.1912 fand hier aus Anlass des Ablebens Seiner
königlichen Hoheit des Prinzregenten Luitpold ein Trauergottesdienst mit
Trauerrede statt, an der sich die freiwillige Feuerwehr Ruhmannsfelden
beteiligte.

\item 1912 erschienen Satzungen für die freiwilligen Feuerwehren des
Bezirks-Feuerwehrverbandes Viechtach.
\end{itemize}

\section*{1913}

\begin{itemize}
\item Am 7.9. war vormittags Gedächtnisfeier für die verstorbenen Mitglieder der
freiwilligen Feuerwehr Ruhmannsfelden, nachmittags Hauptübung.
\end{itemize}

\section*{1914}
\begin{itemize}

\item Bei Kriegsausbruch 1914 war der Stand der Mitglieder 76 Mann. 39 davon
wurden zum Heeresdienst einberufen: Das waren:

Xaver Brem, Müller (Steiger), Joseph Wiesinger, Bäcker (Steiger), Benedikt
Schaffer, Bierbrauer (Steiger), Johann Mühlbauer, Schumacher (Steiger), Karl
Graßl, Binder (Steiger), Leonhard Vierling, Sattler (Steiger), Joseph Friedrich,
Mechaniker (Steiger), Johann Bielmeier, Ökonom (Schlauchleger), Johann
Depellegrin, Steinmetz (Schlauchleger), Alois Geiger, Metzger (Steiger), Ludwig
Biller, Steinmetz (Steiger), Johann Wiesinger, Gastwirt (Steiger), Georg Plank,
Mesner (Steiger), Xaver Dietrich, Bäcker (Steiger), Joseph Holler, Metzger
(Steiger), Johann Ederer, Bader (Adjutant), Xaver Vornehm, Bierbrauer
(Schlauchleger), Georg Pfeffer, Hausbesitzer (Steiger), Michl Wurzer, Schneider
(Steiger), Xaver Kiendl, Schumacher (Spritzenmann), Siegfried Eggl, Kaufmann
(Hydrantenmann), August Högn, Lehrer (Schriftführer), Joseph Zitzelsberger,
Brauer (Spritzenmann), Benedikt Depellegrin, (Steinmetz), Johann Biller, Ökonom
(Spritzenmann), Joseph Karl, Baumeister (Spritzenmann), Martin Götz, Schmied
(Spritzenmann), Michl Baumgartner, Hausbesitzer (Spritzenmann), Ludwig
Hirtreiter, Hausbesitzerssohn (Steiger), Michl Kiesenbauer, Schneider (Steiger),
Karl Raster, Ökonom (Fähnrich), Johann Lippl, Konditor (Steiger), Anton Stadler,
Schumacher (Steiger), Alois Völkl, Steinmetz (Steiger), Georg Kilger, Binder
(Steiger), Joseph Brem, Müller (Steiger), Joseph Schrötter, Metzger (Steiger),
Rudolf Schwannberger, Musiker (Steiger), Joseph Bernbeck, Postbote (Adjutant).

\item Nachdem nun die Mitgliederzahl der freiwilligen Feuerwehr Ruhmannsfelden
infolge der Einberufung seiner Mitglieder immer weniger wurde, erließ die
freiwillige Feuerwehr Ruhmannsfelden nachstehenden Aufruf, der von Haus zu Haus
in der ganzen Gemeindeflur Ruhmannsfelden bekannt gemacht wurde:

\begin{quote}
„Aufruf

an die verehrliche Einwohnerschaft hier.

Viele unserer Braven Kameraden, die auf das Schlachtfeld zogen – mit Gott für
Kaiser und Reich und unser geliebtes bayerisches Vaterland – zu kämpfen – waren
Mitglieder der freiwilligen Feuerwehr Ruhmannsfelden.

Die freiwillige Feuerwehr Ruhmannsfelden braucht für diese Mutigen, die sich
schon in Friedenszeit auf das Kampffeld der freiwilligen Feuerwehr gestellt
haben und jetzt auf dem Kampffeld des Krieges stehen – Ersatz damit, wenn
drohende Feuersgefahr eintreten sollte, die freiwillige Feuerwehr wirksam
eingreifen kann. Es ergeht deshalb vom Verwaltungsrat der freiwilligen Feuerwehr
Ruhmannsfelden der Aufruf zum Neueintritt in die freiwillige Feuerwehr
Ruhmannsfelden.

Unterste Altersgrenze vollend. 16. Lebensjahr, oberste Altersgrenze vollend. 60.
Lebensjahr.

Auf! Männer und Jünglinge! Gott zur Ehr, dem Nächsten zur Wehr!

Freiwillige Feuerwehr Ruhmannsfelden, der Verwaltungsrat. A. Högn,
Schriftführer, Kiesenbauer, Vorstand.“
\end{quote}

Auf diesen Aufruf hin traten 15 neue Mitglieder der freiwilligen Feuerwehr
Ruhmannsfelden bei.

\item Schwer verwundet wurden 3 Mitglieder der freiwilligen Feuerwehr
Ruhmannsfelden: Hr. Johann Gillmeier bei Arras (Verlust des rechten Armes),
Johann Lippl bei Verdun (Lungenschuss), Joseph Karl in den Vogesen (Kopfschuss).

\item Gefallen sind: Joseph Baumann, Schlacht an der Somme, Alois Metzger,
Schlacht in den Argonnen.

\item Vermisst wird: Hr. Michl Hobelsberger.
\end{itemize}

\section*{1915}

\begin{itemize}
\item Hr. Kaufmann Eggl lehnte die Annahme der Kommandantenstelle ab.

\item Die freiwillige Feuerwehr Ruhmannsfelden erhielt vom
Bezirksfeuerwehrausschuss einen Zuschuss von 30.- M.
\end{itemize}

\section*{1916}

\begin{itemize}
\item Am 9.1.1916 fand hier Kommandantenversammlung statt. Ludwig Stern, noch
nicht das vorgeschriebene Lebensalter, meldet sich freiwillig zur freiwilligen
Feuerwehr, die alle Verantwortung hieraus ablehnt.

\item Am 28.9.1916 war Brand bei Fromholzer, Färber.
\end{itemize}

\section*{1917}

\begin{itemize}
\item hätte das 50-jährige Gründungsfest der freiwilligen Feuerwehr
Ruhmannsfelden getroffen. Im Hinblick auf die Kriegszeit wurde das Fest
weiterhin verschoben.
\end{itemize}

\section*{1918}

\begin{itemize}
\item zeichnete die freiwillige Feuerwehr Ruhmannsfelden zur 8. Kriegsanleihe
700.- M., waren schon an 14 Mitglieder der freiwilligen Feuerwehr das eiserne
Kreuz, an ein Mitglied die goldene Verdienstmedaille und an 1 Mitglied die
bayerische Tapferkeitsmedaille verliehen worden.

\item 1918 hatte die freiwillige Feuerwehr Ruhmannsfelden 42 aktive Mitglieder.

\item Nach Beendigung des Weltkrieges wurde auch in Ruhmannsfelden eine
Bürgerwehr gebildet, der alle Feuerwehrmitglieder, die im Felde standen,
beitraten. Die Leitung der Bürgerwehr lag in den Händen der Gendarmerie.
\end{itemize}

\section*{1919}

\begin{itemize}
\item wurde für Hr. Aichinger, Hr. Anton Fronhofer als Kommandant gewählt.

\item Am 29. Juni fand hier die 44. Bezirksfeuerwehrversammlung statt.

\item Am 1. März war der Brand bei Hirtreiter Metzger. Es brannten ab,
Eiskeller, Remise und Stadel. 6 Feuerwehren löschten das Feuer und sicherten die
Nachbarsgebäude.

\item Durch Beschluss vom 14. April wurden die im Besitze der freiwilligen
Feuerwehr Ruhmannsfelden befindlichen Wertpapiere (Kriegsanleihen) verkauft,
weil das Geld benötigt wurde zur Beschaffung einer neuen Fahne, die bei der
Firma Auer in München in Auftrag gegeben wurde.

\item Bei einer am 16.12.1919 stattgefunden Verwaltungsratssitzung wurde
beschlossen, dass die Abgabe und Verwendung von Schläuchen ohne Verständigung
und ohne Erlaubnis der Kommandanten aufs strengste untersagt ist.

\item Am 15.6. war Inspektion der freiwilligen Feuerwehr Ruhmannsfelden.
\end{itemize}

\section*{1920}

\begin{itemize}
\item Bei der Generalversammlung am 26.12. wurden Ehrendiplome für 15-jährige
Dienstzeit ausgeteilt.
\end{itemize}

\section*{1922}

\begin{itemize}
\item Am 25. Juni fand das 55-jährige Gründungsfest (verbunden mit Fahnenweihe)
der freiwilligen Feuerwehr Ruhmannsfelden statt. Am Vorabend war Fackelzug und
Serenade. Der Festtag war vom schönsten Wetter begünstigt. H. Hr. Pfarrer
Fahrmeier nahm die Weihe der neuen Fahne vor, der dabei in einer herrlichen
Ansprache hinwies auf die ideale Berufsaufgabe des Feuerwehrmannes im Dienste
des nächsten und im Dienste seiner eigenen Selbstbestimmung.

\item Bei dem sich anschließenden Festakt sprach den von Jäger Voggendorf
gedichteten Prolog Frl. Frieda Högn. Die Festrede hielt Herr Bezirksschulrat
Aigner. In ergreifenden Worten gedachte er der verstorbenen und gefallenen
Feuerwehrkameraden. Dann folgte durch ihn die Fahnenenthüllung, indem er in
kräftigen Worten die Feuerwehrmänner aufforderte treu zu Fahne und zur
Feuerwehrsache zu stehen. Zum Schlusse sprach er den Dank aus, insbesonders dem
Bezirksfeuerwehrersatzvertreter Hr. Hollmayer in Linden. Dann übergab die
Fahnenmutter Frau Metzgermeistersgattin Holler die Fahne der freiwilligen
Feuerwehr Ruhmannsfelden, dann folgte die Übergabe des Fahnenbandes des
Patenvereins Teisnach II, der Fahnenbänder der Fahnenmutter und der
Festjungfrauen und zum Schluss die Übergabe der Fahnebänder an die 36 von
auswärts erschienen Feuerwehrvereine. Nachmittags war Festzug, bei dem sich auch
Herr Oberregierungsrat Schels beteiligte. Am Schaffer´schen Sommerkeller war
anschließend Kellerfest. Die prachtvolle Fahne, die aus der Fahnenfabrik Auer,
München stammt, wurde viel beachtet. Sie kostete 850 M. 1922 wurde der
Vereinsbeitrag auf 50 M und Aufnahmegebühr auf 20 Mxxi festgesetzt.

\item Am 26.12.1922 erhielten bei der Generalversammlung nachstehende Mitglieder
das 25-jährige bzw. das 40-jährige Ehrenzeichen: das 25-jährige Ehrenzeichen:
Hr. Benedikt Schaffer, Hr. Julius Zinke, Hr. Alois Schwarz, Hr. Ludwig Pongratz.
Das 40-jährige: Hr. Josef Rauch, Hr. Johann Hell, Hr. Ferdinand Fronhofer.
\end{itemize}

\section*{1923}

\begin{itemize}
\item Am 21.4. wurde beschlossen, dass die Fahnenstange der alten Feuerwehrfahne
an den Bürgerverein Ruhmannsfelden abgegeben wird.

\item Als Jahresbeitrag pro 1922 und 1923 wurden 500.- M festgesetzt.
\end{itemize}

\section*{1924}

\begin{itemize}
\item Bei der am 6.1. stattgefundenen Generalversammlung wurde Hr. Vorstand
Kiesenbauer zum Ehrenvorstand der freiwilligen Feuerwehr Ruhmannsfelden ernannt.

\item Der Gemeinde Ruhmannsfelden wurde vom Bezirksamte Viechtach mitgeteilt,
dass sie nicht das Recht habe, die Pflichtfeuerwehr aufzuheben.

\item Für Hr. Kiesenbauer wurde Hr. Georg Plank, Vorstand der freiwilligen
Feuerwehr Ruhmannsfelden und für Hr. Fronhofer, Hr. Michl Zinke Kommandant.

\item Die freiwillige Feuerwehr beteiligte sich an der Fahnenweihe in Teisnach
und Auerkiel.
\end{itemize}

\section*{1925}

\begin{itemize}
\item Hr. Michl Zinke trat als Kommandant der freiwilligen Feuerwehr
Ruhmannsfelden zurück, für ihn wurde Heinrich Leitner als Kommandant gewählt.

\item Am 4.10. war Inspektion und Kommandantenversammlung.

\item Georg Plank legte die Vorstandschaft nieder. Für ihn wurde Hr. Kaufmann
Eggl gewählt.
\end{itemize}

\section*{1926}

\begin{itemize}
\item Am 8. Juli starb der langjährige Vorstand Hr. Wenz. Kiesenbauer, der sich
um die freiwillige Feuerwehr Ruhmannsfelden sehr verdient gemacht hatte. Er war
auch beim Bezirksfeuerwehrausschuss.

\item Bei der Generalversammlung am 26.12 wurde beschlossen, dass in Zukunft die
Feuerwehrmannschaft per Auto zur Brandstätte gebracht werde, soweit sich das
ermöglichen lässt. Herr Schwannberger stellt sein Lastauto zur Verfügung.
\end{itemize}

\section*{1927}

\begin{itemize}
\item Am 6. Juni erklären Hr. Kaufmann Eggl und Hr. Kaufmann Leitner ihre
Chargen bei der freiwilligen Feuerwehr Ruhmannsfelden niederzulegen, wegen des
Sonntagsladenschlusses.

\item Hierauf wurden vorübergehend gewählt als Vorstand Hr. Adolf Sturm und als
Kommandant Hr. Joseph Ernst.

\item Am 11.9.1927 war Opfertag zu Gunsten des Feuerwehrheims, Familienabend mit
Tanz.

\item An Stelle der 1912 erlassenen Satzungen für die freiwillige Feuerwehr des
Bezirks Viechtach tritt die bezirkspolizeiliche Feuerlöschordnung für den Bezirk
Viechtach.

\item 1927 wurde beschlossen, dass alle Mitglieder, die 40 Jahre bei der
freiwilligen Feuerwehr sind, beitragsfrei sind.

\item 1927 wurde eine Alarm-„Sirene“ bei der AEG in Regenburg um 420.-M gekauft
und dieselbe auf dem Dach des Marktsrathauses von Hr. Schlossermeister Adolf
Sturm aufmontiert. Bei Brandfällen gilt das lang anhaltende Zeichen als Brand im
Markt, das kurze unterbrochene Zeichen als Brand in der Umgebung.
\end{itemize}

\section*{1928}

\begin{itemize}
\item Am 1. Januar traten die Satzungen des Bezirksfeuerwehr-Unterstützungs-
Vereins Viechtach in Kraft.

\item Am 19.6. wurde entsprechend dem Antrage beschlussmäßig seitens das
bayerischen Landes-Feuerwehrausschlusses an nachstehende Feuerwehrkameraden das
Feuerwehr-Ehrenkreuz des Verbandes für 50-jährige Dienstleistungen verliehen:
Hr. Leonhard Donauer, Schreinermeister, Hr. Johann Hell, Privatier, Hr. Johann
Rauch, Privatier, Hr. Joseph Rauch, Privatier, Hr. Michael Sixl, Privatier, Hr.
Alois Stadler, Privatier.

\item Am 9.9. dieses Jahres war im Saal der Brauerei Schaffner eine
Familienunterhaltung mit Tanz, wobei den Jubilaren die Auszeichnungen überreicht
wurden. Hr. Schriftführer Högn hielt dabei eine kleine Ansprache.

\item 1928 wird die Schlauchstation von Baumgartner zu Triendl verlegt.

\item Die freiwillige Feuerwehr Ruhmannsfelden beteiligte sich zahlreich an der
Beerdigung des verstorbenen langjährigen verdienten Kassiers der freiwilligen
Feuerwehr Ruhmannsfelden, Hr. Wilhelm Ederer, der am 16.8.1928 starb. Für ihn
übernimmt dessen Sohn Hr. Johann Ederer appr. Bader die Charge als Kassier.

\item Am 22. Juli fand in Zachenberg das Fest der Fahnenweihe der freiwilligen
Feuerwehr Zachenberg statt, bei der sich die freiwillige Feuerwehr
Ruhmannsfelden zahlreich beteiligte.

\item 1928 wurden die „Vorschriften über Maßnahmen bei Bränden durch elektrische
Anlagen“ zur Kenntnis gebracht.

\item Im Frühjahr fand der Führerkurs für den Bezirk Viechtach in Viechtach
statt.

\item Bei der Generalversammlung am 26.12.wurde als Vorstand Herr Adolf Sturm
und als Kommandant Hr. Heinrich Leitner gewählt.
\end{itemize}

\section*{1929}

\begin{itemize}
\item Am 14. April fand hier Bezirksfeuerwehrtag statt.

\item 1929 wurde angeordnet, dass jährlich 6 Übungen abgehalten werden müssen.
\end{itemize}

\section*{1930}

\begin{itemize}
\item Am 30.1. sind Richtlinien erschienen betreffend Richtlinien für Gesuch um
Zuschüsse aus dem Fond für Förderung des Feuerlöschwesens.

\item Am 4.3.1930 war Ladenbrand bei Hr. Kaufmann Eggl. Hr. Johann Lippl hat
sich dabei verletzt.

\item 1930 wurden neu beschafft 150 m neue Schläuche, 2 Strahlrohre, 2
Rauchmasken, Schlauchhalter, Verbandzeug, Pfeifchen etc.
\end{itemize}

\section*{1931}

\begin{itemize}
\item 16 bis 18-jährige können in die freiwillige Feuerwehr eintreten und sind
vorläufig beitragsfrei.

\item Herr Leitner legt die Charge als Kommandant nieder. An seine Stelle tritt
Hr. Michl Kiesenbauer.

\item Am 1. Juli starb Fr. Holler, Metzgermeisterswitwe von hier, welche 1874
und 1907 die Fahnenmutter machte.

\item Aus der großen Zahl der Vereinsmitglieder wurde eine Elite gebildet, der
30 bis 40 junge Leute angehören sollen und eigens ausgebildet werden sollen. Sie
bilden die erste Kompanie.

\item Die Neuuniformierung wurde begonnen, zunächst bei den Chargen.
\end{itemize}

\section*{1932}

\begin{itemize}
\item Am 12. Januar abends brannte das Sägewerk in der Stegmühle des Hr. Xaver
Brem ab.

\item Die freiwillige Feuerwehr erhielt einen Zuschuss von 80.- M.

\item Am 9. September vormittags 9 ½ Uhr brannte das Anwesen des Landwirts
Joseph Hinkofer von Ruhmannsfelden (Kalteck) ab. Der Brand brach im Stadel aus.
Brandursache ist unbekannt. Die Motorspritze Patersdorf hat sich glänzend
bewährt. Joseph Ernst von Rabenstein hat sich eine Verletzung am Bein zugezogen.

\item Am 23.9. beschloss der Gemeinderat einstimmig, „es soll für Ruhmannsfelden
eine Motorspritze angeschafft werden, falls die 40 \% Zuschuss vom Staate und 25
\% vom Bezirksamt Viechtach garantiert sind.“
\end{itemize}

\section*{1933}

\begin{itemize}
\item Am 14.3.1933 fand hier der Führerkurs des Bezirksfeuerwehrverbandes
Viechtach statt. Es beteiligte sich an demselben 45 Feuerwehrmänner der
Feuerwehren Ruhmannsfelden, Gotteszell, Achslach, Patersdorf, Zachenberg I und
II. Von der freiwilligen Feuerwehr Ruhmannsfelden nahmen teil die Herren:
Kiesenbauer Michl, Bielmeier Xaver, Friedrich Joseph, Högn August jun., Tremml
Josef, Ellmann Ludwig, Bielmeier Alois, Krieger Georg, Depellegrin Benedikt,
Bartl Johann, Kopp Michl, Krieger Michl, Glaschröder Franz, Baumgartner Josef,
Stieglbauer Karl.

\item Der Kursus wurde abgehalten auf der Zitzelsberger Wiese neben dem Stadel.
Kursleiter waren die Bezirksausschussmitglieder Kramhöller und Lummer. Zur Übung
wurde die Motorspritze von Patersdorf herbei gefahren.

\item Am 7. Mai 1933 fand in ganz Bayern auf Anordnung der bayerischen
Landesfeuerwehrverbandes ein „Feuerschutztag“ statt, der auch in Ruhmannsfelden
festlich begangen wurde. Um 5 Uhr früh war Weckruf. Um ¾ 9 Uhr war
Kirchenparade, an der auch die freiwillige Feuerwehr Patersdorf teilnahm. Daran
schloss sich das Floriani-Amt. Nach demselben bewegte sich der Zug unter
Vorstand Sturm unter dem Gedenken der im Weltkrieg gefallenen Feuerwehrkameraden
einen Kranz nieder. Dann ging’s zur Herberge zurück zum Frühschoppen. Um 12 Uhr
war Hauptübung der freiwilligen Feuerwehr Ruhmannsfelden mit Pflicht. Um 1 Uhr
war Rückmarsch zum Marktplatz. Dort hielt Schriftführer, Oberlehrer Högn an die
versammelten Feuerwehrmänner eine Ansprache über die „Feuerverhütung“. Hernach
verteilte Herr Bürgermeister Amberger nach einer Ansprache an die
Feuerwehrmitglieder das 40-jährige Ehrenzeichen an Hr. Joseph Hell, Kaufmann, an
Hr. Xaver Brem, Sägewerksbesitzer, Hr. Joseph Baumann, Schlosser und Hr.
Sebastian Vogl, Hausbesitzer. Das 25-jährige Ehrenzeichen an Hr. Joseph Holler,
Metzgermeister und Herrn Georg Plank.

\item Darauf hin schloss sich ein Parademarsch der freiwilligen Feuerwehr und
der Pflichtfeuerwehr an. Dann bewegte sich der Zug auf den Schafferkeller, wo
Gartenkonzert stattfand.

\item Am 24.6.1933 traf die Motorspritze (Firma Paul Ludwig, Bayreuth) in
Station Ruhmannsfelden ein. Bei strömendem Regen wurde dieselbe vom Vorstand Hr.
Sturm, Kommandant Hr. Kiesenbauer, Schriftführer Hr. Högn und Stieglbauer
Michael ausgeladen und an ein Fuhrwerk der Brauerei Amberger angehängt und zum
Feuerhaus transportiert.

\item Am 29.6.1933 fanden die Probe der Motorspritze und die Feierliche
Übernahme derselben statt. Einladung hierzu erging an alle zuständigen Stellen
und an die benachbarten Feuerwehren. Herr Bezirks- und Kreisfeuerwehrvertreter
Schedlbauer und der Bezirksfeuerwehrausschuss Hr. Kramheller, Teisnach konnten
nicht erscheinen, weil sie unmittelbar zuvor in Schutzhaft genommen wurden.
Oberhauptmann Altmann war am Erscheinen verhindert. Als Vertreter des
Bezirksausschlusses erschien Hr. Fischer, Arnbruck. Vertreten waren auch die
Feuerwehren Gotteszell, Achslach, Zachenberg, Patersdorf. Die Probe der
Motorspritze verlief glänzend.

\item Am 25.7.1933 war in Zuckenried nachts 12 Uhr Großfeuer ausgebrochen. Die
Motorspritze von Ruhmannsfelden lieferte von ½ 1 Uhr bis 8 Uhr unausgesetzt
Wasser herbei aus einer Entfernung von cirka 300 m, in eine Höhe von cirka 15 m.
Die Motorspritze bewährte sich vorzüglich und bestand die Feuerprobe glänzend.

\item Am 20.8.1933 beteiligte sich die freiwillige Feuerwehr Ruhmannsfelden an
Fahnenweihe des N.S. Reichsverbandes deutscher Kriegsopfer, Ortsgruppe
Ruhmannsfelden.

\item Am 27.8.1933 fand Inspektion der freiwilligen Feuerwehr Ruhmannsfelden
statt durch Bezirksausschlussmitglied Hr. Kramhöller, Teisnach.

\item Am 1. Oktober 1933 beteiligte sich die freiwillige Feuerwehr bei der
Fahnenweihe des „Stahlhelm, Ortsgruppe Ruhmannsfelden.“

\item Am Freitag, den 27.10.1933, brannte das Schnitzbauer-Anwesen in Zuckenried
ab. Ruhmannsfelden rückte mit Motorspritze aus. Es herrschte Wassermangel. Das
Anwesen brannte total nieder. Der junge Schnitzbauer soll selbst angezündet
haben. Er wurde verhaftet.

\item Am Donnerstag, den 9.11.1933, fand in München die Bürgermeistervereidigung
statt. Am Freitag, den 10.11.33, wurde der neu vereidigte Hr. Bürgermeister
Amberger am Bahnhof Ruhmannsfelden abends feierlich empfangen und zum
Magistratsgebäude in feierlichem Zuge geleitet, woselbst eine
Begrüßungsansprache durch Hr. Apotheker Voit stattfand. Auch die freiwillige
Feuerwehr Ruhmannsfelden hat sich am Zuge beteiligt.
\end{itemize}

\section*{1934}

\begin{itemize}
\item Am 6.1.1934 fand die ordentliche Generalversammlung pro 1933 statt, da im
Dezember 1933 die Generalversammlung nicht stattfinden durfte. Bei dieser
Generalversammlung wurde gewählt als Vorstand: Hr. Adolf Sturm,
Schlossermeister, als Schriftführer Hr. August Högn, Oberlehrer, als Kassier:
Hr. Johann Depellegrin, Kaufmann, als Zeugwart: Hr. Alois Stieglbauer, Kaufmann.

\item Am 9.1.1934 fand der Feuerwehrball in der Brauerei Schaffer statt.

\item Am 9.1.1934 wurde vom Landesbranddirektor Ecker München mit Zustimmung der
Kreisregierung für Niederbayern und Operpfalz Hr. Schedlbauer, Prackenbach mit
sofortiger Wirksamkeit zum Kreisfeuerwehrvertreter für den Kreis Niederbayern
mit der Dienstbezeichnung Kreisbranddirektor ernannt.

\item Im März 1934 wurde Hr. Michael Kiesenbauer zum Kommandanten der
freiwilligen Feuerwehr Ruhmannsfelden ernannt. Außerdem: Hr. Xaver Bielmeier,
Schreiner zum Adjutanten, Hr. Josef Tremml, Schlosser zum Führer der
Steigerjugend, Herr Ludwig Ellmann, Mechaniker zum Führer der Spritzenjugend.
Hr. Alois Bielmeier, Schneider zum Führer der Schlauchwagenjugend, Hr. Heinrich
Linsmeier. Fabrikarbeiter zum Führer der Ordnungsmannschaftsjugend.

\item Am Montag, den 4.6.1934 fand hier Führerkurs unter Leitung des
Brandmeisters Hr. Kramheller, Teisnach statt. Es waren vertreten die
Kommandanten, Adjutanten und jüngeren Feuerwehrkameraden der Feuerwehren
Ruhmannsfelden, Gotteszell, Achslach, Patersdorf (mit Motorspritze), Zachenberg
I und II, Teisnach I und II und IV, insgesamt 9 Feuerwehren mit cirka 60 Mann.
Der Kursus dauerte von 8 Uhr bis 16 Uhr.

\item Nach dem Kursus war Vorbeimarsch. Anschließend hielt Herr Brandmeister
Kramheller eine Ansprache, wobei er die ausgezeichneten Leistungen besonders
erwähnte und die Bitte anknüpfte, es möge auch draußen in den Wehren so
gearbeitet werden. Nachdem er sich noch bedankte für die Überlassung von
Turnplatz, Turnhalle und Feuerwehrgeräten, schloss er mit einem dreifachen „Sieg
Heil“ auf unseren Führer Adolf Hitler.

\item Am Sonntag, den 25.9.1934 war „Feuerschutztag“. Tags zuvor wurden
sämtliche Feuerwehrgeräte mit Blumen geziert. Am Sonntag früh war Weckruf. Um 8
Uhr wurden die Blumengezierten Feuerwehrgeräte auf dem Marktplatz zur
Besichtigung aufgestellt. Um 9 Uhr war Kirchenzug der freiwilligen Feuerwehr
Ruhmannsfelden mit Musik und der freiwilligen Feuerwehr Ruhmannsfelden mit Musik
und der freiwilligen Feuerwehr Zachenberg mit eigener Musikkapelle. Nach dem
Gottesdienst zogen die beiden Feuerwehren zum Kriegerdenkmal. Dort hielt Herr
Lehrer Schultz von hier eine Gedächtnisrede für die Gefallenen. Deutschlandlied
und Lied „Der gute Kamerad“ schlossen die eindrucksvolle Feier. Mittags 1 Uhr
war Übung für die Pflicht- und freiwillige Feuerwehr. Zuerst war Geräteübung,
daran schloss sich eine Schauübung. Das Rathaus und die Nachbarsanwesen galten
als Brandobjekt. Alle Geräte und alle Mannschaften wurden eingesetzt. Die Übung,
unter Kommando des Vorstandes Hr. Sturm verlief tadellos. Nach dieser Übung war
Aufstellung beim Rathaus. Schriftführer Högn sprach über „Brandschaden ist
Landschaden“. Hierauf ergriff Hr. Bürgermeister Zitzelsberger das Wort. Er
sprach zunächst der Pflicht- und der freiwilligen Feuerwehr seine Anerkennung
aus über die Leistungen und ermunterte sie auch fernerhin im Dienste der
Feuerwehr opferbereit und pflichtgemäß arbeiten zu wollen. Die Gemeinde werde
die Feuerwehr jederzeit unterstützen, wie sie das bisher schon getan hat. Dann
brachte er auf den Führer Adolf Hitler ein Dreifaches „Sieg Heil“ aus. Im
Anschluss daran verteilte er die Ehrenzeichen für 40-jährige und 25-jährige
Dienstleistungen bei der Feuerwehr Ruhmannsfelden und zwar an Hr. Privatier
Schaffer und Hr. Josef Wiesinger das 40-jährige und an Herrn Johann Ederer appr.
Bader und Hr. Johann Stracker, Spediteur das 25-jährige Ehrenzeichen.

\item Hierauf war Vorbeimarsch der freiwilligen Feuerwehr auf dem Marktplatze
vor dem Bürgermeister, Hr. Zitzelsberger und dann Propagandamarsch durch alle
Strassen des Marktes. Die blumengeschmückte Motorspritze wurde mitgefahren und
Tafeln mit Aufschriften mitgetragen. Nach Beendigung dieses Marsches
versammelten sich die Feuerwehrmänner zu einem Kameradschaftstreffen im
Herbergslokal. Die Feier verlief in schöner, eindrucksvoller Weise.

\item Am 26.12.1934 fand die 68. ordentliche Generalversammlung statt.
\end{itemize}

\section*{1935}

\begin{itemize}
\item Im März 1935 wurde in Teisnach ein Führerkurs abgehalten. An diesem Kurs
beteiligten sich die Feuerwehrmänner Joseph Friedrich, Wurzer, Linsmeier,
Ellmann.

\item Am 26.12.1935 fand die 69. ordentliche Generalversammlung statt.
\end{itemize}

\section*{1936}

\begin{itemize}
\item Am 1.2.1936 wurde Hr. Kreisbranddirektor und Bezirksbrandinspektor
Schedlbauer von Prackenbach durch Hr. Landesbranddirektor Ecker-München infolge
Überschreitung der Altersgrenze unter Anerkennung seiner Dienstleistungen von
seinen beiden Ämtern (als Kreisbranddirektor und Bezirksbrandinspektor)
entbunden. Kreisbranddirektor für den Regierungsbezirk Niederbayern wurde der
Kreisbrandmeister Stadler von Bärnbach, Bezirksamt Passau. Bezirksbrandmeister
für den Bezirk Viechtach wurde der bisherige Bezirksbrandmeister Hr. Johann
Kramhöller von Teisnach. Demselben wurde von der freiwilligen Feuerwehr
Ruhmannsfelden zu seiner Ernennung zum Bezirksbrandinspektor ein
Glückwunschschreiben zugeschickt.
\end{itemize}

\section*{1937}

\begin{itemize}
\item Am Sonntag, den 21.11.1937 fand in Ruhmannsfelden 62. Bezirksappell des
Bezirksfeuerwehrverbandes Viechtach statt. Neben den Ausschussmitgliedern und
sämtlichen Delegierten der Feuerwehren des Bezirkes waren auch anwesen der
Vertreter der Partei (Ortsgruppenleiter Pg. Schultz), Vertreter des Staates
(Herr Oberamtmann Oetzinger), und Vertreter der Wehrmacht (Hr. Hauptmann
Rutzenstein). Der Bezirksappell, geleitet von Hr. Bezirksbrandinspektor
Kramhöller Teisnach, verlief sehr anregend und interessant.
\end{itemize}

\section*{1938}

\begin{itemize}
\item Am Sonntag, den 16. Januar 1938 war 1. Appell 1938 (Generalversammlung).

\item Aus Anlass der Übertragung der Führerrede aus der Hauptstadt der Bewegung
fand am Samstag, den 2.4.1938 abends 7 Uhr in der Brauerei Schaffer „Pflicht-
Appell“ mit anschließendem Gemeinschaftsempfand statt.

\item Die freiwillige Feuerwehr Ruhmannsfelden ist laut Verordnung des
Landesverbandes Bayern vom 25.5.1938 nach den Bestimmungen für die Einteilung
der freiwilligen Feuerwehren neu eingeteilt worden. Die Neugliederung setzt sich
zusammen unter

\begin{quote}
A = Verwaltung\\
B = Führer und Unterführer\\
C = Löscheinheiten\\
D = Reserve und Mannschaften\\
E = Altersabteilung\\
F = Gesamtmitglieder und Einteilung\\
G = Übungsturnus

Zu A gehören: Vorstand, Schriftführer, Kassier und Zeugwart.

Zu B gehören: beauftragter Kommandant, die Brandmeister, die Löschmeister,
Hornist und Vereinsdiener.

Die Löscheinheiten sind: Löschtrupp, Halblöschzug nach Klasse a und
Normallöschung nach Klasse b.

Zu D gehören alle Wehrfähigen vom 18. bis 40. Lebensjahr soweit in obigen
Gliederungen nicht eingeteilt.

Zu E gehören alle über 40 Jahren alten Feuerwehrmänner und alle körperlich nicht
voll wehrfähigen Grund- und Hausbesitzer.

Zu F: Die Gesamtstärke von 170 Wehrmännern verteilt sich:

A: Aktive Wehr: 75 Mann

B: Reserve I 75 Mann

C: Altersabteilung 20 Mann = 170 Mann

Zu G: Die Übungen für die einzelnen Abteilungen finden getrennt stat.
\end{quote}

\item Am 28.5.1938 war im Nebenzimmer des Schafferkellers Führerappell
betreffend Eingliederung der freiwillige Feuerwehr Ruhmannsfelden.
\end{itemize}

\section*{1939}

\begin{itemize}
\item Am 11.1.1939 war im Schafferkeller Generalappell. Hierbei wurde die
Feuerwehrmannschaft in verschiedene Züge eingeteilt und die Brandmeister und
Löschmeister aufgestellt.

\item Ein ausgezeichnetes Ergebnis brachte die Sammlung am „Tage der deutschen
Polizei“ am 28. und 29. Januar 1939, die von Männern der Feuerwehr
Ruhmannsfelden durchgeführt wurde. Insgesamt wurden gesammelt: aus Verkauf von
Ansteckzeichen 38,60 M, aus Geldspenden der Wehrmänner: 24,40 M, aus sonstigen
Spenden: 29,50 M, aus Haus- und Straßensammlungen 74,24 M Summe 166,74 M.
Sammler und Spender haben damit ihre Opferfreudigkeit bekundet und mitgeholfen
am großen Gemeinschaftswerk, dem WHV des deutschen Volkes.

\item Am 3.7.1939 brannte das Kerschl-Anwesen in Zachenberg nieder.

\item Am 22. Juli 1939 wurden sämtliche Mitglieder der hiesigen freiwilligen
Feuerwehr zu einer Versammlung einberufen. Hr. Brandinspektor Kramheller
referierte über die Pflichten der Feuerwehrmänner und über Forbestand oder
Auflösung der freiwilligen Feuerwehr Ruhmannsfelden.

\item Am 30.7.1939 war Herr Brandinspektor Kramheller bei der Feuerwehrübung
anwesen.

\item Im August 1939 wurde Hr. Joseph Hinkofer als Kommandant aufgestellt.
\end{itemize}

\section*{1940}

\begin{itemize}
\item Am 21.1.1940 war in der Brauerei Schaffer Generalappell. Hr.
Hauptfeldwebel Hinkofer, der auf Urlaub hier weilte, leitete den Appell. Hr.
Kreisfeuerwehrführer Kramheller, Teisnach erstattete ein ausführliches Referat
über alle wichtigen Tagesfragen der Feuerwehr. Hr. Schmiedmeister Wühr wurde
stellvertretender Kommandant. Schriftführer Högn sprach dem
Kreisfeuerwehrführer, dem Kommandanten und allen an diesem Appell beteiligten
Feuerwehrkameraden den Dank aus.

\item Der Feuerwehrkamerad Paul Kern von Hochstrass wurde zu einem Feuerwehrkurs
an die Feuerwehrschule in Regensburg abkommandiert. Dieser Kurs dauerte vom 2.
bis 8.9.1940. Dieser Kurs wurde von 36 Feuerwehrkameraden besucht und von
Hauptbrandmeister Aichhammer geleitet. Nach Beendigung des Kurses wurde der
Feuerwehrkamerad Truppführer der freiwilligen Feuerwehr Ruhmannsfelden.

\item Für Sonntag, den 8.12.1940 mittags 12 Uhr war vom Landrat Viechtach
Alarmübung angesetzt. In Anwesenheit des Kreisfeuerwehrführers Hr. Kramhöller
Teisnach fand eine Angriffsübung statt. Nachdem von den Schlauchstationen
schnellstens die C-Schläuche herbeigeschafft waren und der Anschluss an die
Hydranten betätigt war, konnte in dem kurzen Zeitraum von 4 Minuten gespritzt
werden. Die Motorspritze mit B-Schläuchen, die das Wasser aus einem Graben
entnahm, war in 9 Minuten betriebsfertig. Hr. Kreisfeuerwehrführer Kramhöller
sprach nach der Übung seine vollste Anerkennung und Zufriedenheit aus.

\item Obertruppführer Hr. J. Hinkofer zurzeit Hauptfeldwebel im Heere wurde
wegen besonderer Verdienste mit dem Kriegsverdienstkreuz 2. Klasse mit
Schwertern ausgezeichnet. Obergefreiter Alois Bauer beim Infanterie-Regiment 20,
6. Kompanie erhielt das eiserne Kreuz 2. Klasse und das Verwundeten Abzeichen.
\end{itemize}

\section*{1941}

\begin{itemize}
\item Am 20.1.1941 nachmittags 5 Uhr landete nach mehreren Schleifen um den
Markt ein He-Militärflugzeug mit 2 Mann Besatzung, Maschinengewehren und 2
Kanonen in der Nähe des Färberbrückls in tiefem Schnee. Das Fahrzeug hat auf der
Fahrt von Brüssel nach Memmingen bei dunstigem und nebeligem Wetter die
Orientierung verloren. Die Landung ging glatt vonstatten. Die freiwillige
Feuerwehr Ruhmannsfelden übernahm in der Nacht vom 22. auf 23.1. von 8 Uhr
abends bis 6 Uhr morgens mit 5 Mann und 1 Führer die Wache, und ebenso vom 23.
auf 24. Januar von 7 Uhr abends bis 7 Uhr morgens. Am 24.1 traf militärische
Bewachung ein.

\item Die Besichtigung der HJ-Feuerwehrschar Ruhmannsfelden erfolgte am 19.
Oktober 1941, vormittags 9 Uhr im Schulhof in Ruhmannsfelden. Zur Besichtigung
erschienen: Der Kreisfeuerwehrführer, Hr. Kramhöller, der Stammführer der HJ,
Hr. Meierhöfer, der Bürgermeister, der Ortsgruppenleiter Hr. Karl, die
Gendarmerie, sowie der Führer der hiesigen Feuerwehr Hr. Wühr und dessen
Stellvertreter Hr. Linsmeier. Nach einer kurzen Begrüßung durch den
Kreisfeuerwehrführer begann die Besichtigung der HJ-Feuerwehrschar durch einige
exakte Fußübungen unter der Führung des HJ-Feuerwehrscharführers Linsmeiers.
Anschließend ging die HJ- Feuerwehrschar zur Übung über. Alle Übungen wurden
sehr gut durchgeführt. Durch die Abschlussmeldung des Gruppenführers wurde die
Übung beendet. Der Kreisfeuerwehrführer Hr. Kramheller schloss die Besichtigung
des HJ- Feuerwehrschar Ruhmannsfelden mit einer kurzen Ansprache und wies auf
die Wichtigkeit der HJ-Feuerwehrscharen hin. Der Stammführer des Stammes II
Viechtach Hr. Meierhöfer überreichte dann den HJ-Feuerwehrjungen das HJ-
Feuerwehr Ehrenzeichen. Er sprach im Namen des Landrates Viechtach der HJ-
Feuerwehrschar Ruhmannsfelden die vollste Anerkennung aus.
\end{itemize}

\section*{1942}

\begin{itemize}
\item Am Sonntag, den 1.3.1942, fand in der Brauerei Schaffer der Jahresappell
statt.
\end{itemize}

\section*{1943}

\begin{itemize}
\item Vom 7. bis 14.2.1943 besuchte Feuerwehrführer Hr. Josef Hinkofer die
Feuerwehrschule Regensburg.

\item Am 12.21943 war Planspielbesprechung des Luftschutzes für Ruhmannsfelden.
Diese war von der Regierung von Niederbayern/Oberpfalz angeordnet und von
Luftschutzoffizier der Regierung, Hr. Leutnant Ditsch geleitet worden. Sie war
besucht von den Vertretern des Landrates Viechtach unter anderem Hr. Landrat
Seufert von Deggendorf, dem Kreissicherheitskommissär, dem Kreisgruppenführer
des RLB, dem Kreisfeuerwehrführer und dem Bürgermeister von Ruhmannsfelden.
Außerdem waren erschienen alle Bürgermeister und Feuerwehrführer der 15 km-Zone.
Für den Bürgermeister und für den RLG sprach Rektor Högn, für die Feuerwehr der
Kreisfeuerwehrführer Hr. Kramhöller und der Feuerwehrführerstellvertreter Hr.
Linsmeier, für die Luftschutzwarte Hr. Zadler, für die Post Hr. Brummer, für die
Bahn Hr. Widmann, für NSV Hr. Ernst und für die Gendarmerie Hr.
Gendarmeriemeister Hr. Schindlbeck und für die Glaserinnung Hr. Geiger. Die
Besprechung war nicht öffentlich. Die Aussprache hat gezeigt, dass die Feuerwehr
Ruhmannsfelden im Falle eines feindlichen Fliegerangriffes ihrer Aufgabe bewusst
und gewachsen ist.

\item Am 28.3.1943 fand der Generalappell für die Feuerwehr Ruhmannsfelden
statt. Bei diesem waren auch anwesend der Kreisfeuerwehrführer Hr. Kramheller,
der in längeren Ausführungen referiert über alle wichtigen Fragen, die zurzeit
die Feuerwehr berühren und der Unterkreisfeuerwehrführer Kronner, der seine
Beobachtungen und Eindrücke bei den Visitationen der einzelnen Feuerwehren
schilderte. Hr. Feuerwehrführer Hinkofer leitete den Generalappell und schloss
ihn nach 2 ½ stündiger Dauer mit dem 3-fachen Sieg-Heil auf den Führer.

\item Am 5.9.1943 fand lauf Anordnung der Kreisfeuerwehrführers eine
Dringlichkeitswehrübung statt, zu der 29 Feuerwehrkameraden vorgeladen waren.
Der Kreisfeuerwehrführer erteilte denselben besondere Instruktionen. Anwesend
war dabei auch Hr. Bürgermeister.

\item Am 17.9.1943 mittags 11 Uhr brach bei Schrötter in der Grabsiedlung hier
ein Dachbodenbrand aus. Die rasch herbeigeeilte Feuerwehr löschte in kurzer Zeit
den Brand, wobei sich insbesonders die HJ-Feuerwehr auszeichnete. Auch die
Brandwache wurde von 2 Jungen der HJ-Feuerwehr gestellt. Der Führer der HJ-
Feuerwehr Hr. Johann Linsmeier erhielt eine Kopfverletzung, sodass er sich in
ärztliche Behandlung begeben musste.
\end{itemize}

\section*{1944}

\begin{itemize}
\item Bei der am Sonntag, den 30.1.1944 nachmittags 3 Uhr stattgefundenen
Großkundgebung zum Tage der nationalen Erhebung in hiesiger Turnhalle beteiligte
sich die Feuerwehr sehr zahlreich.

\item Am Freitag, den 11.8.1944 fand in Ruhmannsfelden eine von der Regierung
von Niederbayern/Oberpfalz festgesetzte Luftschutzübung statt, bei der sich auch
die freiwillige Feuerwehr Ruhmannsfelden zu beteiligen hatte. Der anwesende
Kreisfeuerwehrführer Hr. Kramheller von Teisnach sprach sich nach der
Luftschutzübung sehr lobenswert über das schnelle und exakte Eingreifen der
Feuerwehr aus und spendete besonders der tadellos arbeitenden HJ-Feuerwehr ein
ganz besonderes Lob.

\item Am 23.8.1944 nachmittags 3 Uhr brach im Stadel eines kleinen Landanwesens
in Lindenau, Gemeinde Achslach ein Brand aus, bei dem die freiwillige Feuerwehr
Ruhmannsfelden mit der Motorspritze raschestens erschienen war, aber nicht mehr
in Hilfeleistung zu treten hatte, da der Brand teils von den eigenen Leuten,
teils von der erschienen Nachbarsfeuerwehr rasch gelöscht werden konnte.
\end{itemize}

\section*{1945}

\begin{itemize}
\item Bei der Bombardierung der Marktes Ruhmannsfelden durch die Amerikaner
wurde unter anderem das Feuerhaus in Brand geschossen und die darin aufbewahrten
Feuerwehrgeräte und Utensilien restlos zerstört.
\end{itemize}

\section*{1948}

\begin{itemize}
\item Am 3.7.1948 nachts 12 Uhr brannte der Eiskeller der Brauerei Stadler
nieder.

\item Am 12.12.1948 fand eine Generalversammlung der freiwilligen Feuerwehr
Ruhmannsfelden statt.
\end{itemize}

\section*{1949}

\begin{itemize}
\item Am 8.1.1949 war der herkömmliche Feuerwehrball mit Fackelzug und am 23.1.
eine große Tanzunterhaltung im neuen Saal der Brauerei Vornehm.

\item Am 3.2.1949 brach bei Edenhofer Hochstraße ein Werkstättenbrand aus, der
aber rasch gelöscht werden konnte. Maschinen und Werkzeuge konnten in Sicherheit
gebracht werden.

\item Am 29.6.1949 fand der Jahrestag mit Kirchenzug, mit Ehrung der gefallenen
und verstorbenen Feuerwehrkameraden, mit Generalversammlung und anschließender
Tanzveranstaltung statt.

\item Am 8.7.1949 wurde in der Brauerei Schaffer ein Feuerwehrball abgehalten.

\item Am 9. Juli 1949 beteiligten sich die jungen aktiven Feuerwehrkameraden an
einem Maschinistenlehrgang der hiesigen Motorspritze.

\item Bei der am 10.7.1949 stattgefundenen Fahnenweihe der freiwilligen
Feuerwehr Pirka beteiligten sich 15 Mann der hiesigen Feuerwehr.

\item Am 18.9.1949 war unangemeldeter Generalappell sämtlicher Feuerwehren des
Bezirkes Viechtach mittags 1 Uhr in Teisnach.

\item Am 2. Oktober 1949 brach bei dem Bauern Wittenzellner in Prünst mittags 1
Uhr ein Brand aus, dem der Stadel und das Stallgebäude zum Opfer fielen. Das
Wasser musste von der weit entfernten Teisnach hergeleitet werden, weil der nahe
gelegene Weiher wasserleer war.
\end{itemize}

\section*{1950}

\begin{itemize}
\item Zu Gunsten der Feuerwehrkasse fand am 5.6.1950 ein Heimatabend der
freiwilligen Feuerwehr Ruhmannsfelden unter Mitwirkung des Trachtenvereins in
der Turnhalle statt.

\item Am 7.1.1950 fand der herkömmliche Feuerwehrball mit Fackelzug statt.

\item Am 12.3.1950 war der Jahrestag mit Kirchenzug, mit Ehrung der Gefallenen
und verstorbenen Feuerwehrkameraden und mit Generalversammlung.

\item Am 14.3.1950 wurde beim Kriegsentschädigungsamt Passau ein
Entschädigungsantrag in Höhe von 32 192.- DM für kriegsbeschädigtes Feuerhaus
und für restlos zerstörte Feuerwehrgeräte und Utensilien eingereicht.

\item Am 5.6.1950 brach oberhalb Lindenau im Vornehmwald ein Waldbrand aus. Es
waren erschienen die Feuerwehren von Achslach, Ruhmannsfelden, Gotteszell und
Viechtach. Das Feueranmachen auf der Waldblöße bei sengender Hitze war die
Brandursache.

\item Ebenso war am 13.6.1950 ein Waldbrand in Bergern im Zitzelsberger Wald,
der aber im Entstehen gelöscht werden konnte. Die freiwillige Feuerwehr
Ruhmannsfelden brauchte nicht mehr einzugreifen.

\item Am 16. Juli 1950 fand eine außerordentliche Generalversammlung wegen
Neuwahlen statt. Als Vorstand wurde Hr. Josef Brem und als 1. Kommandant Hr.
Johann Linsmeier gewählt.

\item Am 13.8.1950 fand anlässlich des Volksfestes in Ruhmannsfelden ein
„Feuerwehrtag“ statt. Um ½ 11 Uhr gab die Sirene das Signal zur Schauübung. Als
Brandobjekt wurde zuerst das Anwesen des Hr. Ludwig Helmbrecht und dann das
Anwesen der Hr. Karl Raster angenommen. Hr. Kreisbrandinspektor Kramheller
Teisnach, wohnte der Schauübung bei. Beim Festzug mittags 13 Uhr beteiligten
sich die Feuerwehren Ruhmannsfelden, Zachenberg, Gotteszell und Patersdorf.
Ihnen voraus fuhr der Kreisbrandinspektor Hr. Kramheller in einer schön
geschmückten Equipage. Nach dem Testzug versammelten sich die Feuerwehrmänner im
Schafferkeller, wo ein Gartenkonzert stattfand.

\item Am 1. Oktober fand der herkömmliche Jahrtag der freiwilligen Feuerwehr
Ruhmannsfelden statt. Um ½ 10 Uhr war der Kirchenzug und dann Gottesdienst. Nach
dem Gottesdienst war Gedenken und Ehrung der verstorbenen und gefallenen
Feuerwehrkameraden am Kriegerdenkmal. Anschließend fand im Saal der Brauerei
Schaffer Ehrung von 72 verdienten Jubilaren der Feuerwehr für 25. 40. und
50-jährige treue Dienstzeit durch Kreisbrandinspektor Kramheller Teisnach statt.
In seiner Festansprache übermittelte er den verdienten Veteranen der Wehr den
Dank und die Anerkennung des Landrates, um sodann auf die Bedeutung des
Jahrtages einzugehen. Unter anderem wies er darauf hin, dass die erfolgreiche
Tätigkeit des früheren Kommandanten Hr. Hinkofer nicht vergessen werden dürfe
und ermahnte zum treuen Zusammenstehen. Auch der hiesige Kommandant, Hr.
Linsmeier Johann sprach den Jubilaren seine Glückwünsche aus. Hr. Bürgermeister
Muhr rief die jungen Männer aus den Reihen der Neubürger zur Mitarbeit auf und
stattete ebenfalls dem früheren Kommandanten Hinkofer für seine 14-jährige
Tätigkeit im Dienste der Allgemeinheit seinen Dank aus. Im Namen der Jubilare
dankte Hr. Rektor Högn für die zuteil gewordene Ehrung und betonte, dass die
Alten stets der Feuerwehr die Treue halten werden.

\item Am 8.12.1950 wurde ein Kameradschaftsabend der aktiven Mitglieder der
freiwilligen Feuerwehr Ruhmannsfelden im Herbergslokal veranstaltet.

\item Die 83. Generalversammlung der freiwilligen Feuerwehr Ruhmannsfelden fand
am 26.12.1950 im Herbergslokal der Brauerei Schaffer statt. Nachdem der
bisherige Schriftführer Rektor August Högn aus gesundheitlichen Gründen und
infolge eines hohen Alters den Schriftführerposten niederlegte, wurde eine
Neuwahl notwendig. Dabei wurde die Vorstandschaft der freiwilligen Feuerwehr
ergänzt mit dem neuen Schriftführer Hr. Johann Freisinger und dem
Vorstandsmitglied Hr. Michl Wurzer. Gleichzeitig wurde der frühere Schriftführer
Rektor August Högn von der Versammlung in Anerkennung seiner Verdienste für die
freiwillige Feuerwehr Ruhmannsfelden während seiner 40-jährigen Dienstzeit als
Schriftführer der freiwilligen Feuerwehr Ruhmannsfelden zum Ehrenschriftführer
ernannt.
\end{itemize}

\section*{1951}

\begin{itemize}
\item Am 8.1.1951 fand der herkömmliche Feuerwehrball mit Fackelzug statt. Es
spielte die Kapelle Heinrich.

\item Wegen Aussprache über Ankauf eines LKW zum Transport der Motorspritze und
der Mannschaft fand am 23.2.1951 eine Verwaltungsratssitzung statt. Der
Verwaltungsrat beschloss dabei, den Ehrenabend für Rektor Högn am 11. März zu
veranstalten. Weitere 11 verdiente Mitglieder wurden zur Ehrung durch das
Staatsministerium vorgeschlagen. Ebenso wurde der Übungsplan für 1951
aufgestellt.

\item Am 11. März fand der durch den Verwaltungsrat beschlossene Ehrenabend für
Rektor Högn statt. Mitten unter seinen Feuerwehrkameraden und Marktsbürgern
feierte das langjährige Mitglied der freiwilligen Feuerwehr Ruhmannsfelden,
deren Schriftführer und Chronist er seit 1910 ist, das 40-jährige
Dienstjubiläum. Kreisbrandinspektor Kramheller war ebenfalls anwesend. Die
Musikkapelle Heinrich sorgte kostenlos für musikalische Umrahmung der Feier.
Besonderes Bepräge erhielt die Feier durch die Anwesenheit der „Alten“, einer
Vereinigung der alten Bürger, Handwerker und Pensionisten von Ruhmannsfelden,
der auch der Jubilar angehörte. Die Feier wurde um 17 Uhr leider durch einen
Sirenenalarm unterbrochen. In der Gastwirtschaft Wilhelm in Auerbach brach um
diese Zeit ein Zimmerbrand aus, der aber rasch gelöscht werden konnte.

\item Am 1.4.1951 war die erste Feuerwehrübung für das Jahr 1951. Es handelte
sich bei dieser Übung um die Ausprobierung der sämtlichen Oberflurhydranten im
Markte. Bei dieser Überprüfung der 32 Oberflurhydranten wurden Frostrisse
festgestellt und bei 11 Hydranten sind die Kopfdichtungen defekt. Ihre
Ausbesserung kann sofort ausgeführt werden. Überprüfung der restlichen 10
Hydranten soll bei der nächsten Übung erfolgen.
\end{itemize}

\chapter[Reihenfolge der Chargen]{Reihenfolge der Chargen der freiwilligen
Feuerwehr Ruhmannsfelden:}

\section*{Vorstände}

\begin{compactitem}
\item Alois Probst (1867 – 1892)
\item Alois Hochheitinger (1892 – 1897)
\item Joseph Schrötter (1897 – 1901)
\item Friedrich Rauch (1901)
\item Alois Maier (1902)
\item Benedikt Schaffer (1903)
\item Wenz. Kiesenbauer (1903 – 1923), wird Ehrenvorstand
\item Georg Plank (1923 – 1938)
\end{compactitem}

\section*{Kommandanten}

(früher „Hauptmann“ im 3. Reich „Führer“)

\begin{compactitem}
\item Joseph Lukas (1867 bis 1894), ab 1883 auch Bezirksfeuerwehrvertreter
\item Joseph Rauch (1895 – 1896)
\item Andreas Hobelsberger (1897)
\item Joseph Klein (1898 – 1901)
\item Johann Besendorfer (1901 – 1902)
\item Gottfried Bielmeier (1903 – 1907)
\item Alois Bielmeier (1907 – 1911)
\item Aichinger (1912 – 1918)
\item Anton Frohnhofer (1919 – 1923)
\item Michl Zinke (1923 – 1925)
\item Heinrich Leitner (1925 – 27)
\item Joseph Ernst (1927 – 28)
\item Heinrich Leitner (1928 – 1931
\item Michael Kiesenbauer (1931 – 1938)
\item Mathes (1938)
\item Michael Zinke (1939)
\item Bielmeier (1939)
\item Alois Stieglbauer (1930)
\item Joseph Hinkofer (1939 – 1950)
\item Johann Linsmeier (ab 1950)
\end{compactitem}

\section*{stellvertretende während des Krieges}

\begin{compactitem}
\item Wühr (1939)
\item Linsmeier (?)
\end{compactitem}

\section*{Schriftführer}

\begin{compactitem}
\item Raymund Schinagl, Schullehrer (1867 bis ?)
\item August Wimmer, Aufschläger (+1870)
\item Scheibenzuber, Aufschläger (1870 bis 1888)
\item Max Weig, Schullehrer (1888 bis 1897)
\item Joseph Lukas (1898 – 1909)
\item August Högn, Lehrer, Rektor (1910 – 1950)
\item Johann Freisinger (ab 1950)
\end{compactitem}

\section*{Kassier}

\begin{compactitem}
\item Joseph Moosmüller (1867 – 1892)
\item Joseph Meier (1892)
\item Alois Meier (1893 – 1901)
\item Wilhelm Ederer (1902 – 1928)
\item Johann Ederer (1928 – 1934)
\item Johann Depellegrin (1934 – 1939)
\end{compactitem}

\part{Anhang}

\chapter{Quellenangaben}

\begin{itemize}
\item Dienstbücher und Aufschreibungen des Bezirksbrandinspektors Hr. F.
Schedlbauer in Prackenbach

\item Aktennachlass der verstorbenen Hr. J. Lukas und Hr. W. Kiesenbauer

\item Gemeindearchiv

\item Akten der freiwilligen Feuerwehr Ruhmannsfelden
\end{itemize}

\chapter{Anmerkung}

\end{document}