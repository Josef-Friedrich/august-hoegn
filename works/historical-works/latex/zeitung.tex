





























AUGUST HÖGN



HEIMATKUNDLICHE

ZEITUNGSARTIKEL

Meinen Eltern Josef und Anita Friedrich gewidmet

































































Mein Dank gilt:

Herrn Pfarrer Meier, Lotte Freisinger,

Herrn Rektor Roßmeißl



Textgrundlage:

Abschriften von Pfarrer Reicheneder aus der Reicheneder-Chronik

unter der Rubrik „Ruhmannsfelden“: I. – IV., unter der Rubrik „Pfarrkirche St.
Laurentius“: V.

II: 2. stammt aus der Schulchronik der Grundschule Ruhmannsfelden.



Projekt August Högn Geschichtswerk

Ruhmannsfelden, 2003

1. Auflage zu 10 Stück

AUGUST HÖGN

1878 1961





HEIMATKUNDLICHE

ZEITUNGSARTIKEL



























EDITIERT VON

JOSEF FRIEDRICH 2003



INHALTSVERZEICHNIS

INHALTSVERZEICHNIS  4

HEIMTAKUNDLICHE ZEITUNGSARTIKEL 5

.I Geschichtliches vom Markt Ruhmannsfelden,

"Gäu und Wald", Beilage zum Deggendorfer Donauboten, 6.11.1926 und 15.12.1926  
5

.1 Der Name "Ruhmannsfelden"    5

.2 Die Bezeichnung „Markt“  6

.3 Das Wappen von Ruhmannsfelden    7

.4 Schloss und Schlossberg Ruhmannsfelden   8

.II Geschichtliches vom Markt Ruhmannsfelden,

„Gäu und Wald“, Beilage zum Deggendorfer Donauboten, 20.8.1927, Chronik der
Grundschule Ruhmannsfelden  10

.1 Von der Schule in Ruhmannsfelden in früherer Zeit bis 1835*  10

.)a Die ersten Schullehrer in Ruhmannsfelden*   10

.)b Die Schule unter Lehrer Andreas Stern*  12

.)c Das Schul- und Mesnerhaus*  13

.2 Die Schule ab 1835   15

.)a Die Schule benötigt immer mehr Platz*   15

.)b Bau eines 2. Schulhauses*   17

.)c Bau des neuen Schulhauses 1907/08   18

.III Was Ruhmannsfelden für Jubiläen feiern könnte?

Viechtacher Tageblatt, 9.9.1928 19

.IV Wie hat es um Ruhmannsfelden herum ausgesehen vor seiner Entstehung?

Viechtacher Tagblatt, 25.10.1928    22

.V Pfarrkirche St. Laurentius Ruhmannsfelden,

Viechtacher Tagblatt, 1928/29   25

.1 Die Pfarrkirche St. Laurentius bis zum Brand 1820*   25

.2 Die Pfarrkirche St. Laurentius bis zur Einweihung 1828*  27

.3 Die Pfarrkirche St. Laurentius nach der Einweihung bis 1928* 32

ANHANG  38

1 Anmerkung 38



HEIMTAKUNDLICHE ZEITUNGSARTIKEL

AUGUST HÖGN

.I Geschichtliches vom Markt Ruhmannsfelden,

"Gäu und Wald", Beilage zum Deggendorfer Donauboten, 6.11.1926 und 15.12.1926

.1 Der Name "Ruhmannsfelden"

In einer der früheren Nummern dieser Beilage lesen wir in einer geschichtlichen
Abhandlung vom Markte Ruhmannsfelden, dass man sich noch über die Herkunft des
Namens Ruhmannsfelden im Unklaren ist. Ich fragte meinen Tischnachbarn, einen
alten Ruhmannsfeldner Bürger, ob er den nicht wisse, woher der Name
Ruhmannsfelden kommen möge. "Das ist doch sehr einfach. Ruhmannsfelden - ruht
der Mann im Felde," war die schlagfertige Auskunft. Nun musste ich ihm doch zu
bedenken geben, dass er den Namen anders ausspricht, als er ihn deutet, denn er
müsste konsequenter Weise dann Ruhtmannsfeldn oder Ruhemannsfeld aussprechen. So
aber spricht jung und alt den Namen ganz richtig, gemäß seiner Herkunft,
Rumarsfeldn aus. In einer Oberalteicher Urkunde (1184) erscheint der Name
Rumarsfelden, in einer Niederalteicher Urkunde Rudmarsfelden. Im Jahre 1394
schrieb man Rumatzfelden. Und nun kommt für Ruhmannsfelden die schwere
Zerwürfniszeit mit dem Kloster Gotteszell.

Die Ruhmannsfeldner durften nicht mehr ihr bisheriges eigenes Siegel führen. Nur
was das Kloster Gotteszell mit seinem Siegel bestätigte, war gültig. Da
fertigten sich die schlauen Ruhmannsfeldner ein anderes Siegel an. Zwischen zwei
Krummstäben war ein Kissen und auf diesem lag eine Rübe. In einer Urkunde von
1448 erscheint der Name Ruebmannsfelden (Rübmannsfelden). Das Siegel musste
selbstverständlich auf Befehl des damaligen Abtes sofort verschwinden. Aber die
Schreibweise des Namens Rübmannsfelden und die Erklärung, dass der Name
Ruhmannsfelden von Rübe, Symbol des Ackerbaues, des Feldbaues, herstamme,
pflanzte sich noch lange fort. Es ist anzunehmen, dass man später ein weiters
Siegel angefertigt hat und zwar einen im Felde ruhenden Mann. In einer Finkschen
Karte aus der 2. Hälfte des 17. Jahrhunderts erscheint plötzlich der Name
Ruemannsfelden. Seit dieser Zeit mag wohl auch die Namensdeutung "ruht der Mann
im Felde" traditionell geworden sein. Auch H. Hr. Pfarrer Castenauer hat, diese
Deutung des Namens vom Hörensagen in seine Beschreibung vom Markte
Ruhmannsfelden aufgenommen. Die Formen Ruhmarsfelden, Rudmarsfelden, die der
Volksmund bestätigt, die tragen den Stempel der Originalität fort.

Die erste Hälfte des Namens ist der Personenname Hrothmar (Hruothmar= Romar)
Hrot(i) = Ruhm, Sieg; mar(u) = berühmt. Hrothmar = also: der Siegberühmte.
Ruhmarsfelden = das Feld (Bereich, Bezirk) des Siegberühmten.

Mein lieber Tischnachbar! Die Namen Ruebmannsfelden, Ruhemannsfelden, sind
längst verwischt, weil aus Konfliktsstoffen, aus Zank und Streit heraus geboren.
Die Rübe sollte den gestrengen Klosterherren zeigen, dass der unbeugsame Wille
der trutzigen Ruhmannsfeldner noch lange nicht gebrochen, wenn auch die
Krummstäbe drohend über ihrem Haupte sich regten. Und es wurde der Trotz der
Ruhmannsfeldner doch gebrochen. Ruhmannsfelden und seine Bürgerschaft musste die
Widerspenstigkeit gegen das Kloster Gotteszell sehr stark büßen. 30-jahriger
Krieg und österreichischer Erbfolgekrieg schlugen Ruhmannsfelden schwere Wunden.
Erst in der 2. Hälfte des 17. Jahrhunderts gewannen die Bürger, wieder das
Vertrauen zu ihrem geistlichen Oberherrn. Es kehrte Ruhe und Friede ein (ruht
der Mann im Feld), nachdem Jahrhunderte lang Unruhe und Unfriede die Mauern des
einst unter Aldersbacher Klosterherrschaft aufblühenden Marktes Ruhmannsfelden
erfüllte. Die Deutung des Namens können wir also unmöglich aus dem 14. 15. oder
17. Jahrhundert herleiten, nachdem der Ort Ruhmannsfelden seit dem Jahre 1100
besteht und der Name Rumarsfelden sich aus der Mitte des 12. Jahrhunderts
urkundlich nachweisen lässt.

Sind wir froh, dass der Name Rumarsfelden auch keine ändere Deutung zulässt.
Unser Stolz ist der Name unseres Heimatortes Ruhmannsfelden, welch letzterer
einstens der Wohnsitz eines siegberühmten Helden war, der im internationalen
Wettkampf auf Schweizer Boden den Sieg erringen und mit der Qualifikation
"Siegberühmter" an den Ort zurück konnte, der nach ihm den Namen
"Ruhmannsfelden", das ist "Feld des Siegberühmten" erhielt.



.2 Die Bezeichnung „Markt“

Aichinger schreibt in "Kloster Metten und seine Umgebung", dass sich in der
letzten Zeit der Regierung Aldersbachs Ruhmannsfelden zum Markt emporgeschwungen
hat. Wann die Markttitelverleihung und von wem sie stattgefunden hat, ist
urkundlich nicht nachzuweisen, wie eben an so vielen anderen Plätzen auch.
Plötzlich taucht dann in der einen oder anderen Urkunde der Titel „Markt" oder
„Stadt" auf.

Jakob der Rueerer stellte am 26. April 1416 eine Urkunde aus, in welcher er sich
„dy czeitt Richter dez Markehtz zue Ruedmansfelden" nennt. Es dürfte kaum ein
Zweifel bestehen, dass der Urkundenaussteller als landesherrlicher Richter über
die Markteigenschaft seines Wirkungsortes Bescheid wusste. Wir sind deshalb
berechtigt, die Markteigenschaft zu Ruhmannsfelden schon seit dem Beginne des
15. Jahrhunderts in Anspruch zu nehmen. In einem Aldersbacher Codex vom Jahre
1452 ist die Rede von dem forum Rudmansfelden, also Markt, während in einer
Urkunde vom 2. April 1475 das opidum Rudmansfelden erscheint, was mehr an die
befestigte Siedlung als an den Markt gemahnt. Auf Bitten der "Burger unnsers
Margkts zu Rudmansfelden" tut ihnen Herzog Albrecht IV. von Bayern-München die
Gnade: "...Freyen sie auch wissenlich in crafft des briefs, Also das sy und all
Ir nachkomen, sich aller der gnaden vnd freihait geprauchen vud nyessen sollen,
In allermaß als annder vnser Märkt, In Nidern Baiern, von unsern vordern gefreyt
sein."

Das Privileg ist nur in Abschrift erhalten und undatiert, steht aber zwischen
zwei Urkunden desselben Jahres 1469 und darf daher als aus diesem Jahre stammend
angenommen werden. In einem Literale des Klosters Gotteszell vom Jahre 1566 -
1602 kommen vor die „Geschworenen des Rats und Markts R.", "Rat und Gemein des
Markts R.", „Die Geschworenen des Rats und ganze Bürgerschaft des Markts R.",
wie überhaupt seit der Begnadigung von 1496 keinerlei Zweifel an dem Marktrechte
Ruhmannsfeldens mehr aufkommen kann.

.3 Das Wappen von Ruhmannsfelden

Im Laufe der verschiedenen Zeiten hat man in Ruhmannsfelden nicht immer ein und
dasselbe Wappen oder Siegel geführt. Über das ursprüngliche Wappen, das
Ruhmannsfelden zur Ritterszeit, also im 12. Jahrhundert, führte, ist uns gar
nichts mehr bekannt. Später hatte man das Wappen, das Muelich, Apian und das
Wappenbuch der Landschaft bringen, nämlich: "in Blau unter zwei schräg
gekreuzten silbernen Hirtenstäben eine weiße Rübe mit grünen Blättern." Dieses
Wappen ist ein sogenanntes redendes Wappen, welches ohne jede Autorisierung
längere Zeit gebraucht worden zu sein scheint. Die Hirtenstäbe würden lediglich
als Dekoration anzusehen sein. Die Rübe hat man als Wappenbild angenommen und
man hat lange Zeit Rub = Rueb, Rüb = Rüebmannsfelden geschrieben. Dieses Wappen
ist abgebildet als Nr. 541 im Philipp Apians Wappensammlung der altbayerischen
Landschaft, wie des zu seiner Zeit abgegangenen Adels (Oberbayerisches Archiv
XXXIX 471- 498). Als es um 1650 an der Kirche zu Ruhmannsfelden ohne jede
Genehmigung angebracht worden war, erhob dagegen P. Gerard Abt bei dem Kloster
Gotteszell Protest: „Die Ruedtmansfelder haben mit ihrer Unvernuenfftiegen
Rueben auff dem hhüß on Verstandt, ganz Vermeßlich und das Closter gehandlet …
Wer hat ihnen ainmall ain wappen zue fueren erlaubt? Vnd wan sie gleich
Wappenmessig waren, Wer hat ihnen erlaubt solches auf der Kirchen spesa auff
ainem offnen thurm mallen zuelassen, alwo ihnen Ainiges Recht und herschaft nit
zuestehet und gebiertt? Vnd was ist das für ain Verstandt ia phantastische
Einbildund ain Rueben auf einem hhüss?“ Schade, dass eine Fortsetzung der
Korrespondenz über das Wappen und seine Anbringung am Kirchenturm nicht
vorhanden ist. Über das andere Wappen, das einen im Felde ruhenden Mann
dargestellt soll haben, kann kein Aufschluss erholt werden.

Das heutige Wappen des Marktes Ruhmannsfelden weist „in Rot ein von Silber und
blau in 2 Reihen geweckten Schrägrechtsbalken" auf. Zu dieser von 0. Hupp (Die
Wappen und Siegel der deutschen Städte, Flecken und Dörfer, Frankfurt a. M.
1912, S.84) gegebenen Beschreibung fügt der Verfasser noch erläuternd hinzu: „Es
ist gar kein altes Siegel bekannt geworden, sodass es fraglich ist, ob das
beschriebene Wappen, das die Bürgermeistermedaille und ein nach dieser
gefertigtes Magistratssiegel zeigen, das ist, das Ruhmannsfelden früher geführt
hat." Die weiß-blauen Rauten im Schrägrechtsbalken sagen uns, dass das Wappen,
den Ruhmannsfeldnern von einem bayerischen Herzog (weiß-blau) verliehen wurde
für besondere Tapferkeit auf blutgetränktem Schlachtfelde (roter Untergrund im
Wappen.)

.4 Schloss und Schlossberg Ruhmannsfelden

In frühesten Zeiten waren die Höhen hiesiger Gegend mit Urwald bedeckt und die
Täler mit Moosen, Sümpfen und Seen ausgefüllt, Avaren drangen durch dieses
unwegsame Gebiet aus Böhmen heraus bis an die Donau vor, um die Klöster auf dem
linksseitigen Donauufer auszurauben. Die bayerischen Volksherzöge trieben diese
Avaren wiederholt zurück. Erst Karl d. Gr. trat diesem Räubergesindel wirksamer
entgegen. Zugleich schenkte er das ganze Gebiet und Ruhmannsfelden herum
(Grenzline Kohlbach, Voglsang, Köckersried, Eckersberg, Unterauerkiel,
Asbachmündung, Altnußberg, Seigersdorf, Fernsdorf, Frankenried, Hornberg,
Einweging, Schusterstein, Ödwies, Hirschenstein, Kalteck, Voglsang) dem Kloster
Metten. Allerdings war damit dem Kloster Metten eine riesige Arbeitslast
aufgebürdet (Rodungsarbeit) und die Christianisierung der Bewohner die
Urwaldgebietes mag keine leichte gewesen sein. Unverdrossene Möncharbeit des
Rodungsklosters Metten hat aber trotz aller Hindernisse hier in unserem
Heimatgebiet, damals "Nordgau" genannt, die Möglichkeit zur ersten Besiedlung
dieses Gebietes geschaffen. Da plötzlich kommt der Bayernherzog Arnulf, mit dem
Beinamen "der Böse" und nimmt das ganze, dem Kloster Metten gehörige Gebiet,
diesem ab (Säkularisation) und schenkt es seinem Getreuen, dem Grafen von Bogen.
Graf Aswin, ein Sohn des Grafen Hartwig von Bogen und dessen Gemahlin Bertha,
eine Tochter des Ungarnkönigs Bela I., erbte nach dem Tode seines Vaters die
Güter im Nordgau, also auch das ganze Gebiet um Ruhmannsfelden herum. Es wurden
in dieser Zeit auf den Höhen Burgen gebaut, die den Grafen (Schirmvögten,
Adeligen, Edelleuten) als Wohnsitz dienten. Unten in den Tälern wurden auch
Burgen gebaut, abseits vom Wege, ganz versteckt, von Weihern umgeben, die den
Dienstmannen der Grafen als Wohnsitz dienten und die als Verbindungslinie
zwischen den einzelnen den Höhenburgen zu denken sind. (Nussberg = Linden,
Ruhmannsfelden = Weißenstein). Bis jetzt hat man geschrieben und gesprochen, es
sei auf der sogenannten Leithanhöhe bei Ruhmannsfelden ein Schloss gestanden und
man bezeichnet immer noch den Berg als Schlossberg, ja man glaubt mit
Bestimmtheit die Stelle zu wissen, wo einstens das vermeintliche Schloss
geständen sei. Es sind aber nicht die mindesten Anhaltspunkte vorhanden,
annehmen zu müssen, dass hier in Ruhmannsfelden eine Burg, ein Schloss, auf
genannter Höhe gestanden habe. Wir lesen, dass die etlichen Ritter, die hier
ihren Wohnsitz hatten, Dienstmannen, Ministeriale der Grafen von Bogen waren und
diesen in Allem unterstanden. Diese Ministerialen werden nicht freie Auswahl
gehabt haben, sie werden vielmehr ihren Wohnsitz in ihr Arbeitsgebiet verlegt
haben und die Bauart ihres Wohnsitzes wird sich von der des Wohnsitzes des
Grafen, des Adeligen, wesentlich unterschieden haben. Die Sippe, aus der der
Ministeriale stammte, oder der er ihr vorstand, wird ihre Wohnungen in
unmittelbarer Nähe der Wohnung ihres Ministerialen gehabt haben. Wenn das
richtig ist, was Aichinger, Kiendl schreiben, "daß um die Burg Ruhmannsfelden
herum eine kleine Ortschaft entstanden ist", so können wir unmöglich weiterhin
auf der sogenannten Leithenhöhe bei Ruhmannsfelden ein Schloss suchen und finden
wollen, wo niemals eines gestanden. Vielmehr müssen wir aus der baulichen
Entwicklung des Ortes Ruhmannsfelden schließen auf den früheren Standort der
Burg. Die ganze Anlage des Ortes Ruhmannsfelden ist augenfällig nicht von oben
(Leithenhöhe) nach unten, sondern von unten (Bachgasse) nach oben entstanden.
Die tadellos erkennbare Vierecksanlage des Bachgassen-Viertels weist uns
augenscheinlich darauf hin, dass das einstens "die kleine Ortschaft war, die um
die Burg herum entstanden ist." (Aichinger, Kiendl). Die ältesten Gewerbe -
Schmid, Wagner, Gerber, Stricker, Seifensieder usw. - und die ältesten Häuser
finden wir in diesem Teil des heutigen Marktes Ruhmannsfelden. Also müssen wir
doch da, wo die ersten Spuren der baulichen Entwicklung des Ortes und die
Anfänge des gesellschaftlichen und geschäftlichen Lebens und Treibens der ersten
Ansiedler unseres Heimatortes zu suchen sind, auch den Wohnsitz des
Ministerialen des Grafen von Bogen die Burg- und die Häuseranlage der Sippe um
die Burg herum unmöglich auf der Leithenhöhe, sondern im Bachgassen-Viertel,
suchen. Irregeführt hat man wiederholt auf der Leithenhöhe Nachforschungen nach
dem Schlosse angestellt. Besonders Eifrige haben sogar mit Pickel und Schaufel
Grabungen vorgenommen. Was dabei gefunden wurde, das waren einige verroste
Säbeln, Hufeisen, Geldstücke, Sachen, die während der Kriegsjahre späterer
Jahrhunderte versteckt wurden und die an anderen Plätzen auch zu finden sind. Es
ist weder eine bauliche Anlage eines früheren Schlossen zu erkennen, noch sind
sonstige Anhaltspunkte zu finden, die ein ehemaliges Vorhandensein eines
Schlosses auf der vermeintlichen Stelle bestätigen könnten. Der tiefe Brunnen,
der unterirdische Gang, der Schlosskeller mit dem uralten Wein, die eiserne
Kiste voll Gold usw. das gehört alles in das Reich der Sage. Sogar im
Gemeindekataster finden wir auf Grund irrtümlicher Angaben (1835-43) den Eintrag
"am Haus" an der Stelle, an der der Eintrag gar keine Berechtigung hat. Die
Heimatforschung hat sich niemals mit der Frage: Gab’s ein Schloss Ruhmannsfelden
und wo stand dieses? ernstlich beschäftigt, sonst hätte man schon längst darauf
kommen müssen, dass die Benennungen „Schloss Ruhmannsfelden, Schlossberg" keine
Berechtigung haben, dass die Burg Ruhmannsfelden im heutigen Bachgassen-Viertel
stand und sich die Ortschaft Ruhmannsfelden von hier aus entwickelt hat.

.II Geschichtliches vom Markt Ruhmannsfelden,

„Gäu und Wald“, Beilage zum Deggendorfer Donauboten, 20.8.1927, Chronik der
Grundschule Ruhmannsfelden

.1 Von der Schule in Ruhmannsfelden in früherer Zeit bis 1835*i

Märkte und Städte haben sich im 13. und 14. Jahrhundert durch Gewerbefleiß und
Handel mächtig empor geschwungen und diese Regsamkeit in damaliger Zeit hatte
Wohlhabenheit zu Folge und die Überlegenheit im Geschäfte und im Wohlstande des
Einzelnen vor dem anderen macht auch geistige Überlegenheit zur Notwendigkeit.
Das Lernen wuchs aus dem wirtschaftlichen Aufschwunge der damaligen von sich
selbst heraus und macht ohne weiteres das Bedürfnis nach Schulen geltend. Solche
Schulen gab es zunächst nur in den Städten. Auf dem Land gab es nur den
wandernden Volksschullehrer. Ob ein solcher sich hier oder in nächster Umgebung
aufgehalten hat, wissen wir nicht.

.)a Die ersten Schullehrer in Ruhmannsfelden*

Bei Aichinger „Kloster Metten und seine Umgebung“ lesen wir Seite 328, dass 1503
Ruhmannsfelden in den Besitz des Klosters Gotteszell kam und dass von der
incorpierten Pfarrei Geiersthal ein Expositus nach Ruhmannsfelden geschickt
wurde. Dieser Expositus hat hier wohl auch den Kindern den Religionsunterricht
erteilt. 1558/59 fanden auf Anordnung Herzog Albrecht V. überall im Lande
Schulvisitationen statt. Aus Trellingers „Beiträge zur Geschichte des
Schulwesens im Bezirke“, Bayerwald 1925, Seite 105, entnehmen wir, dass um jene
Zeit in Ruhmannsfelden schon eine Schule bestanden hat. Aus den
Visitationsprotokollen schreibt Trellinger in genannter Abhandlung über Schule
Ruhmannsfelden folgendes: „Pfarr Geirstall. Die von Ruebfelden (Ruhmannsfelden)
haben einen Schulmeister, den nennen sie für sich selbst auf. Ist ain
Niernberger; hab (der Pfarrer) mit Im nichts zu thun.--  Zu Ruebenfelden (ist)
ain vacierende meß, von dem einkommen wird der Schulmeister besöldt.“ Die
Zechpröbste von Sankt Lorenz in „Ramffelden Markht, Filialgen Geirstall“ geben
an: „Vaciert ein Meß, welche einkommen bei 4 Jarlang 18 fl. davon hatlten sy ain
Schuelmeister, geben ime jährlich 4 Pfund Regensburger. Schuelmeister zu
Ramanffelden, Casparn Stralnberger von Niernberg pürtig, 6 Jahre allda gewest.
Zu Leipzig studiert, hat mit testimonium. Hat 10 schueler, darunter je zwen, so
gute Ingenia haten, lernen gemeinlich erst lesen. Singt zu Chor. Ist der alten
Religion. Underweist der knaben zur Peicht und Empfahung des Sakraments
catholice, desgleichen zu andern Sacrament, Gottesdienst, Predigt. Hat kein
Superattendenten. Besoldung hat er aus der Bruderschaft 4 Pfund Regensburger. Am
Rath daselbst Ime aufzunemen. Petten alle Morgen das Vaterunser, den grues und
glauben. Singen keinen neuen rueff. Die Knaben seien eines geringen Verstandes,
derhalben sy kain comedi oder declamation halten“

In einer Urkunde von 1550 kommt als Zeuge vor ein Andrä Weißpeckh, Schulmeister
in Ruhmannsfelden und auch im 17. und 18. Jahrhundert berichten Urkunden von
Schulmeistern in Ruhmannsfelden, die zugleich auch Mesner waren. Diese
Schulmeister hatten meist selbst ein Besitztum und unterrichteten die wenigen
Kinder die freiwillig lesen, schreiben und rechen lernen wollten in ihrer
Behausung. So lesen wir in einer Urkunde: „Unserm 5.11.1658 verkauften Georg
Pitter, Bürger und gewester Schulmeister zu Ruhmannsfelden und dessen Frau Eva
ihre Leibgedingsgerechtigkeit auf einem Lehen zu Ruhmannsfelden mit
Bräugerechtigkeit dem Hans Eybeck, Bürger und Metzger daselbst und seiner Frau
Margareta um 365 fl.“ii

Am 5.7.1702 übergibt Rosina Hinderholzer, verwitwete Schulmeisterin zu
Ruhmannsfelden ihre Markbehausung am Kalteck an ihren Tochtermann Martin
Staudenberger, Bürger und Schneider in Ruhmannsfelden.iii

Am 26. April 1779 starb hier Schullehrer Bernhard Hochreiter kunstvoller
Orgelspieler, 52 Jahre alt.iv

1784 wirkte hier ein Schullehrer Adalbert Hermannv und geprüfter Eremit Franz
Pitsch bei Gotteszell als Adstant. Eremit Pitsch hatte auf dem Kalvarienberge in
Gotteszell ein kleines Häuschen und in diesem erteilte er Unterricht an die
Gotteszeller und Ruhmannsfeldner Jugend. Da aber im Laufe der Zeit den
Ruhmannsfeldner Kindern der Weg nach Gotteszell, zumal bei schlechtem Wetter, zu
beschwerlich war, ging der Eremit Pitsch nach Ruhmannsfelden und erteilte hier
mit Hermann den Unterricht. Pitsch war noch 1776 Schulhalter zu Walchsee,
Gericht Kufstein. Nach seinem Tode übernahm der damalige Abt Amadäus selbst im
Gotteszell die Kinder unterrichtet.vi

Handschriftliche Aufzeichnungen im hiesigen gemeindlichen Archiv sagen
folgendes: „bey Aufhebung des Klosters Gotteszell hat man die Wohnung des
zeitlichen Pfarr-Vikars, als auch des Schullehrers und Messners nicht für
zweckmäßig befunden. So kam man auf den unglücklichen Gedanken beyde Wohngebäude
zu verkaufen und dafür ein einziges, zwar größers, aber auch schon sehr
baufälliges Haus anzukaufen und darin den Pfarrer samt Kooperator, Lehrer,
Gehilfen und 200 Schulkinder und den Messner, - im buchstäblichen Sinne des
Wortes – zusammen zu pressen.“

Laut landesgerichtlichen Protokolle vom 27. Juni 1803 wurden die Gebäude von Max
Freiherrn von Antritzky gegen das alte Schulhaus und das Pfarr-Vikariatshäusl
und eine bare Daraufgabe von 1 800 fl erworben und hierauf im Jahre 1804 zur
Pfarr-, Schullehrers- und Messnerwohnung, dann zum Schullokale hergestellt. Der
für die Schullehrers- und Mesnerwohnung, dann Schullokal bestimmte Anteil
infolge Entschließung das Staatsministerium des Innern vom 10. April 1832 um 500
fl. erworben.

Die übrigen Realitäten sind bei der Säkularisation des Klosters Gotteszell
infolge der Verordnung vom 9. September 1803 an den Staat als Eigentümer
übergegangen und bei der Organisation der Pfarrei Ruhmannsfelden als
Pfarr-Widdums Realitäten bestimmt worden. (1803 war Ruhmannsfelden noch keine
Pfarrei. Pfarrprovisor war vom 21. März 1803 bis 1. Oktober 1905 Franz Joseph
Haindl. Für einen Hilfspriester hatte er keine Wohnung. Infolgedessen halten die
Patres von Gotteszell aus und zwar: Hr. P. Stivard Sartor 21. April 1803 bis
Ende Oktober 1803. Hr. P. Maria Triendorfer vom November 1803 bis Ende Februar
1804. Hr. P. Guido Berger März 1804 bis Ende September 1805. Haindl musste diese
Priesteraushilfen selbst bezahlen.)vii

Das sogenannte alte Schulhaus (früher zum Freiherrn von Antritzky´schen Anwesen
gehörig) wurde nach dem Tausch zum 27. Juni 1803 von der Berg´schen Familie
erworben. Heinrich Berger, Bierbräu von Ruhmannsfelden, der dieses alte
Schulhaus schenkungsweise von seiner Mutter an sich gebracht hat, verkaufte
dieses alte Schulhaus an Georg Wurzer, Bauerssohn von March, gelernter
Bräubursch nach Zeugnis des Patrimonial-Gerichts.

.)b Die Schule unter Lehrer Andreas Stern*

1804 erscheint in den Akten der noch vom Kloster Gotteszell aufgestellte
Schullehrer Andreas Stern, ein „sehr würdiger Schullehrer“viii, wie es 1810
heißt. Anläßlich des Verkaufes des sogenannten Schulhauses gab es später einen
öfteren Schriftwechsel mit dem Rentenamte Viechtach betreffs Andrä Stern. Eine
diesbezügliche Urkunde sagt:

„Vom Königlichen Rentamte Viechtach:

Andrä Stern, Schullehrer von Ruhmannsfelden hat im Jahre 1804 bay. Gelegenheit
des Verkaufs vom alten Pfarrhof und Schulhaus für seine eigentümlich auf
Erbrecht besessen Stadl und Garten als Entschädigung hiefür das Mutternackerl
Pl.-Nr. 952 und das zwaymädige Weiher- oder Pfarrweisel Pl.-Nr. 447 erhalten und
solche bis zum Jahre 1833 in Privat Besitz gehabt.

Die Schulstiftung wird hiemit aufgefordert, herkommen zu lassen, auf welche Art
und gemäß welchen Vertrag selbe die der vormaligen Kloster Gotteszell
erbrechtsweise grundbaren Objekte erhalten hat und warum zu dessen Ankunfts Titl
der grundherrliche Konsens nicht erholt wurde, da doch das Kloster Gotteszell
und nun der Staat auf den ganzen Unfang der Burgfriedens von Ruhmannsfelden die
Grundbarkeit zu gaudieren hatt.“

Im Jahre 1804 wurden die Schullokale im neu erworbenen Schulhaus neu
eingerichtet. Laut Ausweis der diesbezüglichen Rechnungen, die von Amadäus
Bauer, Abt und kurfürstlicher Schulinspektor in Gotteszell unterzeichnet und
gesiegelt sind, wurden 5 Spundladen, 8 mittlere Bretter, 7 dünne Bretter, vom
Hafner Lorenz Plötz ein kleiner Ofen, 60 kleine Tintenhaferl, von Johann
Reisinger, Schreinerix, 10 Bänke für Kinder, 2 Tafel zum Zählen, eine Kanzel mit
einer Schublade, ein kleines Schamerl geliefert und vom Wolfgang Geiger,
bürgerlicher Glaser- und Zimmerarbeiter-Meister 6 Winterfenster eingeglast.

Dass es zu damaliger Zeit schon Schulprüfungen gab und dass bei diesen Prüfungen
an die fleißigsten Schüler auch Preise ausgeteilt wurden, besagt eine Urkunde im
gemeindlichen Archiv mit folgenden Worten: „Daß ich die zum Beytrag für die bey
der Schulprüfung der Kinder zu Ruhmannsfelden zu verteilenden Preise gnädigst
bewilligten 12 fl. von dem Bruderschaftsverwalter Baumann richtig erhalten habe,
wird hiermit bescheinigt.

 R u h m a n n s f e l d e n, den 26. Sept. 1804

    Frans Joseph Haindl Pfarrprovisor“.x

Mitteilung vom Bayerischen Staatsarchiv Landshut

1812 wird Andrä Stern ein Adjunkt beigegeben, den er selbst bezahlen soll. Er
ist darüber sehr aufgebracht und anscheinend zog man den Plan wieder zurück.
1814 ist Stern an Lungenentzündung schwer krank, der Adjunkt Georg Lippl von
Böbrach wird zur Aushilfe herbeigerufen. Lehrer und Gehilfe vertragen sich
schlecht. Letzterer muss mit der Stallmagd aus einer Schüssel essen.

In dieser Zeit (1814) war die Industrie-Lehrerin Renate Wuzelhofer und die
Stricklehrerin Magdalena Schweigerin, Schlossermeisterin von hier, an der
Schule.xi

1816 wird ein neuer Gehilfe, Schuldienst-Expektant Alois Rokinger, von
Unterviechtach nach Ruhmannsfelden versetzt. Er ist aber nicht auffindbar. Dafür
kommt der Schulpräparand Andreas Dreseli aus Passau. Sein Nachfolger ist 1820
der Expektant Josef Krieger von Deggendorf, dann ein gewisser Pichlmeier, 1823
ein Thaddäus Esterl, bisher in Langdorf, 1825 folgt Johann Nepomuk Wänninger,
Aushilfslehrer von Langdorf.

1829 finden wir Stern noch als Lehrer in Ruhmannsfelden. In der
Volksschulbeschreibung von 1822 heißt es bei Ruhmannsfelden: „Andreas Stern,
geb. 1766, angestellt 1791, Befähigung: I. – Einkommen: 6533 fl. 56 kr.“xii

.)c Das Schul- und Mesnerhaus*

1819/20: Die am hiesigen Schul- und Mesnerhause notwendigen Baufälle mussten
unumgänglich gewendet werden. Nach der hier in Abschrift vorliegenden
Genehmigung des königlichen Landgerichts Viechtach vom 2. Juni 1820 sind hierauf
59 fl. 42 kr. bestimmt worden. Die Hälfte hiervon ist nach obiger Genehmigung
von den Schulgemeinden zu bestreiten. Von der letzten Hälfte hat die
Marktgemeinde wieder die Halbscheide und die Gemeinde Zachenberg und Patersdorf
auch die Halbscheide zu bezahlen. Da nun diese Baufälle für 1819/20 Kosten von
32 fl. 23 kr. erforderten, worüber die Bescheinigungen der Kirchen Rechnung für
1819/20 angelegt werden, so hat hierzu die Marktkasse den 4. Teilkostenbeitrag
bezahlt mit 8 fl. 5 kr.

Wie es nun in diesem neu angekauften, aus schon sehr baufälligen Pfarr-, Schul-
und Mesnerhaus ausgesehen haben mag, besagt uns eine Mauerer Kostenrechnung vom
22. Mai 1830:xiii

„Bemerkungen.

 Das Mesnerhaus befindet sich im Pfarrhofe linker Hand zu ebener Erde und
 besteht:

1. In einem Vorschlage für eine Dienstmagd

2. In der Wohnstube und Schlafkammer des Mesners

3. In einem Nebenzimmer von welchen man in die Speise kömmt.

4. Unter den letzteren Breden befindet sich der Keller des Meßners.

5. Rechter Hand am Ende der Schlafkammer des Schulgehilfen (das Lehrerzimmer
kömmt zuvor) ist endlich die Küche des Mesners.

6. Der Getreidekoben des Mesners, welcher sich oberhalb der Baustube des
Pfarrers befindet.

7. Die Stallung, nebst einem anstoßenden Schweinestall.

8. Die Scheune  zur Hälfe bis an den Dreschtennd. Daß der Messner auf eben
dieser Tenne sein Getraid auszutreschen berechtig dey, wird nich beweifelt
werden.

9. Die Holzschupfe, welche an die Scheune des Pfarrers anstoßt.

10. Der Abtritt, welcher an jenen der Schule anstoßt.

11. Die Dungställe nächst jener des Pfarrers ist.

12. Das Wasch– und Backhaus, welches für beide gemeinschaftlich gehört.

13. Das kleine Baumgärtlein, welches an die Scheune des Pfarrers anstoßt und mit
einer Blanke versehen ist..“xiv



Die Zustände in dem Pfarr-, Schul- und Mesnerhaus veranlassten die Schulgemeinde
an die Erbauung eines eigenen Schul- und Mesnerhauses heranzutreten. Im Jahre
1828 besichtigten Vertreter der Schulgemeinde das Schulhaus und kamen am 2. Juli
des gleichen Jahresxv zu den Beschluss, dass „nachdem die Schulgemeinde
Ruhmannsfelden vor ung. 20 Jahren das damalige Pfarr- und Schulhaus aus eigenen
Mitteln gekauft hat um als Schul- und Mesnerhaus verwendet zu werden, welches
einen Kostenaufwand von ung. 2500 fl. verursacht und das Staats Aerar hiezu
einen verhältnismäßigen Beitrag leistet, weil das nämliche Gebäude zugleich zur
Pfarrerwohnung bestimmt war, sich das Staatsärar gefallen lasse 1800 fl. zur
Erbauung eines neuen Schul- und Mesnerhauses beizutragen, wogegen das bisherige
Schulhaus die Pfarrerwohnung allein werden würde, folglich dieses Gebäude
ungeteilt dem Staats-Aerar anheim fiel. Wenn nun obiger Beitrag von Seite des
Kirche der Bruderschaft und des Staates geleistet sein werde, so verpflichtet
sich die Schulgemeinde nebst der Herlassung des zum Schulhaus erforderlichen
Platzes, das Defizit der Baulasten zu decken. Zur Bestimmung des
Beitragsquantums nach Abzug des beantragten Aearialbeitrags zu 1800 fl. soll
festgesetzt werden, was im Namen der Kirchen die genannte Bruderschaft, die
Hälfte des Defizits und die Schulgemeinde die 2. Hälfte zu tragen habe.“

Zunächst setzte der Streit ein, welcher Platz für das zu erbauende Schul- und
Mesnerhaus der geeignetste wäre. Der Erfolg dieser langwierigen
Auseinandersetzungen war, dass man auf dem Gedanken kam, das alte Gebäude zu
belassen und ein neues Pfarrhaus zu bauen mit dem zur Erbauung des Schulhauses
treffenden Bau Anteil zu 1650 fl. Laut Protokoll vom 7. August 1830 haben sich
gleichzeitig die Vertreter der Schulgemeinde erheischig gemacht, bei dem neuen
Pfarrhausbau samt nötigen Nebengebäuden die erforderlichen Hand- und Spannführen
zu leisten. Rentamt, Lokalschulinspektor und sämtliche Beigezogene waren mit
diesen Antrag einverstanden. Mauermeister Achatz gab die Erklärung ab, dass er
das in Frage stehende Gebäude für ein Pfarr- und zugleich Schulhaus keineswegs
geeignet finde, doch zum Zwecke einer Wohnung für den Schullehrer und zugleich
Mesner samt Schulgehilfen und für die erforderlichen Lehrzimmer geeigenschaftet
finde. Maurermeister Achatz hat lautxvi Protokoll vom 3. Juni 1831 eine
Erklärung abgegeben, wonach das bisherige Pfarr- und Schulhaus um die Summe von
ungefähr 400 fl. zu alleinigen Zweck eines künftigen Schulhauses hergerichtet
werden könne. Bei näherer Prüfung hat sich aber gezeigt, dass die Kosten auf
eine bei weitem höhere Summe sich belaufen würden. So konnte die Schulgemeinde
das angebotene Gebäude als künftiges Schulhaus nicht übernehmen, weil ihr die
Erbauung eines neuen Schulhauses auch nicht höher zu stehen gekommen wäre. Die
langwierigen Verhandlungen haben, nachdem der Gemeindeanteil an dem
gemeinschaftlichen Pfarr und Schulhause um 500 fl. an den Statt verkauft war,
ergeben, dass ein neues Gebäude als Schul- und Mesnerhaus ausgeführtxvii werde.

„Es ist im Jahre 1834 aus Bruch- und Ziegelsteinen gebaut worden, liegt im
südlichen Endes des Marktes, hart an der nach Gotteszell-Deggendorf führenden
Straße und hat zwei Stockwerke. Im ersten, das ist zu ebener Erde, befindet sich
die Wohnung des Schullehrers und des Schulgehilfen, ein Gewölbe vertritt die
Stelle des Kellers, weil ein solcher wegen zu feuchter Grundlage nicht erbaut
werden konnte.

Im zweiten Stockwerke befinden sich die zwei Schulzimmer und zuvor eines für
Vorbereitung- und I-, dann eines für II. und III. Klasse. Der Dachboden
ordentlich gelegt und das Dach mit Schneidschindeln gedeckt. Die neben dem
Schulhaus in südlicher Richtung stehenden Ökonomiegebäude sind gemauert und
befinden sich im gutbaulichen Zustande, was aber von der am Stalle stehenden
Holzlege ebenso wenig wie vom Schulhause gesagt werden kann. (Beschreibung vom
12.5.1866 von Raymund Schinagl, Schullehrer.)

Mit der Erbauung des neuen Schulhauses und seiner Einweihung und Eröffnung 1835
beginnt eine neue Zeitepoche in der aufwärts sehr reifenden Entwicklung des
Schulwesens im Markte Ruhmannsfelden.

Am 26. Oktober 1830 starb Andreas Stern, Elementarlehrer in Ruhmannsfelden, 63
Jahre und 11 Monate alt. Bemerkung in der Pfarrmatrikel „Dieser würdige Mann war
gegen 43 Jahre lang in Ruhmannsfelden Lehrer und im ganzen 49 Jahre beim
Lehrfach.“

.2 Die Schule ab 1835xviii

Dem verstorbenen Lehrer Andreas Stern folgte als Schullehrer in Ruhmannsfelden
Georg Lippl. (Aushilfslehrer: Anton Ruhstand)xix Der verstorbene H. Hr.
Kooperator Mayer macht zur Schulstiftung 1841 ein Legat von 100 fl. Anna
Roßhaupt vermachte der Schule ein Legat von 200 fl. Außer früheren Reparaturen
mussten im Jahre 1860/61 eine große Baureparatur vorgenommen werden, wozu
Beiträge geleistet werden mussten: von der Gemeinde Patersdorf 47 fl., von der
Gemeinde Zachenberg 107 fl., von der Gemeinde Ruhmannsfelden 175 fl. Im März
1866xx erkrankte der Schullehrer Georg Lippl und starb im April 1866. 1866
wirkten an der Schule: Raimund Schinagl, Max Dierigl, Karl Müller. 1869
verehelichte sich Max Dierigl, Schullehrer von Ruhmannsfelden mit Rosa
Moosmüller, Hutmacherstochter von hier.

.)a Die Schule benötigt immer mehr Platz*

Im Schuljahre 1870/71 zählte die Vorbereitungsklasse 104 Schulkinder, die 1. und
2. Klasse 75, die 3. Klasse 68 Schulkinder, zusammen 247 Schulkinder. Im Laufe
der Zeit wurden die Schulkinder immer mehr, der Platz in den Schulzimmern immer
beschränkter. In einem Beschluss des Gemeinderates Ruhmannsfelden von 7.
November 1878 betr. Schulverhältnisse in Ruhmannsfelden heißt es: „Nach den
schriftlichen Erklärungen der I. und II. Schullehrer Weig und Rubenbauer von
hier sind die hiesigen ebenen Lehrzimmer für das 3. und 4. mit 5., 6. und 7.
Schuljahr geräumig genug, um die zurzeit erforderlichen Schülerzahl zu fassen.
Es erübrigt sich somit nur der Schulgemeinde, für die Herstellung eines
geräumigen Schulzimmers für den Schulgehilfen (Vorbereitungs- und 1. Klasse)
Sorge zu tragen. Nach der beigelegten Erklärung des 1. Schullehrers Weig lässt
dieser die an das untere Lehrerzimmer anstoßende Gewölbe ab und werden diese mit
dem Lehrzimmer dadurch vereinigt, dass die Zwischenmauern herausgenommen und
somit das untere Lehrzimmer den beiden oberen an Größe gleich gemacht wird.“

Im Schuljahre 1879/80 betrug die Schülerzahl bereits 359. Infolge der
Überfüllung der Schulzimmer musste eine weitere Lehrkraft angestellt werden und
um ein weiters Schullokal umgesehen werden. Zufolge Hohen königlichen
Regierungsentschlussxxi vom 31.8.1879 soll mit Beginn des Schuljahres 1879/80
angestellt und demzufolge bis 1. Oktober laufendes Jahresxxii ein passendes
Schullokal entweder durch Umwandlung der Lehrerwohnung zu einem 4. Lehrerzimmer
oder durch Miete eines geeigneten Saales hergestellt werden. Hierzuxxiii wurde
vom Gemeinderat Ruhmannsfelden am 14.9.1879 folgendes beschossen:



1. „Die Umwandlung der jetzigen Lehrerwohnung in ein Schullokal hänge nicht
allein von der Einwilligung des hiesigen Schulsprengels, sonder auch von der
Pfarrkirchenstiftung Ruhmannsfelden ab, da der Lehrer zugleich Mesner und laut
höchster ministerialer Entschließungxxiv vom 22.6.75 die Pfarrkirche zur
Bestreitung der Miete auch die Hälfte zu zahlen hat.

2. bei der vorgeschrittenen Jahreszeit eine Reparatur nicht mehr möglich ist

3. schlechte finanzielle Lage der Gemeindexxv

4. Es sei aus den angeführten Motiven weder die Umwandlung der Lehrerwohnung in
ein Lehrzimmer, noch der Bau eines 2. Schulhauses, doch der Aufbau eines II.
Stocks für 2 Lehrzimmer nach nochmaliger technischer Prüfung durchführbar.“



Laut Beschluss des Gemeinderatesxxvi Ruhmannsfelden betreffendxxvii
Schulverhältnisse in Ruhmannsfelden von 16.10.1875 ist:



1. „Für die aufzustellende 4. Lehrkraft an der hiesigen Schule in unmittelbarer
Nähe des Schulhauses bei dem Schneidermeister Josef Meier hier eine passende
Wohnung gemietet worden und kann dieselbe sofort bezogen werden.

2. Die Unterrichtslokale sowohl bei dem Bierbrauer Kaiser, als Lebzelter
Schreiner von hier sind wegen ihres Einganges durch das Vorhaus als wegen ihrer
nicht zweckentsprechenden Höhe ungeeignet. Der Verkehr mit den Schnapstrinkern
und dergleichenxxviii besoffenen Gästen würde für die Schuljugend ein anstößiges
Beispiel geben.

3. Nachdem die hiesige Schule ohnehin z. Z. weinig Kinder über 300 zählt, so
wäre es zweckentsprechend, wenn vorderhand die hieher anmittierte Lehrkraft
abwechselnd mit der Lehrerin Frl. Elise Fischer Schule halten würde, welcher
Vorschlag auch bereits von dem k. Kreisschulreferenten H. Müller angenommen
wurde.

4. Mit kommendem Jahre wird der Neubau eines 2. Stockes auf das Schulhaus
vollzogen, nachdem an den Bezirksbautechniker die technische Prüfung über die
Tragfähigkeit vorzunehmen das Ersuchen gestellt wird.“



Am 31.3.1880 wurde folgender Gemeindebeschluss gefasst: „In anbetracht der
allgemeinen Stockung jeder gewerklichen Geschäftes u .Verkehr u. der schlechten
u. ungünstigen Erwerbsverhältnisse wäre der Ruin d. Gemeinde in Aussicht
gestellt, deshalb sieht man sich zu der unterlässigen Bitte berechtigt, es möge
der Aufbau eines Stockwerkes auf das derz. Schulhaus vorerst u. bis zum Eintritt
günstigerer Erwerbsverhältnisse sistiert werden“

Am 8. August 1880 wurde folgender Beschluss gefasst: „In anbetracht der hier
bestehenden großen Schülerzahl u. der bereits schon lange bestehenden 4.
Lehrkraft, welche z. Z. ohne Lehrerzimmer ist wurde beschlossen: Es sei bis zu
Beginn der Schulanfangs 1. Okt. d. J. ein 4. Lehrerzimmer für ca. 50 - 60
Schulkinder zu mieten u. dieses mit den nötigen Schuleinrichtungsgegenständen zu
versehen. Hierüber ist d. k. Regierung gehorsamst Mitteilung zu machen u. diese
zu bitten, die bisherigen Schulverhältnisse durch Versetzung der II: Lehrerin zu
ändern, um endlich einmal wieder friedlichere Verhältnisse zu erlangen. In Frage
kam das bei dem Hutmacher Rosenlehner in Aussicht genommene Zimmer. Am 9. August
1880 wurde mit Hutmacher Rosenlehner der diesbez. Mietvertrag geschlossen.

Am 5. Januar 1881 wurde folgendes beschlossen: „In anbetracht, daß in der
Gemeinde Zachenberg Unterschriften von Haus zu Haus gesammelt wurden, welche den
Neubau eines Schulhauses nach Zachenberg od. Auerbach verlangen u. die beiden
Projekte in den nächsten Tagen der k. Distr.-Schul. Insp. u. dem k. Bezirksamte
vorgelegt werden, ist die Schulsprengelverwaltung außer Stande gesetzt worden,
einem defin. Beschluß durch Adoptierung eines 4. Schulzimmers zu fassen.“ Am 4.
August 1881 wurde folgender Beschluss gefasst:



1. „Es sei das bisherige Schulhaus in der Weise zu erweitern, dass die bisherige
Wohnung des 1. Lehrers u. Mesners zu einem 4. Schulzimmer verwendet werde.

2. Als Wohnung d. 1. Lehrers u. Mesners sei an der Südseite des Schulhauses im
Neuanbau aufgeführt, wo auch dem Schulgehilfen eine Wohnung angewiesen wird.“
(Zur Ausführung kam es nicht)

.)b Bau eines 2. Schulhauses*

Am 18. März 1883 wurde beschlossen: „Es sie an d. k. Regierung untertän. die
gehorsamste Bitte zu stellen, daß im heurigen Jahres die Verakkrierung des Baues
u. die Erd- u. Fundamentbauten für das neue Schulhaus noch vorgenommen u. der
übrige Bau im Jahre 1884 vollendet werden dürfe.“

Am 2. September 1883 beschloss die Schulsprengelverwaltung betreffend Neubau des
Mächenschulhauses: „Es sei dem Gesuche d. Bauakkordanten: Johann Aubinger, Josef
Amberger u. Benedikt Ebner die Genehmigung zu erteilen, mit dem Grundbau des
Mädchenschulhauses erst im künftigen Frühjahr beginnen zu dürfen. Zur
Bauaufsicht beim Mädchenschulhausbau wurden ernannt: Anton Bielmeier, Schreiner,
Ruhmannsfelden, Engelbert Treml, Ökonom, Joseph Brandl, Gütler, Zachenberg.“

Die Schuldaufnahme zum Bau des Mädchenschulhauses betrug 20 960 M u. zwar: 12
825 M Kapitalien, 8 000 M Kreisfondzuschüsse, 135 M Zuschüsse und Zinsen. Zu den
12 825 DM kam noch 825 M Schuldaufnahme zum Grundankauf von Johann Sagmeister.
Die Kreisfondzuschüsse wurden geleistet von 1883 – 86 in jährlichen Raten von 2
000 M. Außerdem wurden noch Kreisfondszuschüsse geleistet 1888 ein Betrag von 1
000 M und 1889 ein Betrag von 500 M. 1885 besteht über die Gesamtschulden vom
Mägchenschulhaus in Schuldentilgungsplan. Die sämtlichen Kapitalien werden mit 4
% verzinst.



Schuldenstand:



Johann Sagmeister

5 000 M

Jakob Bielmeier

2 500 M

Lorenz Zitzelsberger

1 000 M

Franz Weiß

1 000 M

Lokalkrankenhaus

   750 M

Lokalmolzaufschlagkasse

   400 M



10 650 M



1884 wurde das Mädchenschulhaus mit einem Kostaufwand von 18 000 M gebaut.xxix
Ein Beschluss der Marktgemeindeverwaltung Ruhmannsfelden vom 29.6.92 betreffend
Einstuhlung der armen Schulschwestern in der Mädchenschule von Ruhmannsfelden
besagt: „Es spricht sich die Schulsprengelverwaltung Ruhmannsfelden zu Gunsten
der Einführung der armen Schulschwestern in der Mädchenschule aus und es seinen
die notwendigen Schritte betreffs Genehmigung der Einführung derselben
einzuleiten.“Zur Durchführung dieses Beschlusses kam es aus verschiedenen
Gründen nicht und eine Mädchenschule unter den „Armen Schulschwestern“ wurde
nicht eröffnet. Am 19.11.1896 wurde der Anschluss des Knaben- und
Mädchenschulhause an die Wasserleitung genehmigt. Nach dem Beschlusse von
2.11.1898 sei der Ziehbrunnen des Knabenschulhauses zu überwölben.

.)c Bau des neuen Schulhauses 1907/08

Nachdem das alte Schulhaus (1835) und das Mädchenschulhaus (1884) den
Anforderungen nicht mehr entsprachen, da die Schülerzahlen von Jahr zu Jahr
größer wurden und dadurch auch mehr Lehrkräfte angestellt werden mussten,
stellte die Gemeinde Zachenberg Antrag auf Erbauung eines Schulhauses in
Auerbach. Aufgrund einer vom damaligen Bürgermeister ausgearbeiteten Denkschrift
über die Schulverhältnisse in Ruhmannsfelden wurde der Bau eines großen neuen
Schulhauses im Markt Ruhmannsfelden mit einem Kostenvoranschlag von 83 000 M
genehmigt, der 1907 begonnen und 1908 vollendet wurde. Die Baukosten beliefen
sich auf fast 100 000 M. Nach Eröffnung und Einweihung des neuen Schulhaus
(1908) wurden zu den bereits angestellten Lehrkräften (4) noch weitere 3
Lehrkräfte angestellt, zusammen also 7 Lehrkräfte.

.III Was Ruhmannsfelden für Jubiläen feiern könnte?

Viechtacher Tageblatt, 9.9.1928

Die erste Ansiedlung in hiesiger Gegend erfolgte am Fuße des Nordabhanges vom
Voglsang. Hier hatte das den Agilolfingern gleichberechtigte Grafengeschlecht
der "Drozza" (nach Hr. Geheimrat Dr. Eberl) einen Hof genannt "Droßlach". In der
Nähe davon befanden sich noch zwei Ansiedelungen, nämlich Achslach und Irlach.
Da wir nirgends um Ruhmannsfelden herum Ortsnamen finden, die auf eine
Entstehung der betreffenden Ortschaft vor dem 8. Jahrhundert schließen ließen,
so müssen wir annehmen, dass die Besiedelung unserer Gegend in der Richtung von
Westen nach Osten, von der Donau her über Metten, Achslach, Droßlach erfolgte.
Die hiesige Gegend, damals noch Urwald, kam später in den Besitz Karl des
Großen. Dieser schenkte einen Teil des sogenannten Nordgaues dem Kloster Metten.
Die Grenze dieses Besitztums lief von Metten über den Voglsang nach Köckersried,
Lämmersdorf, Fratersdorf, Heumühle, Frankenried, Hornwald, Schusterstein,
Kalteneck, Metten. Das Kloster Metten war aber nicht allzu lange im Besitze des
genannten Teiles des Nordgaues. Ein Arnulf, der Böse genannt, hat die Klöster
säkularisiert, d. h. er hat ihnen das Besitztum wieder abgenommen und dieses
dann Grafen gegeben. Und so kommt die hiesige Gegend in den Besitz des Grafen
Aswin von Bogen, Während die Mettner Patres ihre Haupttätigkeit im
Christianisierungswerk erblickten, versuchten die weltlichen Machthaber, die
Grafen, das Land so rasch als möglich urbar und dadurch ertragreich zu machen.
Unter der Mettner Herrschaft entstanden die Ortsnamen mit "dorf", Lämmersdorf,
Fratersdorf, Patersdorf. Als dann unter der Grafenherrschaft die Rodung
einsetzte, entstanden die Ortsnamen mit "ried", Köckersried, Giggenried,
Zuckenried, Kaikenried, Perlesried. Die Grafen stellten dann überall
Ministeriale auf, welche, die Rodungsarbeiten zu leiten und zu überwachen hatten
(nach H. Hr. Pater Fink).



Giggenried    =

Cundachar

Göttlesried    =

Cadoal

Kaikenried    =

Hacco

Lobetsried    =

Luitpolf

Perlesried    =

Perolf

Triefenried    =

Trunolf



Auf diese Weise kam auch hierher ein solcher Ministerialer, der sich der sich
hier ansässig machen musste, genannt Hrothimar (sprich Rumar). Er baute sich
eine turmförmige Weiherburg, kein Schloss, denn er war kein Adeliger, sondern
nur Angestellter des Grafen Aswin von Bogen und kurze Zeit darauf entstand dann
um die Burg herum (siehe Aichinger: "Metten und seine Umgebung") eine Ortschaft,
genannt Hrothimarsfeld, Rumarsfeld, das heutige Ruhmannsfelden. Nach Ansicht der
in Frage kommenden Autoritäten war dies um das Jahr 1100 herum, sodass
Ruhmannsfelden auf ein 800-jähriges Bestehen zurückblicken kann und sein
800-jähriges Geburtstagsjubiläum feiern könnte.

Um die Mitte des 14. Jahrhunderts herum herrschte hier in Ruhmannsfelden die
Pest (nach Schmid, München i. J.1340). Es war die indische Beulenpest, die von
Italien nach Bayern und Österreich eingeschleppt wurde. Diese Pest forderte hier
so viele Opfer, dass zwei neue Begräbnisplätze errichtet werden mussten, Grab
und Siechet, um die Toten beerdigen zu können. Wenn nun die Geschichte schreibt,
dass die Pest hier fürchterlich gewütet hat, so besteht sicherlich Veranlassung
genug zu überlegen, ob man nicht diese beiden 600-jährigen Begräbnisstätten von
so vielen von der Pest dahingerafften Ruhmannsfeldnern als solche auf irgend
eine Weise würdigen könnte und sei es nur eine kurze Inschrift in Stein. Wie
ehren damit die Toten, die vor 600 Jahren dort ihre Ruhestätte fanden und ehren
uns damit selbst. Denn es macht immer einen guten Eindruck, wenn wohl gepflegt
der heimatgeschichtlichen Begebenheiten eines Ortes auch offensichtlich in
Inschriften, auf Tafeln, in Säulen, Denkmälern etc. gedacht wird.

In der letzten Zeit der Aldersbacher Regierung dahier (siehe Aichinger) hat sich
Ruhmannsfelden zum Markt emporgeschwungen. Am 26. April 1416 stellte Jakob der
Rueerer eine Urkunde aus, in welcher er sich "dyczeit Richter dez Markehtz zue
Ruedmansfelden" nennt. Es dürfte kaum ein Zweifel bestehen, dass der
Urkundenausteller als landesherrlicher Richter über die Markteigenschaft seines
Wirkungsortes Bescheid wusste. Es ist nicht genau festzustellen, wann und von
wem Ruhmannsfelden zum Markt erhoben wurde. Aber man kann behaupten auf Grund
der 1416 ausgestellten Urkunde, dass Ruhmannsfelden bei Beginn des 15.
Jahrhunderts, also am das Jahr 1400 herum, zum Markt erhoben wurde, demnach 500
Jahre Markt ist und sein 500-jähriges Marktjubiläum feiern könnte.

Im Jahre 1574 brannte das Pfarrgotteshaus nieder und es könnte möglich sein,
dass sich die Ruhmannsfeldner dadurch zu helfen wussten, dass sie sich bis zum
Wiederaufbau der Pfarrkirche ein Kapellchen bauten, auf einem ganz versteckten
Platz, damit sie nicht zum Gottesdienst bis nach Gotteszell gehen brauchten.

Vielleicht wäre das in Zusammenhang zu bringen mit der Entstehung des
Osterbrünnls. Wenn das zutreffen würde, dann stünde das Osterbrünnl schon 350
Jahre. Wenn nicht, dann ist aber das Osterbrünnl sicher in der Mitte des 17.
Jahrhunderts entstanden, um 1660, sodass das Osterbrünnl sein 250-jähriges
Bestehen feiern kann.

Früher war es der Brauch, dass man das Holz im Walde verbrannte und die Asche
heimfuhr, da das Holz nur einen ganz geringen Wert hatte und für die Asche, die
an die einheimischen Seifensieder oder an auswärtige Händler verkauft wurde,
mehr Geld einnahm als für das Holz. So kam es auch, dass hier beim Berger Bräu
ein großer Aschenhaufen aufgeschichtet war im Hof. In der Nacht zum 1. Juli 1820
kam ein heftiger Wind, der die Glut im Aschenhaufen anfachte und das
Tannenreisig brennend machte. In wenigen Minuten stand der Stadel vom Berger
Bräu in Flammen. Die halbe Marktgasse von der Bachgasse bis zum oberen Markt
samt der schönen gotischen Laurentius-Pfarrkirche wurde ein Raub der Flammen.
Erst im Jahre 1828 konnte die neu aufgebaute Kirche, fertig gestellt und
eingeweiht werden. Demnach kann die hiesige Pfarrkirche, klassizistisch in ihrem
Innern durch Entstauben jugendfrisch gestaltet, als ein wahres Schmuckkästlein
unter den Kirchen des oberen bayerischen Waldes, ihr 100-jähriges Jubiläum
feiern.

Es könnte noch eine Reihe kleinerer Begebenheiten angeführt werden, die auch in
den Jubiläumskranz eingeflochten werden könnten. Aber die oben angeführten
reichen wohl aus überzeugen zu dürfen, dass es sich für Ruhmannsfelden lohnen
dürfte, auch einmal ein Jubiläumsjahr zu begehen, wie es vor kurzem erst
Osterhofen getan.

.IV Wie hat es um Ruhmannsfelden herum ausgesehen vor seiner Entstehung?

Viechtacher Tagblatt, 25.10.1928xxx

Kaiser Karl der Große beabsichtigte ein großes christliches Weltreich unter der
Herrschaft der Franken zu gründen. Dazu brauchte er erstens Soldaten, um die
damals bekannten Völker unterwerfen zu können, und zweitens die Glaubensboten,
um diese unterworfenen Völker, soweit diese noch heidnisch waren, zu
christianisieren. So lernen wir Kaiser Karl kennen als bedeutenden Kriegsfürsten
mit ausgedehnter Machtstellung, die ihm den Beinamen "der Große" einträgt, und
lernen ihn kennen als besonderen Freund der Kirche, sodass ihm Papst Leo III, am
Weihnachtsabend im Jahre 800 als Zeichen der Dankbarkeit die Kaiserkrone auf das
Haupt setzte.

Karl der Große hatte fast alle damals bekannten Völker unterworfen. Da hörte er
auch von dem frechen Eindringen der Avaten und Slawenxxxi der Donau entlang.
Diese Eindringlinge plünderten die Klöster aus, beraubten und belästigten die
Glaubensboten. Eines Tages kam Karl d. Gr. nach Pfelling unterhalb Bogen. Von
dort führten drei Wege in den Wald herein, ein Heeresweg über Kalteck, Achslach,
Gotteszell, ein Saumweg über Berg, Oberhirschberg, Voglsang, Gotteszell, ein
Prüglweg dem Kohlbach entlang über Datting, Hochbühl, Gotteszell. Da kam
bekanntlich Karl d. Gr. den Saumweg benützend, rau dem Einsiedler Utto in
Uttobrunn (bei Berg). Utto bat den Kaiser um die Erbauung eines festen, massiven
Klosters in Metten. Karl d. Gr. erfüllte die Bitte des wundertätigen Einsiedlers
und ließ 792 das Kloster Metten bauen. Karl d. Gr. zog dann Waldeinwerts mit
seinem Heere gegen die heidnischen Slawen, besiegte sie und nahm von Lande,
Besitz. Gleichzeitig gab er einen Teil dieses Gebietes, Nordwald genannt, dem
Kloster Metten. Dieses Besitztum erstreckte sich von Voglsang (Der Kohlbach war
die Grenze.) nach Köckersried, Lämmersdorf, Fratersdorf (Der Flintsbach war die
Grenze.), Neumühle, Frankenried, Hornwald, Schusterstein, Hirschenstein,
Kalteck, Metten. Dieses Gebiet stand unter kaiserlichem Schutz. Das Kloster
Metten hatte in diesem Gebiet seine eigene Gerichtsbarkeit und Immunität
(Steuerfreiheit). Wenn auch das Kloster Metten durch die Erwerbung dieses großen
Gebietes wirtschaftlich gestärkt war, so wurde ihm durch die Zuweisung dieses
Stückes Land eine große Arbeitsleistung zugemutet. Denn das Land war mit Urwald
Sumpf, See usw. bedeckt und musste erst gerodet urbar gemacht werden. Die
Rodungsarbeit und das Christianisierungswerk waren die beiden Hauptaufgaben des
Klosters Metten in hiesiger Gegend gleich zu Beginn des 9. Jahrhunderts. Die
ersten Menschen, die hierher gekommen sind, waren Slawen, die bis zur Donau
vorgedrungen sind. Die Ortschaft Metten soll eine slawische Ansiedlung gewesen
sein. Dann kamen die Römer, die ihre Vorposten bis an den Regen vorschoben. Wäre
zur damaligen Zeit Ruhmannsfelden schon eine Besiedelung gewesen, so wäre der
Name dieser Besiedelung nicht Ruhmannsfelden, sondern mit einem Eigennamen
verbundener "ing"-Name.

Zu Beginn des 9. Jahrhunderts kamen dann hierher die Benediktiner-Patres vom
Kloster Metten. Damals war schon besiedelt die Kalte Herberge = Kalteck (am
Heeresweg), Achslach und Droßlach (Gotteszell), das im Besitz eines ungarischen
Grafengeschlechtes war und später in den Besitz der Grafen von Pfellingxxxii
kam. Die Klosterherren von Metten haben, von der Grenze nach der Mitte des
Besitztums zu besiedelt. So kam es, dass sie sich hier in Ruhmannsfelden nicht
festgelegt hatten. Vielmehr haben die an der Grenze des Besitztums entlang ihre
Ansiedlungen angelegt und haben neben den bereits vorhandenen älteren
Ansiedlungen (durch Slawen oder Römer) neue Höfe (Villen) angelegt, die sie
durch einen Meier oder durch eine Familie bewirtschaften Iießen. Besonders die
Villen, deren Namen die Zusammensetzung mit Eigennamen zeigen, waren an einzelne
Familien zu Bewirtschaftung gegen Entrichtung bestimmter Abgaben vergeben. An
der Grenze des Besitztums entstanden die Ansiedlungen Metten (bei Regen) =
östliche Ansiedlung, March = Mark oder Moar, Geierstal, Markbuchen. An der
Westseite der Besitzung im Nordwald, also von Voglsang bis zum Regen, wurden die
Villen in einer Reihe nacheinander angelegt. (Die althochdeutsche Übersetzung
für Villa = Dorf.) Es sind das Lämmersdorf, Fratersdorf, Aschersdorf,
Hetzelsdorf. An der Ostgrenze von Regen bis Ödwies sind Rannersdorf,
Seigersdorf, Kammersdorf, Fernsdorf, Allersdorf und Schreindorf. In der Mitte
des Besitztums liegt Patersdorf. An der Westgrenze haben sich neben den Villen
(Dorforten) Familien angesiedelt, die sich hauptsächlich bei der Rodungsarbeit
betätigen. Die Namen der Ansiedelungen dieser Familien endigen auf "ried." Bei
Lämmersdorf ist „Giggenried“, bei Fratersdorf "Kaickenried", bei Patersdorf
"Zuckenried", bei Aschersdorf "Bärmannsried." Auffallenderweise finden wir auf
der Ostseite der Besitzung und nur hier, sonst nirgends, ganz nahe beisammen
eine Menge von "ing"-Orten, z. B. Tradweging, Brenning, Zottling, Handling,
Sintweging, weiter östlich gelegen Einweging.

Was nun das Alter dieser Ansiedlungen betrifft, so können behaupten, dass diese
"dorf" und "ried"-Namen im 9.Jahrhundert entstanden sind. Denn erstensxxxiii hat
das Kloster Metten diese Besitzungen im Nordwald zu Karl d. Gr. Lebzeiten
bekommen und hat sich 882 die betreffenden Schenkungs-Urkunde bestätigen lassen,
und zweitens sind die Personennamen, die in diesen "dorf" und "ried"-Namen
enthalten sind, zur damaligen Zeit an das Reichenauer Kloster gesandt und in das
Verbrüderungsbuch eingetragen worden. Lämmersdorf = Lempferstord = Lantfried,
Fernsdorf = Fater, Patersdorf = Patto, Hetzelsdorf = Hesse, Aschersdorf =
Adalschalk, Fernsdorf = Eberwin, Allersdorf = Uodalrat, Schreindorf =
Schrandorf, war Markt, Schrannen- und Gerichtsplatz. Er wurde nicht von einer
Familie, sondern von einem Klostermeier verwaltet.

Die vielen "ried"-Namen weisen auf Rodungsland hin. Köckersried = Coteschalk,
Giggenried = Cundachaz, Kaickenried = Hacco, Bärmannsried = Perhtnand,
Zuckenried = Sigine, Hasmannsried = Hosnod, Lobetsried = Luipolf, Perlesried =
Perolf, Triefenried = Trunolf.

Mit den "dorf"- und "ried"-Orten sind zu gleicher Zeit auch die mit "berg"
zusammengesetzten Ortsnamen entstanden. Dietzberg = Theoto, Witzberg = Witolt,
Wolfsberg = Volficho,

Die "dorf"- und "ried"-Orte sind also alle älter als Ruhmannsfelden. Ihre
Entstehung geht zurück bis zum Jahre 800 und fällt in die Zeit, in der das
Kloster Metten das Besitztum im Nordwalde innehatte. Nicht so ist es mit den
"ing"-Orten, die sich an der Westseite von Ruhmannsfelden dicht zusammendrängen.
Das sind keine echten "ing"-Ortsnamen, weil sie nicht in Verbindung mit einem
Eigennamen auftreten. Sie sind womöglich mit Ruhmannsfelden entstanden. Da waren
vier große Rodungsplätze, an denen das Holz niedergebrannt wurde. Das ist
Voglsang = sengen, Klessing = an der Klippe (Klep) sengen, Prenning = brennen
und Prünst = Punst. Handling = Händlern, Einweging = weging = Weihern = ein
Weiher, Sintweging = sint = drei = drei Weiher, Hartweging = hart = Wald =
Waldweiher, Zottling = Zeidler, Zadler, Zodler, Zodlbauer.

In der Umgebung von Ruhmannsfelden wurde demnach vom Anfang des 9. Jahrhunderts
an gerodet. Es war das Gebiet noch Urwald mit Riesenbäumen und mit großem
Wildbestand. In den Tälern, die früher mit Seen ausgefüllt waren, befanden sich
noch Sümpfe und Auen mit zahllosem Wassergeflügel (Auhof, Au bei Achslach). In
der perlreichen Teisnach musste es gewimmelt haben von großen und kleinen
Fischen. Hier und da führte ein Steg über das Wasser (Stegmühle). In den meisten
Fällen fuhr man an seichten Stellen (Furten, Furthof) durch das Wasser hindurch.
Brücken wurden erst unter der Mettener Herrschaft gemacht (Bruckhof). Zwei
Hauptwege führten durch das Gebiet, der Heeresweg von Süden nach Norden von
Kalteck, Gotteszell, über die Hochstraße über die Starlwiesen Grünbach zu, der
zweite Weg von Osten nach Westen von Metten (bei Regen) über Fratersdorf nach
Schreindorf. Der Kreuzungspunkt dieser zwei Straßen war hier in Ruhmannsfelden.
Dieses wurde von Metten nicht beachtet, denn das Arbeitsfeld der Mettener Patres
lag mehr an der Grenze ihres Besitztums. Dieser Straßenknotenpunkt gab erst den
Grafen von Bogen (die später die Herren dieses Gebietes wurden, nachdem man dem
Kloster Metten das Besitztum im Nordgau abgenommen hatte) den Anlaß, eine
Ansiedelung zu errichten. Auf dieser Ansiedelung entstand dann um das Jahr 1100
herum, also viel später als die umliegenden Ortschaften mit Ausnahme der
"ing"-Orte, die Ortschaft Ruhmannsfelden.

.V Pfarrkirche St. Laurentius Ruhmannsfelden,

Viechtacher Tagblatt, 1928/29

.1 Die Pfarrkirche St. Laurentius bis zum Brand 1820*

In den Tagen vom 31. Oktober bis 2. November dieses Jahres kann die Pfarrkirche
St. Laurentius in Ruhmannsfelden das 100-jährige Jubiläum seiner Einweihung
begehen. In kirchlich feierlicher, aber doch in schlichter würdiger Form werden
die Herzen der gläubigen Pfarrangehörigen in dankbarer Liebe zu Gott empor
schlagen, der ihnen, nachdem am 1. Juli 1820 das frühere Gotteshaus durch
Feuersbrunst in Schutt und Asche verwandelt wurde, wieder eine so schöne Kirche
hat erstehen lassen.

Bei dieser Gelegenheit wollen wir erstens, unsere Gedanken in frühere Zeiten
zurück versetzen und dabei eine kurze Rückschau halten über die
Kirchenverhältnisse von frühester Zeit bis zum Brand am 1. Juli 1820 und
zweitens eine kurze Betrachtung anstellen über die neu erbaute Pfarrkirche nach
dem Brande.

Wir haben schon öfters gelesen, dass in der Karolingerzeit (nach 800) die
Mettener Patres in hiesiger Gegend christianisiert und gerodet haben.
Ruhmannsfelden existierte zu damaliger Zeit noch nicht. An der Stelle des
heutigen unteren Marktes war damals der Kreuzungspunkt zweier wichtiger
Straßenzüge. Hier aber lag nicht das Arbeitsfeld der Klosterherren, sondern es
lag abseits der Heeresstraße und dem Handelsweg, an der Grenze ihres Besitztums.
Erst um das Jahr 1100, als der Nordwaldbesitz in die Hände der Grafen von Bogen
kam, da bekam auch erst diese Straßenkreuzung Bedeutung. Ein Angestellter des
Grafen Aswin von Bogen, Rumar genannt, der musste diesen Punkt sichern, baute,
sich einen kastenförmigen Turm und seine Arbeiter, Rodungsleute und Handwerker,
bauten sich um den Turm herum ihre Wohnstätten. So entstand die Ansiedlung
Hrothimarsfeld (Rumarsfeld). Die ersten Besiedler waren schon christianisiert.
Sie brauchten zu ihrer gemeinsamen Andachtsverrichtung auch ein Kapellchen. Und
dieses bauten sie nicht innerhalb der Besiedlung, sondern erbauten ihr
Kapellchen oberhalb der Besiedlung, inmitten des Bühls, dahin, wo heute die
Pfarrkirche steht. Das können wir daraus entnehmen, weil der Patron der
Pfarrkirche der hl. Laurentius ist. Dieser wurde bei allen jenen Kapellen und
Kirchen als Patron gewählt, die seinerzeit außerhalb der Besiedlung standen.

Das Rumargeschlecht hatte aber bald ausgewirtschaftet gehabt in Ruhmannsfelden.
In der zweiten Hälfte des 13. Jahrhunderts treten als Besitzer der Burg
Ruhmannsfelden die reich begüterten Pfellinger Grafen auf. Nach dem Tode
Heinrichs von Pfelling fiel die Besiedlung den Landesherren zu. Die drei
niederbayerischen Herzöge Otto III., Ludwig III. und Stephan I. verkauften den
Markt an das Kloster Aldersbach, die inzwischen aus dem Hof "Droßlach" das
Zisterzienserkloster Gotteszell gemacht hatten. Nicht bloß um Material für die
Erbauung und Erweiterung des Klosters Gotteszell zu gewinnen, sondern auch
hauptsächlich darum um die Niederlassung irgend einer weltlichen Macht in
Ruhmannsfelden zu verhindern, wurde der Turm in Ruhmannsfelden vollständig
abgebrochen. Ruhmannsfelden stand nun unter der Herrschaft der Aldersbacher
Mönche. Dabei ging es aber Ruhmannsfelden gar nicht schlecht. Im Gegenteil! Da
hat sich Ruhmannsfelden zum Markt emporgeschwungen.

Am 10. März 1408 verpflichtete sich Georg der Parsberger, Chorherr zu Freising
von den Kirchen zu Patersdorf, Geiersthal und Ruhmannsfelden dem Kloster
Aldersbach jährlich 10 Pfund Pfennige zu entrichten. Diese drei Kirchen waren
ihm vom Abte Heinrich zu Aldersbach auf Bitten seines Schwagers, Stephan des
Degenbergers zu Altnussberg, überlassen worden. Die Urkunde ist besiegelt vom
genannten Degenberger, vom Wernherrn, dem Parsberger und Eberhard, dem
Nussberger zu Kollnburg (nach Trellinger, Bayerwald).

1445 fanden Unterhandlungen statt zwischen Aldersbach und Gotteszell die "Villa"
Ruhmannsfelden zu vertauschen. 1496 verkaufte das Kloster Aldersbach den Markt
Ruhmannsfelden notgedrungen an die Degenberger unter Vorbehalt des
Wiedereinlösens, was von Seiten des Abtes Simon von Aldersbach am Ende des 15.
Jahrhunderts geschah. 1503 am Freitag nach Maria Himmelfahrt bestätigte Herzog
Albrecht der Weise einen zwischen den Klöstern Aldersbach und Gotteszell
vollzogenen Tausch, wodurch Ruhmannsfelden an das Kloster Gotteszell kam.
Gotteszell gab dafür mehrere, im Gebiete Georg des Reichen gelegene, Güter.
Aldersbach behielt sich nur den Pfarrhof und die pfarrlichen Rechte in
Ruhmannsfelden und schickte einen Expositus nach Ruhmannsfelden. 1511 begann der
dauernde Streit zwischen Kloster und Markt. 1519 brachen Unruhen aus, die sogar
in offenen Aufruhr ausbrachen. Die Ruhmannsfeldener beschädigten die Güter des
Prälaten. 1522 brach ein neuer Aufstand aus, bei der ganze Markt durch
Feuersbrunst vernichtet wurde. Die Pfarrkirche stand damals wahrscheinlich noch
außerhalb des Marktes, sonst wäre sie sicher auch ein Raub der Flammen geworden.
Hiervon ist aber nichts bekannt. Im Jahre 1574 brannte das Pfarrgotteshaus
vollständig nieder. Für die Wiederherstellung derselben geschah sehr viel von
Seiten des Klosters Gotteszell, das auch die Aufsicht über den Bau leitete. Die
Glocken, die von dem Münchner Glockengießer Girt stammten, wurden erst 1643 auf
den Turm gebracht. 1633 wurde die Pfarrkirche von den schwedischen Kriegsvölkern
vollständig ausgeplündert und Pfarrvikarhaus und Klostertaverne nach geschehener
Plünderung in Asche gelegt.

1658 kam die bisher Kloster Aldersbach'sche Pfarrei Ruhmannsfelden an das
Kloster Gotteszell. Von dieser Zeit an bis zur Aufhebung des Klosters Gotteszell
(1803) blieb der Markt Ruhmannsfelden dem Kloster Gotteszell mit Grund und
niederer Gerichtsbarkeit unterworfen und die Pfarrei wurde von da an nicht mehr
von Geiersthal aus, sondern vom Kloster Gotteszell aus pastoriert.

1659 wurde ein Pfarrhof gebaut. 1686 spuckte das Luthertum in Ruhmannsfelden.
1696 vermachte Paul Huber von Prünst dem Gotteshaus zu Ruhmannsfelden sein
ganzes, nicht unbedeutendes Vermögen. 1745 im österreichischen Erbfolgekrieg
kamen die Panduren nach Ruhmannsfelden und plünderten die Kirche aus.

1803 wurde das Kloster Gotteszell aufgehoben unter Abt Amadäus Bauer. Der
Pfarrhof und die pfarrlichen Rechte von Ruhmannsfelden, die seit 1659 zum
Kloster Gotteszell gehört hatten, wurden von der Aufhebungskommission bei der
Säkularisation verkauft, ebenso die Rechte und Güter in Ruhmannsfelden, die zum
Kloster Gotteszell gehörten. Im Jahre 1800 versah die hiesige Expositur noch Hr.
P. Bernhard Kammerer. Zurzeit der Aufhebung des Klosters war Pfarrprovisor Hr.
P. Joseph Haindl und zwar vom 21.3.1803 bis 1. Oktober 1805. Aushilfe leistete
ihm vom 21.4. bis Ende Oktober 1803 P. Nivard Sator, vom November 1803 bis Ende
1804 P. Marian Triendorfer, vom März 1804 bis Ende September 1805 P. Guido
Berger. Am 14. August 1805 starb in Ruhmannsfelden P. Xaver Sämer. P. Marian
Triendorfer ist dann nach Viechtach übergesiedelt, kehrte aber bald wieder nach
Ruhmannsfelden zurück. 1805 wurde dann in Ruhmannsfelden ein Pfarrer definitiv
angestellt. 1806 wurde die Auspfarrung vollzogen.

Eine Beschreibung vom Jahre 1819 sagt, dass die damalige Pfarrkirche von
Ruhmannsfelden aus rauhen Steinen erbaut war, Langhaus und Chorhaus waren mit
Taschen gedeckt. Der Turm war durchaus von rauhen Steinen erbaut. Die Turmkuppel
war mit Schneideschindeln eingedeckt. Im Kirchenturm hingen drei Glocken und in
der oberen, kleineren Kuppel, die Laterne genannt wurde, ein kleines
Sterbeglöcklein. Die große Glocke wog 12 Zentner, die kleinere 2 ½  Zentner und
das Sterbeglöcklein ½  Zentner. Die drei Glocken hatten einen Wert von 300 fl.
Im Gotteshaus waren vier Altäre und eine Orgel mit 8 Registern von einem
ungewissen Meister. Nächst dem Hochaltar befand sich ein Seitenaltar, auf
welchem der hl. Leib des Märtyrers St. Martin ruhte. Dieser Leib wurde vom
Kloster Gotteszell erkauft und war Eigentum des hiesigen Bierbrauers Martin
Lukas. Der Bruderschafts-Altar war Eigentum der Corporis-Christi-Bruderschaft
und musste von dieser unterhalten werden. Die Kirche war gut versehen mit
Paramenten. Sogar der Chor war gut ausgerüstet mit Instrumenten. Das
Eigentumsrecht von den Trompeten hatten die Wolfgangi-Brüder und die Pauken
gehörten der Bruderschaft.

Am 1. Juli 1820 brach im Hofraum der Berger'schen Bierbrauerei (Amberger) ein
Brand aus. Der damals herrschende Wind entfachte den im Hof aufgestapelten und
mit Tannenreisig zugedeckten Aschenhaufen. Im Nu standen die Berger'schen
Gebäulichkeiten in hellen Flammen und die halbe Marktseite samt der Pfarrkirche
wurde über Nacht in Schutt und Asche verwandelt.xxxiv

.2 Die Pfarrkirche St. Laurentius bis zur Einweihung 1828*

Mit Entsetzen denken wir noch zurück an den Ludwigs-Tag 1894. Der halbe Markt
Ruhmannsfelden war an diesem Tage in wenigen Stunden ein Raub der Flammen
geworden .Ein grauenvolles Bild der Verwüstung steht lebhaft vor unseren Augen,
wenn wir uns zurück erinnern an diese furchtbare Brandkatastrophe. Aber ein viel
größeres und entsetzlicheres Unglück noch brachte der 1. Juli 1820 über
Ruhmannsfelden, da nicht bloß der halbe Markt, sondern mit ihm auch die
Pfarrkirche St. Laurentius und das Osterbrünnl-Muttergottesbild durch Feuer
vollständig vernichtet wurden. Früher hatte das Holz wenig oder gar keinen Wert.
Das sehen wir daraus, dass die Gemeinde Ruhmannsfelden die ganze Ödwies bekommen
hätte, wenn sie nur die Grundsteuerlast für den betreffenden Grundbesitz
übernommen hätte. Ruhmannsfelden hat abgelehnt. Die Berger, Klimmer, Lukas,
Schrötter usw., die damals maßgebend waren, hatten ohnehin soviel Waldbesitz,
als sie brauchten und die kleinen Leute konnten sich Holz heimziehen, soviel sie
brauchten. Mangel an Holz herrschte nicht. Gekostet hat das Holz sehr wenig. Man
hat kurzen Prozess gemacht, man hat das Holz im Walde draußen verbrannt, die
Asche heimgefahren und an die Seifensieder und Aschenaufkäufer verkauft. Beim
Berger-Bräu, bei dem sich genannte Aufkäufer und Händler aufhielten, war im
rückwärtigen Hofraum der Lagerplatz dieser Asche. Ein mächtiger Aschenhaufen war
schon zusammengefahren und mit Reisig zugedeckt. In den nächsten Tagen sollte er
weggefahren werden. In wie viel tausend Fällen mag die Asche wohl schon der
Brandstifter gewesen sein?

Nacht war's. Ein heftiger Wind fing an zu wehen. Der Aschenhaufen kam ins
Glühen. Das Reisig fing zu brennen an. Das Feuer griff über auf Stadel und
Stall. Bis die aus dem Schlaf erweckten Leute kamen, standen die Berger´schen
Gebäulichkeiten in hellen Flammen. Der Wind trug die Funken und die brennenden
Schindeln auf die Dächer der anstoßenden Nachbarhäuser. Die halbe Marktseite
stand in Feuer. Das Feuer griff aber auch auf den Dachstuhl der Pfarrkirche über
und unter lautem Aufschrei der unglücklichen Bewohner fing die mit
Schneideschindeln bedeckte oberste kleine Kuppel zu brennen an. Das Feuer
ergriff dann den Glockenstuhl und ohne dem vernichtenden Treiben des Feuers
entgegentreten zu können, mussten sich die Nichtabgebrannten auf den Schutz des
eigenen Heimes beschränken und mit Jammern und Weinen und Klagen mussten die
unglücklichen Ruhmannsfeldener zusehen und warten bis das Feuer seine
Vernichtung vollendet hatte. 12 bürgerliche Anwesen samt dem Brothaus und die
schöne Laurentius-Kirche sind über Nacht in einen verrußten Stein- und
Trümmerhaufen verwandelt worden. Damit das Maß des Unglücks voll war, ist auch
das Osterbrünnl-Muttergottesbild, das sich in der Pfarrkirche befand, mit
verbrannt.

Auf Veranlassung des Abtes Wilhelm II. von Gotteszell wurde die
Osterbrünnlkapelle 1724 niedergerissen. 1813 erbauten die Marktbürger Josef
Baumann und Anton Schlegel die Kapelle wieder. Die Wallfahrt kam nach Errichtung
dieser Kapelle so in Schwung, dass der Landrichter Beierlein von Viechtach die
Kapelle 1814 schon wieder wegreißen ließ. Die zwei genannten Marktbürger trugen
in der Nacht das Muttergottesbild samt Kasten in das Gotteshaus St. Laurentius,
wo man es unter die Kanzel setzte und mit Silberanhängern beehrt hatte. Und bei
dem großen Brande am 1. Juli 1820 verbrannte auch das schöne Mariahilfbild in
der Pfarrkirche.

Ratlos stand man vor dem Nichts. Zunächst wurde die Brandstätte geräumt. Die
nächste Sorge galt dem Wiederaufbau der 12 abgebrannten Anwesen. Freilich wurde
auch davon gesprochen das Osterbrünnl rasch aufzubauen als vorläufiger Ersatz
für die abgebrannte Pfarrkirche und das „nemliche Mariahilfsbild wieder
herstellen zu lassen, daß wiederum viel Geld einginge." Aber die Kapelle konnte
nicht ausgebaut werden, weil eine Genehmigung hierzu nicht erfolgte, und das
"vorhandene Einlegegeld vom Osterbrünnl von dem damaligen Frühmesser P. Marian
Triendorfer (ein geborener Haidlfinger) nach Viechtach verbracht wurde und von
dort nicht mehr zurückgegeben wurde."

Man kam auf den Gedanken, das beim Osterbrünnl angefahrene Baumaterial zum
Wiederaufbau der abgebrannten Pfarrkirche zu verwenden. Aber das gelang nicht,
"weil die Arbeitsleute lieber bei den Abbrändlern arbeiten, wo sie bare
Bezahlung erhalten. "…" "Die Messopfer unserer Priester geschehen daher auf
offenem Marktplatz an einem Tisch, über welchen einige Bretter geschlagen sind,
während rund herum die Bauleute hämmern und lärmen, Fuhrleute vorübertoben. …"
Die Einnahmen zum Wiederaufbau der abgebrannten Pfarrkirche flossen sehr
spärlich. Zunächst wurde aus dem Kirchenholz das notwendige Bauholz
herausgeschlagen. Das Überholz und das Abfallholz wurde verkauft. Friedrich
Liebl kaufte den Schutthaufen von der abgebrannten Kirche um 9 fl. Durch das
Landgericht Viechtach wurden von den Aichinger'schen Kindern von Schweinberg
2000 fl. und ein Feuerassekuranzgeld von 816 fl. überwiesen. Der Bauerssohn
Andrä Göstl von Patersdorf vermachte freiwillig zum Pfarrgotteshause
Ruhmannsfelden 150 fl. und der Frühmesser Hr. P. Marian Triendorfer 1000 fl. Von
den Handwerkmeistern, die bei dem Wiederaufbau der Pfarrkirche beteiligt waren,
sind aufgeführt: Zimmerermeister Baumgartner von Viechtach, Zimmerermeister
Göstl von der Lindenau, Maurermeister Achatz Jakob von Viechtach, Maurermeister
Moser Lorenz von Zwiesel, der die Ausführung der hinteren Giebelmater innehatte,
Maurermeister Fürg von Straubing, der die Pläne und Überschläge machte.

Lauter fremde Meister, weil die Meister von Ruhmannsfelden und seiner nächsten
Nähe vollauf zu tun hatten, die 12 abgebrannten Anwesen so bald als möglich
wieder aufzubauen. Dann erst sollte die Kirche kommen. Die Wiederaufbauarbeit
setzte aber unter Beiziehung fremder Arbeitskräfte schon am 15. Juli 1820 ein.
Als Maurergesellen werden aufgeführt: Blaßer, Achatz, Span, Stern, Brunner,
Deuschl, Frisch, Weber, Barzinger, Rittmannsberger, Mühlbauer, Seiderer, Sterr,
Haberl, Gleißner. Als Zimmergesellen werden aufgeführt: Eisenreich, Stern,
Volrath, Leichtl, Drinkl, Biller, Kilger, Hinterleuthner, Zadler, Obermayer,
Schmidbauer, Dietl, Drin, Hanghofer, Stöger, Lippl, Daffner, Oberberger,
Englmeier, Amesberger. Der Kalk wurde von Andrä Höller, Zieglmeister von
Schaching bezogen.

Am 5.8.1820 musste Anna Maria Pledlin nach Straubing gehen und musste Pläne und
Überschläge holen. Am 16.8.1820 fuhr der Bürgermeister nach Passau um das
Glockenmaterial schmelzen und reinigen zu lassen. Das Material kam in Kisten
verpackt mittels Fuhrwerk nach Deggendorf und von dort auf dem Wasser nach
Passau, wurde dort auf der Stadtwage gewogen und dann zum Glockengießer Samasa
gebracht. Die Deggendorfer, die "bey der Feuersbrunst beygeholfen haben und die
Schuttaufräumer durften bei Joseph Lukas einen Eimer Nachbier trinken." Für die
Guttäter, welche zur Pfarrkirche Bauholz und Materialien geliefert haben, wurde
bei Bierbrauer Heinrich Bürger um 11 fl. Freibier abgegeben. Am 9. Oktober 1820
ist der Bürgermeister mit zwei Wägen und fünf Pferden nach Deggendorf gefahren,
um die zwei Glocken zu holen, die Glockengießer Samasa in Passau inzwischen
schon gegossen hatte. Herr Joseph Gaim von Deggendorf mit seinen zwei
Schiffsknechten Klober und Sailer hatte die Glocken auf der Donau von Passau
nach Deggendorf gebracht. Die Glockenseile machte Joseph Nagl, Seiler von hier.
Die Glockenschwengel waren nicht arg beschädigt, sie brauchten nur eine kleine
Reparatur. "Die Richtung derselben besorgte Erhard Forstner, Hammermeister von
Böbrach." Die Schindlschneider von Zwiesel, welche 64 300 Schneideschiendeln
gemacht hatten, bekamen hierfür 100 fl.

Inzwischen wandte sich die Pfarrei und die Marktgemeinde Ruhmannsfelden im
September 1820 an den König Max I. mit einem Bittgesuch, das Osterbrünnl
aufbauen zu dürfen unter dem Hinweis, dass infolge des großen Brandes die
Kapelle unschädlich, ja sogar nützlich und notwendig sei. Immer wieder erfolgte
die Abweisung der Bittgesuche, da Landgericht und Rentamt Viechtach gegen den
Wiederaufbau des Osterbrünnls waren. Endlich am 24. Mai 1821 gab das Landgericht
Viechtach seine Zustimmung. "Mit Beyhilfe in und außer der Pfarr wurde eine
schöne Filialkirche aus dem Osterbrünnl erbaut." Am 23. Juli 1821 erhielt Herr
Pfarrprovisor Deischl vom Ordinariate Regensburg die Erlaubnis, die neu erbaute
Kapelle auf dem sogenannten Osterbrünnl nach Vorschrift benedizieren zu dürfen,
was auch geschah. 1822 bekam die Kapelle eine ganz neue Inneneinrichtung,
gefertigt von dem Georg Dachs, Schreiner von Linden, um 116 fl.

Allerdings schreibt der Schätzmann Joseph Schweiger, bürgerlicher
Schreinermeister von Stadtamhof über die Ausführung der geleisteten Arbeit des
Georg Dachs folgendes: "Ich fand, dass die Arbeit in allen Teilen des Altars
weder schön noch fleißig gemacht und hiebei insbesonders alle Symetrie,
Proportion, alle Regeln der Architektur außer acht gelassen, auch dem
herrschenden Zeitgeschmack ganz und gar nicht gehuldigt worden ist." Eine
Filialkirche war nun vorhanden. Es konnte der Gottesdienst wieder gehalten
werden. Aber die Arbeiten an der Vollendung der Pfarrkirche scheinen ins Stocken
geraten zu sein, da die vorhandenen Pfarrakten zwar vom Osterbrünnl aber nichts
mehr von Pfarrkirche St. Laurentius etwas berichten.

Am 11. Mai 1825 stellte das königliche Landgericht Viechtach folgendes Zeugnis
aus: "Der Marktgemeinde Ruhmannsfelden wird auf Ansuchen behufs der Erlangung
außerordentlicher Unterstützung zur Herstellung der abgebrannten
Marktpfarrkirche der Wahrheit und Pflicht gemäß auf den Grund der vorliegenden
Akten bestätigt, daß der am 1. Juli 1820 zu Ruhmannsfelden stattgefundene Brand
in der dortigen Pfarrkirche eine solche Zerstörung stiftete, dass zur
Wiedererbauung und inneren Verzierung derselben nach den Voranschlägen mehr als
15000 fl. erforderlich sind, dass bisher nur 820 fl. aus der Brandversicherung,
3272 fl. Dezimationsbeitrag des Aerars und 58 fl. der dezimationspflichtigen
Privaten flossen und außerdem von den Parochianen bedeutende Hand- und
Spanndienste geleistet wurden.

Das unterfertigte kgl. Landgericht bemühte sich wiederholt für das zerstörte
Pfarrgotteshaus Hilfsquellen zu öffne; allein vergebens. Das Gebäude steht in
seinen Ruinen da, nur in etwa zur Abhaltung des nötigen Gottesdienstes
hergerichtet. Auf ordentlichem Wehe steht eine Abhilfe nicht zu erwarten; denn
die Dezimatoren leisteten das ihrige. Die Brandassekuranz zahlte den
Versicherungsbeitrag. Die Pfarrkinder taten, was sie konnten und sind bei der
Armut hiesiger Gebirgsgegend bei dem erlittenen Unglücke des wiederholten
Hagelschlages und unter dem Druck damaliger Zeitverhältnisse zu einer weiteren
Konkurrenz außerstande. Die Marktkassen zu Ruhmannsfelden sind gleichfalls leer,
da dieser Markt ein so unbedeutendes Communal-Vermögen besitzt, daß aus den
Renten desselben die Kommunallasten nicht bestritten werden konnten."

1825 kam König Ludwig I. zur Regierung. Sicher haben sich die Ruhmannsfeldener
bittend an ihn gewandt, dessen frommer, kirchlicher Sinn sich in Klosterbauten
und Klosterwiederherstellungen bekundete. Dass sich König Ludwig I. um den
Wiederaufbau der Pfarrkirche wärmstens angenommen hat, das ersehen wir daraus,
weil plötzlich auf eine raschere Vollendung der Pfarrkirche gedrungen wurde und
das ersehen wir aus der Auswahl der Bauart der Pfarrkirche.

Aus einer Urkunde vom Jahre 1828 ist zu entnehmen, dass König Ludwig I. dem
Wiederaufbau der Pfarrkirche finanzielle Unterstützung zuteil werden ließ.
Außerdem wandte sich die Pfarrei und Marktgemeinde an verschiedene hohe Gönner,
die den Wiederaufbau durch hochherzige Schenkungen förderten. Eine Sammlung im
ganzen Unterdonaukreis wurde veranstaltet, sodass es möglich wurde, den Bau bis
zum Beginn des Jahres 1828 fertig zustellen. Am 9. Januar 1828 wandte sich die
Gemeinde Ruhmannsfelden wiederholt bittend an den König. In dieser Urkunde heißt
es unter anderem: "Es fehlt das Altarbild für den Hauptaltar. Es ist nichts
darüber festgesetzt, ob dieses Altarbild ein Gemälde oder ob es ein
Bildhauerwerk sein soll - auch der Gegenstand der Darstellung ist nicht genau
vorgeschrieben. Doch wäre es der Sache angemessen und der Gemeinde erwünscht,
wenn das Bild des hl. Laurentius, des Schutzpatrones der Kirche, den Altar
schmücken würde. Alle Mittel ein solches Bild anzuschaffen sind erschöpft. Es
bleibt daher nur die allerhöchste Gnade E. K. H. übrig. …" Am 15. Januar 1823
kam schon die Antwort auf dieses Bittschreiben zurück, dass Galeriedirektor
Dillis beauftragt sei, sofort nach gewünschtem Bilde Umschau zu halten. Am
20.3.1828 kam die Mitteilung, dass ein Laurentius-Bild nicht vorgefunden wurde,
dass an dessen Stelle ein anderes passendes Gemälde gesucht werde. Das
ausgewählte Bild ist das, welches heute den Hochaltar ziert. Es ist aus der
Gemäldegalerie Augsburg, von Joseph de Lens gemalt, ist Staatseigentum, trägt
auf seiner Rückseite diesbezügliche Aufschriften und ist ein Bild von hohem
künstlerischem Wert. Nach den Dimensionen dieses Altarbildes wurde dann der
Hochaltar projektiert und sind in Passau unter Aufsicht und Leitung des
königlichen Kreisbaubüros Passau Hochaltar, Seitenaltäre, Kanzel und
Beichtstühle angefertigt worden.

Inzwischen hat man die Pfarrkirche unter Dach gebracht. Und man war schon froh,
denn das Osterbrünnlkirchlein war ja wegen seiner kleinen Raumverhältnisse kein
vollwertiger Ersatz für die Pfarrkirche. Deshalb errichtete man in der
Pfarrkirche provisorisch einen Altar, um wieder in der Pfarrkirche Gottesdienst
halten zu können. Die heutige Inneneinrichtung der Pfarrkirche kam erst nach
1828. Trotzdem wandte sich die Pfarrei bittend an das Bischöfliche Ordinariat
Regensburg mit der Bitte, die neu erbaute Kirche einweihen zu wollen. Auf dieses
Bittgesuch kam dann vom hochwürdigsten Herrn Johann Nepomuk, Bischof von
Regensburg folgende Mitteilung: "Zufolge der unterm 13. Oktober 1828 anher
übergebenen Bittvorstellung die Einweihung der neuerbauten Pfarrkirche in
Ruhmannsfelden betreffend wird dem Herrn Pfarrer die Bischöfliche Erlaubnis und
Vollmacht anmit erteilt, den Einweihungsakt nach Vorschrift des Diözesan-Rituals
vorzunehmen. Ebenso wird demselben gestattet, die Kirchenparamente zu
benedizieren. Regensburg, den 21. Oktober 1828."

Vor hundert Jahren wurde die Pfarrkirche St. Laurentius Ruhmannsfelden vom H.
Hr. Pfarrer Lienhart eingeweiht. Ruhmannsfelden hatte wieder ein
Pfarrgotteshaus.xxxv

.3 Die Pfarrkirche St. Laurentius nach der Einweihung bis 1928*

Am 5. November 1830 erfolgte der Abtransport der Altäre von Passau nach
Deggendorf auf dem Wasser. Dabei blieb eine Kiste mit Gürtlerarbeiten des
Gürtlermeisters Zisammenschneider von Passau versehentlich zurück, die dann
Schiffermeister Reiter von Deggendorf nachgebracht hatte. Es wurden geliefert
der Hochaltar samt Statuen, 2 Seitenaltäre, 2 Beichtstühle und die Kanzel. Die
Gürtlerarbeiten bestanden aus 6 Leuchtern, 2 Kanontafeln, 2 Opferkandl mit Tace,
ein Rauchfass mit Schifferl, eine Ampel für den Hochaltar und für die zwei
Seitenaltäre 8 Leuchter, 6 Kanontafeln, ferner Zinngießerarbeit im Gesamtbetrage
von 2 322 fl. Am 8. November 1830 wurde dieses alles bei Herrn Vogl in
Deggendorf abgeholt und mittels Fuhrwerk nach Ruhmannsfelden befördert. Zunächst
mussten für die zwei Seitenaltäre Altarbilder beschafft werden. Mit der
Beschaffung dieser Bilder beauftragte der damalige Pfarrherr seinen Verwandten,
den Lithographen Karl Höcherl von München. Nachdem aber dieser plötzlich nach
Italien abberufen wurde, um dort eine leitende Stellung anzutreten, übergab
Höcherl den Auftrag dem Hofmaler, Professor Schlotthauer (oder Schlotthamer).
Dieser hatte zur gleichen Zeit einen hervorragend tüchtigen Schüler namens
Martin Dorner. Da dieser ein armer, bedürftiger Künstler war, übertrug
Schlotthauer die Herstellung der zwei Altarbilder dem Martin Dorner und dessen
Freund Schraudolph, die dann die beiden Altarbilder unter der Leitung der beiden
Hofmaler Schlotthauer und Hauber malten. Am 28. Dezember 1830 schreibt Hofmaler
Schlotthauer "daß die zwei Altarbilder dem Straubinger Boten übergeben wurden,
weil ich befürchte, daß der Deggendorfer zu lange ausbleiben würde möchte und so
könnten selbe dann nicht zur gewünschten Zeit eintreffen. Ich wünsche sehr, daß
sie dem Beifall von Euer Hochwürden entsprechen möchten. Hier sind sie
wenigstens von Kunstkennern sehr gerühmt worden." 1831 wurde eine neue Orgel von
Georg Adam Ehrlich, Orgelmacher zu Passau hergestellt.

1836 wurde der Hochaltar nebst Kanzel gefasst "mittels wohltätiger Beiträge von
Privaten, da die Kirche hiezu keine Mittel hat." Die Vergoldearbeiten machte der
Vergolder Benedikt Brummer von München. Der kleine, in keiner Hinsicht weder zur
Kirche noch zu den Altären passende, dem Einsturz drohende Altar der
Corporis-Christi Bruderschaft gehörend, musste abgeändert werden.

Von den Stiftern sind besonders hervorzuheben: Die Saller'schen Geschwister von
Hinterdietzberg, J. Achatz, Müller von Auerbach, Lorenz Bauer von
Pointmannsgrub, Hofmann von Muschenried, Zitelsberger von Lämmersdorf, Marchl
von Prünst, Achatz von Perlesried, Reithmer von Sintweging, Gößl von Auerbach,
Dienstknecht Keinl, Michael Artmann, Austräger von Fernsdorf, er vermachte für
die abgebrannte Kirche 100 fl.

Inzwischen wurde vom Bildhauer Christoph Itelsperger von Regensburg der
Taufstein nebst einer Statue des hl. Johannes des Täufers von Holz angefertigt
und geschnitzt. Gleichzeitig hat derselbe Meister in Regensburg beim dortigen
Tändler Pflügl eine 6 Schuh hohe Statue, die Immakulata darstellend, um 12 fl.
erworben. Diese Statue, jetzt über dem Taufstein, ist eines der wertvollsten
Stücke in unserer Pfarrkirche und wird von Kunstkennern sehr hoch eingeschätzt.
Diese Statue ist seinerzeit nach dem Kauf sofort von dem bürgerlichen Maler, und
Vergolder F. S. Merz in Regensburg renoviert und gefasst worden.

Am 5. Juni 1837 fand unter H. Hr. Pfarrer Lienhard die bischöfliche KonsekratIon
der Pfarrkirche statt. 1855 wurde ein neuer Ölberg bei der Pfarrkirche nach
einem vom königliche Kreisbüro angefertigten Plan errichtet. 1862 wurden
Reparaturen am Glockenturm vorgenommen, da dieser baufällig war; außerdem wurde
der Turm mit Weißblech von A. Prigelmayer in Viechtach eingedeckt. Im gleichen
Jahre wurde aus dem Tabernakel eine wertvolle Monstranz gestohlen. 1866/67
wurden die Seitenaltäre, 1868 der Hochaltar neu gefasst und am 6. Juli 1869
erfolgte die Konsekration der Altäre unter H. Hr. Pfarrer Uschalt. 1878 erbaute
die Sepulturgemeinde Ruhmannsfelden den neuen Friedhof. Leider hat man dabei die
alten Grabsteine entfernt, die uns so viel erzählen könnten.

Eine Urkunde erzählt von folgender Begebenheit: "Am 17.3.1879 abends beging der
ledige Michael Eisenrichter, 50 Jahre alt, einen blutigen Selbstmordversuch auf
den Stufen des Hochaltares in der Laurentius-Pfarrkirche, indem er sich an
diesem Abend ungesehen in die Kirche einsperren ließ. Plötzlich hörte man Rufe
aus der Kirche und das Schellen mit dem Ministrantenglöcklein. Dem Mesner Plank
und seiner Magd, die sich sofort in die Kirche begaben, bot sich ein
entsetzlicher Anblick. Auf der obersten Altarstufe lag ein Mann in Hemdärmeln,
die Schuhe abgezogen, seine Joppe als Kopfkissen zu Recht gerichtet. Er rief:
"Plank Alois, mich friert so, wo ist mein Rasiermesser, ich habs verloren." Die
Altarstufen, ja selbst der Altartisch waren mit Blut befleckt. Am Halse trug der
Mann zwei mächtige Schnittwunden. Die Mesnermagd Anna Schönberger, welche die
Laterne trug, fand das Rasiermesser am linken Ende der unteren Altarstufe. Am
rechten Ende lag das Taschemesser. Dieses scheint zu stumpf gewesen zu sein.
Inzwischen trafen viele Mannspersonen in der Kirche ein. Der damalige Sergeant
ließ den Mann in das Krankenhaus verbringen. Die Wunden waren nicht tödlich. Den
Beistand des Pfarrers wies er zurück. Der Mann hatte diese Tat mit Absicht und
Vorbedacht begangen. 40 Jahre war er von Ruhmannsfelden fort und war erst kurze
Zeit wieder hier. Er war Lederergeselle und hielt sich größtenteils in
Österreich auf. Er hat viel verdient aber auch alles wieder angebracht. Er war
ein Religionshasser und Gotteslästerer. Die Armenpflege nahm sich seiner an und
besorgte ihm eine Wohnung und ein Bett. Damit für seine Verköstigung gesorgt
war, wurde die Umzeche für ihn angeordnet. Bürgermeister Lederer Lukas
verköstigte ihn gut. Trotzdem wollte er die Umzeche nicht. Weil diese
Angelegenheit nicht nach seinem Wunsche geregelt wurde, beging er die grausige
Tat. Nachdem Eisenrichter ins Krankenhaus gebracht war, wurde das Allerheiligste
sofort in den Pfarrhof gebracht. Alle kirchlichen Funktionen unterblieben. Der
hochw. Hr. Bischof wurde telegraphisch verständigt. Am 19.3.1879 traf auch schon
das gregorianische Wasser ein, Altar und Kirche wurden sofort konsekriert. Mit
einem feierlichen Gottesdienste und Predigt wurden die kirchlichen Funktionen in
der Laurentius-Pfarrkirche wieder aufgenommen. Inzwischen sind die kirchlichen
Verrichtungen im Osterbrünnl abgehalten worden."xxxvi

Zweimal brannte die Pfarrkirche ab 1574xxxvii und 1820. Der 30. April 1889 wäre
derselben bald wieder zum Verhängnis geworden. Um die Mitternachtstunde des
genannten Tages brannten sieben Anwesen im oberen Markte, Dietrich, Sixl,
Weinzierl, Meindl, Hirtreiter, Reisinger und Baumann, die ihre Anwesen in
nächster Nähe um die Pfarrkirche herum hatten, vollständig nieder. Zum größten
Glück hatte die Pfarrkirche um diese Zeit schon eine harte Bedachung (Platten).
Trotzdem fing der Dachstuhl des Presbyteriums schon zu brennen an. Das Feuer
konnte aber glücklicherweise noch bekämpft werden, sodass es nicht weiter
greifen konnte, sonst wäre die Laurentiuskirche sicherlich zum dritten Male ein
Raub der Flammen geworden.

In ihrem jetzigen Gewande ist die Laurentius-Pfarrkirche in Ruhmannsfelden ein
wahres Schmuckkästchen unter den Kirchen des bayerischen Waldes. Wer in die
Kirche tritt, dem fällt sofort der eigentümliche Baustil auf. In der Zeit, als
die abgebrannte Laurentius-Pfarrkirche wieder aufgebaut wurde, hat man unter dem
bestimmenden Einfluss König Ludwig I. die Vorbilder zu den Neubauten der Antike
entnommen und dementsprechend wurde auch die Laurentius-Pfarrkirche unter
Kreisbauingenieur Hochstetters Leitung im klassizistischen Baustil ausgeführt.
Die Seitenschiffe mit den waagrechten Decken sind im griechischen (klassischen)
Baustil. Das Mittelschiff mit seinem Deckengewölbe ist in römisch-griechischem
Baustil. Das ganze Innere der Kirche mit den griechischen Säulen zu beiden
Seiten des Hauptschiffes und dem antiken Aufbau der Altäre klassizistisch.

Der Hochaltar, in elfenbeinweiß gehalten, hat drei Stufen, den Altartisch, den
Altaraufbau mit dem Tabernakel. Zwei jonische, glatte, nicht kanelierte Säulen
mit den Kapitälen tragen ein in Gold reich verziertes Fries, das mit einem
Zahnschnitt nach oben abschließt, auf dem der dreischenkelige Giebel ruht. Im
Giebelfeld ist das strahlende Auge Gottes. Auf den zwei, oberen Giebelschenkeln
sitzen wachende Engel und die Giebelspitze trägt ein goldenes Kreuz. So sind
auch die Seitenaltäre, nur tragen diese gewölbte Giebel. Die Altarbilder vom
Hochaltar und den zwei Seitenaltären haben großen Kunstwert. Das Hochaltarbild
ist Eigentum des Staates. Sehr wertvoll ist auch der Kreuzweg in der
Pfarrkirche. Derselbe ist (siehe Gg. Aichinger) nach Entwürfen von Führich (ein
österreichischer Maler, 1800 bis 1876) auf Messinggewebe von einem Sohn des
Marktes, Leopold Baumann, gemalt. Dieser malte auch das Altarbild vom
Corporis-Christi-Bruderschaftsaltar, im Presbyterium.

1903 wurde die Pfarrkirche einer gründlichen Restaurierung unterzogen, die
Sakristei umgebaut und oberhalb der Sakristei ein Oratorium geschaffen.

Die Deckengemälde, Szenen aus dem Leben des hl. Laurentius darstellend, und
Restaurierung der Seitenwände schuf Malermeister Weber von Amberg. Die Fassung
der Altäre und Statuen führte Malermeister Boroviska von Regensburg aus. Die
Stühle und Türen sowie die Sakristeischränke machte Schreinermeister Kappl von
Linden.

Das Reliefbild oberhalb der Sakristeitüre, die Anbetung Jesu darstellend, wurde
vom verstorbenen Hr. Kammerer Mühlbauer angekauft. Katharina Bielmeier,
Bauerstochter in Schwarzen stiftete den Betrag zur Beschaffung des Pflasters im
Presbyterium. Durch vier Kirchenfenster hinter dem Hochaltar fällt Licht in das
Presbyterium, an dessen Decke die Worte in großen Lettern stehen: "Hic aula Dei
" d. h. das ist die Wohnung Gottes. Das linke von diesen vier Kirchenfenstern,
Herz Jesu darstellend, ist vom Paktistenbund Ruhmannsfelden, das rechte, Herz
Maria darstellen, vom Jungfrauenbund Ruhmannsfelden 1903 gestiftet worden. Durch
drei Kirchenfenster fällt Licht in das rechte Seitenschiff, das eine davon, den
hl. Isidor darstellend vom Bauernverein Ruhmannsfelden 1903 und das andere, den
hl. Nepomuk darstellend, ist vom Bürgerverein Ruhmannsfelden 1903 gestiftet
worden. Leider ist dieses Kirchenfenster von einem Kirchenräuber durch
Eindrücken beschädigt worden. Im rechten Seitenschiff sind 2 Beichtstühle, 9
Seitenstühle, 2 Weihwasserkessel, 2 Opferstöcke und der breite, bequeme
Stiegenaufgang zur Empore und zum Chor.

Von den drei Kirchenfenstern, durch die das Licht in das linke Seitenschiff
fällt, ist das eine, den hl. Franziskus darstellend vom 3. Orden Ruhmannsfelden
1903 gestiftet und das andere, den hl. Georgxxxviii darstellend, vom Krieger-
und Veteranenverein Ruhmannsfelden 1903 gestiftet. Die sämtlichen Fenster
stammen aus der Kunstglaserei Schneider in Regensburg. Im linken Seitenschiff
sind ebenfalls 2 Beichtstühle, 9 Seitenstühle, 2 Weihwasserkessel, ein
Opferstock und der Stiegenaufgang zur Empore. Während des Weihnachtsfestkreises
ist am linken Seitenaltar eine sehr nette Krippe angebracht, die von Jung und
Alt gerne besucht wird.

An den Seitenwänden des Hauptschiffes sind in den breiten Feldern zwischen den
Säulen in kreisrunden mit Verzierungen versehenen Flächen Heiligenbilder. In den
Feldern der Gewölbestützen sind die sieben Sakramente bildlich dargestellt.
Durch je vier große Halbfenster zu beiden Seiten des Hauptschiffes fällt
genügend Licht in den gewölbten Raum. Links vom Hochaltar ist die Statue des hl.
Stephanus, rechts davon die Statue des hl. Laurentius, der zwei Erzdiakone. An
der linken Seite im Presbyterium steht die Statue "Herz-Jesu", an der rechten
Seitenwand die Statue "Herz Maria." Diese beiden Statuen sind erst später von H.
Hr. Pfarrer Fahrmeier nachgeschafft worden und weichen von den Altarstatuen in
Farbe und Größe ab. Links vom Corporis-Christi-Bruderschaftsaltar im
Presbyterium steht die Statue der "St. Barbara", rechts die Statue des "St.
Johannes." Links vom Laurentiusaltar im rechten Seitenschiff steht die Statue
"Herz Maria" und rechts die Statue "Herz Jesu". Links vom Martini-Altar im
linken Seitenschiff ist die Statue der "St. Philomena, rechts ist die Statue des
"St. Joseph." Auf dem Kanzeldach steht die Statue des "Petrus". Die sämtlichen
Statuen sind überlebensgroß und elfenbeinweiß. In der Mitte der Kirche hängt vom
Deckengewölbe herab inmitten der zwei Bilder "Zacharias" und "Ezechiel" das
Kreuz mit dem gekreuzigten Heiland.

3 Gedenktafeln künden uns von verdienstvollen, berühmten Ruhmannsfeldenern. Die
Inschrift der einen Gedenktafel lautet: "Denkmal des Hochw. Herrn Franz Lorenz
Grassl, Missionar aus Ruhmannsfelden, geb. 18.8.1753 als Sohn des hiesigen
Lederermeisters. In seinen Studienjahren zählte ihn der hochselige Bischof
Sailer zu seinem innigsten Freunde. Nach wenigen Jahren priesterlicher Tätigkeit
verließ er mit größtem Schmerze seine liebe Heimat und kam am 10.10.1887 in
Amerika an, wo er als eifriger Missionar recht segensreich wirkte. Wegen seiner
ganz vorzüglichen Natur- und Geistesgaben wurde er auf der ersten Synode der
neuen Republik zum Coadjutor-Bischof von Baltimore gewählt. Während die Tatsache
seiner Wahl zur Bestätigung nach Rom geschickt wurde, versah er immer noch zu
Philadelphia das Amt des unermüdlichen Missionars um eben diese Zeit, als dort
die schreckliche Pest wütete, bis endlich auch er als Opfer der Liebe und des
Seeleneifers, Opfer dieser tückischen Krankheit erliegen mußte am Ende des
Jahres 1793. Sein Andenken bleibt in Segen."

Die Inschrift der zweiten Gedenktafel lautet: "Zur frommen Erinnerung an den
hochw. Herrn Franz Xaver Fromholzer, Pfarrer der Vierzehnnothelferkirche in
Gardenwille, Diözese Buffalo, Nordamerika, geboren zu Ruhmannsfelden, zum
Priester geweiht am 25. Juli 1875 in Brixen, Tirol, wirkte überaus segensreich
16 ½ Jahre in Springwille, Uschford, Scheldorn und Gardenwille, wo er am
4.3.1893 wohlvorbereitet und ergeben in den Willen Gottes im 42. Lebensjahre
verschied. R.I.P."

Die Inschrift der dritten Gedenktafel lautet: "Andenken an die ehrw.
Missionsschwester Mr. Agnes Holler; Metzgermeisterstochter von Ruhmannsfelden,
die am 13. August 1904 bei einem tückischen Überfall der Missionsstation Baining
auf Neupommern, wo sie mit einigen Brüdern und Schwestern zur Erholung und zur
Feier der Einweihung der neuen Kapelle weilte, durch Keulenhiebe getötet wurde
und so als jugendliches Opfer von 23 Jahren für das Reich Gottes, dessen
Ausbreitung ihr als schönste Lebensaufgabe galt, zur unverwelklichen Krone der
Herrlichkeit gelangte. R.I.P."

 Im Jahre 1910 wurde von dem Orgelbaumeister Herrn Ludwig Edenhofer in
 Deggendorf eine neue pneumatische Orgel mit 2 Manualen, 22 klingenden
 Registern, 3 Bassregistern, 2 Koppelungen und 3 Druckknöpfen aufgestellt. Die
 Prospektzinnpfeifen wurden 1917 während des Krieges an den Staat abgeliefert,
 sind aber durch die Bemühungen des derzeitigen Pfarrvorstandes, H. Hr. Pfarrer
 und Schuldekan Fahrmeier wieder neu beschafft. Am 17.8.1925 wurden die 43
 Prospektzinnpfeifen von der Orgelbauanstalt Weise in Plattling geliefert und in
 die Orgel eingebaut. Ebenso wurden die zwei kleineren Glocken von den vier
 Kirchenglocken während des Krieges abgeliefert. Durch das unermüdliche
 Bestreben des H. Hr. Pfarrers Fahrmeier wurde es ermöglicht, dass auch die
 beiden abgelieferten Glocken bald wieder nach beschafft werden konnten. Die
 dritte Glocke kam am 12. Dezember 1924 und die vierte am 1. Juli 1926 wieder
 auf den Turm. Die beiden Glocken stammen aus der Glockengießerei Gugg in
 Straubing und sind 10 und 5 Zentner schwer, während die zwei großen Glocken ein
 Gewicht von 28 und 20 Zentnern haben.

In Jahre 1911 wurde die Friedhofmauer unter Leitung des Baumeisters Gegenfurtner
neu aufgeführt. 1916 kam eine neue Uhr auf den Turm, geliefert von der
Turmuhrenfabrik E. Strobl, Regensburg. 1920 bekam die Laurentius-Pfarrkirche
elektrische Innenbeleuchtung. Am 1. Mai 1920 erstrahlte die "Maienkönigin" auf
dem Hochaltar in der Laurentius-Pfarrkirche zum ersten Male im elektrischen
Lichterkranze. 1920 und 1927 wurden Kuppel, Turm und Außenmauern der Kirche
renoviert. Im April 1928 wurde die Pfarrkirche durch die Entstaubungsfirma
Müller, München entstaubt. Seitdem erscheint das Innere der Kirche wie in ganz
neuem Gewande.

Am 31. Oktober, 1. und 2. November 1928 wurde das 100-jährige Bestehen der
jetzigen Laurentius-Pfarrkirche durch ein Triduum festlich begangen. Die zwei
Redemptoristenpatres H. Hr. P. Braig und H. Hr. P. Waldinger waren eigens von
Deggendorf hierher gekommen. Bei den Predigten und Gottesdiensten war die
festlich geschmückte Kirche immer voll von Gläubigen.

In herrlichen Worten legten die beiden H. Hr. Festprediger Wert und Bedeutung
der katholische Kirche und ihrer Gnadenmittel dar. Überzeugend eiferten sie die
Zuhörer an immer treue Katholiken zu sein, die nicht bloß selbst den Glauben
betätigen, sondern auch stets für ihre kath. Kirche eintreten und für sie
streiten und kämpfen. Überaus groß war der Andrang beim Sakramentenempfang. Am
Schluss des Triduums zog eine feierliche, imposante Prozession durch den Markt,
bei der sich sämtliche Behörden, alle Vereine und die Pfarrangehörigen in sehr
großer Zahl beteiligten. Sollte doch diese Prozession der sichtbare, äußerliche
Ausdruck des großen, innigen Dankes dem lieben Gott gegenüber dafür sein, dass
die Pfarrei Ruhmannsfelden ein so schönes Gotteshaus bekommen hat, wie es die
Laurentius-Pfarrkirche Ruhmannsfelden in ihrem jetzigen, wohl gepflegten
Zustande ist. Am 10. August jeden Jahres begeht die Pfarrei Ruhmannsfelden das
Fest ihres Kirchenpatrons, des hl. Laurentius.

Möchten die Gläubigen der Pfarrei Ruhmannsfelden niemals vergessen, was unsere
Vorfahren für große Opfer gebracht haben, um ein schönes und würdiges Gotteshaus
ihr Eigen nennen zu können. Möchten sie allezeit daran denken, wie viele
Wohltaten und Segnungen dieses Gotteshaus schon vermittelt hat, dann werden sie
nicht bloß zu danken wissen, sondern alles daransetzen, dass die schöne
Laurentius-Pfarrkirche in Ruhmannsfelden für alle Zeiten wahrhaftig ist „domus
et porta coeli et vocabitur aula die“ das Haus Gottes und die Pforte des Himmels
und sein Name ist „aula die“, die Wohnung Gottes.xxxix



ANHANG

1 Anmerkung













