\documentclass[12pt,a4pager]{book}

\usepackage[ngerman]{babel}
\usepackage{fontspec}
\usepackage[sf]{titlesec}
\usepackage{libertine}
\usepackage{longtable}
\usepackage{paralist}
\renewcommand{\labelitemi}{$-$}
\renewcommand{\labelitemii}{$-$}

\renewenvironment{quote}
               {\list{}{\itshape\rightmargin\leftmargin}%
                \item\relax}
               {\endlist}


%\titleformat{⟨command⟩}[⟨shape⟩]
% {⟨format⟩}
% {⟨label⟩}
% {⟨sep⟩}
% {⟨before-code⟩}
% [⟨after-code⟩]

\titleformat{\chapter}[display]
  {\sffamily\bfseries\Large}
  {\filright\Huge\thechapter}
  {1ex}
  {\titlerule\vspace{1ex}\filleft}
  [\vspace{1ex}\titlerule]

  \renewcommand*\descriptionlabel[1]{\hspace\labelsep
                                \sffamily\bfseries #1}


\author{August Högn}
\title{Geschichte von Zachenberg}
\date{1951}

\begin{document}

\maketitle


AUGUST HÖGN

BRIEFE UND

DOKUMENTE

Meinen Eltern Josef und Anita Friedrich gewidmet


Mein Dank gilt:

Herrn Pfarrer Meier, Lotte Freisinger,

Martha Grotz, Herrn Rektor Roßmeißel, Franz Danzinger jun.



Textgrundlage:

Dokumente aus dem Pfarrarchiv: I.;

Reicheneder-Chronik:: II.; III.; IV.1., 4., 5.,  6.,  7.; V.; VI.1., 2.; VII.2.;

Schul-Chronik: VI.3., VI.4.; VII.1.;

Notenmaterial der Josephs-Messe: IV.2.; IV.3.;


Projekt August Högn Geschichtswerk

Ruhmannsfelden, 2003

1. Auflage zu 10 Stück

AUGUST HÖGN

1878 1961





BRIEFE UND

DOKUMENTE



























EDITIERT VON

JOSEF FRIEDRICH 2003





INHALTSVERZEICHNIS

INHALTSVERZEICHNIS  5

BRIEFE UND DOKUMENTE    9

.I Dokumente zur Kirchenmusik   9

.1 Bezirkshauptmann Heerwagen an die Kirchenverwaltung, 18.1.1895   9

.2 Max Rauscher an das Bischöfliche Ordinariat Regensburg, 4.11.1926    9

.3 Max Rauscher an Pfarrer Fahrmeier, 15.11.1926    10

.4 August Högn an den Kirchenrat, ohne Datumsangabe 11

.5 Protokoll der Kirchenverwaltungssitzung vom 30.12.1926   12

.6 Max Rauscher sen. an die Kirchenverwaltung, 14.3.1927    13

.7 Kirchenverwaltung an Max Rauscher jun., 26.3.27  15

.8 Aushändigungs-Nachweis für Max Rauscher jun., 27.3.1927  16

.9 Max Rauscher jun. an die Kirchenverwaltung, 29.3.1927    16

.10 Kirchenverwaltung an die Regierung von Niederbayern, 1.4.1927   16

.11 Kirchenverwaltung an die Regierung von Niederbayern, 22.4.1927  18

.12 Alois Hartl an den Kirchenrat, 15.6.1927    18

.13 Johann Bielmeier an die Kirchenverwaltung, 17.6.27  19

.14 Bischöfliche Ordinariat Regensburg an die Kirchenverwaltung, 28.6.1927  20

.15 Josef Brunner an die Kirchenverwaltung, 29.4.1953   20

.16 August Högn an Pfarrer Reicheneder, 25.1.1954,  21

.17 Pfarrer Reicheneder an August Högn, 6.2.1954    22

.II Dokumente zu den heimatkundlichen Zeitungsartikeln  23

.1 Dr. Keim an August Högn, 30.8.1922   23

.2 Staatsarchiv München an August Högn, 22.9.1924   23

.3 Hauptstaatsarchiv München an August Högn, 7.8.1925   25

.4 Pfarrer Oberschmid an August Högn, 8.9.1926  25

.5 Pater Wilhelm Fink an August Högn, 6.7.1932  25

.6 Expositus Hofmann an August Högn, 30.1.1950  26

.7 Bayerisches Hauptstaatsarchiv an den Marktgemeinderat, 28.2.1951     27

.8 Bayerisches Staatsarchiv Landshut an Herrn Bürgermeister Muhr, 9.3.1951  28

.III Dokumente zur „Geschichte von Zachenberg“  29

.1 Trellinger an August Högn, 25.2.1952 29

.2 Pater Wilhelm Fink an August Högn, 28.1.52   29

.3 Gotthard Oswald an August Högn, 30.1.52  30

.4 August Högn an das Kloster Niederalteich, 26.3.52    30

.5 Damian Merk an August Högn, 27.3.52  31

.6 August Högn an das Pfarramt Grafenau, 26.3.52    32

.7 Pfarrer Rankl an August Högn, 27.3.52    32

.8 Expositus Georg Hofmann an August Högn, 23.10.52 33

.9 August Högn an Pfarrer Max Schefbeck, 17.01.53   34

.10 Pfarrer Max Schefbeck an August Högn, 21.1.53   35

.11 Pater Wilhelm Fink an August Högn, 23.2.54  35

.12 August Högn an den Bürgermeister Bielmeier, 31.2.54 37

.13 Bürgermeister Bielmeier an August Högn, 7.5.1956    38

.14 Exzerpt von August Högn aus Fink, Östl. Grenzmarken 39

.IV Dokumente zur Josephs-Messe 40

.1 Viechtacher Bayerwald-Bote, 15.6.1953    40

.2 Deggendorfer Zeitung, 18.3.1957  40

.3 Plattlinger Nachrichten, ?   41

.4 Gustl Güdner an August Högn, 22.3.1957   42

.5 Viechtacher Bayerwald-Bote, 6.8.1974 42

.6 Regensburger Bistumsblatt, 24.8.1974 42

.7 Viechtacher Bayerwald-Bote, 4.10.1974    43

.V Dokumente zum 80. Geburtstag von August Högn 44

.1 Viechtacher Bayerwald-Bote, 2.8.1958     44

.2 Viechtacher Bayerwald-Bote, 5.8.1958 45

.VI Dokumente zum Tod von August Högn   46

.1 Viechtacher Bayerwald-Bote, 14.12.1961   46

.2 Viechtacher Bayerwald-Bote, 15.12.1961   46

.3 Elfriede Schlumprecht an Lehrer Schambeck, 3.1.1962  48

.4 Elfriede Schlumprecht an Rektor Langesee, 3.1.1962   48

.VII Sonstige Dokumente 49

.1 Viechtacher Bayerwald-Bote, 7.5.1959 49

.2 Artikel in der Reicheneder-Chronik über August Högn  49

.3 Auszug aus den Memoiren von Franz Danziger sen.  50

ANHANG  51

.1 Anmerkung    51





BRIEFE UND DOKUMENTE

.I Dokumente zur Kirchenmusik

.1 Bezirkshauptmann Heerwagen an die Kirchenverwaltung, 18.1.1895

N 69.               Viechtach, den 18. Januar 1895



Kgl. Bezirksamt

Viechtach



Betreff:



Verwertung des Meßner und

Chorregentdienstes in Ruhmannsfelden



An

das k Pfarramt

Ruhmannsfelden



In Erwiderung sehr geschätzter Zuschrift vom 7. d. M. beehre ich mich ergebenst
mitzuschreiben, daß nach der Erklärung des k. Distiktsschulinspektion Viechtach
II rein Rechtlich der Hilfslehrer Gratzl zum Chorregenten wegen seinervon
liturgischen Erfahrungen mehr  eignet als der Aushilfslehrer Krieger zum
zweifachen Chor und Organistendienste.

Bei  Weigerung des Lehrers Weig für die Dienste seitens des Chores eine
geeignete Persönlichkeit aufzustellen dürfte zunächst die kgl.
Distriktschulinspektion Viechtach II zur Einschreitung gegen Weig zuständig
sein.



der k. Bezirkshautmann



Heerwagen

.2 Max Rauscher an das Bischöfliche Ordinariat Regensburg, 4.11.1926

Ruhmannsfelden, 4. November 1926

An das

    Hochwürdigste bischöfliche Ordinariat

                    Regensburg.



Der ergebenst unterfertige Max Rauscher, Chorregent in Ruhmannsfelden gestattet
sich dem höchwürdigsten Ordinariat nachstehendes höfl. und vertrauensvoll zu
unterbreiten.

Ich habe meiner Kirchenverwaltung die Bitte unterbreitet meine Anstellungs- und
Gehaltsverhältnisse nach den Richtlinien der hohen Kirchenbehörden zu regeln.
Herr Pfarrer Fahrmeier hat mein Ersuchen abgelehnt. Er hat im Anschluß an seine
Ablehnung evt. Kündigung in Aussicht gestellt.

Mein heutiges Fixum ist monatlich Mk: 33.- Außerdem beziehe ich die Stdgebüren.
Ich bitte das hochwürdigste Ordinariat ebenso höflich wie dringend sofort eine
kirchenbehördliche Entscheidung zu fällen.

Ich habe von meiner Kirchenverwaltung die Gruppe VI der St. B. O., sowie Mk:
30.- Ortszulage zu beanspruchen in der Höhe von 180.- und Mk: 30,- = 210 Mk
monatlich. Kirchensteuer hat die Gemeinde Ruhmannsfelden noch nicht erhoben. Daß
die Kirchensteuer entrichtet werden muß, haben nicht die Pfarrer oder die
kirchlichen Behörden angeordnet sondern der bayerische Landtag durch das
religionsgesellschaftliche Steuergesetz vom 1. August 1923.

Ich bitte daher die Gehaltssache innerhalb 8 Tagen zu regeln. Sollte Herr
Pfarrer Fahrmeier von sich nichts hören lassen, so müßte ich die Sache
gerichtlich übergeben und meine bürgerlichen Rechte in Anspruch nehmen. Der
Ausgang der Rechtsstreites würde für mich auf alle Fälle der denkbar beste sein,
denn die bischöflichen Behörden haben selbst erklärt, daß es den Kirchenbeamten
frei steht, die Gemeinden zu Erreichung ihres Zieles vor den ordentlichen
Gerichten zu verklagen.

Möchte auch noch bemerken, daß die Gemeinde Ruhmannsfelden auch Parochiamen Wert
auf gute Kirchenmusik legt und mit meiner Leistung voll und ganz zufrieden ist.

Vertrauensvoll der sofortigen Bescheid der hohen Kirchenbehörde erwartend,
zeichne ich



mit vollster Hochachtung

ergebenst

Max Rauscher, Chorregent.

.3 Max Rauscher an Pfarrer Fahrmeier, 15.11.1926

Ruhmannsfelden, den 15.XI.26



Hochwürdiger Herrn Pfarrer Fahrmeier!



Über Ihre gestrige Zurrechtweisung, die ich gerade nicht vorbildlich nennen
kann, erlaube ich mir Ihnen nahe zu legen, daß - wenn Sie gegen mein Verhalten
am Kirchenchor Kritik üben wollen, sie jederzeit das Recht haben mich persönlich
rufen zu lassen.

Öffentlich vor meiner versammelten Sängerschar auffälligen Streit, oder Rügen zu
erteilen, verbiete ich mir und wiese selbe ganz energisch zurück.

Es wäre besser am Platze gewesen, diese zur Zeit von Lehrer Högn, in Anwendung
gebracht zu haben, nachdem damals nicht bloß Lektüre während des Gottesdienstes
gelesen, sondern von seiner Tochter Liebeleien getrieben und Schokolade gegessen
wurde. Wenn ich fremde Gäste anstandsgemäß begrüße dürfte Ihr Verhalten wie
allgemein unter den versammelten Mitwirkenden laut wurde, ganz unangebracht
gewesen sein.

Ich würde bei evtl. Wiederholung sofort Beschwerde an das Bischöfl. Ordinariat
einleiten.

Ferner erlaube ich mir die Steuerforderung vom Finanzamt beizulegen, nachdem es
mir mit dem derzeitigen Gehalt nicht möglich ist selbe zu bezahlen und auch kein
Recht besteht, daß von mir Steuer gefordert werden kann, solange meine
Gehaltssache nicht erledigt ist.

Wollen Sie das weitere mit dem Finanzamt selbst in Erledigung bringen.



Dies zur Kenntnis.



    Ergegenster



        Max Rauscher

            Chorregent

.4 August Högn an den Kirchenrat, ohne Datumsangabe



An den sehr verehrlichen

Kirchenrat der Pfarrei Ruhmannsfelden



Erw. Hochwohlgeboren!



Die ungezogenen Anschuldigungen des Max Rauscher jr. in dem Brief an H. H.
Pfarrer Fahrmeier drücken mir die Feder in die Hand mit der Bitte an den sehr
verehrlichen Kirchenrat die unqualiflizierbare Frechheit des Max Rauscher jr.
gegenüber H. H. Pfarrer Fahrmeier mit den dem Kirchenrat zu Gebote stehenden
Mitteln zu führen. Ein Mensch, der so rücksichtslos vorgeht wie Max Rauscher
jr., der verdient keine Berücksichtigung, ein Angestellter, der so vorgeht gegen
seinen Vorgesetzten, der wird an allen Orten u. zu allen Zeiten in allen Landen
von seiner Dienststelle entlassen.

In der Wahrung der Autorität u. im Ansehen unser hochverdienten u. beliebten H.
H. Pfarrer Fahrmeier ersuche ich den Max Rauscher zu kündigen, damit endlich
diesem unerhörten Treiben dieses jungen Mannes Einhalt geboten ist u. nicht
derselbe u. noch andere mit in der Selbstüberhebung u. Geringeinschätzung
anderer gestärkt werden. Das Verhalten beider Rauscher Söhne gegenüber H. H.
Rauscher wird allgemein aufs schärfste verurteilt, eine dementsprechende
Maßnahme seitens des sehr verehrlichen Kirchenrates von der Gesamtheit erwartet.



    Mit vorzüglicher Hochachtung!

        Ergebenster!

            August Högn

                Hauptlehrer

.5 Protokoll der Kirchenverwaltungssitzung vom 30.12.1926

P R O T O K O L L

über die

Sitzung der Kath. Kirchenverwaltung Ruhmannsfelden vom

30. Dezember 1926.



Gegenwärtig:

Der Kirchenverwaltungsvorstand

Fahrmeier

die Kirchenverwaltungsmitglieder

Fenzl

Kraus

Schwarz

Härtl

Zur Sitzung nicht erschienen:

____________

Zu der auf heute Vormittags 8 1/4 Uhr im Pfarrhof anberaumten Sitzung wurden die
sämtlichen Mitglieder der Kirchenverwaltung (Artikel 37 Absatz I der
Kirchengemeindeordnung) richtig geladen.

Von den Geladenen sind die Nebenbezeichneten erschienen.

Die Mitgliederzahl der Kirchenverwaltung nach dem Sollstande (Art. 37 Abs. I der
Kirchengemeindeordnung) beträgt 6; an der Beratung und Abstimmung haben die
Erschienen - 6 an der Zahl - , also mehr als die Hälfte aller Mitglieder
teilgenommen.

Auf Vortrag der Kirchenverwaltungsvorstandes und nach eingehender Beratung wurde
folgende Beschlüsse - soweit bei ihnen nichts anderes vermerkt –

e i n s t i m m i g und in öffentlicher Sitzung (KGO. Artikel 63 Absatz II,
Allgem VV. § 13 Absatz II) gefaßt:



Der mit dem Chorregenten Max Rauscher von Ruhmannsfelden am 14. Dezember 1924
aufgestellte Dienstvertrag wird ab 1. Januar 1927 gekündigt aus folgenden
Gründen:

a) Wegen des pflichtwidrigen und ungehörigen Betragens gegenüber Herrn Pfarrer
Fahrmeier - dieser  Kündigungsgrund wurde ohne Mitwirkung des Herrn
Kirchenverwaltungsvorstandes ausgesprochen - , ferner wegen seines nicht völlig
einwandfreien sittlichen Betragens.

b) wegen seiner jeder Rechtsgrundlage entbehrenden Beschwerde bei der Regierung
von Niederbayern u. bei dem Ordinariat Regensburg,

c) wegen übertriebener Gehaltsforderungen,

d) wegen nachgewiesener Überforderung von Stolarien.

Die Kirchenverwaltung ist bereit einen neuen Dienstvertrag ab 1. April 1927 im
März 1927 mit Herrn Rauscher Max abzuschließen unter folgenden Bedingungen:

1. In der Kirchenratssitzung Abbitte zu leisten,

2. auf einen neuentworfenen Dienstvertrag einzugehen.

V. g. u. u.



Genaue Kenntnis genommen

Am 30. XII. 26

Lt. U. (gez.)

Max Rauscher.



(gez.) Fahrmeier, Vorstand

   "     Fenzl,

   "     Kraus,

   "     Schwarz,

   "     Stadler,

   "     Härtl.

.6 Max Rauscher sen. an die Kirchenverwaltung, 14.3.1927

Ruhmannsfelden, den 14. März 1927



An die verehrl. Kirchenverwaltung hier!



Betreff.

Besoldungsfrage des Chorregenten.



Wenn ich mir in dieser Frage, als Vater des Chorregent, Stellung zu nehmen
erlaube, so wollen die Herren Hösl entschuldigen, denn ich erkenne mit Recht,
hierin seit 2 1/2 Jahren gewißen Einblick in diesen Bereich genommen zu haben +
meine finanziellen Leistung zur Hebung unserer Kirchenmusik dürften weit über
den der Kirchenratsmitglieder stehen.

An der Auflösung des früheren Dienstleistungsvertrag wäre ja an sich selbst
nichts zu beanstanden, da er sich schon Anfangs als wertlos erwiesen hat, doch
bin ich über dessen Begründung noch nicht ganz im Klarem, nachdem ich eine
Forderung um gewollte Besold als Grund nicht ansehen kann, somit dieses Urteil,
daß sehr viel Ähnlichkeit mit Versailes zeigt, mir bekanntlich Kläger + Richter
in einer Person fungirten, wirklicher Reformierung bedürftig erscheint.

Ich muß offen gestehen, daß ich über das Verhalten der Kirchenratsmitglieder in
dieser Sache sehr enttäuscht bin. Einige ältere Kirchenratsmitglieder werden
sich noch errinnern können, daß z. Z. Herrn Lehrer Weig unser Kirchenchor ein
guten Namen hatte. Es kam dann die Zeit der Anna Auer, wo R-felden den
Niedergang der Kirchenmusik erlebte, der unseren Ort wahrlich nicht zum Ruhme
gereichte.

Es ist deshalb sehr zu begrüßen, daß selbe durch Herrn Högn eine Verbesserung
erfuhr.

Hat man bei Übernahme dieses Posten meinen Sohn gegenüber, wenig Vertrauen
gesetzt, so glaube ich heute sagen zu dürfen, daß er seine Aufgabe voll + ganz
erfüllte, die im als Berufschorregent zur Pflicht auferlegt ist + die es möglich
machte unseren Kirchenchor in vollendeter Höhe zu sehen.

Gestatten Sie mir die Frage, hat nun dies zu ermöglichen, auch der Kirchenrat
seine Bestand beizutragen?

Ich muß es mit - Nein - beantworten, sonst wäre es nicht möglich, daß das bisher
geleistete so schlechte Würdigung ungeeignete Besoldung gefunden hätte.

Wo Pflichte sind, müssen auch Rechte bestehen + wenn der Chorregent die Pflicht
hat, das beste in Kirchenmusik für eine Marktgemeinde zu leisten, auf die es ein
Anrecht hat, so dürfte es auch Plicht der verehrl. Kirchenverwaltung sein, den
Gehalt eines Chorregent nicht auf die Stufe eine Polizeidieners zu stellen.

Es zeigt sich, daß bei uns das sozialle Verständnis fehlt, möcht ich hierbei nur
an einer Oster- oder Weihnachtsaufführung errinnern, die uns jedesmal mit Ankauf
von neuen Messen über M 100 zu stehen bekommt, daß genau dem Gehalt von 1/4 Jahr
entsprach.

Was sonst an direkten Auslagen enstehen, will ich gar nicht erwähnen, obwohl
selbe schon oft bedeutend in die Waagschale fielen.

Oder denkt die verehrl. Kirchenverwaltung, daß eine Zahl von 30 - 35
Mitwirkenden, so ohne alles seinen Dienste anbietet? - Nein - einen Chor
hierorts auf der Höhe zu halten, stellt ganz andere Forderungen + sind unsere
wenigen Kräfte oft schneller verscheut als gewonnen, was sich bekanntlich kurz
vor Weihnachten zeigte.

Ist dem verehrl. Kirchenverwaltung auch bekannt, wie notwenig es ist, neue
Kräfte für den Chorgesang auszubilden.

Was dann? - wenn die Frl. Raster einmal dem Gesang den Rücken kehren?

Mein Sohn wäre auf Ausbildung besorgt gewesen. Allein ich konnte meiner Frau
nicht zumuten wöchtenlich 2 mal den Reinigungsdienst bei Gesangsstunden der
Jugend zu übernehmen, wenn weder Dank noch Entschuldigung von Seiten der Eltern
noch Behörden vorhanden ist.

Unsere Rückständigkeit im Orte ist überall hinreichend bekannt. Deshalb meine
ich, ist es besondere Pflicht das Erreichte nicht wieder verkommen zu lassen.
Ich, wie meine Söhne, haben bisher alle Opfer gebracht, um Ruhmannsfelden ich
welt + kirchliche Musik zu heben. Wen es leider zu Unmöglichkeit gemacht wird,
fällt die Schuld auf jene, die auf eine Verbesserung dieser Zustande keinen
Anteil haben wollen.

Anfügend nur davon ein Beispiel.

Mein Sohn Siegfried der allsontäglich in seiner zuständigen Kirche Münchens beim
Gottesdienst mitwirkt, wurde vom dortigen Pfarrer M 20 angeboten, wenn er die
Weihnachttage seine Mitwirkung zusagen würde. Nein - er lehnte dankend ab, mit
der Begründung in seinem Heimatort helfen zu müssen.

Und in welcher Form in hier der Dank gebracht? Er wollte keinen Dank, was aber
glauben die Herren, welchen Eindruck auf ihn die Mitteilung machte, das man am
Jahrerschluß, wo es sonst in hochtönenden Worten üblich war, den Mitwirkenden am
Kirchenchor zu danken, seinen Bruder die aufopfernde Mühe, Kosten und Arbeit
eines Neujahrswunsch, in Form eines Mißtrauensvotum + Degradierung brachte,
gegen die er sich nicht einmal verteidigen konnte, da er schon vor vollendeter
Tatsachen gestellt war.

Wenn man auf der einen Seite durch das Getöse von Herrn Amberger, einer
Übertretung der festgelegten Taxe glaubt feststellen zu können, was in
Wirklichkeit nicht der Wahrheit entspricht, warum sieht man auf der anderen
Seite nicht die vielen Aushilfen von Beerdigungen und sonstigen Verrichtungen,
die dem Chorregenten durch Nichtbezahlung entgehen?

Ich meine, wo ein Wille ist, ist auch ein Weg zu finden, wodurch der Chorregent
in der Lage versetzt wird, mit voller Kraft mit gerechter Besoldung in seinem
Amt zum Ansehen unseres Orts arbeiten können.



    In aller Hochachtung

        zeichnet

        M. Rauscher

            Conditor



NB. Eine Erklärung meines Sohns dürfte sich nach oben angeführten erübrigen.

.7 Kirchenverwaltung an Max Rauscher jun., 26.3.27

Ruhmannsfelden, den 26. 3. 27

Kath. Kirchenverwaltung Ruhmannsfelden.



An Herrn   M a x   R a u s c h e r   jun.



R U H M A N N S F E L D E N.



Betreff:



Vollzug des Kirchenverwaltungsbeschlusses vom 30. 12. 26.



Nachstehend bringen wir Ihnen den Kirchenverwaltungsbeschluß im Teilauszug zur
Kenntnis und Darnachachtung:



".................

Die Kirchenverwaltung ist bereit, einen neuen Dienstvertrag ab 1. April 1927 im
März 1927 abzuschließen unter folgenden Bedingungen:

1. in der Kirchenratssitzung Abbitte zu leisten,

2. auf einen neu entworfenen Dienstvertrag einzugehen."

Im Vollzug dieses Beschlusses findet am Sonntag den 27. März 1927 Nachmittags
Kirchenratssitzung im Pfarrhofe statt. Sie werden hiermit geladen, zu dieser
Sitzung pünklich um 3 1/4 Uhr Nachmittags zu erscheinen.

Sollten Sie der Vorladung nicht Folge leisten und zudem nicht ob Ihres
Verhaltens dem Kirchenverwaltungsvorstand gegenüber und wegen Ihrer
herausfordernden Schreibereien an die Kirchenverwaltung und andere Behörden vor
den versammelten Kirchenratsmitgliedern öffentlich Abbitte leisten, so würde die
Kirchenverwaltung die Angelegenheit als erledigt betrachten und die
Chorregentenstelle ab 1. April 1927 anderweitig neu besetzten.



.8 Aushändigungs-Nachweis für Max Rauscher jun., 27.3.1927

A u s h ä n d i g u n g s - N a c h w e i s



Der Unterzeichnete bestätigt, heute ein Schreiben von der Kath.
Kirchenverwaltung Ruhmannsfelden ausgehändigt erhalten zu haben.



Ruhmannsfelden, den 27. März 1927



Max Rauscher, Chorregent.

.9 Max Rauscher jun. an die Kirchenverwaltung, 29.3.1927

Ruhmannsfelden, den 29.III.1927



An die Kirchenverwaltung hier!



Die juristische Prüfung, meiner an die Kirchenverwaltung gerichteten Briefe hat
ergeben, daß sämtliche einwandfrei von Beleidigung sind.

Es besteht somit die Lösung meines Dienstvertrags zu Unrecht, nachdem ich mir in
meinem Dienste nichts zu schulden kommen ließ.

Ich kann schon heute, die Verwaltung darauf Aufmerksam machen, daß ich mein
Recht, nach Ablauf von 8 Tagen durch Klage am Amtsgericht Viechtach erzwingen
werde.

Wir werden dann sehen, ob das Kirchenverwaltungsmitglied Herr Härtl dem
juristischen Wissen & Rechten standhält.



Max Rauscher

Chorregent.

.10 Kirchenverwaltung an die Regierung von Niederbayern, 1.4.1927

Ruhmannsfelden, den 1. April 1927

Katholische Kirchenverwaltung

Ruhmannsfelden.



An

die Regierung von Niederbayern

in

    L A N D S H U T



    Betreff: Neubesetzung der Chorregentenstelle.



./. 3 Beilagen

Chorregent Max Rauscher von Ruhmannsfelden hat bei der Regierung sowohl als beim
Bischöflichen Ordinariate Regensburg Beschwerde erhoben gegen die
Kirchenverwaltung wegen zu geringer Besoldung für seine Leistungen auf dem
Kirchenchore. (Beilage 1) Nebenbei bemerkt betrug die Entlohnung im Jahre 1926
ca 1400 M bei durchschnittlich 5 stündiger Arbeitsleistung in der Woche, die
widerrechtlichen Überforderungen nicht mit eingerechnet.

Die Entscheidung der kirchlichen Oberbehörde, wonach kein anderer Ausweg sich
finden lasse, als den Dienstvertrag zu lösen liegt bei. (Beilage 2)

Aus diesem Grunde und noch anderen Gründen, wie aus der Protokollabschrift des
Kirchenverwaltungsbeschlusses ersichtlich ist, hat die Kirchenverwaltung den
Dienstvertrag mit Rauscher ab 1. Januar 1927 gekündigt, ihm aber noch die
Möglichkeit geboten, ab 1. April einen neuen Dienstvertrag einzugehen unter
folgenden Bedingungen:

    1.) In der Kirchenverwaltungssitzung Abbitte zu leisten.

    2.) einen neu entworfenen Dienstvertrag einzugehen. (Beilage 3)

Rauscher wurde zur Sitzung frühzeitig geladen und ist zu derselben rechtzeitig
erschienen. Er weigerte sich trotz dreimaliger Anfrage Abbitte zu leisten. Zudem
war sein Benehmen so ungeziemend, daß dies allein schon hinreichend gewesen
wäre, keinen neuen Vertrag mehr mit ihm abzuschließen. Die Angelegenheit war
hiemit für die Kirchenverwaltung erledigt, Rauscher entlassen. Die
Kirchenverwaltung könnte auch mit Rauscher nie mehr sich einigen. Sie weiß sich
gedeckt durch die ganze Kirchengemeide.

Die Kirchenverwaltung stellt nunmehr an die hohe Regierung von Niederbayern die
ergebenste und dringende Bitte:

"Es wolle gnädigst genehmigt werden, daß die Chorregentenstelle in
Ruhmannsfelden bis auf weiteres von einem Lehrer übernommen werden dürfe."

In Betracht könnte nur Hauptlehrer Högn kommen, der allein die zu Übernahme des
Kirchenchores notwendigen musikalischen Fähigkeiten und Kenntnisse besitzt und
vor Rauscher schon 3 Jahre den Kirchenchor zu größten Zufriedenheit der ganzen
Kirchengemeinde versehen hat.

Der unterzeichnete Vorstand erklärt sich bereit, bei dienstlicher
Inanspruchnahme des Herrn Hauptlehrer Högn während der Schulzeit dieselbe mit
Religionsunterricht auszufüllen. Bei levit. Leichen, die äußerst selten sind -
schon volle 8 Monate war keine mehr - würde der Ausfall der Stunden durch
gewissenhafte Nachholung derselben wieder hereingebracht werden.

Die politische Gemeinde Ruhmannsfelden sowohl, als auch die größtenteils hier
eingepfarrte Gemeinde Zachenberg schließt sich der Bitte der Kirchenverwaltung
an; erstere verspricht sobald die Verhältnisse es gestatten eine Wohnung bereit
zu stellen und es so zu ermöglichen, einen pensionierten Lehrer als Chorregent
anzustellen. Andere Chorregenten könnten angesichts der mißlichen finanziellen
Lage der Kirche wie auch der Kirchengemeinde nicht in frage kommen.

Die Kirchenverwaltung wie auch die beteiligten Gemeinderäte sehen mit vollem
Vertrauen der wohlwollenden Gewährung ihrer vorgetragenen Bitte entgegen.



Der Kirchenverwaltungsvorstand

.11 Kirchenverwaltung an die Regierung von Niederbayern, 22.4.1927

Ruhmannsfelden, den 22. April 1927

Kirchenverwaltung Ruhmannsfelden



An die Regierung von Niederbayern

Kammer des Innern in

Landshut



Betreff: Nebenämter und Nebengeschäfte der Volksschullehrer.



Zur Regierungsentschließung v. 9.4.27, Nr. 2306 g 52



Der Kirchenchor Ruhmannsfelden ist ab 1. Mai ohne Chorregenten. Die
Kirchenverwaltung stellt daher an die hohe Regierung die ergebenste und
dringende Bitte:

"Es wolle gütigst genehmigt werden, daß Herr Hauptlehrer Högn von Ruhmannsfelden
die Führung des Kirchenchores bis zu den großen Ferien weiterbehalten darf."



Die Chorregentenstelle Ruhmannsfelden wird demnächst zur Bewerbung öffentlich
ausgeschrieben werden.

Der wohlwollenden Gewährung der vorgetragenden Bitte wird vertrauensvoll
entgegengesehen.



Der Kirchenverwaltungsvorstand:

.12 Alois Hartl an den Kirchenrat, 15.6.1927

Ruhmannsfelden 15. Juni 1927



Einlauf: An den Kirchenrat Ruhmannsfelden



Ich stelle hiermit Antrag den früheren Chorregenten Max Rauscher dahier wieder
einzustellen & seine Ansprüche die er macht, welche auch voll & ganz berechtigt
sind zu genehmigen.



Begründung: Max Rauscher ist & wahr der Beste Chorregent, dem ich seit
Lebenszeit kennen lernte, derselbe führt eine Kirchenmusik die noch nicht da war
in Ruhmannsfelden & und auch nicht mehr kommen kann, denn ein Mann wie Rauscher
der Tag & Nacht übte & lernte & nur am Fortschritt war, macht ein anderer nicht,
geschweige erst ein alter Pensionist, wann es Leute gibt die glauben ein
Chorregent hat bloß 1 - 2 Stunden täglich Arbeitszeit, ist also Grundfalsch &
somit belöhnt zu werden, scheinbar soll der Chorregent die andere Zeit
Taglöhnerarbeit verrichten, oder sonstiges welches ihm von seinem Musikalischen
Übungen unterbinden würde, es ist also auch nicht & fast gar nicht zum rechnen
wann eine jährliche Umlage erhoben wird, die höchstens für die mittleren
Steuerzahler 15 Pfg. ausmacht, wann die Kirchenverwaltung mit diesem kleinem
Sparsinn anfängt, einem eifrigen tüchtigen einheimischen Chorregenten was ich
kaum glaube nicht zu gönnen geneigt ist, hört sich doch alles auf.

Freilich sind die Leute geschreckt wann es heisst es muss eine Umlage erhoben
werden, wann es aber gesagt wird, daß es nicht der Mühe wert ist davon zu reden
wegen 15 Pfg. dann streibt sich kein Mensch dagegen, wie auch in Viechtach
nicht. Habt Ihr Herren schon gehört wie es in Böbrach zugeht, habt Ihr Herren
schon bemerkt, daß bei uns nicht mehr soviel in die Kirche gehen, der Volksmund
sagt dem Rauscher ist keiner gewachsen &  ist auch die Wahrheit, wir waren an
der Höhe hier in Ruhmannsfelden mit der Kirchenmusik, soll es jetzt wieder
abwärts gehen? Nein Musik & Gesang ist eine Kunst wir verlangen Rauscher wieder
in seine frühere Stellung eingesetzt, mein Antrag ist rein Objektiv, ich habe
mit Rauscher kein Wort davon geredet & kann es auf Eid versichern, wir
verzichten auf einem fremden Chorregenten, wir haben selbst einem besseren & das
ist & bleibt Rauscher!



Hochachtend!

Alois Hartl

.13 Johann Bielmeier an die Kirchenverwaltung, 17.6.27

Ruhmannsfelden, den 17. Juni 1927



An die Kirchenverwaltung Ruhmannsfelden



Ich bringe hiermit einen Antrag Rauscher wieder als Chorregenten einzusetzen,
zwar deshalb weil Rauschen ein ganz hervorragender Musik u. Gesangtechnisch gut
ausgebildeter Mann ist. Er wäre traurig wenn man denselben für seine Leistung
auch nicht belohnen könnte, was überhaupt für die Steuerzahler in der
Pfarrgemeinde nur ein kleines Almosen ist, eine Umlage nur im Namen etwas
ausmacht, aber der Geldbeutel davon nichts kennt, wegen dieser paar Pfennige
jährlich.

Ich kann ihnen sagen ganz aufrichtig, ich bin ein Musikfreund zumal für die
Kirche, aber bedauern muß ich, wenn es wieder zurück geht, statt vorwärts! & das
kommt, denn ein alter Pensionist, hat doch diesen Eifer nicht so wie ein Junger
meines Erachtens. Bemerke auch, was mir eine Chorsängerin sagte bei einem Besuch
dahier. sie hätte auch gerne Lust auf den Chor hier zu singen, weil ein junger
tüchtiger Regent waltet, aber mit einen alten Regenten will sie nichts anfangen
weil der Eifer fehlt. Die Chorsängerin welche Alt & Sopran singt & zwar schon
acht Jahre, kommt wahrscheinlich zu mir, wenn meine Tochter heirtatet, weil es
eine Schwester ist zum Bräutigam & ich versichere heute schon, dieselbe
meineswegen zu Dienst steht, aber ob dieselbe bei einem alten Chorregenten
mithilft glaube ich kaum. Ich ersuche nun die Kirchenverwaltung meinen Antrag zu
billigen, welchen auch der Volksmund spricht.



Grüße Hochachtend!

Joh. Bielmeier

.14 Bischöfliche Ordinariat Regensburg an die Kirchenverwaltung, 28.6.1927

Ad.Num:Exh. 6565.                               Regensburg, den 28. Juni 1927.



D a s

B i s c h ö f l i c h e   O r d i n a r i a t



R E G E N S B U R G



Dem Vater des entlassenen Chorregenten Max Rauscher ist auf seine Eingabe vom
16./23. ds. Mts. durch Übermittlung dieser Entschließung im Original alsbald zu
eröffnen, daß Chorregent Max Rauscher nicht durch oberhirtliche Stelle, sonderen
durch die Kirchenverwaltung entlassen wurde, und da die oberhirtliche Stelle
gegen das Vorgehen der Kirchenverwaltung keine Erinnerung hat, weil der
Entlassene bezüglich seine Entlohnung unerfüllbare Forderungen stellte und zum
vorgesetzten Pfarrvorstande in ein dienstlich unerträgliches Verhältnis sich
setzte.



gez. Kiefl.

.15 Josef Brunner an die Kirchenverwaltung, 29.4.1953

Josef Brunner                   Achslach 29.4.1953

Chorregent

Achslach



An die

Kirchenverwaltung Ruhmannsfelden



Betrifft. "Bewerbung als Chorregent in der Pfarrei Ruh´felden"



Wie ich aus maßgebenden Personenkreis erfahre, will Herr Rektor Högn bei
Amtsantritt eines neuen Hochw. H. Pfarrers sein Amt als Chorregent aus
gesundheitlichen Gründen niederlegen.

Da ich als Chorregent in Achslach schon länger die Absicht hege, mich um einen
größeren Wirkungskreis umzusehen, will ich diese Gelegenheit nicht verpassen und
möchte mich auf diese Weise um die Stelle als Chorregent in der Pfarrei
Ruh´felden bewerben. Möchte aber ausdrücklich betonen, das ich die Stelle erst
bei einer völlig freiwilligen Kündigung durch H. Rektor Högn zu übernehmen
gewillt bin.

Sollten Sie an dieser meiner Bewerbung interreßiert sein, so würde ich Sie
freundlich ersuchen, mir einen kurzen Bescheid zukommen zu lassen. Außerdem wäre
ich, wenn Sie es für notwendig betrachten, zu einer persönlichen Aussprache
jederzeit bereit.



Hochachtungsvoll Jos. Brunner

.16 August Högn an Pfarrer Reicheneder, 25.1.1954,

Ruhmannsfelden, den 25. Januar 1954



Sehr geehrter H. H. Pfarrer Reicheneder!



Gestern Sonntag mittags besuchte mich mein lieber Nachbar H. Rudolf
Schwannberger. U. a. kamen wir auch zu sprechen auf die derzeitigen Verhältnisse
auf den hiesigen Kirchenchor.

Sie wissen es, daß ich es selbst am lebhaftesten bedauert habe, daß ich
unmittelbar vor dem Weihnachtsfeiertagen krank wurde u. daß dies der einzige
Grund war, warum ich nicht mehr weitermachen konnte auf dem Kirchenchor. Eine
andere Ursache bestand bei dieser Sache für mich bestimmt nicht. Alles Andere -
z. B. Kündigung seitens des H. H. Pfrs., udgl. ist unrichtig u. unwahr u.
höchstens ein blödes Weibergeschwätz. Ich habe ja den Kirchenchordienst nur von
Tag zu Tag aushilfsweise gemacht. Wohl habe ich oft u. oft H. H. Pfr. Bauer
gebeten - auch im Beisein von H. H. Koop. Neugebauer - er möge doch vom nächsten
Tage an einen anderen Chorregenten bestimmen, da ich mich voll u. ganz in den
Ruhestand begegen möchte. "Kommt nicht in Frage," so hieß es da immer. Nach dem
Tode des H. H. Pfr. Bauer habe ich H. H. Koop. Huber - der damals Pfarrprovisor
war - ersucht, einen anderen Chorregenten für den hiesigen Kirchenchor zu
bestimmen. Diese Bitte wurde auch abgelehnt. Ebenso habe ich Sie, H. H. Pfr.
Reicheneder, nach Ihrer Ankunft hier in Ruhmannsfelden davon verständigt, daß
ich den Kirchenchordienst hier in Ruhmannsfelden weiterhin nicht mehr versehen
kann. Auf Ihr Ersuchen hin habe ich aber - wenigstens für kurze Zeit - wiederum
zugesagt. Nun aber muß ich Sie, H. H. Pfr. Reicheneder herzlichst bitten davon
Kenntnis zu nehmen, daß ich von meinem 20. Lebensjahr bis zum 76. auf den
verschiedensten Kirchenchören der beiden Diözesen Passau u. Regensburg tätig war
als Sänger, als Organist u. als Chorregent. Das dürfte reichen zu der
berechtigten Forderung auf Ruhestand auch im Kirchenchordienst. Es wird wohl im
ganzem Pfarrsprengel Ruhmannsfelden u. weit hinaus niemanden geben, der da
anderer Ansicht sein könnte. Und die ewigen Nörgler u. Bessermacher im
Ruhmannsfelden sehen jetzt ihren Wunsch erfüllt, das sie jetzt endlich einen
anderen Kirchenchor haben, der ihre kirchenmusikalische Wünsche u. Ansprüche
besser befriedigen dürfte als es unter Rektor Högn der Fall war. Mein Alter
(geb. 2.8.78) u. meine 55 jährige Tätigkeit auf dem Gebiete der Kirchenmusik
bestimmen mich, daß ich von nun an mich nicht mehr beteilige u. bestätige auf
dem Kirchenchor, damit ich von dem wohlverdienten Ruhestand während der kurzen
Lebenszeit, die mir noch beschieden sein mag, auch ein klein bischen was habe.





Möchte noch befügen, daß eine vertragliche Abmachung über Kirchenchordienst
zwischen Pfarramt Ruhmannsfelden u. mir niemals bestanden hat.



Ersuche H. H. Pfarrer Reicheneder von Vorstehenden gütigst Kenntnis nehmen zu
wollen.

Mit vorzüglichster Hochachtung

Ihr Ergebenster!   A. Högn



Habe von Ehrw. Frau Oberin erfahren, daß Sie kränklich sind. Wünsche Ihnen recht
baldige Genesung.

.17 Pfarrer Reicheneder an August Högn, 6.2.1954

Ruhmannsfelden, den 6.2.1954



Sehr geehrter Herr Rektor!



Mit aufrichtigem Dank bestätige ich den Empfang Ihrer geschätzten Zeilen vom 25.
Januar 1954.

Wiederholt hätte ich Sie gerne besucht, um mich nach dem gesundheitlichen
Zustand des H. Rektors zu erkundigen, doch leider konnte ich infolge einer
Stimmbandlähmung nicht mehr sprechen und so war es nicht möglich. Auch jetzt ist
es noch nicht so, wie es sein soll, aber immerhin schon wieder bedeutend besser,
sodass ich hoffe Ihnen bald wieder einen Besuch anstatten zu können. Ich hoffe
und wünsche, daß Ihre Gesundheit bis dahin weiterhin sich bessere.

Vom Übrigem Inhalt Ihres werten Schreibens habe ich Kenntnis genommen. Ich
bedauere diese Tatsache zwar sehr, kann aber Ihren Standpunkt voll und ganz
verstehen und möchte daher, zunächst auf diesem Wege, bis mir ein persönlicher
Besuch möglich ist, für Ihre Tätigkeit auf dem hiesigem Kirchenchor, sowohl vor
der Zeit meines Hierseins, wie auch bes. für die Monate meines hiesigen Wirkens
meinen und der Kirchenverwaltung aufrichtigen Dank zum Ausdruck bringen. Möge
Ihnen der Herrgott vergelten, was Sie so zu seiner Ehre und zum Besten der
Pfarrgemeinde geleistet haben, bes. dadurch, daß er Ihnen bald wieder recht gute
Gesundheit schenkt und Ihnen den erwünschten Ruhestand noch recht lange genießen
läßt.



Mit diesem aufrichtigem Wunsche

in aller Hochachtung

und mit den besten Grüßen

in Dankbarkeit



Ihr



Pfr.

.II Dokumente zu den heimatkundlichen Zeitungsartikeln

.1 Dr. Keim an August Högn, 30.8.1922

Straubing, den 30.8.22

Sehr geehrter Herr Hauptlehrer!



Die Formen Rumarsfelden, Rudmarsfelden, die der Volksmund bestätigt, tragen den
Stempel der Originalität.

Die erste Hälfte ist der Personenname Hrotmar (Hruotmar, Romar).

Hrot(i) = Ruhm, Sieg. --mar(u) = berühmt. Also: der Siegberühmte. Ich glaube,
daß der Ort bei Gründung des Klosters Gotteszell schon längst vorhanden war und
halte den Mann, der dem Ort den Namen gegeben hat, für einen Dienstmann der
Grafen von Bogen, die ja Rodungen veranstaltet haben. Wir kämen so in das 11.
bis 12. Jahrhundert und dürfen die Entstehung Ruhmannsfelden um 1100 ansetzen.
Mehr läßt sich einstweilen nicht sagen.



Mit vorzüglicher Hochachtung ergebenst



Dr. Keim.

.2 Staatsarchiv München an August Högn, 22.9.1924

Nr.1390 Bay. Hauptstaatsarchiv München           München, den 22. Sept. 1924



An Herrn Hauptlehrer Högn in Ruhmannsfelden, Bayer. Wald.



Betreff: Marktrecht und Wappen von Ruhmannsfelden



Zum Schreiben vom 15/18. Sept. 1924.



Was die Führung des Markttitels und eines Wappens von Ruhmannsfelden anbelangt,
vermag es zwar keine erschöpfende, aber immerhin wesentlich aufklärende Antwort
gegeben zu werden.

Jakob der Rueerer stellte am 26. April 1416 eine Urkunde aus, in welcher sich
"dy czeitt Richter dez Markchts zue Ruedmarnsfelden" nennt. Es dürfte kaum ein
Zweifel bestehen, daß der Urkundenaussteller als Landesherrlicher Richter über
die Markteigenschaft seines Wirkungsortes Bescheid wußte. Wir sind deshalb
berechtigt, die Markteigenschaft zu Ruhmannsfelden schon seit dem Beginn des 15.
Jahrhunderts in Anspruch zu nehmen.

In einem Aldersbacher Codex v. J. 1452 ist die Rede von dem forum Rudmansfelden,
also Markt während in einer Urkunde vom 2. April 1475 das opidum Rudmansfelden
erscheint, was mehr an die befestigte Siedlung als an den Markt gemahnt.

Auf Bitten der "Burger unnsers Margkts zu Rudmansfelden" tuet ihnen Herzog
Albrecht IV. von Bayern-München die Gnade: "....Fryen sie auch wissennlieh in
crafft des briefs, Also das ayvnd all Ir Nachkomen, sich aller der gnaden und
freihait geprauchen vnd nyessen mügen vnd die haben sollen, In allermass als
annder vnnser Märkt, In Nidern Bairn, von vuunsern vordern gefreyt sein." Das
Privileg ist nur in Abschrift erhalten und undatiert, steht aber zwischen zwei
Urkunden desselben Jahres 1469 und darf daher als aus diesem Jahre stammend
angenommen werden.

In einem Literale des Klosters Gotteszell vom Jahre 1566 - 1602 kommen vor die
"Geschwornen des Rate und Markts R.", "Rat und Gemein des Markts R." ,wie
überhaupt seit der Begnadigung von 1469 keinerlei Zweifel an dem Marktrechte
Ruhmannsfelden mehr aufkommen kann.

Das heutige Wappen des Marktes Ruhmannsfelden weist "in Rot ein von Silber und
Blau in 2 Reihen geweckten Schrägrechtsbalken" auf. Zu dieser von O. Hupp (die
Wappen und Siegel der deutschen Städte, Flecken

und Dörfer. Frankfurt a. M. 1912, S. 84) gegebenen Beschreibung fügt der
Verfasser noch erläuternd hinzu: "Es ist gar kein altes Siegel bekannt geworden,
sodaß es fraglich ist, ob das beschriebene Wappen, das die Bürgermeistermedaille
und ein nach dieser gefertigtes Magistratsiegel zeigen, richtig ist, oder ob das
Wappen, das Mueelich, Apian und das Wappenbuch der Landschaft bringen, nämlich
in blau unter 2 schräggekreuzten silbernen Hirtenstäben eine weisse Rübe mit
grünen Blättern". Dieses letztere Wappen ist ein sog. redendes Wappen, welches
ohne jede Autorisierung längere Zeit gebraucht worden zu sein scheint. Im 16.
und 17. Jahrhundert hat man vielfach Rub- (Rueb-, Rüb--, Rüeb-) mannsfelden
geschrieben und hat dafür die Rübe als Wappenbild angenommen; die Hirtenstäbe
würden lediglich als Dekoration anzusehen sein. Es ist abgebildet als Nr. 641 in
Philipp Apians Wappensammlung der altbayerischen Landschaft wie des zu seiner
Zeit abgegangenen Adels. (Oberbayerisches Archiv XXXIX, 471 - 498). Als es um
1650 an der Kirche zu Ruhmannsfelden ohne jede Genehmigung angebracht worden
war, erhob dagegen P. Gerard, Abbt bei dem Kloster Gotteszell Protest: "Die
Ruedtmannsfelder haben mit ihrer unnernünftigen Rueben auff dem khüss an
Verstandt, ganz vermesslich vnd vuuernuenftig ia vnuerantworttlich wider mich
und das Closter gehandlet.... Wer hat ihnen ainmal ain wappen zue, fueren
erlaubt? Vnd wan sie gleich wappenmessig waren, wer hat ihnen erlaubt solches
auf der Kirchen spesa auff einen offnen thurn mallen zuelassen, alwo ihnen
ainiges Recht und herschaft nit zuestehet vnd gebiertt? Vnd was ist das für ein
Verstandt ia phantastische Einbildung ein Rueben auf einem Küss?"

Schade, daß es Fortsetzung der Korrespondenz über das Wappen und seine
Anbringung der Kirche nicht vorhanden ist.



I. V. Unterschrift.

.3 Hauptstaatsarchiv München an August Högn, 7.8.1925

E. Nr. 1408                 München, den 7.8.1925

Bay. Hauptstaatsarchiv



An Wohlgeb. Herrn Hauptlehrer Högn in Ruhmannsfelden, bay. Wald.



Betreff: Wallfahrtskapelle Osterbrünnl bei Ruhmannsfelden; hier die 1550 von
Hans v. Milloph, in Regensburg gegossene Glocke. z. Ges. vom 24./25. vor. Mts.



Die hierorts gepflogenen Nachforschungen über eine im Jahre 1550 von Hans v.
Milloph in Regensburg gegossene, nunmehr in der…. (weiter leider nicht
überliefert)

.4 Pfarrer Oberschmid an August Högn, 8.9.1926

Straubing, den 8.9.1926



Hochwohlgeb. Herrn Hauptlehrer Högn, Ruhmannsfelden.



Betreff: Die Glocke in der Kirche Osterbrünnl.



Es gereicht mir zu großem Vergnügen, Euer Hochwohlgeboren den gewünschten
Aufschluß über den Meister der Glocke im Osterbrünnl an der Hand der mit Ihrer
gütigen Beihilfe selbst am letzten Freitag 3.9. aufgenommenen Schriftpause mit
aller, Zuverlässigkeit geben zu können. Wie Sie ja aus der beigegebenen
Pausenkopie ersehen und Jedermann klar beweisen können, ist die Glocke ein Werk
des seinerzeit hochberühmt gewesenen und noch jetzt mit zahlreichen Glocken im
weitesten Umkreis vertretenen Meisters Hans Durnknopf zu Regensburg, 1550.

Zu allen mir möglichen Aufschlüssen jeder Art gerne bereit, beehre ich mich
unter besten Grüßen zu zeichnen Euer Hochwohlgeboren ganz ergebener



Joseph Oberschmid, Pfr.

.5 Pater Wilhelm Fink an August Högn, 6.7.1932

Metten, 6.7.1932



Sehr geehrter Herr Oberlehrer!



Besten Dank für Ihren Brief. Halten Sie an dem fest, was Dr. Keim und ich über
Namen und Entstehung von Ruhmannsfelden geschrieben. Diese Artikelschreiber, die
ihren Namen verbergen, wollen etwas besseres bringen, können es aber nicht. Für
die Entstehung Ruhmannsfeldens legen Sie Ihre Beobachtungen zu Grunde. Ich habe
es auch so gemacht. Ob der oder die Artikelschreiber, weiß ich nicht. Man kann
aus dem Kartenbild gerade so gut die Geschichte eines Platzes ablesen wie aus
dem Namen. Oft sind das die ältesten und einzigen Zeugen. Wir müssen für das
Werden Ruhmannsfeldens drei Ansätze annehmen:

1. Die älteste Siedlung mit der Kirche aus der Zeit vor den Bogener

Grafen, wo ein Rudmar den Wald rodete und Felder anlegte.

2. Die Burg aus der Zeit der Bogener Grafen.

3. Der Markt, der jüngste Teil, eine viereckige Anlage, was für das 13. oder 14.
Jahrhundert bezeichnend ist, also aus der Zeit des Klosters Gotteszell.

Auf diese Weise hat sich Ruhmannsfelden entwickelt, das zeigt der Augenschein.



Es grüßt Sie mit treuem Waldlergruß



P. Wilhelm Fink O.S.B.

.6 Expositus Hofmann an August Högn, 30.1.1950

Schönau, 30. Jan. 1950

Sehr geehrter Herr Rektor!

Entschuldigen Sie, daß ich erst heute auf Ihre beiden Anfragen antworten kann.
Vorige Woche war ich verreist und am Freitag hatte ich einen Vortrag in Böbrach
zu halten.

I. Im Hauptstaatsarchiv München sind 35 Urkunden über den Markt Ruhmannsfelden
von 1295 bis 1622. (Urkunden des Gerichts Viechtach, Fasc. 38 - 43 Nr. 445 a -
497), meist sind es Kaufbriefe von Häusern und Grundstücken. Schon in der ersten
Urkunde Nr. 445 a vom 28. April 1295 (Eberl hat irrtümlich 1294), ausgestellt zu
Regensburg wird Ruhmannsfelden "Markt" genannt. Die Herzöge Otto, Ludwig und
Stephan verkaufen drückender Schulden halber dem Abt und Konvent der Kirche zu
Aldersbach für 400 Pfund Regensburger Pfennig Schloß und Markt zu Rudmarsveld,
Mühle und Hof zu Brugk (Bruckmühle), das Dorf Arnoldesried (Arnetsried), das
Dorf Lawandried (Labersried) das Dörflein Viechleinsod (Weichselsried?),die Höf
zu Zierberg (Zierbach) und Leyfnaldstorf (Lämmersdorf), Hof und Mühle zum Steg
(Stegmühle) und alle Besitzungen und Leute, welche Heinrich von Pföling sel.
inne gehabt hat.

Diese Urkunde ist zwar nicht mehr im Original vorhanden, sondern in einer lat.
Papierkopie aus dem 16.Jahrhundert. Eine eigentliche Marktrechtverleihung für
Ruhmannsfelden ist unter diesen Urkunden nicht. Vom Markt Viechtach habe ich die
älteste Verleihung gefunden von Jahre 1474, doch mit Berufung auf ältere
Verleihungen, da die Bürger von Viechtach angaben solche verloren zu haben. Es
stehen diese und auch weitere Erneuerungen von Jahre 1529,1599 u. a. in einem
eigenen Akt. "Privilegien des Marktes u. Spitals Viechtach."

Ob für Ruhmannsfelden ebenfalls ein solcher Akt vorhanden ist, kann ich zur Zeit
nicht angeben. Viechtach wird aber schon im Urbarbuch von 1280/1310 ebenfalls
bereits als "Markt" bezeichnet.

II. Haus- und Hofbesitzer, die bereits über 100 Jahre auf demselben Hof sind,
kann ich leider nicht angeben, da ich beim Grundbuchamt noch nichts
herausgeschrieben habe. Ich habe nur die älteren Güterverzeichnisse, die im
Hauptstaatsarchiv in München sind, herausgeschrieben, so von 1536, 1555, 1565,
1577, 1668, 1752 u. 1760. Auch habe ich einzelne Familien bezw. Sippen
bearbeitet, so die Hacker, Hinkofer, Muhr, Steinbauer, Kufner, Geiger die
teilweise auch in der Ruhmannsfeldener Gegend vorkommen, doch sind wohl noch
mehrere Familien mehr als 100 Jahre auf ihrem Hof, die ich nicht näher kenne.
Habe gegenwärtig auch nicht Zeit, aus dem Grundbuchamt Auszüge zu machen, da ich
mit allerlei anderen Arbeiten in Anspruch genommen bin. Bedauere daher, daß ich
Ihnen damit nicht behilflich sein kann.



Mit freundlichen Grüssen

Ihr ergebenster

Georg Hofmann

Expositus.

.7 Bayerisches Hauptstaatsarchiv an den Marktgemeinderat, 28.2.1951

Bay. Hauptstaatsarchiv Nr. 318/1052

Arcisstr. 12

München 22 Ludwigstr. 23/0              28.2.1951



An den Marktgemeinderat Ruhmannsfelden, Niederbayern.

Betr. Marktverleihung z. Schreiben v. 21.2.1951.



Die Nachforschungen über die Marktgemeindeverleihung an Ruhmannsfelden sind
leider bisher ohne Erfolg geblieben. Der Forschung hinderlich ist namentlich der
Umstand, daß ein Akt des Klosters Gotteszell, zu dem Ruhmannsfelden seit der
Mitte des 15. Jahrhunderts gehörte, über Streitigkeiten zwischen Kloster und
Markt wegen der Marktrechte im 17. Jahrhundert nach auswärts verlagert und daher
z. Z. nicht greifbar ist. In ihm könnte ein Hinweis auf die Verleihung oder
sogar eine Abschrift der Marktverleihung enthalten sein. Im Übrigen muß
Ruhmannsfelden sich schon sehr früh des Marktprivilegs gefreut haben. Denn schon
in einer hier vorhandenen Urkunde vom 28.4.1295 über den Verkauf von Schloß und
Ort durch die Bay. Herzöge Otto, Ludwig und Stephan an das Kloster Aldersbach
wird es als Markt bezeichnet. Auch in einem der Bayerischen Urbare von 1318
(Mon. Boika 362, 417) kommt ein "Ruhmannsfelder Metzen" (Metrete Roudmarsvelder)
vor, was das Bestehen eines Marktes anzeigt. Ob in den hier befindlichen "Tomi
privilegiorum" das Marktprivileg enthalten ist, mußte eine genaue, Durchsicht
dieser umfangreichen Bände erst ergeben. Nach den Vorschriften müsste dies
jedoch den daran interessierten Benützern selbst überlassen werden. Ebenso müßte
noch eine Kartei über sämtliche vorhandene Urkunden bayerischer Herzoge, die
leider mit keinem Register versehen ist, herangezogen werden.

Erfahrungsgemäß sind aber Privilegienverleihungen vor dem 14. Jahrhundert im
Original nicht mehr vorhanden, da sie meist sehr früh verloren gegangen sind.



I. A. Unterschrift Staatsarchivdirektor.

.8 Bayerisches Staatsarchiv Landshut an Herrn Bürgermeister Muhr, 9.3.1951

Bay. Staatsarchiv Landshut             Landshut, 9.3.1951 Schloß Trausnitz

An H. Wolfg. Muhr, Bürgermeister

d. M. G. Ruhmannsfelden



Betreff: Marktrechtsverleihung an die Gemeinde Ruhmannsfelden

Z. Schreiben vom 15/17 2. des H. A. Högn Oberlehrer und Rektor i. R.
Ruhmannsfelden.

Einen urkundlichen Beleg über die Marktrechtsverleihung an Ruhmannsfelden
konnten wir im hiesigen Staatsarchiv nicht feststellen. Auch der uns zur
Verfügung stehenden Literatur konnten wird das Datum der Priveligierung nicht
entnehmen.

Wir möchten Sie jedoch zwecks etwaiger näherer Beantwortung Ihrer Anfrage an das
Bay. Hauptstaatsarchiv in München, Arcisstr. 12 verweisen bei dem der staatliche
Urkundenbestand vor dem Jahre 1400 liegt und das in alten Kopialbüchern mitunter
Abschriften von Privilegien der Städte und Märkte besitzt, die im Original nicht
mehr vorhanden sind.



I. V. v. Rehlinger, Staatsarchivrat.

.III Dokumente zur „Geschichte von Zachenberg“

.1 Trellinger an August Högn, 25.2.1952

Landshut, den 25. Februar 1952



Sehr geehrter Herr Rektor!



Meine Zusammenstellung über die Geschichte der angegeben Orte der Gemeinde
Zachenberg habe ich heute der Gemeinde Zachenberg übersandt. Mein Material
stammt fast nur aus Archivalien des hiesigen Staatsarchives. Ich habe Teile des
Gemeindebezirks nur einige Male flüchtig durchwandert, 2 Mal als ich vom Bahnhof
Gotteszell nach Oberbreitenau ging u. einmal als ich nur die zwei Riesenbäume in
Giggenried ansah.

Ich bin froh, daß ich mit meiner Arbeit so gut es ging fertig bin, ich leide
immer noch an den Folgen meiner schweren Erkrankung u. stehe immer noch in
ärtzlicher Behandlung .... (wegen Unleserlichkeit der Handschrift keine
Fortsetzung)

.2 Pater Wilhelm Fink an August Högn, 28.1.52

Metten, 28.1.52



        Sehr geehrter Herr Rektor!

Es ist begrüßenswert, daß eine Beschreibung einer Gemeinde im Walde beabsichtigt
wird. Gült- und Stiftsverzeichnisse haben weniger der Gaben, die gereicht
wurden, wegen, Wert als weil sie uns die Hofbesitzer und besonders die
Herrschaften namhaft machen, die die Reichnisse empfangen haben. Die Reichnisse
bleiben ja auch fast ständig gleich, wenigstens nach 1300. Sie meinen, der
Wandlhof hätte früher Wandldorf geheißen, gehörte aber der Zeit der Vorsiedlung
an. Die Möglichkeit gebe ich zu. Aber zwar müßten noch einige Fragen geklärt
werden. Welche Lage haben die Felder? sommerseitige? Spätere Siedlungen begnügen
sich mit weniger günstigen Lagen. Es müßten dann noch wenigstens in der Urzeit
zwei kleine Höfe dabei gelegen sein. Es ist das erkenntlich an der Zahl der
Tagwerke des Ackerlandes. Als ich meine Studien vor 30 Jahren aufnahm, daß der
Wandlhof eine ritterliche Siedlung neben den alten Dorforten ist. Dieser Zeit
gehört auch der Ried-ort Hafenried an. Hat er ursprünglich aber zum Wandlhof
gehört? Das ist die Frage. Wegen des Namens könnte man an den alten Namen
Wantila - Wentila denken. Als Grenzbach wäre er auch gut zu erklären. Das Gebiet
war Grenzland der alten Mettener Gemarkung. Daher auch der Ort March. Grenzland
ist es, da es von Seiten von Bischofsmais und Regen umfaßt wird und scheiden
sich hier im Mettener Grenzland zwei Gaue, Donaugau und der Schweinachgau. So
wäre die Annahme, daß Wandlbach Grenzbach bedeutet wohl berechtigt.

Interessant ist, daß der Wandlhof den Namen Hafenried angenommen. Das wäre wohl
bei einer Umsiedlung kaum möglich gewesen. So ist anzunehmen, daß der Wandlhof
wohl schon anfänglich zur Rodung des Hadubet - abgekürzt - Habe oder Hawe
gehörte.



    Mit freundliche Grüßen

        Ihr ergebener  P. Wilhelm Fink O.S.B

.3 Gotthard Oswald an August Högn, 30.1.52

Rinchnachmündt, 30.I.52



Sehr geehrter Herr Oberlehrer!



Zu Ihrer Anfrage v. 25.I. teile ich Ihnen mit:



Die Ortsnamen sich vielfach so entstellt, daß es unmöglich ist, sie zu erklären,
wenn nicht aus Urkunden, aus alten Güter= u. Zehentverzeichnisse die
ursprüngliche Schreibweise herauszulesen ist. Beispielsweise hierfür könnte ich
Ihnen eine Menge nennen. Wenn der Ortsname Zachenberg nicht entstellt ist, dann
wird ihn wohl zugrunde liegen der Personenname Zacha, die volksmundartige
Abkehrung des Namens Zacharias, der im Mittelalter nicht selten als Taufname
vorkommt. So hieß z. B. im 14. Jahrhundert ein Besitzer des Edelsitzes March
Zacharias. Der Personenname Zacha kommt jetzt noch öfter vor als Familienname in
der Form "Zacher" (gesprochen "Zacha")



Der Ortsname "Hafenried" kommt vielleicht, wenn er nicht entstellt ist, daher,
daß sich hier einst ein Hafner ansiedelte. (Oder, wenn hier einst ein Hof stand,
könnte es einst geheißen haben "Hofenried" (Hofaried?).

Da kommt es auf die Aussprache im Volksmund an.

Wenn P. Fink Ihnen keine Antwort gab, dann wußte er offenbar keine Erklärung.

Für Ihre "Geschichte von Ruhmannsfelden" recht vielen Dank.



Mit freundlichen Grüßen



Ihr ergebenster



Gotthard Oswald

.4 August Högn an das Kloster Niederalteich, 26.3.52

A. Högn, Oberlehrer u. Rektor i. R.             26.3.52

Ruhmannsfelden



An

H. H. Klosterbibliothekar

des Klosters

Niederalteich



Sehr geehrter H. H. Pater!



Es handelt sich um die Chronik der Gemeinde Zachenberg, die sich erstreckt zu
beiden Seiten der Bahnlinie Station Gotteszell - Station Triefenried. Das Gebiet
- angrenzend an das seinerzeitige Waldgebiet zum Kloster Niederalteich - dann im
Besitz des Kloster Metten - war in der Hauptsache u. die längste Zeit grundbar
u. zehentpflichtig zu Kloster Gotteszell. Bis die unglückliche Zeit für die
Klöster kam - das Jahr 1803.

Bei dieser Chronik der Gd. Zachenberg arbeiten mit H. H. Dr. P. Amandus
Bielmeier, Metten - H. H. Pfarrer Oswald Rinchnachmündt und G. Zollrat
Trellinger, Landshut, welcher schon ein 63 Seiten umfassendes Scriptum - alles
aus dem Staatsarchiv Landshut entnommen - an die Gemeindeverwaltung Zachenberg
gesandt hatte. In dieser Aufzeichnung kommt unter: "Vorderdietzberg" folgendes
Stelle vor: "Ein Sohn von Vietzgerg war der am 18.11.1774 geborene Thadäus
Saller. Wir finden ihn als Mönch in Niederalteich, er starb als Kommorant in
Grafenau am 2. Sept. 1848"



H. H. Pater! Ich würde Sie frdl. bitten um Mitteilung, ob über diesen
Benediktinermönch Thadäus Saller etwas Bemerkenwertes bekannt ist, das der
obigen kurzen Mitteilung beigefügt werden könnte. Es ist mir leider nicht
bekannt, ob die Geschichte des Klosters Niederalteich im Druck erschienen ist u.
in welchem Verlage. Da wäre ich sicher beim Durchlesen auf Thadäus Saller
gestoßen. Ich kenne nur die diesbez. Artikel im Donauboten (Gäu und Wald)

H. H. Pater! Ich bitte Sie freundlichst um Ihre gütigste Unterstützung.



        Für Ihre Bemühung herzlichst dankend!



            Hochachtungsvollst!

                A. Högn

.5 Damian Merk an August Högn, 27.3.52

In meinem Professbuch III/364 steht Folgendes:



P. Thadäus (Franz) Sailler (Saller) geb.18.XI.1774 zu Dietlsperg oder
Vorderdietzberg.

Profess in NA 23. Juni 1799,

Ordin. in  Passau 6. Juli 1800,

Primiz in NA 10. August 1800,

Kooperator in Gravenau 1803 bis 1837,

Kommorant in Gravenau 1837 bis 1848,

Gestorben in Grafenau 2. September 1848.



NA, am 27. März 1952



Besten Gruss!!                  P. Damian Merk OSB

                            (Archivar)

.6 August Högn an das Pfarramt Grafenau, 26.3.52

An das
Ruhmannsfelden, 26.3.52

Katholische Pfarramt

Grafenau

H. H. Pfarrervorstand!



Es handelt sich um die Chronik der Gemeinde Zachenberg. H. H. D. P. Amandus
Bielmeier, Metten (ein Bruder des Zachenberger Bürgermeisters) H. H. Pfr. Oswald
Rinchnachmündt u. G. Zollrat Trellinger Landshut arbeiten an der Erstellung
dieser Chronik mit. Letzterer hat der Gemeinde Zachenberg ein 63 Seiten
umfassendes Schreiben zu gesandt, in dem er alles Wissen verwertet über die
einzelnen Ortschaften u. Höfe der Gd. Zachenberg aus den Archivalien des
Staatsarchivs Landshut herausgeschrieben hat. Dabei kommt unter
"Vorderdietzberg" folgende Stelle vor: "Ein Sohn von Vietzgerg war der am
18.11.1774 geborene Thadäus Saller. Wir finden ihn als Mönch in Niederalteich,
er starb am 2. Sept. 1848 als Kommorant in Grafenau!"

Würde Sie, geehrter H. H. Pfarrvorstand, freundlichst bitten um Mitteilung, ob
vielleicht in der dortigen Pfarrmatrikel etwas Bemerkenswertes über H. H.
Thadäus Saller zu finden ist.



Für Ihre Bemühung herzlichst dankend!

Hochachtungsvollst!



A. Högn

Rektor i. R.

Ruhmannsfelden

.7 Pfarrer Rankl an August Högn, 27.3.52

(auf die Rückseite des vorangehenden Briefes geschrieben)



Vonkurzer Hand zurück mit dem Bemerken, daß in der Sterbematrikel hierorts sich
der Eintrag findet:

Franz Thaddä Sailer (wohl verschrieben für Saller) Exkonventual des Klosters
Niederalteich Benediktinerordens u. von 1804 bis 1840 Cooperator in Grafenau,
verstorben in Grafenau HsNr. 47 (dem heutigen Mesnerhaus, das der Kirche gehört)
an Brand am 2. September 1848 um 8 1/2 morgens, beerdigt am 4. September durch
Pfarrer Stephaner. Er war 74 Jahre alt und wurde vom Pfarrer Stephaner versehen.



In der hiesigen Allerseelenkapelle ist in der Nordwand die Gedenktafel dieses
Priesters eingelassen. Sie trägt die Inschrift:

Denkmal des Hochwürdigen Herrn

Franz Thaddä Sailer (merkwürdig heißt er auch hier wieder Sailer)

Conventual des Benediktiner-Klosters Niederalteich und nach Auflösung desselben
Cooperator in Grafenau bis zum Jahre 1840.

Er war ein Muster der Demut und ein Freund der Armen

geboren am 18. November 1774

zum Priester geweiht am 6. Juli 1800

gestorben am 2. September 1884

R.I.P



Selig sind die Toten, die im Herrn sterben, sie ruhen aus von ihren Leiden und
ihre Werke folgen ihnen nach (Apocal 14,13)



P.S. Im Taufbuch, wo der taufende Priester eingetragen ist heißt es ständig
Saller, nicht Sailer bis zum Jahre 1830. Dann auf einmal Sailer mit dem Beginn
eines neuen Bandes. (Unerklärlich!)



L. Rankl, Pfr.

.8 Expositus Georg Hofmann an August Högn, 23.10.52

Schönau, 23. Oktober 1952.



Sehr geehrter Herr Rektor!

Entschuldigen Sie, dass ich nicht gleich auf Ihren Brief vom 13. geantwortet
habe, ich hatte in diesen Tagen viel zu tun.

Betreff des Namens Zachenberg kann ich mitteilen, das Studienprofessor Willibald
Schmidt (ein Lehrersohn aus Moosbach, jetzt an der Realschule in Straubing) in
seinem Ortsnamenverzeichnis des Bezirksamts Viechtach (1924) angibt: 1273
erscheint ein Pabo von Zachenperg. "Berg eines Zacco", also mit einem alten
Eigennamen zusammengesetzt, wie auch z. B. Blossersberg = um 1120 Plassansperch,
Siedlung eines Plassan auf einem Berg. Den Eigennamen könnte man nach Schmeller,
Bayerisches Wörterbuch, erklären entweder von "zochen" = langsam oder schleppend
einhergehen oder von "zoch" = ein grober, roher, bengelhafter Mensch, dagegen
bedeutet in Innertirol wieder ein Zoch Mehrzahl Zocher, einen Burschen voller
Kraft und Saft. Natürlich sind auch noch andere Deutungen möglich. Es lässt sich
da schwer etwas bestimmtes sagen.

Vom Bürgermeister der Gemeinde Zachenberg habe ich ein Schreiben von Herrn
Zollfinanzrat Anton Trellinger nicht erhalten. Herr Trellinger hat
jahrzehntelang, an der Quelle in Landshut sitzend, sehr viel Material über die
Gemeinden des Bezirkes Viechtach gesammelt, leider hat er im Frühjahr einen
Schlaganfall erlitten und kann daher nicht mehr so viel arbeiten.

Durch Umfragen in den einzelnen Häusern kann man gar manches Interessante
herausbringen, aber leider sind die Angaben über die Sesshaftigkeit der
einzelnen Familien nicht immer zuverlässig. Ich habe die alten
Güterverzeichnisse, namentlich von 1668 und 1752/60 benützt und dann mit der
Pfarrmatrikel gearbeitet. Zuerst habe ich mir die sämtlichen Trauungen
herausgeschrieben und dann verglichen, wie weit ich sie zurückverfolgen konnte,
dann erst die Taufen und Sterbefälle. Die Pfarreien Teisnach-Geierstahl und
Böbrach habe ich zum Teil verzettelt und kann daher hier alles leichter
übersehen. Ruhmannsfelden ist mir zu sehr abgelegen. Doch habe ich auch von dort
einige Geschlechter bearbeitet, die mit anderen zusammenhängen, so die
sämtlichen Steinbauer, Muhr, Hinkofer, Hacker.

Am besten beginnen Sie wohl mit dem Urkataster von ca. 1843, den Ihnen wohl Herr
Trellinger abgeschrieben hat, von da können Sie dann weiter zurückkommen. Ich
bin leider mit den dortigen Anwesen nicht ganz auf dem laufenden, bin aber gern
bereit Ihnen so weit möglich behilflich zu sein. Auch weiss ich nicht genau,
welche Ortschaften zur Gemeinde Zachenberg gehören. Ich habe nur die
Verzeichnisse vor 1800 mit den alten Grundherrschaften.

Nächsten Sonntag ist in Viechtach ein Sippentag der Pritzl, die ich bearbeitet
habe, wobei ich einen Vortrag halten soll. Bis zum Januar soll ich die
Geschichte der Nussberger fertig gedruckt haben, damit sie in Jahresbericht des
historischen Vereins Straubing gedruckt werden können. Leider kommen immer
wieder andere Sachen dazwischen, daß ich von dieser Arbeit abgehalten werde.
Vormittags habe ich fast alle Tage Schule.

Wünsche Ihnen recht viel Erfolg bei Ihrer Arbeit, ich weiss, es ist eine recht
mühsame Geduldsarbeit, aber man darf es sich nicht verdriessen lassen.



    Recht herzliche Grüsse Ihr ergebener

Georg Hofmann, Expositus

.9 August Högn an Pfarrer Max Schefbeck, 17.01.53

Ruhmannsfelden, 17. Januar 1953



Sehr geehrter hochwürdiger Herr Pfarrer!



Es handelt sich hier um die Heimatgeschichte der Gemeinde Zachenberg. Da sind
bei dem Bahnbau Deggendorf-Eisenstein im Gemeindebezirk Zachenberg deutsche u.
ausländische Bahnarbeiter teils tödlich verunglückt oder an Typhus, udgl.
gestorben. Ich habe im hiesigen Pfarrhof aus den damaligen Sterberegistern
entnommen, daß bei dem damaligen Bahnbau - soweit der Gemeindebezirk Zachenberg
in Frage kommt - tödlich verunglückt sind und hier begraben wurden: 1 Italiener,
2 Böhmen u. 2 Tiroler

u. daß gestorben sind von diesen ausländischen Bahnarbeitern: 1 Tiroler, 2
Italiener u. 4 Böhmen. Die Namen aller derer habe ich angeführt.

Sehr geehrter hochwürdiger Herr Pfarrer! Dürfte ich Sie bitten - falls   S i e
gelegentlich einmal das Sterberegister 1874 bis 1877 in Händen hätten,
nachschauen zu wollen, ob da sich auch solche Einträge vorfinden. Wenn   j a ,
dann würde ich   S i e   herzlichst bitten, mir das gütigst mitteilen zu wollen.
Die anfallenden Gebühren entrichte ich ganz gerne. Für   I h r e   Bemühung im
voraus bestens dankend, zeichnet



Hochachtungsvollst!

Ergebenster!

A. Högn Ruhmannsfelden



z. B. tödlich verunglückt:      Stefano Palaera, 32 Jahre alt, aus

                Nevaledo im Südtirol - 1877

gestorben an Typhus:        Martinus Preghenella, 45 Jahre alt, aus

                Pregherr in Italien  -1875

.10 Pfarrer Max Schefbeck an August Högn, 21.1.53

(auf die Rückseite des vorangehenden Briefes geschrieben)



March/Ndby.21.I.53



Sehr geehrter Herr Rector!



Gerne habe ich Ihren Wunsch zu erfüllen gesucht, aber leider oder Gott sei Dank
nur ein Todesopfer entdeckt, das italienischen Geblütes ist. Ich lege Ihnen
gerne den Auszug bei und hoffe Ihnen so gedient zu haben.



Bitte Familie Hertl recht herzlichst zu grüssen!



Ergebenst



Max Schefbeck, Pfr.

(auf eine kleinem Formular beigefügt:)



Auszug aus dem Sterberegister der kath. Pfarrei March Ndb.

__________



Am    12. Februar 1876     ist in    March    verstorben und wurde am      13.
II. 1876    in March Ndb. nach römisch-katholischen Ritus beerdigt

               Peter Del Fabero                               Jahre alt,
               röm.-katholisch

Erdarbeiterskind z. Zt. in Triefenried (frühgetauft 1/4 Std. alt)



Kath. Pfarramt March Ndb.

M. Schefbeck, Pfr.

.11 Pater Wilhelm Fink an August Högn, 23.2.54

BIBLIOTHEK DER ABTEI METTEN



Metten, den 23. 2. 54



        Sehr geehrter Herr Oberlehrer!



Sie werden entschuldigen, daß ich Sie so lange auf die Folter gespannte habe.
Meine Zeit ist knapp. Da ich Ihre Arbeit genauer durchstudieren wollte, brauchte
ich etwas länger. Am besten ist der zweite Teil geraten. Er wäre druckreif mit
einigen kleinen Änderungen. Sie erzählen von den Bewohnern der einzelnen Häuser;
der Volkskundler würde es aber gerne sehen, wenn auch einmal so ein Haus
dargestellt würde (Bauzeit, Einteilung der Räume, Verteilung von Stallungen,
Scheunen, Schupfen, Brunnen usw., Aussehend der alten Bauernstubeck).

Im ersten Teil wie auch im dritten vermisse ich die rechte Anordung. Bei solchen
Arbeiten wie dieser wird am besten mit einer Schilderung der Lage des
betreffenden Ortes begonnen. Und Fremde, nicht Einheimische sollen die
Geschichte lesen. Sie müssen aber wissen wo der Ort bzw. die Gemeinde liegt, z.
B. allgemein zwischen Ruhmannsfelden und Regen, den Flüssen Teisnach und Regen,
den Bahnen, Bergen etc. Überleitung: im bayer. Wald. Späte Besiedlung. Abschnitt
über die Landnahme der Bayern fällt weg, ist auch nicht richtig dargestellt.
Bay. Wald: bis 800 unbewohntes Waldgebiet. Hieß Nordwald, nicht zu verwechseln
mit dem Nordgau, der nordwestlich von Regensburg liegt. Gegend von Zachenberg
gehörte noch zum Donaugau. Um 900 älteste Siedlung: Kloster Metten: Warum hier
gesiedelt? Bodenbeschaffenheit? Wo liegen die ersten Siedlungen. Sonnen,
günstige Seite. Gebundene Wirtschaftsform: Größe der Äckerflur, davon die eine
Hälfte der Meier oder Großbauern in den Dörfern, die andere Hälfte den übrigen
Höfen zugeteilt. Gerade auf die Darstellung der Größe der Äckerflur wäre mehr
Aufmerksamkeit zu schenken. 2. Siedlung: 1000 - 1200. Mittelalter: Wo liegen
Burgen? Wem gehören sie? Nach den Arnulfingern waren die Babenberger, die
zugleich Markgrafen in Österreich waren, Herren des Donaugaues. Ihnen gehörte
jetzt der Donaugau. Ihre Nachfolger sind 1080 die späteren Grafen von Bogen.
Ihre Ministerialen waren die Inhaber der Burgen Altnußberg, Linden,
Ruhmannsfelden, Weißenstein, die Nußberger, Degenberger, Pfellinger. Ried-orte.
Zu beachten, daß damals auch durch die Bischofe von Passau und den Burggrafen
dort die Siedlungen bis an die Grenzen der beiden Diözesen Regensburg und
Passau, der beiden alten Gaue Donau und Schweinachgau. Beide Siedlungsperioden
eigen sind Burg- und Bachnamen. Aussterben der Bogener 1242, Erben
Wittelsbacher, Gründung von Märkten Viechtach, Regen. Viechtach Sitz eines
Gerichtes, Böbrach und Regen Schergenämter. Wichtig die Gründung von Gotteszell.
Eingehen auf seine Geschichte Zachenberg, Ort und Gemeinde in Frage kommen.
Bedeutung: Seelsorge, geistiger Mittelpunkt. Aufhebung. Jetzt. Neu-Ordnung des
bay. Staates: Aufhebung der Grundherschaft. Abgaben blieben bis 1848 (Ablösung
durch die Bodenzinnse) Hof und Flurfreies Eigentum des Besitzes. Schaffung der
Steuergemeinden, Zusammenfügen der Siedlungen 1818: Gemeinden
Selbstverwaltungskörper (1. Gemeindeordnung, Schaffung der Gem. Zachenberg im
Landgerichte Viechtach. Jetzt Statistiken: Größe, Zahl der Einwohner in
verschiedenen Perioden. Und jetzt der 2. Teil! Das Ganze muß gar nicht
aneinander wachsen. Im 3. Teil: Organisation der Gemeinde: Bürgermeister,
Gemeinderat, -schreiber, -diener u. s. w. Aufgaben der Gemeinde: Baupolizei,
Gewerbeaufsicht (Darstellung wann und wie viel in den einzelnen Perioden und was
gebaut oder ernannt wurde welche Gewerbe alte wie Wirte, Bäcker, Metzger nun
nach Einführung der Gewerbefreiheit, Schuster, Schneider, Weber etc.)
Armen-Fürsorge: Kranken: Invaliden: Arzt, Sicherheit: Gendarmerie etc. Schulen:
Hat die Gemeinde Besitz? Verkehrswesen: Straßen: Kraftfahrwesen, Bahn, Post.
Telephonzugang. Gemeinnützige Vereine: Feuerwehr etc. Politk ändert sich.
Kirche: wenig Veränderung. Pfarrei Ruhmannsfelden. Kirchliches und religiöses
Leben, Volkskunde, Geistliche, die aus der Pfarrei hervorgegangen, Wallfahrten,
Kapellen u. Volksfrömmigkeit, prf. Volkskunde: Heimats-, Faschingsbräuche etc.

Zum Schlusse die Schattenseiten des Lebens: Krieg, Hunger und Pest, Seuche.

Ich glaube, daß in einer solchen Anordnung die Geschichte Zachenbergs ein
angenehmes, weil übersichtliches und ein nützliches Lesen sein wird. Zum Schluss
darf ich auf einige Irrtümer .... (Seitenende das Blattes, weiteres Blatt nicht
überliefert)

.12 August Högn an den Bürgermeister Bielmeier, 31.2.54

Ruhmannsfelden, 31. März 1954



        Sehr geehrter Herr Bürgermeister!



Erlaube mir Ihnen hiemit die druckfertige schriftliche Arbeit über
"Heimatgeschichte der Gemeindeflur Zachenberg mit seinen Ortschaften" vorläufig
einmal zur Einsichtnahme u. zum Durchlesen in Vorlage zu bringen.

Diese Arbeit wurde von H. H. Professor P.   F i n k   in Metten durchgesehen u.
einer gründlichen Korrektur unterzogen. Alles - was er dabei - als " nicht
richtig dargestellt " gefunden hat, wurde weggelassen u. ist in dieser Arbeit
nicht mehr enthalten. Was H. H. Expositus Hofmann von Schönau noch ergänzend
gewünscht hatte, wurde in diese Arbeit nachträglich aufgenommen.

In dieser Form dürfte die beiliegende Arbeit doch allgemeinem Interesse
begegnen, da es ein wirklich interessanter Lesestoff ist. Ich bitte H.
Bürgermeister, diese Arbeit vielleicht an den Osterfeiertag durchlesen zu wollen
u. sich selbst darüber ein Urteil zu bilden. H. H. Expositus Hofmann fragt, ob
die Gemeinde Zachenberg diese Heimatgeschichte veröffentlichen lassen will.
Damit ich auch darüber gleich Aufschluß geben kann, habe ich bei Firma Laßleben
in Kallmünz angefragt. Dort ist auch das Büchlein der Geschichte von
Ruhmannsfelden gedruckt worden. Laßleben teilte mir mit, daß 500 Bücher auf 900
DM kommen, also das Buch auf 1,80 DM, so daß dann im Verkauf das Buch auf 2,50
DM oder 3 DM käme. Als Absatzgebiet kämen Gd. Zachenberg, Gd. Ruhmannsfelden,
Gd. March u. Stadt Regen in Frage. Freilich müßte zuvor Reklame gemacht werden
in der Presse u. durch Anschlag einer Bekanntmachung, vielleicht in der Form:
Heimatgeschichte der Gemeindeflur Zachenberg mit seinen sämtlichen Ortschaften
erschienen. Interessanter Lesestoff. Preis ........ Bestellung in der Gd.
Kanzlei Z. - Gd. Achslach hat den Verkauf eines solchen Büchleins selbst
besorgt. Es könnte zuvor auch eine Vorbestellung in die Wege geleitet werden, so
daß man sich über den Absatz des Buches einigermaßen orientieren kann.

Das Buch sollte wenigstens in der Gd. Zachenberg in keinem Haus u. in keiner
Familie fehlen. Die 3 DM können in Zigaretten wieder hereingespart werden. Der
Gemeinde Zachenberg verblieb doch ein ansehnlicher Gewinn.



Hochachtungsvoll

A. Högn

.13 Bürgermeister Bielmeier an August Högn, 7.5.1956

Ruhmannsfelden, den 7. Mai 1956



Herrn

Rektor A. Högn

R u h m a n n s  f e l d e n



Betreff: Heimatgeschichte der Gemeinde Zachenberg



Hochverehrter Herr Rektor!

In der ersten Sitzung der Legislaturperiode 1956/57 hat sich der Gemeinderat mit
Ihrem Schreiben vom 30.4.1956 eingehend befasst und größtes Interesse für Ihre
heimatkundliche Arbeit bekundet. Es wurde zum Ausdruck gebracht, daß Ihre Arbeit
für das heimatkundliche Geschehen unserer Gemeinde begrüßt und anerkannt wird.
Der Gemeinderat übernimmt für die Akten und die Chronik der Gemeinde gerne Ihre
umfangreichen heimatkundlichen Aufschreibungen und sorgt bis zur Drucklegung für
sichere Aufbewahrung.

Über die Drucklegung wird der Gemeinderat in der nächsten Sitzung beschliessen.
Der Gemeinderat hat beschlossen, Ihnen für Ihre Arbeit eine finanzielle
Anerkennung zukommen zu lassen. Ich darf Sie daher bitten, mir für die nächste
Sitzung einen Betrag zu nennen, der Ihre Arbeit einigermassen entschädigen
würde. Wir wissen, daß Ihre Arbeiten aus Idealismus entstanden sind. Sollten Sie
mir keinen Betrag nennen, wird der Gemeinderat von sich über die Höhe der
Anerkennung für Ihre vorbildliche Arbeit beschliessen.

Ich möchte mich auf diesen Wege im Namen des Gemeinderates und auch in meinen
Namen für Ihre bisherige Arbeit auf diesem Gebiete für unserer Gemeinde
aufrichtig bedanken und bin überzeugt, daß Sie auch weiterhin der Gemeinde mit
Rat und Hilfe bei derartigen Arbeiten zu Verfügung stehen.



Hochachtungsvollst!

Ihr

Bielmeier, 1. Bürgermeister



.14 Exzerpt von August Högn aus Fink, Östl. Grenzmarken

Köckersried



Von P. W. Fink: Östl. Grenzmarken 1922 Heft 8



Kökesried: vergl. Maurer, Ortsnamen des Hochstifts Passau 57



Goggesreut - P. M. Cotessschalk (R. V. 122 6,3)

(Vorstehendes aus Östl. Grenzmarken 1922 Heft 8

In der Karte gleichen Heftes S. 139

Köckersried = Eigenname, ew. im R. V.







Vorstehe Karte im gleichem Heft S. 139

Das Ganze Gebiet um Köckersried war vor der Schenkung an Metten oder bei
derselben, nicht besiedelt, war also ausgesprochenes Rodungsland. Köckersried -
wie alle anderen "ried" Namen sind erst später entstanden. Zuerst kamen die mit
Eigennamen zusammengesetzten "ing" Namen, die echten. Solche sind hier nicht.
Später kommen die "dorf" Namen (Mettner Gründungen) u. "berg" Namen. Zachberg
dürfte älter sein wie Köckersried. In einer Niederalteicher Urkunde v. J. 1273,
in einer päpstlichen Bulle v. J. 1274 wird Zachenberg erwähnt. Die Nußberg, dann
die nächtigen Degenberger hatten in Zachenberg Besitzungen

Von dem bereits i. J. 1404 als teilweise Nußberger´sches Besitztum aufgeführten
Köckersried, so bemerkt, daß nach dem herzoglichen Salbuche v. J. 1577 2 Güter
Frau Rosina von Hauff auf Neu-Nussberg inne hatte u. diese Güter waren dem
Kaplan zu Neu Nußberg gültpflichtig. Es muß also im dortigen Schoß ein eigener
Kaplan aufgestellt gewesen sein. 4 Güter von Köckersried gehörten nach dem
beredten Salbuche mit der Gült dem Kloster Gotteszell. Das in Köckersried sich
befindliche an Stelle einer 1803 abgebrochenen Kapelle, erbaute Kirchlein,
entstand in der Jahren 1830 - 33. Es ist dem Erzengel Michael geweiht. Messe
wird in derselben nicht gelesen. Köckersried das bis 1806 zur Pfarrei
Ruhmannsfelden gehörte, ist jetzt nach Gotteszell eingepfarrt. (Vorstehendes v.
G. Trellinger Landshut in Bayerwald)

.IV Dokumente zur Josephs-Messe

.1 Viechtacher Bayerwald-Bote, 15.6.1953

Installation von Pfarrer Reicheneder



Feierliche Installation des neuen Pfarrherrn

Pfarrer Reicheneder empfing aus der Hand des Dekans den Kirchenschlüssel



Ruhmannsfelden. Der neue Pfarrherr von Ruhmannsfelden Pfarrer Franz       R e i
c h e n e d e r,    wurde am Sonntagvormittag unter großer Anteilnahme der
Pfarrbevölkerung durch den Dekan des Dekanats Unterviechtach, Geist. Rat   L i p
f,  offiziell in seine neue Pfarrei eingeführt. Vom Pfarrhof holten die
Ortsvereine den Geistlichen ab und geleiteten ihn im feierlichen Zuge unter dem
Geläute der Glocken in die festlich geschmückte Pfarrkirche. Den Zug begleiteten
Landrat   K a u e r,  die Bürgermeister   P i e h l e r   und   B i e l m e i e
r,   der Marktgemeinderat die kath. Kirchenverwaltung, die Feuerwehr, alle
Ortsvereine und viele Gläubige. ....

.... Das feierliche Hochamt zelebriete der neue Pfarrherr bei festlichem Gesang
des Kirchenchores. Ergreifend trug der klangvolle Chor die „Missa St. Josephi"
von Rektor Högn vor. Mit dem feierlichen Te Deum, in das die gesamte
Pfarrgemeinde einstimmte, schloß diese kirchliche Festfeier für die Pfarrei
Ruhmannsfelden. Anschließend übergab Dekan Lipf noch den, Friedhof an den
Geistlichen, der die Gräber mit Weihwasser besprengte.

.2 Deggendorfer Zeitung, 18.3.1957

Ankündigung der Deggendorfer Erstaufführung der Josephs-Messe



Josephsmesse zum Josephsfest

Morgen Erstaufführung einer Komposition August Högns



Am morgigen Josephstage erlebt die Josephsmesse eines gebürtigen Deggendorfers
ihre Deggendorfer Erstaufführung: Rektor August Högn ist der Komponist, und in
St. Martin wird seine Festmesse zum Pfarrgottesdienst um 9 Uhr erklingen. Im
Hause des 1913 verstorbenen Buchhändlers Andreas Högn erblickte dessen Sohn
August am 2. August 1878 das Licht der Welt. Der Umgang mit Büchern im
elterlichen Hause und eine auffallende musikalische Begabung mochten in der
Berufswahl den Ausschlag zum Lehrerberuf gegeben haben. Im Straubinger
Lehrerseminar hierfür vorbereitet, ließ sich der 30jahrige nach kurzen
Zwischenstationen an niederbayerischen Dorfschulen endgültig in Ruhmannsfelden
nieder, seiner Heimatstadt nicht allzufern und doch entrückt dem städtischen
Betrieb und umwoben von der Stille der Bayerwaldberge, die gerade für einen
schaffenden Musiker so wichtig und - wie auch in diesem Fall - so ersprießlich
ist.

Die Messe trägt die Opuszahl 62, und das will - da es sich immerhin um eine
nebenberufliche Liebhaberei handelt - schon etwas heißen. Das musikalische
Lebenswerk Högns tragt den Stempel echter und guter Gebrauchsmusik, d. h. es ist
Musik, die zum Gebrauch bestimmt und auch zu "brauchen" ist und nicht in den
luft- leeren Raum hineinkomponiert wurde. Die ländliche Stille und der Umgang
mit den einfachen, unkomplizierten Menschen, mit denen er in der Schule, auf dem
Kirchenchor und in der Gesellschaft zu tun hatte, hat ihn auch vor sogenannten
modernen Experimenten bewahrt. Trotzdem ist er auch hier Erzieher, der den Boden
unter den Füßen zwar nie verliert und doch, den ethischen Bildungswert der Musik
erkennend, Blick und Ohr für Höheres weiten möchte.

So ist auch diese Messe ein ehrliches Selbstzeugnis, des Musikers sowohl wie des
Menschen. Sie verrät gediegene Handwerkskunst, die sich in einer sauberen
satztechnischen Handschrift äußert, Sinn für harmonische Farbigkeit hat, ohne in
spätromantische Chromatik abzugleiten, und sich der Mittel des Kontrastes in der
Gegenüberstellung kraftvoller Unisoni und motivisch aufgelockerter Chorsätze
bedient. Doch das Wichtigste; der Mensch, der all dies zutiefst im Glauben
bejaht, steht hinter dem ganzen Werk. Mag die morgige Aufführung darum eine
herzliche Gabe "postfestum" an den 80-jährigen Komponisten sein. Daß es auf den
Tag genau der 60. Geburtstag von Joseph Haas ist, ist ein schöner Zufall. Auch
ihn hat die musikalische Begabung zum Lehrberuf geführt, den Grundsätzen seiner
Jugend treu im Glauben, und in der Liebe bis heute geblieben. Eine Brücke zu
Anton Bruckner, dem Volksschullehrer von Windnhaag, ließe sich von hier aus
leicht schlagen. Mag die Musik dies selbst besorgen, wenn als Offertorium
Bruckners "Os justi" und Joseph Haas' Lied zum hl. Josephs zum Eingang der Messe
erklingt, die in Anwesenheit des Komponisten morgen in St. Martin erstaufgeführt
wird. Ihm und seinem großen "Kollegen" Joseph Haas möchten wir ein herzliches
"Ad multos annos" mit in die Töne hineinverweben. Fritz Goller

.3 Plattlinger Nachrichten, ?

Ankündung der Erstaufführung der Josephs-Messe in der Magdalenen-Kirche



Heute gibt es eine Erstaufführung



Zum heutigen Hochamt in der Magdalenenkirche bringt der Kirchenchor eine
Festmesse zu Ehren des hl. Josef in Erstaufführung. Es handelt sich um ein Werk
des Rektors a. D. A. Högn, der in Ruhmannsfelden lebt. Högn ist kein Neutöner,
sondern pflegt einen romantischen, liebenswerten Kompositionsstil. Das
Benediktus für Sopransolo und Chor wird sicher allen gefallen, die die Musik
nicht erst über den Verstand, sondern gleich ins Herz fließen lassen wollen.

.4 Gustl Güdner an August Högn, 22.3.1957

Plattling, 22.3.57

Sehr geehrter Herr Rektor Högn!

Anbei folgt Ihre Josephsmesse mit bestem Dank zurück. Sie hat am Josephsfest gut
gefallen. Ich habe verschiedene Stimmen darüber gehört. Unserer geistl. Rat hier
hat sich sehr gefreut über die Aufführung. Daß es so lange bis zur Rücksendung
gedauert hat, bitte ich nochmals zu entschuldigen.

Mit besten Grüßen bin ich Ihr erg.

Gustl Güdner?

.5 Viechtacher Bayerwald-Bote, 6.8.1974

Verabschiedung von Pfarrer Reicheneder



Ein schwerer, ein ehrender Abschied für den scheidenden Priester

Abschiedsgruß an die Pfarrgemeinde: “Der Friede sei mit euch"



Ruhmannsfelden. Mit dem Gruß: „Der Friede sei mit Euch", mit dem er vor 21
Jahren die Pfarrgemeinde begrüßt hatte, verabschiedete sich am vergangenen
Sonntag Geistlicher Rat Franz S. Reicheneder von seiner Pfarrgemeinde. In einem
feierlichen Abschiedsgottesdienst, der vom Kirchenchor unter Leitung von Karl
Geiger mit der “Josefi-Messe” von A. Högn und den Hymnen “Die Himmel rühmen”,
“Nun danket alle Gott", „Herr unser Gott” festlich umrahmt wurde, unterstrich
der scheidende Seelsorger die Bedeutung des Friedens in unserer Zeit. ....

.6 Regensburger Bistumsblatt, 24.8.1974

Verabschiedung von Pfarrer Reicheneder



Nach 21 Jahren Abschied von Ruhmannsfelden



Ruhmannsfelden. Mit dem Gruß „Der Friede sei mit Euch", mit dem er vor 21 Jahren
die Pfarrgemeinde begrüßt hatte, verabschiedete sich nun Geistl. Rat Franz S.
Reicheneder von seiner Pfarrgemeinde. In einem feierlichen
Abschiedsgottesdienst, der vom Kirchenchor unter Leitung von Karl Geiger mit der
„Josefi-Messe" von A. Högn und den Hymnen „Die Himmel rühmen", „Nun danket alle
Gott", „Herr unser Gott" festlich umrahmt wurde, unterstrich der scheidende
Seelsorger die Bedeutung des Friedens in unserer Zeit. Die Fähigkeit Frieden
nach außen weiterzugeben, besitze aber nur der, der selbst Frieden im Herzen
trage. ....

.7 Viechtacher Bayerwald-Bote, 4.10.1974

Installation von Pfarrer Krottenthaler



Mensch bleiben in einer Zeit der Hektik und des Fließbandes

Pfarrer Otto Krottenthaler feierlich installiert - Dank an Geistlichen Rat
Reicheneder



Ruhmannsfelden. Am vergangenen Sonntag feierte die Pfarrgemeinde die fei­erliche
Installation des neuen Pfarrers. Ein stattlicher Kirchenzug, angeführt vom
Spielmannszug der Freiwilligen Feuerwehr, begleitete Pfarrer Otto Krot­tenthaler
zur Pfarrkirche. Neben den kirchlichen und weltlichen Vereinen der Marktgemeinde
bekundeten als Vertreter des öffentlichen Lebens Landrat Helm. Feuchtinger sowie
die Gemeinderäte der Marktgemeinde Ruhmanns­felden und der Gemeinde Zachenberg
mit ihren Bürgermeistern Alois Zellner und Georg Artmann ihre Anteilnahme und
Verbundenheit zur hiesigen Pfarr­gemeinde.

Zu Beginn des Festgottesdienstes richtete Pfarrer Krottenthaler, nach seiner
Vorstellung durch Dekan Günthner aus Bodenmais, Begrüßungsworte an die
Pfarrgemeinde. Er versicherte, er wolle alles in seiner Macht Stehende tun, um
der Gemeinde ein guter Seelsorger zu sein. Im Verlauf des Festgottesdiens­tes,
vom Kirchenchor umrahmt mit der „Josefi-Messe" v. A. Högn und den Chören „Groß
ist der Herr" und „Preiset mit feurigem Danke den Herrn" wurde mit der
feierlichen Übergabe des Evangelienbuches durch Dek. Günthner die symbolische
Übertragung der Pfarrei vollzogen. Während der Opferung übergaben Dekan Günthner
u. Franz Hacker, Vorsitzender des Pfarrgemeinderates, Kelch und Hostienschale an
Pfarrer Krottenthaler. …

.V Dokumente zum 80. Geburtstag von August Högn

.1 Viechtacher Bayerwald-Bote, 2.8.1958

80. Geburtstag von August Högn



Rektor Högn feiert 80. Geburtstag

Der Jubilar ist Ehrenbürger der Gemeinde



Ruhmannsfelden. Heute kann Rektor August Högn seinen 80. Geburtstag feiern. Der
Jubilar wurde am 2. August 1878 in Deggendorf als Sohn der Buchhändlerseheleute
Andreas und Helene Högn, letztere eine geborene Zöpfl, Kaufmannstochter von
Geiselhöring, geboren. Er besuchte die Knabenschule in Deggendorf, die damalige
Lateinschule in Metten und dann die Präparandenschule in Deggendorf. Nach dem
Absolutorium der Lehrerbildungsanstalt in Straubing in Jahre 1898 praktizierte
er an der Knabenschule in Deggendorf unter Hauptlehrer Buchner und Edelmann, kam
dann als Aushilfslehrer nach Neukirchen bei Haggn, Bezirk Bogen, später nach
Schaufling (Deggendorf) Geratskirchen (Eggenfelden) und wurde Hilfslehrer in
Zeilarn bei Simbach am Inn. Von dort wurde Rektor Högn nach Wallersdorf
versetzt, legte 1902 an der Regierung von Niederbayern die Anstellungsprüfung ab
und wurde an 1. Januar 1903 Schulverweser in Wallersdorf. Dort verehelichte er
sich im Juli 1904 mit der Bierbrauerstochter Emma Gerstl. Aus der harmonischen
Ehe (die Gattin starb erst 39jähring im Jahre 1926) gingen zwei Kinder hervor,
ein Sohn und eine Tochter.

Am 1. Juni 1905 kam Högn in den Bayerischen Wald nach Eberhardsreuth im Kreis
Grafenau. Seit 1. Januar 1910 wirkt der Jubilar an der Volksschule
Ruhmannsfelden, die er bis 1945 zur vollsten Zufriedenheit der Schulbehörde
leitete. Im Jahre 1929 wurde Högn zum Oberlehrer und 1940 zum Rektor ernannt.
Nach Beendigung des Krieges mußte er den Schuldienst quittieren und kam erst
wieder 1947, jedoch nur kurze Zeit bis zu seiner Pensionierung in den
Schuldienst.

Außer dieser, nahezu 50jährigen Erziehungsarbeit war er seit 1898 ununterbrochen
bei den Kirchenchören, teils als Aushilfe, teils als Chorleiter tätig. Er galt
als guter Sänger und tüchtiger Organist sowie auch Kirchenkomponist. Außerdem
führte er die Gemeindeschreiberei der Gemeinde Zachenberg von 1913 bis 1920, war
während dieser Zeit auch Rechner des Darlehenskassenvereins Ruhmannsfelden und
40 volle Jahre Schriftführer der Freiwilligen Feuerwehr Ruhmannsfelden.
Gleichzeitig war er auf dem Gebiete der Heimatforschung schriftstellerisch
tätig. Er schuf das Heimatbüchlein "Geschichte des Marktes Ruhmannsfelden" und
die "Geschichte der Gemeinde Zachenberg", die demnächst in Druck erscheinen
wird.

Der Jubilar ist Ehrenbürger der Gemeinde Ruhmannsfelden und Zachenberg und wurde
für treue Mitarbeit in der Mundartforschung von der Wörterbuchkommission der
Bayerischen Akademie der Wissenschaften in München bereits im Jahre 1933 mit
einer Urkunde geehrt. Während des ersten Weltkrieges wurde der Jubilar 1915 zum
Heeresdienst eingezogen und erwarb sich neben schon verliehenen Orden wie dem
König-Ludwig-Orden auch das Militär-Verdienstkreuz. Sein Lieblingshobby ist die
Jägerei.



.2 Viechtacher Bayerwald-Bote, 5.8.1958

Nachbericht zum 80. Geburtstag von August Högn



"Schön die Abendglocken klangen"

Rektor Högn wurde zum 80. Geburtstag geehrt



Ruhmannsfelden. Ein großer Teil der Bevölkerung nahm Anteil an dem 80.
Geburtstag von Rektor Högn, um so das reiche Schaffen des Jubilars zu würdigen.
Viele schöne Geschenke gingen zu diesem Ehrentag im Haus in der Schulstraße ein
und gaben Zeugnis davon, daß Rektor Högn weit über die Grenzen des Landkreises
hinaus bekannt und geachtet ist. Wie im Vorjahr brachte der Männerchor auch
diesmal am Vorabend des Festes unter Leitung von Franz Danziger seinem
Ehrenmitglied ein Ständchen. Nach dem Abendläuten sang der Chor "Schon die
Abendglocken klangen" und Vorstand Hans Czech gratulierte für die Sangesbrüder.

Bürgermeister Hans Muhr sprach anschließend für die Marktgemeinde die besten
Glückwünsche aus und überreichte ein Geschenk. Damit sollte ausgedrückt werden,
daß man das Wirken des Rektors als umsichtigen Pädagogen in Ruhmannsfelden nicht
vergessen hat. Den Gratulanten schloß sich dann Bürgermeister Ludwig Bielmeier
aus Zachenberg an. Auch er überbrachte ein Geschenk. Der Vorstand der
Freiwilligen Feuerwehr, Max Süß, wünschte zum Fest viel Glück und überreichte
eine Ehrenurkunde. Gleichzeitig überbrachte er dem Jubilar die Goldene
Ehrennadel des Vereins, die höchste Auszeichnung, die die Ruhmannsfeldener Wehr
zu vergeben hat. Bei der Feuerwehr ist Rektor Högn Ehrenschriftführer. Für den
Krieger- und Veteranenverein sprach Alois Steinbauer dem Ehrenmitglied die
herzlichsten Glückwünsche aus und übergab ein Geschenk.

Rektor August Högn dankte allen Vereinsvorständen für die guten Wünsche und
berichtete mit Humor kleine Episoden die er bei diesem oder jenem Verein während
seiner jahrzehntelangen Tätigkeit erlebt hat. Dem Männerchor versprach er,
weiteres Notenmaterial zur Verfügung zu stellen. Am Festtag selbst marschierte
auch noch die Ruhmannsfeldener Blaskapelle auf, um dem Musiker Högn eine
Geburtstagsfreude zu bereiten. Unter der Stabführung von Ludwig Heinrich erklang
eines der Lieblingslieder des Jubilars "Grün ist die Heide".

.VI Dokumente zum Tod von August Högn

.1 Viechtacher Bayerwald-Bote, 14.12.1961

Todesanzeige



Am Mittwoch, den 13. Dezember 1916, früh 5 Uhr, nahm Gott der Herr nach Empfang
der hl. Sterbesakramente im 84. Lebensjahr zu sich in die Ewigkeit den
ehrengeachteten



Herrn August Högn



langjähriger Lehrer und Rektor an der Volksschule Ruhmannsfelden



Der Verstorbene versah viele Jahre hindurch mit unermüdlichem Fleiß auch den
Dienst des Organisten an der hiesigen Pfarrkirche.



Es ist uns daher ehrende Verpflichtung, dem Verstorbenen auch auf diesem Wege
für diese Tätigkeit ein aufrichtiges "Vergelts Gott" in die Ewigkeit
nachzurufen.



Zu äußeren Zeichen der Dankbarkeit wird für ihm am kommenden Donnerstag, 9 Uhr,
in der Pfarrkirche ein feierliches Requiem angehalten werden, an dem sich auch
die hiesige Schule beteiligen wird. An alle Pfarrangehörigen, die ihn ja zum
großen Teil auch als Lehrer in der Schule hatten, ergeht hiermit Einladung zur
Teilnahme



Ruhmannsfelden, den 14. Dezember 1961



Die kath. Kirchenverwaltung Ruhmannsfelden

Georg Krieger, Kirchenpfleger                      Franz Ser. Reicheneder,
Pfarrer



.2 Viechtacher Bayerwald-Bote, 15.12.1961

Nachruf



Er war Mitstreiter für alles Gute und Wahre

Rektor a. D. August Högn starb im 84. Lebensjahr



Ruhmannsfelden. Im 84. Lebensjahr stehend, starb am Mittwoch nach langer
schwerer Krankheit Rektor a. D. August Högn. Ein aufrechtes, arbeitsames und in
Güte gereiftes Leben fand damit seinen Abschluß. Es war das Leben eines Mannes,
des sein umfangreiches Wirken auf humanistischen Grundsätzen aufgebaut und der
reiches Wissen und Können sehr oft und mit selbstloser Begeisterung in den
Dienst seiner Mitmenschen gestellt hatte. In seiner lauteren und geläuterten
Lebensart war er nicht nur ein Mitstreiter für alles Gute und Wahre, sondern er
vermittelte auch still und bescheiden seinen vielen Freunden und Kameraden seine
edle Anschauung über die vielfältigen Belange des Lebens. Mit dem Tod von August
Högn, der weit über die Grenzen seiner Heimat bekannt, geachtet und geschätzt
war, verlieren die Marktgemeinde und auch die Gemeinde Zachenberg, deren
Ehrenbürger der Verstorbene war, einen ihrer Besten.

Nicht nur als umsichtiger Pädagoge, der sich seinen Schützlingen stets sehr
gewissenhaft und mit verständnisvoller Liebe gewidmet hatte, sondern auch als
anerkannter Heimatkundler war Rektor Högn bekannt. "Die Geschichte von
Ruhmannsfelden" und auch "Die Geschichte von Zachenberg" entstammen nach langer
und gründlicher Forschung seiner Feder. Aber auch als Komponist, - er hatte
mehrere schöne Messe geschrieben, - war er hervorgetreten. Jahrzehnte hatte er
auch als Chorleiter und Organist in der hiesigen Pfarrkirche gewirkt, bis er
dieses Amt, das ihm immer so viel bedeutet hatte und dem er zu jeder Zeit von
ganzem Herzen zugetan war, krankheitshalber hat aufgeben müssen.

Seit 1910, dem Jahr seiner Ankunft in Ruhmannsfelden nach seinem Wirken in
Eberhardsreuth, hatte er sich in den örtlichen Vereinen mit seiner vitalen Kraft
und mit dem seine Person kennzeichnenden Idealismus zur Verfügung gestellt. Er
war maßgebend am Aufbau der Turnhalle beteiligt, war seit 1904 Mitglied der
Feuerwehr. Unterstützte stets aktiv und passiv, die jeweiligen
Gesangsvereinigungen, diente Jahrzehnte im Krieger- und Veteranenverein und war
fast 60 Jahre Mitglied des Bayerischen Lehrervereines.

Alle diese genannten Verbände hatten "ihren" August Högn in Würdigung seiner
außerordentlichen Verdienste zum Ehrenmitglied oder Ehrenfunktionär ernannt. Der
Bayerische Lehrerverein hatte ihm das Goldene Ehrenblatt für 50jähriges
Mitwirken verliehen, das Landesamt der Bayerischen Feuerwehren hatte ihm die
höchste Auszeichnung überreicht, er war Inhaber der Goldenen Ehrennadel des
Bayerischen Turnerbundes und war auch im Besitz der Goldenen Plakette des
Veteranenvereins. Als bezeichnend sei noch erwähnt, daß August Högn um 1910 bis
1920 ehrenamtlich alle schriftlichen Arbeiten für die Gemeinde Zachenberg
erledigt hatte.

Das Herz eines Menschen, der schlicht und einfach, und doch so stark, sein in
sich abgerundetes Leben immer wieder selbstlos in den Dienst der Gemeinschaft
gestellt hatte, schlägt nicht mehr. Die sterblichen Überreste wurden noch am
Sterbetag nach Deggendorf überführt, um neben seiner, am 19. Juli 1926
verstorbenen Frau, die August Högn am 20. Juli 1904 geehlichte hatte, beigesetzt
zu werden. Trotzdem werden sich die Fahnen der Ruhmannsfeldener Vereine am Tag
der Beerdigung symbolisch vor dem Grabe ihres Rektors senken zum Zeichen dafür,
daß Rektor Högn allen seines edlen Wesens wegen Vorbild bleiben möge. Das Lied
"Vom guten Kameraden" möge ihm letzter, herzlicher Gruß sein.



.3 Elfriede Schlumprecht an Lehrer Schambeck, 3.1.1962

Dankesschreiben nach der Beerdigung von August Högn



München, den 3.1.1962



Sehr geehrter Herr Lehrer Schambeck! Sie sprachen so herzliche Worte des
Abschieds am Grabe meines verstorbenen Vaters, daß ich auf das tiefste bewegt
war. Es hat mir wohlgetan zu hören, daß mein Vater bei seiner Kollegenschaft der
Ruhmannsfeldener Schule geachtet und beliebt gewesen ist. Ihnen für die
freundlichen Worte des Mitgefühl und für den schönen Kranz zu danken, den Sie im
Namen Ihrer Kollegen am Grabe niederlegten, ist mir ein aufrichtiges Bedürfnis.



Mit meinen besten Grüßen und nochmaligen Dank



Ihre

Elfriede Schlumprecht

.4 Elfriede Schlumprecht an Rektor Langesee, 3.1.1962i

Dankesschreiben nach der Beerdigung von August Högn



München, den 3.12.1961?



Sehr geehrter Herr Rektor! Ich möchte mich besonders bei Ihnen für die
herzlichen Worte des Abschieds und der Anerkennung bedanken, die Sie meinem
Vater am Grabe gewidmet haben. Es war für mich ein besonderer Trost, wiederum zu
hören, daß mein lieber Vater als Mensch wie als Pädagoge bei allen, die mit ihm
zu tun hatten, beliebt und geachtet gewesen ist. Es ist mir ein besonderes
Bedürfnis, Ihnen für die Herzlichkeit, mit der Sie der Person und dem Wirkens
meines Vaters gedacht haben, nochmals meinen Dank zu sagen. Diesen Dank bitte
ich auch Herrn Kreisschulrat Botschafter zu übermitteln.



Mit meinen besten Grüssen und Dankesworten bin ich



Ihre



Elfriede Schlumprecht

.VII Sonstige Dokumente

.1 Viechtacher Bayerwald-Bote, 7.5.1959

Einweihung des Schulanbaus



Gemeinde hat Verständnis für ortsgebundene Landschule

Der Anbau zum Schulhaus wurde feierlich von Pfarrer Reicheneder geweiht



R u h m a n n s f e l d e n. Bei schönstem Sommerwetter wurde am
Christihimmelfahrtstag der Zubau zum Schulhaus von Pfarrer Franz Reicheneder
eingeweiht. Eine Festmesse war vorausgegangen. Architekt Ritz übergab den
Schlüssel zum neuen Gebäude an Bürgermeister Muhr. Nach dem Weiheakt folgten
Ansprachen der Vertreter, der Behörden und Ämter. Ein Festmahl schloß die Feier.
....

... Der verstärkte Kirchenchor unter Leitung von Chorleiter Danziger und die
Ruhmannsfeldener Blaskapelle sorgten für die musikalische Umrahmung des
Festgottesdienstes.

Zwanglos und ohne Festzug begab man sich dann das kurze Stück Weg von der Kirche
zum Festakt beim Schulhaus. Bürgermeister Wolfgang Muhr konnte
Oberregierungsschulrat Dr. Limmer, Landshut, MdB Dr. Stefan Dittrich, MdL Alois
Rainer, Bezirksrat Ferdinand Kollmer, Landrat Rudolf Kauer, Schulrat
Bothschafter, Architekt Ritz, Kreisbaumeister Pfeiffer, die Bürgermeister der
benachbarten Gemeinden Glasschröder, Gotteszell, Bielmeier, Zachenberg und
Jungbeck, Patersdorf, sowie Rektor Högn und Rektor Langesee mit den übrigen
Lehrkräften begrüßen. Nach einem Gedicht, das drei Mädchen vortrugen, sang der
Mädchenchor unter Leitung von Lehrerin Piehler den Chor „Brüder reicht die Hand
zum Bunde" von Mozart. Architekt Ritz, der in der Bauleitung auf den
umfangreichen Vorarbeiten von Kreisbaumeister Pfeiffer fußen konnte, übergab den
Schlüssel zum neuen Gebäude an Bürgermeister Wolfgang Muhr mit dem Wunsch, daß
das Gebäude Lehrern wie Kindern eine gute Erziehungsstätte sein möge. …

.2 Artikel in der Reicheneder-Chronik über August Högn

Rubrik: Schul- und Bildungswesen, Lehrerpersonal geordnet nach Rektoren



VI. August Högn

1921 bis 1945 bzw. 1947



August Högn wurde am 2. 8.1878 in Deggendorf geboren. Sein Lebenslauf bis zu
seinem Wirken in Ruhmannsfelden ist enthalten in dem Berichten des Viechtacher
Bayerwaldboten, die anlässlich seines 80. Geburtstages erschienen sind.

Nach Ruhmannsfelden kam er am 1.1.1910 und wirkte dann hier als 2. Lehrer unter
Auer, bis er am 1.10.1921 dessen Nachfolger als 1. Lehrer antrat. Er wurde hier
Oberlehrer und Hauptlehrer und schließlich 1940 Rektor der Schule
Ruhmannsfelden.

Im Jahre 1926 starb in Ruhmannsfelden seine Frau im Alter von erst 39 Jahren.

Als 1. Lehrer übernahm er dann in Ruhmannsfelden auch den Organistendienst (wenn
auch Schul- und Kirchendienst bereits getrennt waren) und übte denselben auch
noch in seinem Ruhestand bis zum Jahre 1953 aus.

Einen Namen hat er sich auch gemacht als Forscher an der Heimatgeschichte.
Zahlreiche geschichtliche Abhandlungen sind von ihm in Zeitschriften und
Zeitungen erschienen.

Die Geschichte der Gemeinde Zachenberg und die der Feuerwehr Ruhmannsfelden
liegen in Schreibmaschinenschrift vor.

Über den Markt Ruhmannsfelden gab er ein Büchlein heraus: "Geschichte von
Ruhmannsfelden"

Aus Anlaß seines 25.jährigem Dienstjubiläums wurde er am 21.7.1923 in dankbarer
Anerkennung seiner großen Verdineste um Gemeinde und Schule zum Ehrenbürger der
Marktgemeinde Ruhmannsfelden ernannt.

Wie alle Lehrkräfte an den Schulen, auch an der Schule Ruhmannsfelden wurde er
nach dem 2. Weltkrieg und dem Ende des Hitlerreiches seines Dienstes enthoben.
Wurde aber dann 1947 nochmals kurz angestellt und ging dann noch im gleichen
Jahr in den endgültigen Ruhestand. Er verbrachte denselben in Ruhmannsfelden. Am
13.12.1961 ist er im Alter von 84 Jahren gestorben. Die Beerdigung fand in
Ruhmannsfelden statt.

.3 Auszug aus den Memoiren von Franz Danziger sen.

... In der Zeit von 1921 bis 1924 galt mein besonderes Interesse der Arbeit im
Katholischen Gesellen - und Meisterverein, heute Kolping. Besonders das
Theaterspielen hatte es mir angetan. Wir spielten große Sing- und Schauspiele
wie: "Bettelprinzessin", "Henkersohn und Zigeunerin", "Zunftmeister von
Nürnberg", "Über Land und Meer", viele Volksstücke wie: "Austragsstüberl", "Das
Glück vom Riedhof" u. andere. Auch bei dem vom Radlerverein aufgeführten Stück:
"Der Tatzelwurm" im Mai 23 und die vom Turnverein inszenierte Operette "Der
Postillion" wirkte ich mit. Als zweiter Geiger war ich auch in dem von Rektor
Högn geführten Orchester bei weltlichen und kirchlichen Veranstaltungen tätig. …

ANHANG

.1 Anmerkung

\end{document}