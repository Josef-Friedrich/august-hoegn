%!TEX program = lualatex
\documentclass[12pt,a4paper]{book}

\usepackage[ngerman]{babel}
\usepackage{fontspec}
\usepackage[sf]{titlesec}
\usepackage{libertine}
\usepackage{longtable}
\usepackage{paralist}
\renewcommand{\labelitemi}{$-$}
\renewcommand{\labelitemii}{$-$}

\renewenvironment{quote}
               {\list{}{\itshape\rightmargin\leftmargin}%
                \item\relax}
               {\endlist}


%\titleformat{⟨command⟩}[⟨shape⟩]
% {⟨format⟩}
% {⟨label⟩}
% {⟨sep⟩}
% {⟨before-code⟩}
% [⟨after-code⟩]

\titleformat{\chapter}[display]
  {\sffamily\bfseries\Large}
  {\filright\Huge\thechapter}
  {1ex}
  {\titlerule\vspace{1ex}\filleft}
  [\vspace{1ex}\titlerule]

  \renewcommand*\descriptionlabel[1]{\hspace\labelsep
                                \sffamily\bfseries #1}


\author{August Högn}
\title{Geschichte von Ruhmannsfelden}
\date{1949}

\begin{document}

\maketitle

Meinen Eltern Josef und Anita Friedrich gewidmet

Mein Dank gilt:

Herrn Pfarrer Meier, Lotte Freisinger,

Franz Danzinger jun.

Textgrundlage:

Erstdruck im Verlag Kallmünz bei Regensburg,1949

\tableofcontents

\newpage

AUGUST HÖGN

1878 1961

GESCHICHTE VON

RUHMANNSFELDEN

EDITIERT VON

JOSEF FRIEDRICH 2003

\part{Geschichtliches}

\chapter{Wie hat es vor seiner Entstehung ausgesehen?}

Wo unser Heimatort Ruhmannsfelden ist, da schaute es früher ganz anders aus als
heute. Es gab keine Ortschaften, keine Häuser, Straßen und Wege, keine Brücken
oder gar Eisenbahnen. In den undurchdringlichen Wäldern mit den Riesenbäumen von
Eiben, Buchen und Eichen, mit den hohen Sträuchern, Gift- und Heilkräutern und
den dichten Mooren und Moosen, da hausten Tiere der verschiedensten Art und
Größe, Tiere, die schon längst ausgestorben sind. In den Felshöhlen und
Steinriegeln lebten die Bären, Wölfe, Luchse und Füchse, die unheimlichen
Salamander, Eidechsen und Nattern. In den Gewässern schwammen unzählige Fische
und Wassertiere. In hohlen Bäumen wohnten die Wildkatzen und Marder und
großmächtige Adler, Bussarde, Habichte und Sperber hatten ihre Nester in den
dichten Kronen der hohen Bäume, Menschen gab es in dieser Wildnis anfänglich
noch nicht. Es fehlten ja alle notwendigen Lebensbedingungen für sie und die
ersten Menschen, die aus Böhmen über den Arber herüber kamen und in der Richtung
zur Donau und zur Donauebene unser Gebiet hier durchzogen, die mussten sich erst
mühsam einen Weg durch diesen Urwald suchen und nur da, wo es sich um eine von
Natur aus lichte Waldstelle handelte und wo sie einigermaßen von den wilden
Tieren sicher waren, konnten sie ihre Rast- und Ruheplätze anlegen, die dann von
ihren großen und kräftigen Hunden bewacht wurden.

\chapter{Der Nordgau kommt in den Besitz des Klosters Metten.}

Diesen Urwald lernte auch der damalige Kaiser Karl der Große kennen bei seinen
Jadgstreifen in den Vorbergen des bayerischen Waldes der Donau entlang. Bei
einem solchen Jagdausflug in die wildreichen Wälder traf er bei der Ortschaft
Berg in der Nähe von Deggendorf den frommen und wundertätigen Einsiedler Utto,
dem er versprach, ein Kloster in dieser Gegend erbauen zu wollen. Durch die
tätige und verständnisvolle Arbeit des Einsiedlers Utto und mit Hilfe der
kaiserlichen Unterstützung entstand das Kloster Metten, das sich immer der Gunst
höchster und weitester Kreise erfreuen und sich durch alle Stürme und Drangsale
des fehdereichen Mittelalters und der kriegerischen Jahrhunderte der Neuzeit
hindurch bis auf den heutigen Tag erhalten konnte und in allen Ländern und auf
allen Erdteilen als erstklassige Pflanzstätte der geistigen Bildung der Jugend
bekannt ist.

Da Kaiser Karl der Große Kirche und Klöster reichlich beschenkte, bekam auch das
neu entstandene Kloster Metten von ihm das ganze Gebiet rund um unseren
Heimatort herum, das damals „Nordgau“i genannt wurde. Dieser Nordgau umfasste
das Gebiet innerhalb der Grenze: Metten, Edenstetten, Faßlehen, Voglsang,
Kohlbach, Köckersried, Zachenberg, Klessing, Eckersberg, Unterauerkiel,
Asbachmündung, Altnussberg, Seigersdorf, Fernsdorf, Frankenried, Hornberg,
Einweging, Schusterstein, Ödwies, Hirschenstein, Kalteck, Edenstetten, Metten.

Innerhalb dieser Gebiets, das nun dem Kloster Metten gehörte errichteten die
damaligen Klosterherren von Metten große Höfe, welche Sie bewirtschafteten mit
Hilfe von Arbeitern, die sie von Metten und der Donauebene mitgebracht hatten.
Ein solcher Hof wurde benannt nach dem Namen des Hofverwalters. Da es damals nur
Taufnamen gab und der Hof „Villa d. h. Dorf“ genannt wurde, so hieß der Hof
Patos der Patoshof = Patersdorf, der Hof Fatos der Fatoshof = Fratersdorf, der
Hof Lempfers (Landfrit) Lempfershof = Lämmersdorf. Bei diesen Höfen fehlten auch
nicht Obst-, Gemüse- Hopfengärten, auch nicht die Weinberge (Patersdorf).
Allerdings konnten sich die Mettener Klosterherren nicht lange des Besitzes vom
Nordgau, wie das hiesige Gebiet früher geheißen hat, erfreuen. Alles, was sie
hier gesessen haben, wurde ihnen von Arnulf dem Bösen (911 - 937) enteignet und
dem Grafen Hartwig von Bogen zugesprochen für seine dem Arnulf geleisteten
Hilfsdienste. Überdies konnte Graf Hartwig dieses Gebiet notwendig brauchen, da
er sehr viele Kinder hatten (angeblich 38) und jedem von seinen Söhnen ein
möglichst großes Besitztum geben und eine ertragreiche und einflussreiche
Stellung mit Amt, Macht und Würde (Grafensitz, Bischofsitz und dergleichen)
sichern wollte.

\section{Um 950 unter Graf Aswin von Bogen entstanden Siedlungen}

Auf diese Weise gelangte nun die Landschaft um unseren Heimatort herum in den
Besitz von Aswin, einem Sohn des Grafen Hartwig von Bogen. Dieser schickte
sofort Arbeiter hierherii, die das Land urbar machen mussten. Dabei musste vor
allem der Urwald aus dem Wege geräumt werden. Mit Säge und Hacke hätte man aber
die kolossal dicken und riesig hohen Bäume nicht fällen können. Infolgedessen
musste der Wald niedergebrannt werden. Das Holz war ja damals vollständig
wertlos, weil es ohnehin zu viel gab. Nur die Asche konnte später verkauft
werden an die Seifensieder.

Die Namen der Orte Prünst, Brenning, Klessing, usw. und alle -sang, -seng und
-sing Namen deuten heute noch hin auf die Plätze der damaligen großen
Waldbrände.

Der Waldboden wurde dann von den Baumstöcken und -wurzeln gerodet und
anbaufähiger Boden daraus gemacht. Zugleich wurden an diesen Orten Wohnstätten
für die Rodungsarbeiter errichtet. So entstanden zu dieser Zeit die
„ried“-Namen, z. B. Giggenried bei Lämmersdorf, Kaikenried bei Fratersdorf,
Zuckenried bei Patersdorf. Diese „ried“-Namen sind wie die Dorfnamen auch mit
dem Taufnamen des Rodungsvorarbeiters verbunden, z. B.:

Köckersried= Choteschalk (ried)

Kaikenried = Hacco

Giggenried = Chundahar

Zuckenried = Sigine

Hasmannsried = Hasnolf

Lobetsried = Luipolf

Perlesried = Perolf

Triefenried = Trunolf

Die „dorf“-Namen (Mettener Gründungen) sind also die ältesten. Dann kommen die
„ried“-Namen (Bogener Gründungen). Gnänried, auch Gnadenried genannt =
Kleinried, Gollersried oder Köhlersried = Gottlesried, usw. zählen nicht zu den
echten „ried“-Namen. Zu gleicher Zeit erscheinen auch die „berg“-Namen, die aber
auch mit Taufnamen verbunden sein müssen, z. B. Dietsberg = Theotosberg,
Wolfsberg = Wolfichosberg, usw. Harnberg = Heidenberg zählt nicht zu den echten
“berg“-Namen. Dieser Name sagt uns, dass dort Harn oder Flachs gebaut wurde. Die
an der Westseite von Ruhmannsfelden dicht zusammengedrängten „ing“-Namen sind
keine echten „ing“-Namen, weil sie nicht in Verbindung stehen mit einem
Taufnamen, z. B. Einweging = wag, wog, wöge, weging = Wasser in einem Graben
oder Weiher - Sintweging = viel Wasser, Tradweging = drate = eilig, schnell,
reißend = Stelle des reißenden Wassers. Dazu gehören auch Handling, Zottling
(Zeidler- oder Imkerort).

Wie bereits erwähnt, ließ Graf Aswin den Waldboden in hiesiger Gegend roden. Ein
Vorarbeiter leitete die Rodungsarbeiten. Der Mann, der diese Arbeiten hier
leitete, war ein Bediensteter (ministeriale, also kein Adeliger) des Grafen
Aswin von Bogen und führte den schönen Namen „Hrothimar oder Hrothmar oder
Rumar.“ Hrot(i) = Ruhm, Sieg und mar(u) = berühmt, also Rumar = der
Siegberühmte.

Dieser Rumar errichtete sich eine Siedlung, bestehend aus Wohnhaus, Stall und
Stadel. Seine Dienstmannen bauten sich anschließend an diesen Hof (Villa oder
Dorf) in schönem Viereck (siehe Vierecksanlage der heutigen Bachgasse) ihre
damals noch üblichen niederen Wohnhäuser. Zum Zeichen der Wehrhaftigkeit wurde
dieser Siedlung ein viereckiger Turm aus Findlingssteinen, Erdreich und Lehm
errichtet (siehe Turmruine in Linden). Die ganze Siedlung erhielt den Namen der
Vorarbeiters, wie bei den „dorf“- und „ried“-Namen, also den Namen Rumars und
wurde deshalb Ruhmarsfeld, Rumarsfelden genannt. („felden“ nicht im Sinne eines
einzigen Feldes, sondern im Sinne eines größeren Gebietes).

Der Name „Rumarsfelden“ wird verschiedentlich geschrieben. In einer
Oberaltteicher Urkunde von 1184 steht noch „Rumarsfelden“. In einer späteren
Niederalteicher Urkunde heißt es schon „Rudmarsfelden“ und 1394 schrieb man
„Rumatzfelden“. In einer Urkunde von 1448 erscheint der Name „Ruebmannsfelden“.
Rübe im Trutzwappen) und auf der Fink´schen Karte aus der 2. Hälfte des 17.
Jahrhunderts kann man „Ruemannsfelden“ lesen. Nach dieser Zeit hat man an die
Stelle des „e“ in diesem Worte ein „h“ gesetzt und seit dieser Zeit schreibt man
Ruhmannsfelden. Die Formen „Rumarsfelden“ oder „Ruhmarsfelden“, so wie sie das
Volk hier noch heute ausspricht, die tragen den Stempel der Originalität.

\section{Ruhmannsfelden unter den Nachfolgern Rumars.}

Rumars Nachfolger waren fleißige Leute, sodass bald um die Siedlung herum schöne
Felder und Wiesen entstanden. Wege gebaut, Stege (Stegmühle) und Brücken
(Bruckhof und Bruckmühle) errichtet und Weiher (Weiherwiesen) angelegt wurden.
Sie waren auch tüchtige Jäger und gingen in Spiel und Wettkampf
(Alkovenschlüsselrennen, Hürdenspringen, Steinheben und Steinwerfen,
Armbrustschießen, Speerwerfen, Lanzen- und Ringelstechen zu Fuß und auf dem
Pferd, mit und ohne Sattel) meistens als Sieger hervor. Arnold von Rumarsfelden
wird gefeierter Sieger bei einem Ritterspiel in Zürich in der Schweiz. Aber die
Nachfolger Rumars waren nicht lange hier. Das Geschlecht der Rumar starb bald
aus. Die Siedlung Ruhmannsfelden kam in der 2. Hälfte des 13. Jahrhunderts in
den Besitz der an der Donau zwischen Plattling und Bogen reichbegüterten Grafen
von Pfelling. Nach dem Tode Heinrichs von Pfelling, der neben Burg und Ortschaft
Ruhmannsfelden auch einige Ortschaften um Ruhmannsfelden herum sein Eigen
nannte, fiel dies alles dem Landesherren Heinrich XIII. zu. Nach dessen Tode kam
Ruhmannsfelden in den Besitz seiner Söhne Otto III., Ludwig III. und Stefan I.
(1290). Diese verkauften und vertauschten Grundstücke und Höfe, die zur Siedlung
Ruhmannsfelden gehörten, an reiche Grafen z. B. an die Degenberger, Parsberger
usw. Der Wehrturm wurde abgebrochen.

Erst später gelangte die Siedlung Rumarsfelden in den Besitz des
Zisterzienserklosters Aldersbach durch Kauf (1295). Lauf Brief vom 28. April
1294, ausgestellt in Regensburg, verkaufte Herzog Otto III. mit Zustimmung
seiner beiden Brüder Ludwig und Stefan die Ortschaft Ruhmannsfelden „um die Last
ihrer Schulden zu erleichtern“ und aus Anhänglichkeit an den Zisterzienserorden
mit allen Wäldern, Fischwassern, Wegen und Stegen, Viehweiden, Mühlen, dem
Bruckhof mit der Mühle, dem Dorfe Arnetsried, dem Weiler Labertsried, der
Ortschaft Weichselsried, dem Hofe zu Lemersbach und zu Zierbach, einschließlich
Gerichtsbarkeit (mit Ausnahme über Straßenraub, Notzucht und Totschlag) und
Grundbarkeit um 400 Pfund Regensburger Pfennig an das Kloster Aldersbach. Im
Jahre 1292 am 4. Mai wurde der Verkauf abgeschlossen.

\chapter{Ruhmannsfelden unter den Zisterziensern von Aldersbach}

\section{Unter der Herrschaft des Klosters Aldersbach ging es Ruhmannsfelden gut.}

Unter dieser Herrschaft blühte Ruhmannsfelden auf. Am Bühl stand eine kleine
hölzerne Kapelle zu Ehren des hl. Laurentius. Die Sebastiani-, Laurentius- und
Magdalenen-Kapelle standen überall außerhalb der eigentlichen Siedlung. Bei
dieser Laurentius-Kapelle am Bühl war hie und da eine Andacht oder ein
Gottesdienst, zu dem ein Geistlicher von Geiersthal kommen musste. Vermutlich
war bei der Laurentius-Kapelle auch die erste Begräbnisstätte.

Nachdem die Siedlung dem Kloster Alderbach gehörte, wurde aus der Kapelle ein
schönes Kirchlein gemacht, wenn auch anfänglich noch ein kleines und es fanden
nun öfters und regelmäßig hier Gottesdienste statt. Da kamen auch Händler und
Krämer von Viechtach, Regen und Deggendorf. Diese schlugen zu beiden Seiten des
Weges von der Siedlung zum Kirchlein hinauf und um das Kirchlein herum ihre
Verkaufsstände und Buden auf. Später bauten sich die Händler und Krämer feste
Wohnsitze hier und auf diese Weise entstand anschließend an die Siedlung, in der
sich die Lederer, Gerber, Stricker usw. bereits vorher sesshaft gemacht hatten,
der untere Markt mit den Bäckereien, Metzgereien, Brauereien und Fragnereien und
zwar wieder in Vierecksanlage, was für das 13. und 14. Jahrhundert bezeichnend
ist.

Die Ortschaft Ruhmannsfelden hat sich von unten nach oben, von West nach Ost,
also von der ursprünglichen Siedlung (der heutigen Bachgasse) aus zum
Laurentius-Kirchlein am Bühl hin entwickelt. Von einem Schloss auf der Anhöhe
steht weder in den Urkunden etwas, noch sind auch nur die spärlichsten Überreste
oder Spuren von einem solchen Schloss oder einer Ansiedlung um ein solches herum
auf der Anhöhe zu finden. Wohl sprechen die Urkunden von einer Villa und von
einer Burg Ruhmannsfelden. Darunter ist aber wohl nur die anfängliche Siedlung
mit oder ohne Wehrturm und den dazu gehörigen Grundstücken, Höfen und Rechten zu
verstehen.

\section{Die Gründung des Klosters Gotteszell 1295}

Wo heute die Ortschaft Gotteszell steht, war früher nur ein einziger großer Hof,
der einem ungarischen Grafen, namens „Drozza“ gehörte und „Drozzalach“, kurz
“Droßlach“ heiß. Dieser Hof kam in den Besitz des Grafen von Pfelling bei Bogen,
dem auch früher die Siedlung Rumarsfelden gehörte. Dieser Graf schenkte dieses
“Droßlach“ im Einvernehmen mit seiner Gemahlin Mechthildis, Kinder hatten sie
nicht, dem Kloster Aldersbach (Zisterzienserklosters) mit der Absicht, aus
diesem Hof ein schönes Kloster zu machen, was auch geschah. Anfänglich waren nur
zwei Klosterherren in Gotteszell. Bischof Heinrich II. von Regensburg, der mit
dem Grafen von Pfelling verwandt war (Mechthildis war eine Schwester zu diesem
Bischof), genehmigte die neue Klosterniederlassung in Gotteszell, machte sie
selbstständig, indem der sie von Pfarrverbande Geiersthal löste und gab ihr den
schönen Namen „Cella Dei“ = „Zelle Gottes“ oder Gotteszell. Das war in der
gleichen Zeit, als das Zisterzienserkloster Aldersbach die Siedlung Rumarsfelden
von den damaligen Landesherren käuflich erwarb (1295).

\section{Die Pest in Ruhmannsfelden (1354 - 1357 nach H. Geheimrat Eberl).}

„Unter dem Krumstab des Klosters Aldersbach war gut zu leben“, berichten die
Urkunden aus damaliger Zeit. Die Siedlung Ruhmannsfelden lag am Kreuzungspunkt
der Straße von Cham, Viechtach und Deggendorf, also von Nord nach Süd und der
Straße von Regen, Fratersdorf, Kalteck, Bogen und Straubingiii, also von Ost
nach West. Da es damals Eisenbahnen noch nicht gab, war auf diesen Straßen ein
reger Verkehr von Fußgängern und Fuhrwerken, was sich auf das Geschäftsleben und
die Entwicklung von Ruhmannsfelden nur günstig auswirken konnte. Die bevorzugten
Waren der tüchtigen Handwerker von dort wurden gerne gekauft und der große
Durchgangsverkehr brachte den Gewerbetreibenden ein blühendes Geschäft.
Ruhmannsfelden wurde weit über den bayerischeniv Wald hinaus bekannt. Unter der
Herrschaft der Klosters Aldersbach brach für die Ruhmannsfeldner wirklich eine
glückliche, hoffnungsvolle Zeit an. Aber! Mit des Schicksals Mächten ist kein
ew`ger Bund zu flechten. Nur ein halbes Jahrhundert dauerte diese Zeit
ungestörten Schaffens und belohnten Fleißes.

Plötzlich kam für Ruhmannsfelden in der Mitte des 14. Jahrhunderts etwas ganz
Schreckliches. Die indische Beulenpost, die von Italien hierher eingeschleppt
wurde, suchte auch in Ruhmannsfelden und Umgebung ihre Opfer und kein Haus und
keine Familie blieben davon verschont. Ganze Familien wurden von dieser Seuche
hinweggerafft. Den Pestkranken reichte man die Nahrung mit Hilfe von langen
Stangen an ihre Fenster, um ja nicht mit ihnen in Berührung zu kommen. Jede
Möglichkeit einer Ansteckung wurde vermieden. Der Friedhof wurde zur Aufnahme
der Toten zu klein. Es mussten zwei neue Begräbnisplätze mit Massengräbern
errichtet werden, nämlich „das Grab“ auf der Anhöhe und „das Sichet“ von der
Ziegelei bis zum Mittelzellner Holz. Nur ganz wenige Einwohner blieben von
dieser furchtbaren Krankheit verschont.

In dieser Zeit kamen von auswärts böse Menschen, welche die Not und das Unglück,
das die Pest über die Einwohner von Ruhmannsfelden gebracht hatte, auszunützen
suchten. Sie machten den noch Übriggebliebenen allen möglichen Schwindel vor,
raubten und plünderten in den menschenleeren Wohnungen, wobei es auch zu Streit
und Raufereien kam. Dabei wurde auch das schöne Laurentius-Kirchlein am Bühl ein
Raub der Flammen.

\section{Ruhmannsfelden wird um das Jahr 1400 herum „Markt“.}

Aber trotz Post und Krieg entwickelte sich in einem Zeitraum von 100 Jahren
wieder ein reger Handel und Verkehr. Auch das Handwerk kam wieder in großen
Aufschwung. Handwerker, die bis dahin in Ruhmannsfelden nicht vertreten waren
(Nagelschmied, Seiler, Konditor, Chirurgen und Binder), bauten sich ihre
kleineren Häuser anschließend an den unteren Markt in die unmittelbare Nähe der
Kirche, woraus dann der obere Markt entstand. Die Tüchtigkeit der
Ruhmannsfeldner Handwerker und der Gewerbefleiß seiner Bürger brachten es
schließlich so weit, dass Ruhmannsfelden zum Markt erhoben wurde.

Wann die Markttitelverleihung erfolgte und von wem sie vollzogen wurden, ist
urkundlich nicht nachzuweisen, wie eben an anderen Plätzen auch. Jakob der
Rueerer stellte am 26. April 1416 eine Urkunde aus in welcher er sich „Dn czeitt
Richter des Markehtz zue Ruedmansfelden“ nennt. Es dürfte kaum ein Zweifel
bestehen, dass der Urkundenaussteller als landesherrlicher Richter über die
Markteigenschaft seines Wirkungsortes Bescheid wusste. Es ist deshalb
berechtigt, die Markteigenschaft zu Ruhmannsfelden schon seit dem Beginne des
15. Jahrhunderts in Anspruch zu nehmen. Am 15. Februar 1431 bekannte der Jakob
der Degenberger von Altnußberg „vielleicht derselbe, der 1418 bis 1431 Bachvogt
von Ruhmannsfelden war“ dass ihn das Gotteshaus zu Aldersbach zum Vogte über
seine armen Leute im „Markte“ zu Ruhmannsfelden und den Gütern in Viechtreich
eingesetzt habe. In einem Aldersbacher Codex vom Jahres 1452 ist die Rede von
dem „Forum Rudmannsfelden“, also Markt, während in einer Urkunde vom 2. April
1475 des „Opidum Rudmannsfelden“ erscheint, was mehr an die befestigte Siedlung
als an den Markt gemahnt. Auf Bitten der „Burger unnsers Margkts zu
Rudmannsfellden“ tut ihnen Herzog Albrecht IV. von Bayern-München die Gnade:
„.....Freyen sie auch wissenlich in crafft des Briefs, Also das sy vund all Ir
nachkommen, sich aller der gnaden vnd freihait geprauchen vnd nyessen mügen vnd
die haben sollen, In allermaß als vnnder vnser Märkt, in Nider Baiern von
vnnsern vordern gefreyt sein.“

Dieses Privileg ist nur in Abschrift erhalten und undatiert, steht aber zwischen
zwei Urkunden desselben Jahres 1469 und darf daher als aus diesem Jahre stammend
angenommen werden. In einem vom Jahres 1803 heißt es: „Es erhelle aus den
älteren Umständen Literale des Klosters Gotteszell vom Jahre 1566 bis 1602
kommen vor die „Geschworenen des Rats und ganze Bürgerschaft des „Markts“
Ruhmannsfelden, wie überhaupt seit der Begnadigung von 1469 keinerlei Zweifel an
dem Markrecht Ruhmannsfelden mehr aufkommen kann. In der Rechnung des Marktes
Ruemannsfelden sehr deutlich, daß der Markt Ruemannsfelden eine vollständige
Marktgerechtigkeit mit der angemessen Jurisdiktion besessen habe. Es war
derselbe schon in der Jahren 1500 mit einer förmlichen und gegenwärtig noch
vorhandenen, die Wahrheit des Altertums bezeigenden Raths Glocke ersehen.“

Ruhmannsfelden war zurzeit der Markttitelverleihung mit Toren versehen und die
Tapferkeit seiner Bürger war rühmlichst bekannt. Das waren ja die zwei
Voraussetzungen für die Marktitelverleihung einerseits und für die weißblauen
Rauten im Wappen andererseits. Ruhmannsfelden war wieder auf seiner früheren
wirtschaftlichen Höhe angelangt, auf die es unter der Herrschaft der
Aldersbacher Klosters gekommen ist.

Besonders zeigte sich dieser Aufschwung bei den Märkten. So strömten z. B. zum
Laurentius-Markte von weit und breit die Leute herbei und kaum konnte der
Marktplatz und der Kirchenplatz die zahlreichen Leute, Stände und Buden fassen.
Dass auch damals schon bei solchen Gelegenheiten Ordnung und Disziplin
herrschten, ersehen wir aus einer Verordnung für das Feilbieten von Waren. Die
diesbezügliche Marktständeordnung vom Freitag nach den Pfingstfeiertagen unseres
lieben Herrn Christo anno 1503 lautet: „- - es soll gehalten werden, wie hernach
folgt: Schmalz, Käse, Garn, Inschlit, Schmier, Wildpret, das Gehstuhl und das
Rauchleder soll man vor dem Schenkhaus, zwischen dem Kasten und des Leonhard
Annthallers Behausung feilhalten. Item die Kramer vor der Kirchen am Bühl, item
die Bäcker im Markt. Salzhäfe,. Eisen, Eisengeschir,. Lederer, Schuhmacher,
Seiler sollen zwischen Hofmanns, Weningers Enzlens, Lorenz Segens und Achazens
Behausung feilgehalten werden. Item der Schweinemarkt vor des Penzkofers und
Michael Grüters Behausung feilgehalten werden. Item die Fremden Bäcker vor dem
Pfarrhof, Item die Tuechschneider und Huterer vorn am Platz am Pranger. Doch
steht es in des Prälaten Willen, solche Ordnung zu ändern und zu verbessern.“

\section{Ruhmannsfelden unter der Herrschaft des Klosters Gotteszell}

Der Ausbau des Klosters Gotteszell mit der Kirche, den Konventshäusern und der
Wirtschaftsgebäuden ging rasch vor sich. 1297 wurde das Kloster Gotteszell zu
einem Aldersbacher Priorat erhoben und beherbergte schon 13 Klosterherren. Im
Jahre 1320 wurde Gotteszell selbstständige Abtei mit 20 Mönchen. Der erste Abt
in Gotteszell hieß Berthold.

Da sich aber die Ausgaben des Klosters seiner raschen Aufwärtsentwicklung
entsprechend erhöhten, musste auch für die Erhöhung seiner Einnahmen gesorgt
werden, was aber große Schwierigkeiten machte. Deswegen fanden fortgesetzt
Unterhandlungen zwischen Aldersbach und Ruhmannsfelden statt, die sich viele
Jahre hindurch zogen. 1445 fanden solche Unterhandlungen statt zwischen
Aldersbach und Gotteszell die „Villa Ruebmannsfelden“ zu vertauschen.

1496 verkaufte das Kloster Aldersbach den Markt Ruedmannsfelden notgedrungen an
die Degenberger unter Vorbehalt des Wiedereinlösens, was von Seiten des Abtes
Simon von Aldersbach Ende des 15. Jahrhunderts auch geschah. Erst nach 50 Jahren
gelang eine Einigung in diesen Verhandlungen. Am Freitag nach Maria Himmelfahrt
im Jahre 1503 bestätigte Herzog Albrecht der Weise einen zwischen den Klöstern
Aldersbach und Gotteszell vollzogenen Tausch, nach welchen das Kloster
Gotteszell den Markt Ruhmannsfelden bekam mit Ausnahme des Pfarrhofes und der
pfarrlichen Rechte, die vom Expositus in Geiersthal ausgeübt wurden, da
Geiersthal noch zu Aldersbach gehörte. Ruhmannsfelden stand nun vollständig
unter der Herrschaft und der Gerichtsbarkeit des Klosters Gotteszell.

Bald trübte sich aber das gute Einvernehmen zwischen dem Bürgern von
Ruhmannsfelden und den Klosterherren von Gotteszell. Die Ruhmannsfeldener Bürger
wollten selbständig sein wie unter der Aldersbacher Herrschaft und 1511 schon
brach der Streit aus wegen der Grenzen, beim Hüten, wegen der Abgaben an das
Kloster, wegen Unterhalt des Prälatenhauses, usw. Von den Straubinger Räten
sollte dieser Streit geschlichtet werden, was zwar auf kurze Zeit gelang. Aber
der gleiche Streit ging bald wieder los, weil sich die Ruhmannsfeldener der
Klosterherrschaft von Gotteszell einfach nicht fügen wollten, weil sie auf ihrer
eigenen Verwaltung und der Führung ihres eigenen Siegels hartnäckig bestanden.
Bei diesen Streitigkeiten ging es auch nicht ohne große Unruhen ab, die sogar in
offenen Aufruhr ausarteten, bei denen das Prälatenhaus in Ruhmannsfelden in
Flammen aufging (1519).

Die Ruhmannsfeldner hatten einen unabhängigen Rat gewählt, brachten die
Landsteuer eigenmächtig ein. Das Gericht in Viechtach musste dagegen
einschreiten. Der Markt mit seinen Bürgern wurde hart bestraft. Die
Ruhmannsfeldner Bürger mussten an das Kloster Gotteszell Entschädigung zahlen,
die Bürger durften auf keinem Landtag mehr erscheinen und die Leibesstrafen
wurden ihnen angedroht. Dessen ungeachtet brachen drei Jahre darauf (1522) schon
wieder Unruhen aus, bei denen der Markt durch Brände einen bedeutenden Schaden
erlitt. Auch die Kirchentrennung und Glaubensspaltung, die 1517 ihren Anfang
nahm, machte sich mit allen ihren Folgen in Ruhmannsfelden bemerkbar, sodass die
Unruhen und Streitigkeiten nicht mehr auszugehen schienen. 1540 brach in
Ruhmannsfelden eine neue Revolution aus. Die Leibesstrafen mussten erhöht
werden. Trotzdem gingen Handel und Wandel bei den Ruhmannsfeldnern ihren
gewohnten Weg weiter und in einer friedlichen Zwischenzeit wurde der Markt sogar
mit neuen, festen Turmwerken versehen und zugleich entstand (1566) ein neues
Marktviertel, das Kalteck, das damals aus 12 Häusern bestand. Mit der Zeit haben
sich die Ruhmannsfeldner Bürger auch etwas mehr Recht und Freiheit erworben.
Freilich fehlte noch viel zur Erreichung ihres ersehnten Zieles. In der 2. Hälfe
des 1. Jahrhunderts brannte des Pfarrgotteshaus St. Laurentius zum größten
Schmerz der Ruhmannsfeldener zum wiederholten Male ab. Für die Wiederherstellung
des abgebrannten Gotteshauses geschah viel von Seiten des Kloster Gotteszell.
Die Glocken, die von einem Münchner Glockengießer stammten, konnten erst 60
Jahre später auf den Turm angebracht werden. Das leider im letzten Weltkrieg zu
Verlust gegangene Osterbrünnl-Glöcklein mit seinem herrlich reinen Ton stammte
aus dieser Zeit. Es trug die Aufschrift: „hans durnknopf Regenspurg 1550.“ Von
dem gleichen Glockengießer waren solch tonreine Glocken noch auf 7
niederbayerischen Pfarrtürmen und dazu die Feuerglocke auf dem Stadtturm in
Straubing. Ob diese herrlichen Glocken nicht auch die Kriegfurie von den Türmen
herab gerissen hat wie das Osterbrünnl-Glöcklein, ist uns nicht bekannt.

Vieles hatte sich in Ruhmannsfelden seit seiner Entstehung (um 950) und seiner
Erhebung zum Markt (um 1400) ereignet. Aber in der nun folgenden Zeit des
30-jährigen Krieges (1618 - 1648) sollte Ruhmannsfelden unendlich mehr zu
erdulden bekommen. Schwedische Kriegsvölker kamen 1633 auf ihren Durchzügen von
Deggendorf über Viechtach nach Cham und später auf ihren Rückmärschen auch nach
Ruhmannsfelden und Umgebung (Schwedentrunk). Die Pfarrkirche wurde nebst den
übrigen Häusern im Markte ausgeplündert und das Pfarrvikarhaus und die
Klostertaverne in Schutt und Asche gelegt.

Genau so erging es dem Kloster Gotteszell, das während dieser Zeit dreimal
gebrandschatzt wurde. Selbst als der Westfälische Friede (1648) schon
abgeschlossen war, rückte eine Abteilung schwedischer Soldaten von Böhmen her in
Bayern ein. Auf ihrem Durchzug in hiesiger Gegend blieb eine Kompanie in der
Gegend von Achslach zurück und ließ sich in Wolfertsried häuslich nieder und
machte von da aus die Streifzüge. Die Bewohner von Achslach schlossen sich aber
zusammen und vernichteten die letzten Schweden in dortiger Gegend in einer
Nacht. Diese sollen im sogenannten Schwedenloch bei Wolfertsried beerdigt sein.
Der Ort Ruhmannsfelden erlitt in der Zeit von 1633 bis 1643 den für jene Zeit
ungeheuren Schaden, nach damaligen Schätzungen, von 50 000 Gulden, wenn man
unter anderem in Betracht zieht, dass unmittelbar vor dieser unglücklichen
Kriegszeit Ruhmannsfelden heimgesucht wurde von der Pest (1613), dem Viehfall
(1620), und anderem mehrv. Von der Bevölkerung waren nur mehr wenige am Leben,
das Geld und die Wertsachen waren gestohlen, das Vieh tot, die Felder und Wiesen
30 Jahre nicht mehr bewirtschaftet. Dass es da in unserem Vaterland und auch
hier sehr traurig ausgesehen haben muss, kann man sich leicht vorstellen. Die
Leute mussten die Arbeit wieder von vorne anfangen und Verdienst und
Einnahmequellen suchen.

Zur dauernden Erinnerung an den 30-jährigen Krieg und zum Gedenken an all die
vielen Gefallen auf den Kriegsschauplätzen und die vielen Toten in der Heimat
währen dieses unseligen Krieges haben die Ruhmannsfeldner im Jahrevi 1687 am
östlichen Marktausgang einen Gedenkstein errichtet (Stegmühle).

Damit das Kloster Gotteszell mehr Einnahmen erhielt, um die erlittenen
Kriegsschäden wieder gut machen zu können, wurde 4 Jahre nach Beendigung des
30-jährigen Krieges unter Abt Gerhard die pfarramtliche Seelsorge in
Ruhmannsfelden, die bisher von der Pfarrei Geiersthal ausgeübt wurde, auf das
Kloster Gotteszell übertragen (1652). Von dieser Zeit an bis zur Aufhebung des
Klosters Gotteszell blieb der Markt Ruhmannsfelden dem Kloster Gotteszell
unterstellt. Allmählich traten auch an Stelle von Unglauben, Aberglauben,
Zauberei und Hexerei wieder religiöse Frömmigkeit und echter Gottesglaube.
Besonders pflegten aber damals die Leute in ihrer seelischen Not und in ihrem
wirtschaftlichen Elend die Marien-Verehrung. An vielen Orten wurden in der
Nachkriegszeit der 30-jährigen Krieges Marien-Kapellen erbaut oder schon
bestehende Kapellen in Marien-Kapellen umbenannt. So entstanden in damaliger
Zeit auch die Osterbrünnl-Kapellen. Ob das hiesige Osterbrünnlkirchlein in
dieser Zeit entstanden ist oder schon früher, darüber berichten uns Urkunden
nichts. Es könnte sein, dass das Osterbrünnlkirchlein schon früher erbaut wurde
als Notkirchlein in einer Zeit, da die Pfarrkirche in Schutt und Asche lag und
das war ja öfters der Fall. Für alle Fälle ist es aber nicht die herabgesetzte
Schlosskapelle, weil es ein Schloss auf der dortigen Anhöhe niemals gab.

Nachdem die Errichtung der Osterbrünnl-Kapelle und gleichzeitig die vielen
wundertätigen Heilungen an diesem Gnadenort weit und breit bekannt wurden,
setzte ein so großer Besuch dieser Stätte ein, sogar bis aus Böhmen und
Österreich kamen Wallfahrer hierher, dass das Kloster Gotteszell sich durch den
geringer Besuch seiner kirchlichen Veranstaltungen und die Geschäftsleute der
Ortschaft Gotteszell sich durch den dadurch entstanden Einnahmeausfall
benachteiligt fühlten. Urkundlich erscheint das Osterbrünnl im Jahre 1724.

Dort kann man folgendes lesen: „Da ist die hölzerne Kapelle beim Osterbrünnl mit
den Votivtafeln verbrannt. Das darin sich befindliche Mutter Gottesbild von
Altenötting wurde in das Kloster Gotteszell abgeliefert - auch die bei der
Kapelle gestandene, steinerne Martersäule mit dem Bildnis der allerheiligsten
Dreifaltigkeit und statt derselben wurde ein Schaup Stroh aufgestellt.“ Der Abt
Wilhelm II. von Gotteszell berichtet, dass das Bild der Muttergottes von
„Altenötting“ an einem Baum ohne seine Erlaubnis aufgestellt wurde und dass dazu
eine Kapelle aus Holz erbaut wurde und ein Opferstock aufgestellt wurde. Er habe
das Bild abnehmen und die Kapelle verbrennen lassen. Die Bürger von
Ruhmannsfelden hetzten dagegen eben einquartierte Reiter auf, welche wieder ein
Bild aufstellten. Darauf ließ der Abt den Baum umhauen. Die Bürger hängten aber
wieder Tafeln auf, welche der Abt immer abnehmen und die bereffenden Bäume
umhauen ließ und ihnen unter Strafe verbat, diese Osterandacht weiter zu
treiben. Die Ruhmannsfeldener erklärten, sie ließen sich das Beten nicht nehmen.
Der Abt erklärte ihre Andacht als Baum- und Stockverehrung. Das Bischöfliche
Konsistorium verfügte, dass die Kapelle im Osterbrünnl nicht mehr errichtet
werden dürfe. Erst nach dem Brand des Marktes im Jahre 1820 ging man
notgedrungen an die Erbauung des Osterbrünnl-Kirchleins.

Heute ist das Osterbrünnl-Kirchlein, ganz neu renoviert und mit einer neuen
Glocke versehen, die von Hr. J. Ederer appr. Bader von Ruhmannsfelden gestiftet
wurde, das Ziel vieler, vieler Wallfahrer aus nah und fern, die alle mit ihren
Bitten und Dankgebeten zu dem in jetziger Nachkriegszeit gestohlenen und wieder
zurückgebrachten Gnadenbilde im stillen Heiligtume der Osterbrünnl-Kapelle
wallfahren.

Kaum waren die Schäden des 30-jährigen Krieges einigermaßen behoben, erfolgte
der mit argen Quartierlasten verbundene Rückmarsch von österreichischen Truppen
aus Richtung Cham (1704). Am Weihnachtstag 1705 war die Sendlinger
Bauernschlacht und am 8. Januar 1706 wurden die niederbayerischen Bauern von den
Österreichern bei Aidenbach unweit Vilshofen besiegt. 2000 bis 3000
niederbayerische Bauern verloren dabei ihr Leben, darunter viele Bauern aus dem
Bezirke Viechtach und auch von Ruhmannsfelden. Zum ehrenden Gedenken an die bei
Aidenbach 1706 gefallenen Ruhmannsfeldener wurde ein Denkmal errichtet, das die
Jahreszahl 1710 trägt (Holler Garage). Möchte dieses Denkmal doch mehr gewürdigt
werden!

1729 wurde ein Wilderer vom Klosterrichter abgeurteilt, weil er auf dem
Vogelsang ein Wildschwein erlegt hatte. 1745 kam General Bärenklau mit den
Panduren aus Richtung Grafling hierher. 1762 herrschte hier wiederum die
Viehseuche und im gleichen Jahre vernichtete ein starker Reif alles auf dem
Felde und Wiese. Bei der Hungersnot im Jahre 1770/71 wurde in den königlichen
Jagden im Gebirge das Wild abgeschossen und das Wildpret im München kostenlos an
die hungernde Bevölkerung abgegeben. Im Auslande wurde Getreide aufgekauft und
zu verbilligtem Preise verteilt. So wechselten im Laufe der verflossenen
Jahrhunderte gute und schlimme Zeiten ab. Die Bevölkerung von Ruhmannsfelden
verlor aber trotz der härtesten Bedrängnisse den Glauben und das Vertrauen auf
sein Können und seinen Fleiß nicht.

Hat es in damaliger Zeit auch schon Schulen gegeben? Ja! Klosterschulen,
Lateinschulen, in denen talentierte Knaben zu Geistlichen herangebildet wurden.
Aber es gab auch damals noch nicht die Schulhäuser, wie jetzt überall. Lehrer,
damals Schulmeister genannt, gab es wohl. Wanderlehrer und sesshafte. Urkunden
aus dem 16. Jahrhundert weisen nach, dass es solche damals auch schon in
Ruhmannsfelden gegeben hat. In einer Urkunde von 1550 kommt als Zeuge vor ein
Andrä Weißpech, Schulmeister in Ruhmannsfelden und auch im 17. und 18.
Jahrhundert berichten Urkunden von Schulmeistern in Ruhmannsfelden, die zugleich
auch Mesner waren. Diese Schulmeister hatten meist selbst ein Besitztum und
unterrichteten die wenigen Kinder, die freiwillig lesen und schreiben und
rechnen lernen wollten, in ihrer Behausung. So lesen wir in einer Urkunde:
„Unterm 5.11.1658 verkaufte Georg Pitter, Bürger und gewester Schulmeister zu
Ruhmannsfelden und dessen Frau Eva ihre Leibgedingsgerechtigkeit auf einem Lehen
zu Ruhmannsfelden mit Bräugerechtigkeit dem Hans Enbeck, Bürger und Metzger
daselbst und seiner Frau Margareta um 365 fl.“ Am 5.7.1702 übergibt Rosina
Hinderholzer, verwitwete Schulmeisterin zu Ruhmannsfelden ihre Markbehausung am
Kalteck an ihren Tochtermann Martin Staudenberger, Bürger und Schneider in
Ruhmannsfelden. 1784 wirkte hier der Schulmeister Herrmann und der geprüfte
Eremit Franz Pitsch von Gotteszell. Dieser hatte auf dem Kalvarienberg bei
Gotteszell ein kleines Häuschen. In diesem unterrichtete er die Gotteszeller und
Ruhmannsfeldner Jugend. Da aber im Laufe der Zeit den Ruhmannsfeldner Kindern
der Weg nach Gotteszell, zumal bei schlechtem Wetter, zu beschwerlich war, ging
der Eremit Pitsch nach Ruhmannsfelden und erteilte hier mit Herrmann den
Unterricht. Nach Wegzug des Eremiten Pitsch von Gotteszell hat Amadäus in
Gotteszell (1803) hat man die Wohnung des Pfarrvikars und die des Schulmeisters
und Mesners in Ruhmannsfelden verkauft und dafür ein einziges, zwar größeres,
aber sehr baufälliges Haus angekauft und darin den Pfarrer und Kooperator, samt
Lehrer und Gehilfen und das Schullokal für 200 Schulkinder und den Mesner
untergebracht. Das war aber nur ein Notbehelf und die unhaltbaren Zustände, die
sich im Laufe der Zeit in diesem Gemeinschaftshaus für Pfarrer, Lehrer und
Schule ergaben, mussten zu einer Änderung dieser Verhältnisse führen.

\chapter{Ruhmannsfelden wird selbständig}

Kaum hatte sich das Kloster Gotteszell nach all den Wirrnissen, Drangsalen und
Kriegsschäden wieder einigermaßen erholt, kam plötzlich die Aufhebung des
Klosters im Jahre 1803. Die Klostergeistlichen mussten sich um Unterkunft oder
Wiederverwendung als Geistliche in den umliegenden Pfarreien umsehen. Die
Klostergebäulichkeiten, die Äcker, Wiesen und Wälder wurden öffentlich
versteigert und die Kirchengewänder, Messgeräte, Bücher und Urkunden verkauft,
wobei auch die für den Markt Ruhmannsfelden wertvollen Urkunden schubkarrenweise
um billiges Geld veräußert wurden. Im Jahre 1804 wurde dem Markt Ruhmannsfelden
das Selbstverwaltungsrecht übertragen. Der erste Bürgermeister war Josef Liebl,
bürgerlichervii Bierbrauer.

1805 wurden auch ein Pfarrer, ein Kaplan und ein Schullehrer angestellt. Das
Schulzimmer und die Lehrerwohnung waren aber in einem so schlechten Zustande,
dass die Schulkinder wieder lieber nach Gotteszell zur Schule gingen. Am 15.
August 1806 wollte die in Ruhmannsfelden liegende französische Besatzung ein
großes Fest abhalten. Auf dem Marktplatze war ein großes Transparent
aufgestellt. Als abends die Lichter angezündet waren, warf jemand aus der Mitte
der vielen Zuschauer heraus einen Stein gegen das Transparent, sodass es in
Flammen aufging. Die betreffende Person blieb bis heute unbekannt.

Im Jahre 1812 gehörten zur Pfarrei Ruhmannsfelden 9 Dörfer, 6 Weiler, 13 Einöden
und 3 Neusiedlungen. Die Bevölkerung der Pfarrei setzte sich zusammen aus 79
Inleuten und Taglöhnern, 29 Ganzbauern, 53 Halbbauern, 20 Viertelbauern, 2
Achtelbauern und 3 Sechzehntelbauern. Unter den Namen der einzelnen Ortschaften
steht auch ein „Hutweging“. Nicht weniger als 94 Handwerker und Gewerbetreibende
waren damals hier ansässig , darunter 6 Brauer, 5 Bäcker, 5 Schmied, 5
Schuhmacher, 4 Schneider, 2 Tischler, 3 Küfer, 4 Zimmererleute, 4 Maurer, 9
Leinweber, 3 Wagner, 3 Krämer, 2 Metzger, 2 Müller, ein Wirt, Sattler,
Weißgerber, Rotgerber, Hutmacher, Strumpfstricker, Drechsler, Färber, Kürschner,
Glaser, Zinngießer, Gürtler, Schlosser, Bartscherer. Das Gewerbe der Leinweber
und Zeugmacher war in hiesiger Gegend besonders vertreten. Im Markt
Ruhmannsfelden stand ein eigenes Zunfthaus.

1813 kam die Wallfahrt zur Osterbrünnl-Gnadenstätte in verstärktem Maße wieder
auf. Da ließen Josef Baumann und Anton Schlögl von Ruhmannsfelden eine neue
hölzerne Kapelle errichten. Außerdem ließen sie ein großes Marienbild
anfertigen, das in feierlicher Prozession von der Pfarrkirche in die neue
Kapelle gebracht wurde.

In dem Opferstock dortselbst wurde viel Geld eingelegt. Als Landrichter
Beyerlein von Viechtach dies hörte, ließ er die Kapelle im Oktober 1814
niederreißen. Das Muttergottesbild wurde dann in der Pfarrkirche unter der
Kanzel aufgestellt.

Im Jahre 1817 herrschte große Not. Das Getreide, das nur sehr wenig war, hatte
furchtbar hohe Preise. In dieser Notlage mussten sich die einen Lins und die
anderen bloß Schoßwürze oder Moos, das man zum Einstreuen nicht gut brauchen
konnte, mahlen lassen. Das gab sicher kein gutes Brot. Aber als das beste Brot
in dieser Zeit galt noch das Kleinenbrot.

Das Jahr 1820 war eines der unglücklichen Jahre für Ruhmannsfelden. Wie schon
erwähnt, wurde früher das Holz nicht verkauft, sondern im Walde verbrannt und
dafür die daraus gewonnene Asche verkauft an Seidensieder, usw. Der Aufkäufer
hatte schon einen ziemlich hohen Haufen Asche im Hofe beim Berger Bräu
(Amberger) aufgestapelt und mit Tannen- und Fichtenzweigen zugedeckt. In der
Nacht zum 1. Juli 1820 ging ein heftiger Wind, entfachte die glühende Asche zum
lodernden Feuer und schon brannte die Pfarrkirche. 11 Häuser samt dem Brothäusl
wurden ein Raub der Flammen, ebenso auch die zwei kleinen hölzernen
Feuerspritzen und die „meßingerne wurde verdorben.“ Leider verbrannte auch das
schöne Marienbild, das 1814 vom Osterbrünnl hierher verbracht wurde. Da an einen
raschen Wiederaufbau der Pfarrkirche nicht gedacht werden konnte, ging man
daran, die Osterbrünnl-Kapelle als Filialkirche auszubauen, was auch mit Hilfe
aller Pfarrangehörigen bald gelang, sodass die neu erbaute, aber etwas tiefer
herabgesetzte Osterbrünnl-Kapelle schon im Jahre 1821 eingeweiht werden konnte.

Früher stand diese Kapelle auf der Anhöhe beim hohen Kreuz, zu dem die 14
Stationen des Kreuzweges hinführen. Auch ein Muttergottesbild, wie das frühere,
wurde wieder neu angeschafft. Die 1820 abgebrannte Pfarrkirche konnte erst 1828
fertig gestellt und seiner Bestimmung übergeben werden. Das Hochaltarbild ist
ein sehr wertvolles Gemälde von Josef von Lens. Die beiden Seitenaltarbilder
stammen von Münchner Künstlern und wurden in der Hauptsache ausgeführt von
Martin Dorner, der ein armer, bedürftiger, aber hervorragend tüchtiger Schüler
der beiden Hofmaler Schraudolpf und Hauber von München war. Sehr wertvoll sind
auch die Madonnenstatur über dem Taufstein und der Kreuzweg aus Messinggewebe
gemalt von Leopold Baumann, einem gebürtigen Ruhmannsfeldener. 1831 bekam die
Pfarrkirche auch wieder eine Orgel. Am 21.4.1821 starb H. Hr. Pfarrer Blaim von
hier an Lungenentzündung im 49. Lebensjahr.

An die Stelle der 1819 eingeführten magistratischen Verfassung trat am
14.10.1825 eine „Marktgemeinde“ unter dem Landgerichte Viechtach. Da die
damaligen kleinen Zimmer, die für Unterrichtszwecke benützt wurden (im Haus des
Bielmeier Eierhändlers), den Erfordernissen nicht mehr entsprachen, wurde 1834
ein neuer Schulhaus aus Bruch- und Ziegelsteinen an der Straße nach Gotteszell
erbaut 1835 eröffnet, Dieses Schulhaus ist das erste und damit das alte
Schulhaus in Ruhmannsfelden. In diesem Schulhaus waren zwei Schulzimmer, die
Wohnung der Lehrers und Hilfslehrers (Schulgehilfen).

1841/44 wurde der jetzige Pfarrhof gebaut. 1842 bekam Ruhmannsfelden eine
Feuerspritze, die sich dann darauf bei dem Brande des Moosmüller-Zeugweberhauses
bestens bewährte. Eine Feuerwehr gab es damals noch nicht. Diese wurde erst im
August 1867 gegründet. Josef Fromholzer bürgerlicher Färberssohn von hier erlitt
bei einem Theaterbrande in Karlsruhe (Baden) derartige Brandwunden, dass er
daran sein junges Leben einbüßen musste. Er wurde in Karlsruhe feierlichst
beerdigt. Im Februar 1849 trug sich beim Steinbauer in Haberleuthen ein ganz
seltener Fall zu. Da kam nämlich ein angeschossener Hirsch auf den Stadelfirst
hinauf, stürzte auf der anderen Seite des Daches herunter auf einen Apfelbaum
und dort zu Boden, wo er dann von dem Sohn der Steinbauern mit dem Haustürriegel
erschlagen wurde. In einer Nacht des Jahres 1851 hat man den 37-jahrigen
Gemeindediener Josef Moosmüller ermordet. 3 Jahre später hat man den
verheirateten Bauern Achatz von Perlesried tot (infolge eines Sturzes von einem
Baume) aufgefunden.

In der Mitte des Marktplatzes stand früher eine überlebensgroße barocke Statue
des Johannes von Nepomuk inmitten von 4 Allebäumen. Diese Statue wurde im Jahre
1885 von diesem Platz entfernt und zunächst an das Haus des Hr. Rauch gestellt.
Hier blieb sie bis Hr. Rauch die Läden in sein Haus einbaute. Dann kam die
Statue an das Haus des Hr. Fromholzer, wo sie heute noch steht. Die 4 Allebäume
kaufte Bierbrauer Sagstetter und in die Mitte des Marktplatzes kam ein schöner
Marktkorbbrunnen aus Granit mit der Statue der Patrona Bavaria 1859. Alle
Marktbürger steuerten bei zur Deckung der Kosten dieses Marktbrunnens und Hr.
Alois Fromholzer stellte seinen Bürgermeisterjahresgehalt gleich für 2 Jahre zu
diesem Zwecke zur Verfügung.

Im Jahre 1862 wurden am baufälligen Glockenturm der hiesigen Pfarrkirche größere
Reparaturen vorgenommen und der Turm mit Weißblech eingedeckt. Im gleichen Jahre
wurde aus dem Tabernakel der Pfarrkirche eine wertvolle Monstranz gestohlen.

1867 wurde aus Krankenhaus hier (jetzige Kinderbewahranstalt) gebaut, was sich
gleich als sehr nützlich erwiesen hat, da im gleichen Jahr der Typhus herrschte
und von 9 Todesfällen dabei berichtet wird.

Auch das 19. Jahrhundert ging nicht ohne Krieg ab. So gab es 1866 kriegerische
Auseinandersetzung zwischen Preußen und Bayern und 1870/71 zwischen Frankreich
und Deutschland.

Am 11. März 1871 war in hiesiger Pfarrkirche ein Trauergottesdienst für die
Gefallenen dieses Krieges und anschließend fand eine Friedensfeier statt, von
der die Teilnehmer noch nach Jahrzehnten erzählten.

Vom 26. auf 27. Oktober 1870 war ein so gewaltiger Sturm, dass ganze Wälder
umgelegt wurden und kein Baum mehr davon stand. Auch der Signalturm auf dem
Hirschenstein wurde dabei umgerissen. Am 21. Mai 1871 fand in Deggendorf ein
großer Katholikentag statt, bei welchem bei der Prozession 20 000 Katholiken,
jung und alt, teilnahmen. 1871 wurde auch mit der Vermessung der Eisenbahnlinie
Plattling-Deggendorf begonnen und 1872 der Bau dieser Bahn über Deggendorf nach
Gotteszell und Zwiesel beschlossen und genehmigt. Bei diesem Bau waren neben den
deutschen Arbeitern auch viele Italiener beschäftigt. Dass es bei den vielen
Felsdurchbrüchen auf dieses Bahnstrecke nicht ohne Unfälle abgehen konnte, ist
leicht erklärlich. Bei einem Sprengschuss im Tunnel bei Ulrichsberg wurden 3
Arbeiter getötet. Bei einer Dynamitexplosion in der Nähe von Zwiesel wurden
gleich fünf Bahnarbeiter getötet. Im Ruhmannsfeldner Friedhof liegen viele
Italiener begraben. Am 1. Februar 1877 war die erste Probefahrt von Plattling
über die neue Eisenbahnbrücke Deggendorf nach Eisenstein. Am 15. November 1877
wurde diese Bahnstrecke dem öffentlichen Verkehr übergeben.

1874 ging während des 2. Evangeliums bei der Fronleichnamsprozession plötzlich
ein Haus in Flammen auf. Am Dreifaltigkeitssonntag des Jahres 1875 wurde in
Perlesried ein Raubmordversuch durch die Hilfe der Nachbarsleute von Sintweging
verhindert. 1878 erbaute die Sepulturgemeinde Ruhmannsfelden den neuen Friedhof
bei der Pfarrkirche. Leider hat man dabei die alten Grabsteine vollständig
beseitigt, die uns vielen Aufschluss über die alte Geschichte von Ruhmannsfelden
hätten geben können. Am 17.3.1879 ließ sich der ledige Eisenrichter von hier in
die Kirche abends einsperren und beging dann auf den Stufen des Hochaltars einen
Selbstmordversuch. Da er mit dem Messglöcklein läutete, wurde er vom Mesner
gehört und aus seiner peinlichen Lage befreit. Die Kirche wurde darauf wieder
eigens konsekriert. Das 700-jährige Wittelsbacher Jubiläum wurde auch hier
feierlich begangen im August 1880.

Am 20. Oktober 1883 war die Eröffnung der Telegraphenstation Ruhmannsfelden.
1884 wurde das Mädchenschulhaus mit einem Kostenaufwand von 18 000 Mark gebaut,
da sich das alte Schulhaus als zu klein erwies für die große Schülerzahl. In der
Zeit von 1885 bis 1891 wurde die Pflasterung des Marktes vorgenommen. Die
Pflastersteine wurden vom Zeitlhof hierher gebracht. In der Nacht vom 30. April
auf 1. Mai 1889 war wieder ein großer Brand. Dieser Brand wäre beinahe unserer
Pfarrkirche zum Verhängnis geworden. Es brannten 7 Anwesen ab im oberen Markt
(Dietrich, Sixl, Weinzierl, Meindl, Hirtreiter, Reisinger und Baumann), alle in
nächster Nähe der Pfarrkirche. Der Bauplatz des abgebrannten Dietrich wurde um
3000 Markt angekauft und damit der Friedhof erweitert. Seit dieser Zeit
existiert auch die Friedhofsmauer.

Am 20. November 1890 fuhr der erste Zug von Viechtach nach Ruhmannsfelden. Dabei
gab es einen Dammrutsch bei Mariental, sodass wieder 8 Tage lang die Postkutsche
die Personen von Viechtach nach Ruhmannsfelden und Bahnhof Gotteszell fahren
musste.

Beim Maimarkt 1891 in Deggendorf gab es zum ersten Mal die Vorführung einer
„englischen Sprechmaschine“ – Grammophon. In der Ankündigung hieß es: „Die
Maschine betet, singt und lacht.“ Die ganz kleinen Hörröhrchen an langen
Gummischläuchen musste man sich in beide Ohren stecken, um etwas hören zu
können. 1892 bekam die Fr. Feuerwehr eine neue Feuerspritze. Im Oktober 1892 hat
der Gemeinderat Ruhmannsfelden beschlossen, den Bau einer Wasserleitung in
Angriff zu nehmen. Es blieb aber nur beim Beschlusse. Erst als sich die
Notwendigkeit einer Wasserleitung für den Markt erwies, ging man an den Bau
derselben.

Der 25. August 1894 war für Ruhmannsfelden wieder einmal ein ganz großer
Unglückstag. Es war ein sehr heißer Sommertag. Die Leute des Marktes waren zum
größten Teil auf den Feldern. Ob nun Kinder, die mit Zündhölzern ihr Spiel
trieben, schuld waren oder ob eine ältere Frauensperson absichtlich angezündet
hat, darüber ließ sich bis heute nichts Genaues feststellen. Plötzlich stand der
Stadel des Wagnermeisters Metzger in hellen Flammen. Das Feuer griff über auf
die Wilhelm-Brauerei und dann auf die andere Marktseite. Bis die Leute von den
Feldern heim kamen, war ihre ganze Habe abgebrannt. Obwohl die Feuerwehren bis
von Straubing, Plattling, Deggendorf, Zwiesel, Regen und alle Feuerwehren des
Viechtacher Bezirkes kamen, konnten sie nicht mehr verhüten, dass 51 Firste,
darunter 21 Wohnhäuser, ein Raub der Flammen wurden. Für die Abgebrannten von
Ruhmannsfelden wurde eine öffentliche Sammlung angeordnet. Die abgebrannten
Häuser wurden von italienischen Baumeistern und Bauarbeitern aufgebaut. Das
ersieht man an dem italienischen Baustil der ganzen Häuserfront von Stadler bis
Hirtreiter, nämlich an der horizontalen Fassade dieser Häuser, die zum echten
Waldlerstil, die Giebelfassade, der gegenüberliegenden Häuserseite gar nicht
passt.

Am 9. Januar 1895 traf ein „Aufsehen erregendes Vehikel“, nämlich ein
Motorzweirad zum ersten Mal in Deggendorf ein. Eine riesige Menschenmenge stand
zu beiden Seiten der Straße von der Donaubrücke zum Stadtplatz, um dieses
seltsame Wunderding zu sehen.

Der Brand 1894 hätte die Dringlichkeit des Baues einer Wasserleitung im hiesigen
Markte wohl zur Genüge erwiesen. Trotzdem zogen sich diese Verhandlungen immer
noch in die Länge. Hauptsächlich wegen des Erwerbes der Grundstücke in der
Muschenrieder Ortsflur in welcher das Quellengebiet für die Ruhmannsfeldner
Wasserleitung liegt. 1896 konnte dann endlich mit dem Bau der Wasserleitung
begonnen werden. Eine Münchener und eine Ludwigshafener Firma führten die
Bauarbeiten aus. Im gleichen Jahre (1896) bekam die freiwillige Feuerwehr
Ruhmannsfelden eine fahrbare Schubleiter.

Jahrzehntelang wurde schon immer prophezeit, dass eine Zeit wird kommen, in
welcher die Wägen nicht mehr geschoben oder gezogen werden brauchen, sondern von
selbst laufen und dass dann das Ende der Welt nicht mehr ferne sein wird. Ein
solcher Wagen, Motorwagen oder Automobil, kam zum ersten Mal über die
Donaubrücke nach Deggendorf am 2. Februar 1899. Die Leute waren voll des
Staunens und wichen dem geheimnisvollen Fahrzeug weit aus.

Nachdem das Schulhaus (1835) und das Mädchenschulhaus (1884) den Anforderungen
nicht mehr entsprachen, das die Schülerzahlen von Jahr zu Jahr größer wurden und
dadurch auch mehr Lehrkräfte angestellt werden mussten (1906 waren es 4),
stellte die Gemeinde Zachenberg Antrag auf Erbauung eines Schulhauses in
Auerbach.

Aufgrund einer vom damaligen Bürgermeister ausgearbeiteten Denkschrift über die
Schulverhältnisse in Ruhmannsfelden wurde der Bau eines großen neuen Schulhauses
im Markt Ruhmannsfelden mit einem Kostenvoranschlag von 83 000 Mark genehmigt,
der 1907 begonnen und 1908 vollendet wurde. Die Baukosten beliefen sich auf fast
100 000 Mark. Im Jahre 1908 wirkten an der Schule Ruhmannsfelden schon 7
Lehrkräfte. Die Konzession zum Betriebe der Apotheke in Ruhmannsfelden wurde am
12. Dezember 1910 dem appr. Apotheker Hr. Vitus Voit von München vom
Staatsministerium der Innern verliehen. Hr. Apotheker Voit von München kaufte
das alte Bauernhaus des Michl Achatz an der Bahnhofstraße samt 112 dzm. Grund.
Im April 1911 wurde mit den Abbrucharbeiten und dem Neubau der Apotheke
begonnen. Die Baupläne fertigte der Münchner Architekt Deininger. Den
Kostenvoranschlag zu diesem Bau machte der hiesige Baumeister Helmbrecht. Dieser
starb aber plötzlich an einer heftigen Lungenentzündung, sodass dessen Bruder
Ludwig Helmbrecht von Metten den Bau ausführen musste. Am 7. September 1911 war
der Bau vollendet und am 1. November 1911 konnte nach umfassenden Vorarbeiten
die Marien-Apotheke in Ruhmannsfelden eröffnet werden.

Im Jahre 1910 wurde in hiesiger Pfarrkirche von Orgelbaumeister Edenhofer in
Deggendorf eine neue pneumatische Orgel mit 22 klingenden Registern aufgestellt.
1917 mussten die schönen Prospektpfeifen an den Staat abgeliefert werden,
konnten aber 1925 wieder ersetzt werden. Durch die lange Trockenheit im Sommer
1947 hat auch diese Orgel großen Schaden erlitten. Um diese Schäden zu beheben
musste die Orgel von Orgelbaumeister Kratochwill von Plattling vollständig
auseinander genommen werden. Heute repräsentiert diese neu renovierte Orgel
einen großen Wert.

1911 feierte Hr. Kaufmann Probst von Ruhmannsfelden sein 90-jähriges
Geburtstagsfest. 1914 stellte Hr. Max Zadler von hier den Antrag auf
Konzessionverleihung zur Belieferung von elektrischem Strom für den Markt
Ruhmannsfelden. 1915 kam auf den Glockenturm der hiesigen Pfarrkirche eine neue
Turmuhr, geliefert von dem Turmuhrenfabrikanten E. Strobl von Regensburg und
kostete 1 720 Mark.

1918 endet nach 4-jähriger Kriegsdauer der erste Weltkrieg. 150 Gefallene hatte
die Pfarrei Ruhmannsfelden zu betrauern. Zunächst wurde für 17 in diesem
Weltkrieg gefallene Steinhauer von den Steinarbeitern der Firma Haberstumpf
rechts vom Schulhausaufgang ein Gedenkstein errichtet, der aber auf Anordnung
der Bauamtes Straubing wieder entfernt wurde. Am 23.1.1919 starb H. Hr. Pfarrer
und Kammerer Mühlbauer. Nach diesem wurde hier Pfarrer der damalige
Krankenhauskurat von Deggendorf H. Hr. Karl Fahrmeier.

1919 begannen auch wieder Unterhandlungen hauptsächlich mit auswärtigen Firmen
wegen Errichtung eines Elektrizitätswerkes. Nachdem sich Ruhmannsfelden für den
Anschluss an das Überlandwerk Niederbayern entschlossen hatte und die
Installationsarbeiten erledigt waren, erstrahlte am 1. Mai 1920 zum ersten Mal
die Maienkönigin in der Laurentius-Pfarrkirche im elektrischem Lichte.

1921 feierte H. Hr. Kammerer Fahrmeier sein 25-jähriges Priesterjubiläum im
Wilhelmsaale. 1921/22 wurde die Marktsiedlung auf der ehemaligen Voglwiese,
bestehend aus 8 Häusern, gebaut. 1925/26 fanden Theateraufführungen statt zu
Gunsten der Schule Ruhmannsfelden. Von dem Erlös wurde ein Lichtbildapparat
„Janus“ gekauft. Die zum größten Teil sehr schönen farbigen Bilder dienten
unterrichtlichen Zwecken und bereiteten den Kindern große Freude.

Am 25.1.1926 wurde das von Steinlieferant Klein in Frankenried angefertigte
Kriegerdenkmal, das ein Gewicht von 150 Zentnernviii hat, um den Preis von 1 000
Mark gekauft. Nachdem die viel umstrittene Platzfrage zur Aufstellung dieses
Denkmals dadurch gelöst wurde, dass Oberlehrer Högn die Hälfte des Schulgartens
beim alten Schulhaus zu diesem Zwecke abtrat, wurde das Kriegerdenkmal 1929 auf
diesem Platze aufgestellt und am 10.11.1929 eingeweiht ohne jegliche weltliche
Feier.

Am 5.5.1926 kaufte der damalige 2. Bürgermeister Glasl im Auftrage der
Gemeindeverwaltung Ruhmannsfelden das Anwesen der Privatiers Klimmer von hier um
den Preis von 10 000 Mark. Seit dieser Zeit befindet sich die Marktkanzlei und
die Wohnung des Kanzleisekretärs in diesem Haus.

Am 29. Januar 1928 wurde die neu gebaute Turnhalle feierlich eröffnet. Der Name
des Turmvereins Ruhmannsfelden, gegründet 1894, hatte früher einen guten Klang.
Gebrüder Bielmeier waren als beste Gewichtsheber und Schwarz als bester
bayerischer Ringer im ganzen Bayerland bekannt. Die Turner von Ruhmannsfelden
brachten von den Wettkämpfen stets ehrenvolle Siegerkränze mit nach Hause. Der
Turmplatz befand sich früher zwischen Feuerwehrhaus und Zitzelsberger-Stadel.
1904 wurde die neue Turnvereinsfahne geweiht.

1927 wurde eine Alarmsirene bei der A.E.G. in Regensburg um den Preis von 420
Mark gekauft und dieselbe auf dem Dach des Marktrathauses von Hr.
Schlossermeister Sturm aufmontiert. Jeden Samstagmittag wird dieselbe auf ihre
Zuverlässigkeit ausprobiert.

1929/30 wurde die Huberweidsiedlung mit 10 Häusern gebaut. Am 23. September 1932
beschloss der Marktgemeinderat die Anschaffung einer Motorspritze. Am 24. Juni
1933 wurde dieselbe von der Firma Paul Ludwig in Bayreuth geliefert. Am 25. Juli
desselben Jahres brach in Zuckenried nachts 12 Uhr ein Großfeuer aus. Die
Motorspritze von Ruhmannsfelden lieferte bei diesem Brande von ½ Uhr nachts bis
8 Uhr morgens aus einer Entfernung von 300 Meter und bei einer Steigung von 15
Meter unausgesetzt Wasser und bestand dabei ihre Feuerprobe glänzend.

Am 1. Oktober hat H. Hr. Kammerer Fahrmeier freiwillig resigniert und ist nach
Deggendorf, seinem früheren Wirkungsort, verzogen. Am 13. November das gleichen
Jahres hat der frühere langjährige 1. Stadtpfarrkooperator von Deggendorf H. Hr.
Pfarrer Bauer die Pfarrei Ruhmannsfelden übernommen. 1938 feierte dieser sein
25-jähriges Priesterjubiläum und bei der Feier seines 60. Geburtstages am 2.
April 1947 erhielt er von der Marktgemeindeverwaltung ein Ölgemälde, den hl.
Florian darstellend als Anerkennung für sein mutiges und tatkräftiges Eingreifen
bei dem Brande im Pfarrhof, der verursacht wurde bei der Beschießung des Marktes
am 23. April 1945.

1937 veranstaltete der Bezirkslehrerverein Regen-Viechtach eine
Wanderausstellung. Die mit viel Fleiß und Geschick von Lehrern und Schülern
gefertigten Ausstellungssachen, die in sämtlichen Räumen des neuen Schulhauses
zur Schau ausgestellt waren, erregten die Bewunderung aller Besucher dieser
schönen und lehrreichen Ausstellung. 1939/40 entstand die Grabsiedlung mit 9
Häusern.

Bei der Beschießung des Marktes am 23. April 1945, bei der hauptsächlich die
Kirche, der Friedhof und die Häuser des oberen Marktes in Mitleidenschaft
gezogen wurden, brannte das Feuerhaus ab, wobei die wertvollen
Feuerwehrutensilien (ein ganz neuer, modern ausgerüsteter, motorisierter
Requisitenwagen, die fahrbare Schubleiter, usw.) ein Raub der Flammen werden.
Hr. Hausbesitzer Veit hat bei dieser Beschießung leider sein Leben einbüßen
müssen und eine Flüchtlingsfrau, die im Hause des Hr. Veit Zuflucht suchte,
erlitt so schwere Verletzungen, dass ihr ein Bein amputiert werden musste. Die
Brände bei Bierbrauer Stadler und im Pfarrhof konnten rechtzeitig gelöscht
werden.

Am 25. April 1945 wurde der damalige Bürgermeister Hr. Sturm von SS-Männern
verhaftet und mit unbekanntem Ziel fortgeführt. Angeblich haben ihn einige
Männer, die bei den Sprengungsarbeiten am Hochbühl beschäftigt waren, bei der SS
denunziert. Hr. Sturm wurde nach Plattling gebracht und wäre sicherlich der
Vollstreckung des härtesten Urteils, das über ihn bereits gefällt war, kaum
entgangen, wenn ihm nicht in letzter Minute noch die Flucht gelungen wäre. So
ist er wieder glücklich und wohlbehalten nach Ruhmannsfelden zurückgekommen.

Das gleiche Schicksal ereilte auch einen Sohn des Marktes Ruhmannsfelden,
nämlich Hr. Studienrat Leonhard Donauer, Straubing, der Ende 1944 wegen
antinationalsozialistischer und antimilitaristischer Äußerungen vom
Zentralgericht des Heeres in Berlin zum Tode verurteilt wurde. Er sollte am 24.
April 1945 mit noch einigen Leidensgenossen außerhalb Berlin erschossen werden,
aus dem Transport zum Exekutionsplatz sperrten aber die Russen den Weg zu diesem
Platze ab. Infolgedessen wurde sie nach Spandau zurückgebracht. Den raschen
Einmarsch der Russen in Spandau wurde die Vollstreckung des Urteils unmöglich
gemacht. Hr. Donauer kam zwar in russische Gefangenschaft, wurde aber nach
kurzer Zeit aus derselben entlassen und kam wieder glücklich in die Heimat
zurück.

\part{Allgemeines}

\chapter{Sagen}

\section{Wie Kaiser Karl Schulvisitation hielt (Nach einem Gedicht von Karl
Gerok).}

Kaiser Karl besuchte auch die Schulen und prüfte das kleine Volk im Schreiben,
Buchstabieren, Vaterunser, Einmaleins und was es damals sonst noch zu lernen
gab. Nach der Prüfung wurden die Fleißigen zur rechten und die Faulen zur linken
Seite des Kaisers aufgestellt. Die Fleißigen und Braven, unter denen sich
manches Kind eines armen Knechtes des kaiserlichen Hofgesindes und manches
Bürgerkind in einfachem Leinenkittel befand, lobte der Kaiser und versprach
ihnen stets gütiger Vater und gnädiger Herr zu sein. Die Faulen aber, unter
denen sich mancher feiner Herrensohn in Pelz oder Bändergeschmücktem Kleide
befand, diese tadelte er und ermahnte sie, dass es bei ihm nicht auf den Namen,
sondern auf das Verdienst des Einzelnen ankomme. Und so soll es im ganzen
Menschenleben und überall gehandhabt werden, zuerst die Kunst (das Können) und
dann erst die Gunst (die Bevorzugung).

\section{Gründung des Klosters Metten}

Kaiser Karl der Große kam bei seinen Jagdausflügen in die wildreichen Wälder des
Bayerischen Waldes einmal zur Klause des Einsiedlers Utto, die sich in der Nähe
der Ortschaft Berg bei Deggendorf befand. Utto war eben bei der Arbeit, sich
einen Balken zu zimmern. Das Beil, das er zu dieser Arbeit benützte, brauchte er
nicht auf den Boden zu legen, sondern schwebte frei in der Luft von den
Sonnenstrahlen getragen. Als der Kaiser einen Trunk frischen Wassers verlangte,
klopfte Utto mit dem Beil auf den nächsten Stein und schon sprudelte eine Quelle
mit klarstem Wasser hervor. Kaiser Karl erkannte in Utto einen wunderbaren Mann.
Er gestattete dem Einsiedler einen Wunsch zu äußern. Dieser wünschte sich die
Entstehung eines Klosters. Kaiser Karl sagte: „Werfe das Beil in die Luft und wo
dieses hinfällt, da will in ein Kloster erbauen. Utto tat wie ihm befohlen. Das
Beil schwebte in der Luft fort und blieb erst eine halbe Gehstunde von diesem
Ort in einem Baume stecken. Hier wurde dann das Kloster Metten errichtet. Der
erste Abt dieses Klosters war Utto, der in der Klosterkirche zu Metten begraben
liegt und später heilig gesprochen wurde.

Uttobrunn hat ein schönes Kirchlein, das von vielen Fußwanderern gerne besucht
wird.

\section{Graf Aswins Tanne (gekürzt nach Adalbert Müller)}

Die Grafen von Bogen hatten im Osten des Nordgaues in den slawischen Völkern
gefährliche Nachbarn, die immer wieder über die Arberkette herüberkommend in das
Gebiet der Bogener Grafen einfielen. Aber die Grafen von Bogen waren tapfere
Kämpfer und so konnte Graf Aswin die Slawen in drei Feldschlachten besiegen.
Nach Beendigung der 3. Schlacht am Einfaltersberg an der Landstraße von Cham
nach Straubing rastete der Graf unter einer mächtigen, hohen Tanne. Zum Zeichen
des Sieges und zur Erinnerung daran auch noch in späteren Jahren hieb er mit
wuchtigen Schlägen in den Stamm dieser Tanne das Zeichen des Kreuzes mit seinem
eigenen Schwerte. Graf Aswin war ein gefeierter Sieger und wurde genannt: „der
Schreck der Böhmen.“ Die Tanne stand viele Jahrhunderte. Altersschwach wurde sie
von einem Sturmwinde gebrochen. Aber alle, die an dem Stock vorübergingen,
wunderten sich über den ungeheuren Umfang der einstigen riesengroßen
Aswins-Tanne.

\section{Der Schatz in Ruhmannsfelden}

Es war im Frühjahr zurzeit der Feldbestellung. Da musste ein Knecht einen Acker
am Hang, der seinerzeit zur Burgsiedlung Ruhmannsfelden gehörte, umackern. Auf
einmal blieben aber Ross und Pflug und Knecht stehen. Der Pflug hatte sich in
einen starken Eisenring eingehängt, der an einer großen Kiste gefestigt war. Wie
kam nun diese Kiste in den Ackerboden und was mag wohl in derselben verborgen
sein? Der Knecht dachte sofort an einen versteckten Geldschatz. Zunächst löste
er den Eisenring von seinem Pflug, dann versuchte er, die Kiste zu öffnen. Es
gelang. Was musste er nun zu seinem größten sehen? Das reinste Gold und die
kostbarsten Edelsteine blitzten ihm entgegen. Schon wollte er darnach greifen.
Aber im selben Augenblick ertönte ein starker Pfiff und er zog rasch seine Hand
wieder zurück. Sein Herr war es, der pfiff, weil er glaubte, sein Knecht
faulenze. Und der Knecht ackerte weiter, als ob gar nichts vorgefallen sei,
zumal er ja seinem Herrn nichts wissen lassen wollte von dem verborgen Schatz in
der Kiste. Kaum knallte die Peitsche und kaum hatte das Rösslein wieder
angezogen, da ertönte ein donnerähnliches krachen und knallen. Der ganze Acker
bebte und zittere. Die Kiste mit dem kostbaren versank in die Tiefe. Übrig davon
blieb nur mehr das schallende Hohngelächter der höllischen Geister, die mit der
Kiste für immer in der Unterwelt verschwanden. Hätte der Knecht mit Bortkrummen
oder mit seinem Roßenkranz die Kiste gebannt, so wäre ihm der sicher gewesen und
er wäre ein reicher Mann geworden.

\section{Die Perle in der Teisnach (M. Waltinger)}

In der Christnacht fuhr einmal ein Mann von Ruhmannsfelden nach Gotteszell
zurück. Als er auf der Teisnacher Brücke angelangt war, sah er aus dem Wasser
ein Lichtlein schimmern. Er beugte sich über das Geländer und gewahrte, dass
dasselbe eine herrliche Perle beleuchtete, die inmitten einer geöffneten Muschel
lag. Da er die Muschel nicht erreichen konnte, fuhr er rasch nach Hause und
kehrte so schnell als möglich mit einigen Leuten, die ihm helfen sollten, wieder
zurück. Gerade kam er auf der Brücke an, als man in Gotteszell zur Mette
läutete. Da verschwanden Licht und Perle.

\section{Die Pest in Ruhmannsfelden}

Zurzeit, als die Pest auch in Ruhmannsfelden wütete, lebte in einem Häuschen der
damaligen Hofmark Ruhmannsfelden auch ein fleißiger, tüchtiger Weber mit seinem
Gesellen und seiner Familie. Da er mit ansehen musste, wie man einen Nachbarn
nach dem andern auf dem Karren hinausfuhr ins Siechet, hatte er nicht mehr viel
Lust zu Arbeit und er und sein Geselle saßen statt an dem Webstuhl an dem Tisch
und beratschlagten, wie man am besten der Pest entrinnen könne. Da hatte der
Geselle einen glücklichen Einfall. Er sagte: „Meister! Nichts hilft, außer wir
sperren die Pest ein!“ Mit diesem Vorschlag war alles einverstanden. Die Weberin
war das Strickzeug weg. Die Kinder klatschten und alle riefen voll Freude und
Zuversicht: „Ja! Die Pest wird eingesperrt!“ Der Geselle nahm einen Bohrer und
machte damit ein tüchtiges Loch in die Holzwand der Stube und sprach: „Pest, ich
will dich bannen!“ Dann schlug er einen festen hölzernen Pfropfen darauf. Und
siehe! Von diesem Augenblick an verschwand die Pest. Niemand starb mehr und
alles konnte wieder seiner Arbeit nachgehen. Nach einiger Zeit ging der Geselle
auf die Wanderschaft, wie es damals üblich und vorgeschrieben war. Der Weber
bekam einen anderen Gesellen, einen leichtfertigen und neugierigen Burschen. Am
zweiten Tag sah er schon den Holzpfropfen an der Stubenwand. Als man ihn
aufklärte über den Zweck dieses Pfropfens, sagte er lächelnd: „Lassen wir die
Pest wieder heraus!“ Er nahm ein Scheit Holz und schlug damit den Pfropfen aus
der Wand heraus. Und siehe! Am nächsten Tag ging das Sterben an der
Pestkrankheit schon wieder los. Der erste, der starb, war der leichtfertige
Webergeselle.

\section{Von der Entstehung des Osterbrünnls}

Es war um die Mitte des 17. Jahrhunderts. Da hatte der Bruckhof-Bauer einen
Hütbuben. Das war ein etwas schwächlicher und dazu fußkranker Knabe. Mit dem
Laufen und dem Springen, was man bei einem Hütbuben besonders voraussetzt,
konnte er nicht viel machen. Er musste viel sitzen während des Hütens. Da saß er
wieder einmal an einem schönen Herbstnachmittag auf einem Baumstock am Ufer der
Teisnach. Plötzlich sah er im Wasser der Teisnach ein Muttergottes-Bild
schwimmen. Schnell holte er es aus dem Bach heraus und lehnte dasselbe an den
nächsten Baum. Abends erzählte er es dem Bruckhof-Bauern. Dieser verständigte
davon den Hr. Pfarrer. Diese machte das Bild am Baume fest. Der Hütbube ging
alle Tage zu diesem Bilde hin und betete. „Siehe Himmelmutter, mach meine Füße
wieder gesund!“ Und siehe! Die Himmelmutter erhörte das Bitten des Hütbuben und
er wurde schnell gesund. Nach kurzer Zeit war am nächsten Baum die erste
Votivtafel angebracht mit der Inschrift: „Maria hat geholfen.“ Diese
wundertätige Heilung wurde bald weit und breit bekannt. Viele Leute aus nah und
fern strömten an diese Gnadenstätte um Hilfe in ihrem Anliegen und Nöten bei
Maria, der Helferin und Retterin, zu ersehen. Später wurde aus der Anhöhe eine
hölzerne Kapelle errichtet und Osterbrünnl genannt.

\section{Das Schwedenloch}

Am Ende des 30-jährigen Krieges kamen schwedische Truppen von Böhmen her auf
ihrem Rückmarsche auch in die Gegend von Achslach. In Wolfertsried, in der Nähe
von Achslach, haben sich einige Schweden eingenistet und haben von dort aus ihre
Streifzüge gemacht. In ferner Nacht versammelten sich nun die Männer von
Achslach und Umgebung bei ihrem Ortsvorstand auf dem Hienhart und geschlossen,
sich dieser Bedrücker zu entledigen. Um die Mitternachtsstunde zogen sie dann,
von ihrem Ortsvorsteher geführt, mit Sensen, Acker- und Hausgeräten als Waffen
gegen die noch ansässigen Schweden. In einem hinteren Wald hergestellten
gemeinsamen Grab sollen sie beerdigt sein. Diese Stätte wird heute noch
Schwedenloch genannt.

\section{Der Hirschenstein}

In jener Zeit, als es in hiesiger Gegend noch viele Hirsche gab, spürten
anlässlich einer Jagd in den Achslacher Forsten die Jagdhunde einen
Sechzehnender auf und jagten denselben über den ganzen Bergrücken des
Kälberbuckels hinüber der Steinwand zu. Es ging bei dieser Jage durch Dickicht
und Beerengesträuch, durch Buchen- und Eichenwald hindurch. Plötzlich stand der
Hirsch, der König des Waldes, auf einem hohen Felsen. Da die großen Jagdhunde
schon ganz nahe hinterdrein waren, stürzte sich der Hirsch von dem hohen Stein
hinab in die Tiefe und blieb unten tot liegen. Seitdem heißt dieser Berg der
Hirschenstein.

\section{Der Teufel auf der Ödwies}

Die Knechte des Sägmüllers mussten mit dem Blöcherwagen, der von 2 kräftigen
Pferden gezogen wurde, die bereits ausgesetzten Blöcher von der Ödwies holen.
Das Aufladen dieser Blöcher ging ausgerechnet an diesem Tage nicht ohne
Schwierigkeiten ab und dauerte auch viel länger als sonst. Als sie mit dem
Holzbeladenen Wagen zum Platzl kamen, zogen die beiden Pferde keinen Schritt
mehr. Alle Bemühungen waren vergebens. Der Baumer schimpfte und fluchte und
sagte: „Wenn nur grad der Teufel käm!“ Im selben Augenblick trat ein kleines
Männlein, ein hochbeiniges Pferd hinter sich herführend, aus dem Wald heraus. Es
blieb stehen, schaute einige Augenblicke Wagen, Pferde und Knechte an. Dann nahm
es eines der eingespannten Pferde am Zügel und sofort zogen die beiden Pferde
mit Leichtigkeit den schwer beladenen Wagen weiter. Das Männlein sagte aber
noch: “Ihr werdet nicht weit kommen und schon war das Männlein mit seinem Ross
verschwunden. Als die Pferde wieder anzogen, rutschte der Blöcherwagen in den
Graben. Jetzt haben die Knechte nicht mehr geflucht, sondern spannten eiligst
die Pferde aus und verschwanden rasch von der unheimlichen Stelle, wo ihnen der
Teufel leibhaftig in Gestalt eines kleinen Männleins mit einem Pferd erschienen
ist. Sie gelobten in Zukunft den Teufel nicht mehr herbeizuwünschen auf der
Ödwies.

\chapter{Ruhmannsfeldner Söhne, die sich dem Priesterstande widmeten}

\section{Franz Lorenz Graßl}

H. Hr. Franz Lorenz Graßl, Missionär von Philadelphia, geb. 18.8.1753 als
Lederermeisterssohn in Ruhmannsfelden (Völkl-Haus), wanderte 1787 nach wenigen
Jahren priesterlicher Tätigkeit in der Diözese Regensburg nach Amerika aus, wo
er als eifriger Missionär segenreich wirkte. Wegen seiner vorzüglichen Natur-
und Geistesgaben wurde er zum Coadjutor Bischof von Baltimore gewählt. Bis zum
Eintreffen der päpstlichen Bestätigung dieser Wahl aus Rom versah er immer noch
den Dienst der Missionärs in Philadelphia, wo damals die Pest entsetzlich
wütete. Im Jahre 1793 erlag auch er als Opfer der Liebe und des Seeleneifers der
schrecklichen Pestkrankheit. R.I.P.

\section{Franz Xaver Fromholzer}

H. Hr. Franz Xaver Fromholzer, Pfarrer der 14-Nothelfer-Kirche in Gardenwille,
Diözese Buffalo, Nordamerika, wurde am 25.7.1775 in Brixen, Tirol, zum Priester
geweiht und wirkte überaus segensreich 16 Jahre lang in Springwille, Aschford,
Scheldon und Gardenwille, wo er auch im 42. Lebensjahr am 4.3.1793 starb. R.I.P.

\section{A. C. Helmbrecht}

H. Hr. A. C. Helmbrecht, Monsignore in Hoven, St. Dakota (USA) geb. 13.5.1870 in
Ruhmannsfelden als der Sohn von Alois und Franziska Helmbrecht (Baumgartnerhaus
Kalteck), war 2 Jahre Hütbube in Zuckenried, dann 3 Jahre Färbergehilfe bei
Fromholzer hier. Hierauf begann er das Studium in Metten, das er noch einigen
Jahren in Mt. Calvarn Seminar in Milwaukee in Nordamerika fortsetzte, dort auch
seine philosophischen und theologischen Studien beendet und auch dort von
Erzbischof Katzer am 29.6.93 zum Priester geweiht wurde, Ein Jahr lang war er
Assistent an der St. Franziskuskirche in Milwaukee und 4 Jahre lang
Hilfspriester an der Muttergottes-Kirche in New York City. Dann war er 4 Jahre
als Missionär tätig und wurde darauf Hilfspriester an der Engelskirche in New
York. Im Jahre 1904 ging Msgr. Helmbrecht auf Einladung des Bischofs O. Groman
nach Süd Dakota, weil dort Mangel an Deutschsprechenden Priestern war. Er wurde
Pfarrer in Frank und Eden in der Provinz Day und arbeitete da sehr erfolgreich,
bis er die Pfarrei Hoven übernahm. Eben war die Bahn nach dort gebaut und für
die aufstrebende Stadt war Pfarrer Helmbrecht gerade der richtige Mann, der
durch seine Initiative für den Aufbau der Stadt verantwortlich zeichnete. Die
Kirche in Hoven war in schlechtester Verfassung, ebenso die Pfarrei als Ganzes
gesehen. Er überbrückte die Feindschaften unter seinen Pfarrkindern und ging
dann an das Aufbauwerk. Zuerst erbaute er die Schule (27 000 Dollar) 1908. 1909
war deren feierliche Einweihung. 1912 fing er mit dem Bau der großen, schönen
Kirche an. Zu Weihnachten des gleichen Jahres konnte er in der neuen Kirche
schon die hl. Messe lesen. 1916 erbaute er ein neues Pfarrhaus. (17 000 Dollar)
1918/19 ging er an den Weiterausbau der Kirche und am Gründonnerstag 1921 war
der Bau vollendet, der 250 000 Dollar gekostet hat. 1921 gönnte er sich Erholung
und fuhr in seine Heimat nach Deutschland. Sein Neffe H. Hr. Pfarrer Ludwig
Brunner, ein geborner Ruhmannsfeldner, versah währen seiner Abwesenheit von
Hoven dessen Pfarrei. Am 30. Mai 1923 feierte H. Hr. Pfarrer Helmbrecht sein
Silberjubiläum. 31 Priester feierten mit ihm und Bischof Mahong hielt dabei das
das Pontifikalamt und ehrte den verdienten Priester in glühenden Worten. 1930
wurde H. Hr. Pfarrer Helmbrecht Diözesan-Consultator und erhielt von Papst Pius
XI. 1932 den Titel Monsignore (Vorstehens ist entnommen der großen Zeitung „The
Hoven Review“, die unterm 6. Mai 1948 die ganze riesengroße Titelseite dieser
Zeitung H. Hr. Msgr. Helmbrecht und seiner Wirkungsstätte mit Bildern widmete.).

\section{Peter Fenzl}

H. Hr. Peter Fenzl, geistlicher Rat und Oberpfarrer an der Strafanstalt
Straubing, geboren am 29.6.1866 als der Sohn der Bauerseheleute Josef und Anna
Fenzl von Bruckhof, Gemeinde Zachenberg, besuchte die Volksschule in
Ruhmannsfelden, studierte in Metten und Regensburg und feierte sein 1. hl.
Messopfer am 8.5.1892 in der Pfarrkirche zu Ruhmannsfelden. Er wirkte dann als
Priester in Kallmünz, Neukirchen-Balbini, Schwandorf und Straubing und wurde bei
Eröffnung der Strafgefangenenanstalt Straubing 1920 zum Oberpfarrer dieser
Anstalt ernannt, an der er bis zu seiner Pensionierung wegen Krankheit Jahre
1927 wirkte. Am 17.5.1932 feierte H. Hr. geistlicher Rat Fenzl sein 40-jähriges
Priesterjubiläum und am 8.5.1942 sein 50-tes. Er starb am 23. Januar 1945 in
Straubing. R.I.P.

\section{Alois Auer}

H. Hr. Alois Auer, Pfarrer in Hörgering bei Neumarkt, geboren am 15.1.1880 in
Oberpöring als der Sohn der Oberlehrerseheleute Alois und Anna Auer (von 1895
bis 1920 in Ruhmannsfelden), wurde am 23.7.1906 in Freising geweiht, feierte
sein 1. hl. Messopfer am 31.7.1906 in Ruhmannsfelden, war Coadjutor in Buch a.
Erbach und Fridolfing, dann Kaplan in Landshut St. Jodok, wurde am 2.8.21
Pfarrer in Hörbering, verzog am 1.9.32 als freires. Pfarrer nach Wiesau, Diözese
Regensburg. Ab 1.10.35 war er Kommorant in St. Weit Neumarkt, wo er am
17.11.1939 starb. R.I.P.

\section{Ludwig Brunner}

H. Hr. Ludwig Brunner, Pfarrer in Herreid, Staat Dakota, Nordamerika, geboren am
30.7.1897 in Ruhmannsfelden als der Sohn des Postoberschaffners Anton Brunner in
Ruhmannsfelden, besucht die Volksschule Ruhmannsfelden, studiert dann 5 Jahre
lang im Kloster Metten und wanderte 1913 nach Amerika aus. Dort setzte er sein
Studium am Gymnasium und an der theologischen Hochschule fort, das er in seinem
21. Lebensjahr vollendete. Da aber die Weihe zum Priester erst nach Vollendung
der 23. Lebensjahres erfolgen konnte, war H. Hr. Pfarrer Brunner zunächst als
Gymnasialprofessor 3 Jahre lang tätig. Am 15.6.1921 feierte er dann sein erstes
hl. Messopfer und kam nach Herreid im Statt Dakota, Nordamerika, wo er 15 Jahre
als Priester wirkte. Von da weg wurde er versetzt nach St. Mary.

Da die Pfarrangehörigen seiner früheren Pfarrei die Zurückversetzung an seine
früher Wirkungsstelle forderten, kam er wieder wunschgemäß nach Herreid, wo H.
Hr. Pfarrer Brunner heute noch segensreich wirkt.

\section{Amand Bielmeier}

H. Hr. Dr. Pater Amand Bielmeier, Professor und Direktor im Benediktinerkloster
Metten, geboren am 16.9.1902 in Prünst, Gemeinde Patersdorf als der Sohn der
Zimmermannseheleute Josef und Maria Bielmeier von Prünst, besuchte die
Volksschule Ruhmannsfelden, studierte in Metten (1912 bis 21), in Regensburg (21
und 22) und an der Universität München (23 bis 26). Nach der Priesterweihe in
Metten (10.4.1926) feierte er sein 1. hl. Messopfer in Ruhmannsfelden am
19.4.1926. Hierauf setzte er die Universitätsstudien fort in Würzburg und macht
dort das Doktorexamen der Philosophie (1930). Hierauf war H. Hr. Dr. Pater Amand
zunächst Professor am Gymnasium Metten und Direktor des Klosterseminars und
währen der Zeit der Unterrichtssperre (Nazizeit) Leiter der Klosterverwaltung
Metten. Seitdem aber am dortigen Gymnasium wieder Unterricht erteilt wird,
waltet er in seiner früheren Stellung als Professor und als Direktor des
Klosterseminars Metten seines Amtes.

\section{Johann Bielmeier}

Sein Bruder, H. Hr. Johann Bielmeier, Pfarrer in Michaelsneukirchen, geboren am
8.2.1904 in Prünst, Gemeinde Patersdorf, besuchte ebenfalls die Volksschule in
Ruhmannsfelden, studierte in Metten und Regensburg, wurde dort zum Priester
geweiht und feierte am 7.7.1928 in Ruhmannsfelden sein 1. hl. Messopfer. Jetzt
ist derselbe Pfarrer in Michaelsneukirchen.

\section{Fritz Kiendl}

H. Hr. Fritz Kiendl, Pfarrer in Weisenfelden, geboren am 26.9.1908 als Sohn der
Schuhmacherseheleute Xaver und Anna Kiendl, studierte in Metten und Regensburg,
wurde am 29.6.1934 zum Priester geweiht und feierte am 10.7.1934 in
Ruhmannsfelden sein 1. hl. Messopfer. Zurzeit betreut er als Pfarrvorstand die
Pfarrei Wiesenfelden, Dekanat Bogen.

\section{Gedenktafel an Agnes Holler in der Pfarrkirche}

Auf einer Gedenktafel in der Laurentius-Pfarrkirche in Ruhmannsfelden steht
folgende Inschrift:

„Andenken an die ehrwürdige Missionsschwester Mr. Agnes Holler,
Metzgermeisterstochter von Ruhmannsfelden, die am 13. August 1904 bei einem
tückischen Überfall der Missionsstation Baining auf Neupommern, wo sie mit
einigen Brüdern und Schwestern zur Erholung und zur Feier der Einweihung der
neuen Kapelle weilte, durch Keulenhiebe getötet wurde und so als jugendliches
Opfer von 23 Jahren für das Reich Gottes, dessen Ausbreitung ihr als schönste
Lebensaufgabe galt, zur unverwelklichen Krone der Herrlichkeit gelangte. R.I.P.“

Wie aus der Schrift: „Die Hiltruper Märtyrer von St. Paul“ zu entnehmen ist,
wurden bereits die notwendigen Vorarbeiten zur Seligsprechung dieser Märtyrer
von St. Paul unternommen und wie aus Rom berichtet wird, sind dort die
diesbezüglichen Arbeiten schon soweit gediehen, dass wir mit Zuversicht auf eine
Seligsprechung der ehrwürdigen Mr. Agnes Holler rechnen dürfen.

\chapter{Verwaltung}

\section{Pfarrer}

Nach der Aufhebung des Klosters Gotteszell löste sich im Stillen auch der
Klosterkonvent allmählich auf. Das Verlassen des Klosters ohne Erlaubnis der
Regierung war den Mönchen bei Strafe des Verlustes des Anspruches auf Pension
bzw. des provisorischen Unterhaltsbeitrages verboten. Im September 1803 suchte
H. Pater Bernhard Kammerer nach, Beihilfe in der Seelsorge in Ruhmannsfelden
leisten zu dürfen, starb aber im September 1804. Dann versah hier das Amt eines
Pfarrprovisors H. Hr. Frz. Josef Heindl bis 30.11.1805. Er starb 19.10.1821 in
Gotteszell (Gedenkstein dort). Ab 1.12.1805 wirkte dann als erster definitiver
Pfarrer in Ruhmannsfelden H. Josef Castenauer, der im Jahre 1817 die Pfarrei
Regen verliehen erhielt. Ihm folgte als 2. Pfarrer in Ruhmannsfelden H. Peter
Blaim, der am 21. April 1821 an Lungenentzündung im 49. Lebensjahr hier
verstarb. 1806 wurde die Auspfarrung vollzogen.

Am 14.8.1806 starb hier der „Hochwürdige und Hochgelehrte H. Anton Xaver Sämer,
gewester Prior des Klosters Gotteszell.“ Der nach dem Ausscheiden aus dem
Kloster Gotteszell in hiesiger Pfarrei Seelsorgebeihilfe leistete. P. Marian
Triendorfer zog nach seinem Ausscheiden aus dem Kloster Gotteszell zuerst nach
Viechtach, kehrte aber bald wieder nach Ruhmannsfelden zurück und war hier als
Frühmesser beschäftigt. Er starb 1824. Nach dem Tode des Pfarrers Blaim versah
das Provisorat H. Hr. Kooperator Wagner. Von 1821 ab folgten als Pfarrer: Dieß
1821 – 27, Linhard 1828 – 41, Wagner 1841 – 45, Wandner 1846 – 57, Hösl 1857 –
61, Uschalt 1861 – 75, Rötzer 1875 – 81, Englhirt 1881 – 89, Neppl 1890 – 1902,
Mühlbauer 1902 – 18, Fahrmeier 1918 – 35, Bauer ab 13. November 1935.

\section{Lehrer}

Über Schulverhältnisse in Ruhmannsfelden vor der Aufhebung des Klosters
Gotteszell wurde in diesem Büchlein bereits berichtet. Einen Blick in die
früheren Schulverhältnisse geben uns die noch vorhandenen Urkunden aus damaliger
Zeit. So lesen wir z. B. in einer Urkunde folgendes: „Die Zechpröbste von Sankt
Lorenz in Ramffelden Markth. Filialgem. Geirstall geben an. Variert ein Meß,
welche einkommen bei 4 Jarlang 18 fl. davon halten sy ain Schuelmaister, geben
Ime jährlich 4 pfund Regensburger.“ In einer weiteren Urkunde heißt es:
„Schuelmaister zu Rumanffelden, Casparn Stralnberger von Niernberg pürtig, 6 Jar
allda gewest. Zu Leipzig studiert, hat mit Testimonium. Hat 10 Schueler,
darunter je zwen, so gute Jngenia haben, lernen gemeinlich erst lesen. Singt zu
Chor, Ist der alten Religion. Besoldung hat er aus der Bruderschaft 4 Pfund
Regensburger.“ Am 26. April 1779 starb hier Schullehrer Bernhard Hochreiter
(artificosus organista = kunstvoller Orgelspieler). 1830 starb Andrä Stern, der
49 Jahre lang im Schulfach wirkte. Nach diesem wirkte als 1. Lehrer in
Ruhmannsfelden: Lippl, Schinagl, Pongratz, Weig, Auer, Högn und Albrecht.

\section{Bürgermeisterix}

Vor 1803 stand Ruhmannsfelden unter der Jurisdiktion (niedere Gerichtsbarkeit)
des Zisterzienserklosters Gotteszell, das auch das amtliche Schreibwesen der
Marktgemeinde Ruhmannsfelden und damit auch das Siegel führte. Erst 1804 wurde
dem Markte Ruhmannsfelden das Selbstverwaltungsrecht übertragen. Der 1.
Bürgermeister in Ruhmannsfelden war der bürgerliche Bierbrauer Josef Liebl (1804
– 1806). Ihm folgten bis heute 41 Bürgermeister. Von allen diesen hatte der
Bürgermeister Lorenz Schreiner, Konditor, mit 12 Dienstjahren als amtierender
Bürgermeister. 1945 war sogar ein gebürtiger Holländer kurze Zeit
Marktbürgermeister.

\section{Gemeindeschreiber}

Nach Aufhebung des Klosters Gotteszell erscheint in den Urkunden und
Protokollbüchern der Gemeindeverwaltung Ruhmannsfelden ein Schullehrer, namens
Andrä Stern, als Gemeindeschreiber, der dann vom Schullehrer Tauschek von
Gotteszell abgelöst wurde. Nach diesem wurde Andrä Voerge aus Vilshofen durch
Beschluss des Magistrates vom 23.3.1823 gegen eine jährliche Besoldung von 325
Gulden als Marktschreiber aufgenommen. Er starb aber schon ein Jahr darauf. Es
gab dann einen häufigen Wechsel bei den Gemeindeschreibern bis im Juni 1834 der
Schullehrer Georg Lippl das Amt des Marktschreibers übernahm, der es bis
15.4.1866 führte. Nach diesem gab es auch wieder viele Veränderungen bei den
Marktschreibern, bis im November 1878 der Schullehrer Max Weig Marktschreiber
wurde, der es bis zu seinem Tode am 16.3.1895 war. Der Bezirksoberlehrer Alois
Auer versah das Amt das Marktgemeindeschreiberei Ruhmannsfelden vom 26.5.1899
bis zum 1.9.1920. Nach ihm folgten Büttner, Brunner, Klingseisen, Neueder,
Schroll, Achatz, Kilger und Linsmeier.

\section{Polizei}

In frühester Zeit versah den Polizeidienst der Scherge, Schergenkerl, Scharnkarl
genannt (schergen, schürgen, verschürgen = verklagen). Es war das in der Regel
ein bewaffneter, himmellanger, kräftig gebauter Kerl, der stets einen großen
Fanghund mit sich führte. Das Schergenhaus befand sich zuletzt im Meindl-Haus
(Winkler). Das jetzige Ellmann-Haus in der Bachgasse war das frühere
Klostergerichtshaus. Dort befanden sich auch die zwei Arreste, ein ganz kleiner,
niederer und finsterer Raum, in dem man nur liegen, aber nicht stehen konnte und
ein klein wenig größerer Raum, in den nur durch eine ganz schmale Mauerscharte
wenig Luft und Licht Zutritt hatten. Wer in einem solchen Arrest eingesperrt
war, der hatte wirklich nichts zu lachen. Die Schergen in Ruhmannsfelden standen
im Dienste des Klosterrichters von Gotteszell. Nach der Aufhebung des Klosters
Gotteszell erhielt Ruhmannsfelden das Selbstverwaltungsrecht und damit auch eine
eigene Ortspolizeibehörde, bestehend aus dem jeweiligen Bürgermeister und dem
Polizeimann, der zugleich auch das Amt des Gemeindedieners versah. Außer der
Ortspolizei besteht in Ruhmannsfelden auch eine staatliche Gendarmeriestation.
Ursprünglich waren die Gendarmen (Schandarmen) Edelleute, die Dienst machten als
Leibgarde am Hofe der französischen Könige.

In Deutschland waren es die Kürassiere, die diesen Dienst an den deutschen
Fürstenhöfen versahen. Der Name Gendarm übertrag sich dann in Deutschland auf
die militärisch organisierte Polizeitruppe, welche wir mit dem Namen Gendarmerie
(Schandarmerie) bezeichneten. Seit 1945 führt sie den Namen Landpolizei, das ist
eine auf Schutz der öffentlichen Ordnung und Sicherheit gerichtete Gewalt der
inneren Verwaltung. Stationsführer der hiesigen Gendarmerie waren: Bogenreither,
Bauer, Gmeiner, Adrian, Baumgartner, Hetzenecker, Sponrast, Wallner und
Schindlbeck. Den Landpolizeiposten Ruhmannsfelden führten: Schaffer und zurzeit
Oberkommissär Piehler.

\section{Post}

Die Post ist eine Einrichtung zur regelmäßigen Beförderung von Sendungen und
Personen. Sie wurde von Thurn und Taxis, einem italienischen Adelsgeschlecht,
das sein Schloss in Regensburg hat, eingeführt und das seit 1516 alle
Hoheitsrechte über die Post besaß. Die letzten Postgerechtsamen verlor das
Fürstenhaus Thurn und Taxis im Jahre 1867. Seit dieser Zeit ist die Post eine
öffentliche Staatsanstalt. Die deutsche Reichspost besteht seit 1871. Die
bayerische Posterverwaltung aber wurde erst am 1.4.1920 vom Reich übernommen.
Postsendungen und Personen wurden in früheren Zeiten mit der gelben Postkutsche
befördert. Der Postillon war bekleidet mit hohen Stiefeln, weißer Lederhose,
blauem Frack mit Silberknöpfen, Zylinder mit hohem, weißblauem Büschl und führte
das Posthorn an weißblauer Schnur mit sich. Da die Strecke von Viechtach bis
nach Deggendorf täglich hin und zurück für die Postpferde zu anstrengend gewesen
wäre, wurden die Pferde in Ruhmannsfelden gewechselt. Der Poststall war damals
bei Kaufmann Probst. Am 20. Oktober 1883 war hier die Eröffnung der
Posttelegraphenstation. Nachdem im November 1890 die neu erbaute Regentalbahn
Viechtach-Gotteszell dem öffentlichen Verkehr übergeben wurde, übernahm diese
Bahn neben Güter- und Personentransport auch die Beförderung der Postsendungen.

1906 wurde hier eine öffentliche Telefonstelle errichtet. Früher war das
Postlokal in der Brauerei Amberger. Seit 1939 befindet sich dasselbe im Haus des
Hr. Professors Hieke. Als Leiter der Postagentur Ruhmannsfelden sich noch in
Erinnerung: Hr. Alois Sagstetter, Frl. Laufenbeck und Hr. Rödl. Seitdem die
Postagentur in eine Postanstalt umgewandelt wurde, versahen den
Verwaltungsdienst der hiesigen Postanstalt: Frl. Mathilde Sagstetter, Frl.
Wirthensohn und Hr. Pilch. Zurzeit ist damit Hr. Brummer betraut.

\section{Aufschlageinnehmerei}

Früher war in Ruhmannsfelden auch eine königliche Aufschlageinnehmerei. Sie war
eine Nebenstelle des Zollamtes Zwiesel. Der Aufschläger hatte die Kontrolle und
die Verrechnung über die aufschlag- und zollpflichtigen Waren. Davon waren in
Ruhmannsfelden hauptsächlich die Brauereien betroffen (Bier- und Malzaufschlag).
Als Aufschläger fungierten hier: Wimmer, Eisenreich, Strobl, Sturm und Hörner.
1924 wurde die hiesige Aufschlageinnehmerei aufgehoben und dem Zollamte Zwiesel
zugeteilt. In frühester Zeit hieß der Aufschlag oder Zoll: „die Maut.“ Schiffe
mussten von ihrer Schiffsladung die Maut entrichten (Mäuseturm oder Mautturm).
Von den schwer beladenen Wägen der Kaufleute, die auf den großen Handelstraßen
von Süd nach Nord oder von West nach Ost ihre Waren transportierten (Eisenbahnen
gab es damals nicht), wurde die Maut heruntergeholt. Im Landkreis Wolfstein im
unter bayerischen Wald führt eine Grenzortschaft den Namen Mauth.

\chapter{Flurnamen}

Im Gemeindebezirk Ruhmannsfelden gibt es zirka 195 verschiedene Flurnamen (in
der Regel 10 – 20\% der vorhandenen Plannummern), die uns hinweisen:

\begin{itemize}
  \item auf den Besitzer des Grundstückes z. B. Englmeierfleck, Föderlwies,
  Moosmüllerwies, Woferlacker, usw.

  \item auf die berufliche Tätigkeit des Besitzers z. B. Hirtagärtl,
  Schinderwies, Schmiedhöh, Farberackerl, Scheideracker, Lederergraben,
  Tuchhauserloch, usw.

  \item auf Ausnahmgrundstücke z. B. Häuslwies, Leibthumacker, Ahnlwies,
  Ausnahmsacker usw.

  \item auf den Ort des Grundstückes z. B. Kalteneckerackerl, Grabackerl,
  Bühelacker, Rabensteineracker, usw.

  \item auf angrenzende Ortschaften z. B. Auhofwies, Achslacheracker,
  Gotteszelleracker, Perlesleithenacker, usw.

  \item auf Wasser z. B. Frauendümpflacker, Rothseignwies, Weiherwies usw.

  \item auf Wiese und Garten z. B. Häupflwies, Heuweg, Krautgarten,
  Hopfengarten, usw.

  \item auf Bäume z. B. Reiserackerl, Größlingacker, Laubbergacker,
  Stockholzwies. Aspernackerl, usw.

  \item auf Tiere z. B. Ziegenglöcklacker, Fuchsacker, Stierwies,
  Lerchenfeldacker, usw.

  \item auf Eigentum der Kirche z. B. Pfarrerwies. Pfarreracker, Pfarrermoos,
  Priesterweg, usw.

  \item auf Form und Lage des Grundstückes z. B. Spitzackerl, Zipfelwies,
  Gabelacker, Winklackerl, Wellen- oder Wiegenacker, Lang- und Breitacker,
  Hochfeldacker, Multernackerl, usw.
\end{itemize}

\chapter{Gassen, Gassl, Wege von früher}

\section{Gassen}

Bachgasse, Wohlmuthgasse, Hauptgasse, Hirtagasse, Grabgasse, Urtlgasse,
Leithengasse, Brechhausgasse.

\section{Gassl}

Schlossergassl, Pfarrergassl, Konditorgassl, Badergassl, Strickergassl,
Färbergassl, Glasergassl, Geinergassl, Schrollgassl, Rauchschmiedgassl,
Lieblschneidergassl.

\section{Wege}

Osterbrünnlweg, Zachenbergerkirchenweg, Lämmersdorferkirchenweg, Stegweg,
Mühlweg, Baderweg, Siechangerweg, Handlingerweg, Multernweg, Huberwaidweg,
Sintwegingerweg, Geigerbergweg, Angerweg, Hamperholzweg, Sagholzweg,
Hochstrassweg, Sägmühlweg, Priesterweg, Dürrweg, Föderlweg, Weiherwiesweg,
Bodenweg, Schmiedhöhweg.

\section{Straßen}

Der Weg von Viechtach über Ruhmannsfelden wurde schon im 13. Jahrhundert
„Strass“ genannt, wie man aus der Bestätigungsurkunde Herzogs Otto von 1294
ersehen kann, in welcher das „Straßholz“ bei Gotteszell vorkommt (Prata apud
silvam, quae vulgariter dicitur Straßholz. Mon. Boic. Fol. V. p. 401). Die
Straße von Ruhmannsfelden nach Deggendorf führte ehedem über die Hochstrass.

Die Straße von Ruhmannsfelden über Bergerhäusl und Stockerholz zum Bahnhof
Gotteszell wurde erst nach dem Bahnbau gebaut.

Die Strass von Ruhmannsfelden nach Giggenried führte durch eine Furt, die inx
eine Stelle, an der die Fuhrwerke durch das Wasser der Teisnach fahren mussten.
Für die Fußgänger war an dieser Stelle ein Steg angebracht (Mühle beim Steg =
Stegmühle). Eine solche Furt war auch in unmittelbarer Nähe der heutigen
Kotmühle, früheren Probst-Mühle.

\chapter{Höhenlagen in unserem Heimatgau und Barometerstand}

Meereshöhe von Ruhmannsfelden = 536,9 m.

Die nachstehend angeführten Höhen sind in Metern angegeben:

\begin{longtable}{ll}
  Hochstraße (Birkenholz) & 592 \\
  Teisnacherbrücke nach Gotteszell & 538 \\
  Gotteszell Ortschaft & 568 \\
  Kalvarienberg & 659 \\
  Gotteszell Bhf. & 553 \\
  Bergerhäusl & 569 \\
  Stockerholz & 572 \\
  Harnberg & 560 \\
  Hampermühle & 571 \\
  Achslach & 591 \\
  Lindenau & 641 \\
  Schneeberg & 882 \\
  Ödwieser Forsthaus & 1029 \\
  Hirschenstein & 1092 \\
  Kälberbuckel & 1057 \\
  Senke zum Kälberbuckel und Hirschenstein & 904 \\
  Schusterstein & 897 \\
  Senke zum Hirschenstein und zum Rauhen Kulm & 952 \\
  Rauhe Kulm & 1049 \\
  Grün & 719 \\
  Kalteck & 752 \\
  Hochgart & 675 \\
  Wittmannsberg & 737 \\
  Hochweid & 740 \\
  Klosterstein & 1019 \\
  Regensburgerstein & 951 \\
  Vogelsanghaus & 900 \\
  Schwarzenberg & 766 \\
  Fasslehen & 645 \\
  Oberhalb Köckersried & 651 \\
  Grub & 617 \\
  Wühnried & 727 \\
  Loderhard & 724 \\
  Oberbreitenau & 1037 \\
  Einödriegel & 1128 \\
  Wachtstein & 978 \\
  Bocksruck & 972 \\
  Gemeindeberg & 843 \\
  Bumsenberg & 695 \\
  Zusammenfluss von Wandlbach und Teisnach & 507 \\
  Oberhalb der eisernen Bücke & 601 \\
  Wandlhof & 537 \\
  Lämmersdorf & 589 \\
  Breitenstein (Brigidenstein) & 766 \\
  Wolfsberg & 693 \\
  Rabenholz & 736 \\
  Vierzehn Nothelfer & 734 \\
  Mühlholz & 645 \\
  Dietzberg & 632 \\
  Bärwinkel & 788 \\
  Höllholzberg & 847 \\
  Berg bei Zottling & 813 \\
  Schön & 790 \\
  Hochebene zwischen Perlesried und Mooshof & 693 \\
  Handling & 538 \\
  Handlingberg & 580 \\
  Masselsried & 608 \\
  Mausmühle & 557 \\
  Schönberg & 554 \\
  Armenhaus Prünst & 499 \\
  Schinderhöhe & 515 \\
  Großer Riedelstein & 1134 \\
  Arnbrucker Forst & 1083 \\
  Sattelhöhe & 1153 \\
  Hochstein & 1069 \\
  Enzian & 1287 \\
  Kl. Arber & 1389 \\
  Gr. Arber & 1457 \\
  Falkenstein & 1372 \\
  Bodenmais & 691 \\
  Harlacherspitz & 918 \\
  Platte bei Böbrach & 881 \\
  Kronberg & 983 \\
  Scheuereck & 773 \\
  Weißenstein & 758 \\
  Zwieselberg & 686 \\
  Rachel & 1454 \\
\end{longtable}

Barometerstand von Ruhmannsfelden = 711,2 mm. Zum vorstehenden Barometerstand
ist Folgendes zu bemerken: Das Barometer fällt, je höher wir steigen. Am Meer
beträgt der normale Barometerstand 760 mm.

Von 0 – 700 m fällt das Barometer um 1 mm bei 11 m Steigung.

Von 700 – 1450 m fällt das Barometer um 1 mm bei 12 m Steigung.

Von 1450 – 2000 m fällt das Barometer um 1 mm bei 13 m Steigung.

Es ist demnach für jede vorstehende angeführte Höhe der Barometerstand selbst
leicht zu berechnen.

\chapter{Ruhmannsfelden auf der Erdkugel}

Ruhmannsfelden liegt am 49. Grad nördliche Breite und am 13. Grad östliche
Länge. Am 49. Grad nördlicher Breite ist ein Längengrad vom anderen noch 72,9 km
entfernt. Der 49. Breitengrad ist demnach 72,9 km mal 360 = rund 26 244 km lang.
Die Erde dreht sich in rund 24 Stunden um ihre eigene Achse. Die Geschwindigkeit
der Erde in Ruhmannsfelden (am 49. Grad nördlicher Breite) ist demnach:

Stundengeschwindigkeit: 26 244 km : 24 = 1093 km/hxi

Minutengeschwindigkeit: 1093 km : 60 = 18,2 km/mxii

Sekundengeschwindigkeit 18,2 km : 60 = 300 m/sxiii

\chapter{Bevölkerungsbewegung}

Volkszählung

Einwohner der Gemeinde Ruhmannsfelden
\begin{longtable}{l|ccc}
  Datum & männlichxiv & weiblich & Gesamt \\
  1.12.1885 & 508 & 587 & 1095 \\
  2.12.1895 & 666 & 703 & 1369 \\
  1.12.1910 & 661 & 758 & 1419 \\
  1.12.1916 & 519 & 750 & 1269 \\
  8.10.1919 & 693 & 801 & 1494 \\
  16.6.1933 & 797 & 821 & 1618 \\
  17.5.1939 & 795 & 857 & 1652 \\
  29.10.1946 & 1177 & 1434 & 2611 \\
\end{longtable}

Seelenzahl der Pfarrei Ruhmannsfelden = 3800

Erkenne in Deiner Heimat die Allmacht und Weisheit Gottes – dann wirst Du Sie
beide über alles lieben – Deinen Gott und Deine Heimat!

Wenn Du Deine Heimat über alles liebst, dann stehe auch treu zu ihr in Sprache
und Brauchtum!

Heimatliebe fordert Heimattreue!

\part{Anhang}

\section{Quellenangaben}

\begin{itemize}
  \item Pfarrarchiv
  \item Gemeindearchiv
  \item Aichinger: „Metten und seine Umgebung“
\end{itemize}

\section{Anmerkungen}

\end{document}
