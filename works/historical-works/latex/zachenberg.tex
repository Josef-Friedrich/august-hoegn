\documentclass[12pt,a4pager]{book}

\usepackage[ngerman]{babel}
\usepackage{fontspec}
\usepackage[sf]{titlesec}
\usepackage{libertine}
\usepackage{longtable}
\usepackage{paralist}
\renewcommand{\labelitemi}{$-$}
\renewcommand{\labelitemii}{$-$}

\renewenvironment{quote}
               {\list{}{\itshape\rightmargin\leftmargin}%
                \item\relax}
               {\endlist}


%\titleformat{⟨command⟩}[⟨shape⟩]
% {⟨format⟩}
% {⟨label⟩}
% {⟨sep⟩}
% {⟨before-code⟩}
% [⟨after-code⟩]

\titleformat{\chapter}[display]
  {\sffamily\bfseries\Large}
  {\filright\Huge\thechapter}
  {1ex}
  {\titlerule\vspace{1ex}\filleft}
  [\vspace{1ex}\titlerule]

  \renewcommand*\descriptionlabel[1]{\hspace\labelsep
                                \sffamily\bfseries #1}


\author{August Högn}
\title{Geschichte von Zachenberg}
\date{1951}

\begin{document}

\maketitle

Mein Dank gilt:

Herrn Pfarrer Meier, Lotte Freisinger,

Textgrundlage:

Abschrift von Pfarrer Reicheneder aus der

Reicheneder-Chronik unter der Rubrik „Zachenberg“

\tableofcontents

\newpage

Projekt August Högn Geschichtswerk

Ruhmannsfelden, 2003

1. Auflage zu 10 Stück

AUGUST HÖGN

1878 1961


GESCHICHTE VON

ZACHENBERG




EDITIERT VON

JOSEF FRIEDRICH 2003

GESCHICHTE VON ZACHENBERG

AUGUST HÖGN

\part{Vorwort}

Die vorliegende Arbeit über „die Geschichte der Gemeindeflur Zachenberg“ erhebt
weder Anspruch auf Vollständigkeit, noch wolle das Geschriebene vom
wissenschaftlichen Standpunkte her beurteilt werden. Mit dem vorliegenden
Büchlein will ja nur der Anfang gemacht sein, allen denen, die sich für die
Geschichte der Gemeinde Zachenberg interessieren, einmal in zusammenhängender
Form über die früheren Zeiten und Geschehnisse in der Gemeinde Zachenberg etwas
zu schildern und gleichzeitig alle diejenigen aufzufordern, das bisher
Geschriebene zu ergänzen und zu vervollständigen, wenn dies möglich sein sollte.

Im Übrigen will das Büchlein den Heimatgedanken wecken durch Kennenlernen des
Unbekannten in der Heimatflur, durch Wiederaufrütteln des einmal Gewesenen, aber
längst Vergessenen. Das Büchlein will die Liebe zur Heimat und die Achtung vor
derselben wecken und beleben. „Ehre, achte und liebe deine Heimat!“

\part{Erklärung}

In der Abhandlung über die Geschichte der Gemeinde Zachenberg kommen die
Ausdrücke „Zehent“ und „Gilt“ häufig vor. Es waren dies grundherrschaftliche
Leistungen. Die Grundherren konnten von ihren Untertanen nach Herkommen oder
Bedarf Dienste verlangen und zwar Fronden oder Arbeitsdienst und Scharwerk oder
Hand– und Spanndienste. Dazu gehörte vor allem Mithilfe auf dem Gutshofe in der
Erntezeit, oder auch beim Anbauen im Herbst und im Frühjahr, Holzfuhren und
Holzarbeit im Winter, Zehentfuhren, Treiben bei Jagden und dergleichenii. Die
grundherrschaftlichen Abgaben waren vor allem die jährliche Gilt und Stift. Die
Gült oder Gilt ist eine verhältnismäßig geringe Abhabe in Geld, die meist in
zwei Raten an Georgi und Michaeli bezahlt werden musste. Die Stift bestand in
einer jährlichen genau bestimmten Abgabe von Leghühnern („Fastnachtshuhn“),
Eiern, Schmalz und Käse. Letztere wurde schon lange in Geld („Käsegeld“)
verlangt. Von manchen Höfen auch Wachs, Honig oder Fische. Schwerer war das
sogenannte Laudemium oder der Handlohn, das bei jeder Besitzveränderung
(Todesfall, Übergabe, Verkauf) bezahlt werden musste und bei den kirchlichen
Grundherren 5% des Schätzwertes betrug (bei Verkauf) und bei Todesfall oder
Übergabe 7 1/2 %. Weltliche Grundherren verlangten mehr (10%, oft 20%). Der Zehent
war anfangs eine rein kirchliche Abgabe. Im Bezirk Viechtach gehörte jede 10.
Getreidegarbe dem Pfarrer, jede 20. dem Kloster Oberalteich, die 30. Garbe einer
Kirche, Spital oder auch einem weltlichen Herrn. Außer diesem sogenannten
Großzehent von Getreide gab es den Kleinzehent vor allem von Flachs in der
Pfarrei Geierstal, wozu ja auch noch bis 1803 Ruhmannsfelden und March gehörten.
Das war eine Vereinbarung, dass jeder ganze Hof 2 Pfund, der halbe Hof 1 1/2
Pfund, die Sölde 1 Pfund gehechelten Flachs geben sollte. Als Blutzehent erhielt
der Pfarrer von Geierstal von jedem ganzen und halben Hof und der Bausölde
(Viertelhof) 2 Hennen oder Hähnchen, von einem Grassöldner (Achtelhof) 1 Henne.

In dieser Abhandlung ist auch von ganzen, halben, Viertel-, Achtel-,
Sechzehntel- und Zweiunddreißigstelhöfen. Die ganzen Höfe waren die ganz großen
Höfe. Die Größe des Wiesen– und Waldbesitzes spielte dabei keine Rolle. Der Wald
hatte gar keinen besonderen Wert. Die Wälder wurden nur niedergebrannt und die
Asche an die Seifensieder verkauft. Die halben Höfe waren die Huben (daher der
Name Huber). Die Viertelhöfe waren die Lehen (Name: Lehner). Die Achtelhöfe
waren die Sölden (Name: Söldner). Es gab Bausölden und Grassödlen. Die
Sechzehntel- und Zweiunddreißigstelhöfe hießen Leerhäusl, waren Tagwerkerhäuser
oder Handwerkerhäuser, ohne Grundbesitz. Dieser „Hoffuß“ durfte bis 1805 nicht
verändert werden.

Der Inhalt der dieser Abhandlung beigefügten Erzählungen stammt von alten Leuten
aus der Gemeinde Zachenberg.iii

\chapter{Geschichtliches über die ganze Gemeindeflur Zachenberg}

\section{Der Nordgau im Besitz des Kloster Metten}

Die Siedler, die sich auf dem rechtsseitigen Donauufer ansiedelten, haben sich
derartig übervölkert, dass wenigstens ein Teil dieser Siedler nach neuen
Wohnplätzen suchen musste. Was lag da näher, als dass die Leute dieser Sippen,
die ja hauptsächlich Ackerbauern, Jäger, Fischer und Händler waren, über das
linke Donauufer in den ihnen noch vollständig unbekannten Donaugau und
Schweinachgau und darüber hinaus bis in den Nordwald vordrangen. Dieses Gebiet
war bis in die Zeit um 800 herum noch ein fast unbewohntes Waldgebiet. Kaiser
Karl der Große hat damals dem Kloster Metten ein großes, wenn auch zur damaligen
Zeit wertloses Gebiet im Vorwalde, das ein Teil des Nordwaldes war, als
Schenkung vermacht. Dieser Besitz war verbunden mit der mühevollen Rodung dieser
Waldwildnis und mit der Herstellung von Wegen zwischen Rodungsgebiet und dem
Kloster Metten.

In dieser Zeit entstanden, wohl zuerst an den sonnigen Hängen und auf den
trockenen Plätzen, die Siedlungen mit den „dorf“-Namen (Patersdorf, Fratersdorf,
Lämmersdorf, Wandldorf). Das „Dorf“ bildete damals nur einen einzigen Hof, der
Eigentum des Klosters Metten war und von einem „Meier“ verwaltet wurde. Freilich
wurden im Laufe der Zeit zu diesem einem Hof zwei oder mehrere Anwesen dazu
gebaut, sodass aus dem einen Hof dann ein Weiler oder ein ganzes Dorf entstand.
Umgekehrt konnte ein solcher Meierhof auch Einzelhof bleiben, wenn eine weitere
Bebauung des Geländes nicht möglich war. Der Name eines solchen Meierhofes wurde
dann später, wohl im 14. Jahrhundert in „hof“ umbenannt, z. B. Wandldorf in
Wandlhof, Rugendorf in Rugenhof (Hof eines Rugo).

Die Gemeindeflur Zachenberg, 2731 ha = 8015,84 Tagwerk groß, erstreckt sich
zwischen Ruhmannsfelden und Regen im bayerischen Wald vom Hochbühl bis zur
Hausermühle unterhalb Triefenried, verläuft zu beiden Seiten des Wandlbaches,
der von Osten nach Westen fließend in die Teisnach mündet, als ein schmaler
Streifen zu beiden Seiten der Staatsbahnlinie Gotteszell-Triefenried und wird in
zwei Hälften eingeteilt, in die obere Hälfte und in eine untere Hälfte.

Die erste Hälfte ist unwirtliches Berg-, Holz-, Stein- und Ödland und liegt auf
der Schattenseite. Die zweite Hälfte liegt auf der Sonnen beschienen Seite und
ist in seiner landschaftlichen Struktur und seiner Bodenbeschaffenheit nach mehr
für Acker- und Wiesenland geeignet. Daraus ist ersichtlich, dass von diesen zwei
Gemeindehälften schon bei der anfänglichen Besiedlung der Gemeindeflur
Zachenberg die untere oder nördliche Gemeindehälfte zur Ansiedlung bevorzugt
wurde. Hier hatte das Kloster Metten leichtere Kolonisationsarbeit als auf der
anderen Seite der Gemeindeflur Zachenberg. Das Gelände ist auf dieser Seite
nicht steil, mehr flach, wellig, sonnig, ist nach Süden und Westen leicht
abfallend und hat an seinen Hängen einen sandig, lehmigen Boden, der vorzüglich
geeignet war als Ackerboden, als Wiese und Weide, zum Flachsbau, zum Garten-,
Obst- und Weinbau. Hier entstanden wohl als die beiden ersten Meierhöfe in
dieser Gegend Patosdorf (= Patersdorf) und Fatosdorf (= Fratersdorf).

In diesem Rodungsgebiet der unteren Gemeindehälfte waren noch zwei andere
Musterhöfe des Klosters Metten, nämlich Lampertsdorf (=Lämmersdorf) und Wantila
oder Wantilentorf (=Wandlhof).

Der Wandlbach (Wandl = Grenze) war der Grenzbach für das Rodungsgebiet des
Klosters Metten in dieser Gegen und hier im Mettener Grenzland schieden sich
auch zwei Gaue, der Donaugau und der Schweinachgau. Südlich dieses Grenzbaches,
also in der oberen Gemeindehälfte der Gemeindeflur Zachenberg, hat das Kloster
Metten nicht gerodet, was auch daraus zu ersehen ist, dass in diesem Teil der
Gemeindeflur Zachenberg nicht ein einziger „dorf“-Name sich vorfindet. Das
Kloster Metten übte seine Kolonisationstätigkeit auf der nördlichen Seite des
Wandlbaches aus, um Fratersdorf, Lämmersdorf, Wandlhof herum, also am sonnigen
Rücken des Südabhang des Bergrückens, der sich von Lämmersdorf bis Lobetsried
herunterzieht und da an das Rodungsgebiet des damaligen Klosters Rinchnach, das
zum Kloster Niederalteich gehörte, angrenzte.

Auf der Höhe dieses Bergrückens war ein Hochmoor zwischen Wolfsberg, Eckersberg
und der Weilnhöhe und ein mit viel Steingeröll und Steinriegeln versehener
Hochwald (Rabenholz). Dieses Gelände wurde von der Siedlung Eckersberg aus
gerodet. Diese Siedlung dürfte unmittelbar nach Fratersdorf entstanden sein, da
sie urkundlich bis in das 9. Jahrhundert zurückreicht.

Dieser untere, nördliche Teil der Gemeindeflur Zachenberg war also in der Zeit
von Anfang des 9. Jahrhunderts bis 10. Jahrhunderts schon besiedelt, wenn auch
nur in seinem westlichen Teil (um Fratersdorf und  Lämmersdorf herum) und wenn
auch nur von den wenigen Bewohnern der einzelnen vom Kloster Metten begründeten
Meierhöfe.

Diese nachkarolingisch-klösterliche Kolonisationstätigkeit sollte bald von einer
anderen Wirtschaftsform abgelöst werden. Die Hofverwalter auf den früheren
klösterlichen Meierhöfen hatten ihr Amt als Hof-“meier“v zu einer Art Herrschaft
ausgebaut. Sie oblagen mehr den Aufträgen und Beschäftigungen des Adels als
ihrer Klosterherren und versäumten mit Jagd und Spiel die Wirtschaftsführung.
Dazu kamen die dauernden Ungarneinfälle in Bayern (955 Ungarnschlacht auf dem
Lechfelde, Ulrichsberger-Kirchlein), wodurch einerseits die Macht der Ritter
gestärkt und andererseits der wirtschaftliche Einfluss der Klöster verdrängt
wurde.

\section{Der Nordgau im Besitz des Herzogs Arnulf des Bösen}

So verlor das Kloster Metten durch die Säkularisationen allmählich auch im
Nordgau die Früchte seines Fleißes und die Rodung und der Besitz im Nordgau
wurde von dieser Zeit an eine Angelegenheit der weltlich-ritterlichen Seite.
Herzog Arnulf der Böse (911 bis 937) hatte dem Kloster Metten den ganzen Wald
enteignet. Zur Befestigung seiner Macht, zur Behauptung seiner Unabhängigkeit
und zur Durchführung seiner Verteidigungspläne gegen die Ungarn vergab er die in
den Ungarnkriegen verwüsteten Klostergüter als herrenloses Gut an ihm ergebene
Lehensleute. Wegen der Säkularisation der Klostergüter legten die Mönche dem
Herzog Arnulf den Beinamen der „Böse“ bei.

\section{Unter Graf Aswin von Bogen entstandene Siedlungen}

Nach den Arnulfingern waren die Babenberger, die zugleich auch Markgrafen von
Österreich waren, Herren des Donaugaues. Ihnen gehörte damals der Donaugau, also
auch das Gebiet nördlich der Donau mit dem Nordwald, das Gebiet der Gemeindeflur
Zachenberg mit inbegriffen. Die Nachfolger der Babenberger wurden dann um 1080
herum die Grafen von Bogen. Der kinderreiche Graf Hartwig von Bogen übergab das
Nordwaldgebiet seinem Sohne Aswin. Dieser war in der Donauebene von Wörth bis
Deggendorf, im Wald bis zum schwarzen Regen und dessen Seitentälern sehr
begütert und vergab deshalb diese Besitzungen zum Teil als Lehen zum Teil als
Schenkungen an seine Ministerialen.

Man unterscheidet zwei Arten von diesen Ministerialen, nämlich die Adeligen
(Nobiles) und die Dienstmannen (Ministeriales). Diese letzteren verwalteten den
gräflichen Grundbesitz und leisteten vor allem Heeresdienst. Die Grafen erbauten
für ihre Ministerialen, oder diese auch fürvi sich selbst, Wirtschaftsgebäude
(Hof mit Stallung und Stadel) und bauten auch manchmal einen Burgturm, kein
Schloss, sondern einen Wehrturm aus Lehm und Findlingssteinen dazu (Haus
genannt, siehe die Burg Linden, Neunussberg). Ministerialen der Grafen von Bogen
waren die Inhaber der Burgen Nussberg, Linden, Ruhmannsfelden, Weißenstein, es
waren die Nussberger, die Degenberger, die Pfellinger.

Von diesen Ministerialen wurden die Rodung ihres Gebietes und die
Bewirtschaftung ihrer Siedlung geleitet und geführt. Da galt es nun zunächst die
ausgedehnten Gemarkungen zwischen den einzelnen „dorf“-Orten zu kultivieren und
dann zu besiedeln. Diese Siedlungen erhielten dabei die Bezeichnung „ried“ und
diese Namen waren, wie die „dorf“-Namen mit einem Personennamen oder Eigennamen
zusammengesetzt. So entstanden

bei Patersdorf Zuckenried = Siginenried,

bei Fratersdorf Kaickenried = Haccoried,

bei Lämmersdorf Giggenried = Gundachazried

und, nachdem in Klessing (Klepsing) der Weg nach Osten durch Wegbrennen des
Urwalds frei gemacht wurde, die Siedlungen

Lobetsried = Luipolfsried und Triefenried = Trunolfsried.

In dieser also im Halbbogen um March (Moar) herum, nördlich an Triefenried
vorüber, hinunter zur Hausermühle und Furth, hinauf auf den 5 Bartenstein, der
Habischrieder Gemeindegrenze entlang, hinüber zum Bocksruck, der Grenze der
Gemeinde Bergern folgend, auf dieser ganzen Strecke dürfte die heutige äußerst
unregelmäßige, verzwickte Gemeindegrenze der Gemeinde Zachenberg dieselbe Grenze
sein, wie die einstige Rodungsgrenze der Rodungsgebiete der Klöster Metten,
Rinchnach, und Niederalteich.

Man kann beobachten, dass sich die Bogener Grafen bei der Neugründung dieser
Siedlungen wenig um die Rechte der früheren geistlichen Grundherrschaften
kümmerten, nachdem sie ersten „ried“-Siedlungen unmittelbar in der Nähe der
früheren vom Kloster Metten gegründeten „dorf“-Orte anlegten. Das können wir
auch daraus ersehen, dass sie ihre Dienstmannen in die dem Kloster Metten
gehörenden Waldungen einfallen ließen, den Wandlbach (Grenze) überschritten, um
auch südlich des Wandlbaches, in der anderen Hälfte der Gemeindeflur Zachenberg
durch die Rodungsleute neues Siedlungsland und neue Siedlungsorte zu erwerben.
So entstanden wohl im 11. und 12. Jahrhundert dort die „ried“-Orte:

Gottlesried = Godilesried,

Weichselsried = Wigilesried,

Hasmannsried = Hosnodsried, und im südlichen Teil dieses Gebietes

Köckersried = Coteschalksried.

Die mit einem Personen- oder Eigennamen zusammengesetzten „ried“-Namen sind
lauter echte „ried“-Namen und die Gründer dieser „ried“-Siedlungen gehören alle
einer großen Grundherrschaft an. Das jüngere Alter der „ried“-Orte gegenüber den
“dorf“-Orten ergibt sich auch daraus, dass diese „ried“-Orte im Gegensatz zu den
viel älteren Dorfsiedlungen nicht an der Verkehrsstraße sondern vielfach in
ungünstiger Lage zu finden sind. Kleinried ist kein echter „ried“-Name, weil er
nicht mit einem Personennamen zusammengesetzt ist. Kleinried hieß ursprünglich
“Gnänried“. Gnän bedeutete im Althochdeutschenvii klein.

Den beiden Siedlungsperioden der „dorf“- und „ried“-Orte gehört auch die
Gründung der „berg“- und „burg“- und „bach“-Siedlungen an, und zwar wiederum in
der unteren Gemeindehälfte zuerst, wie dort die „dorf“-Orte und in der oberen
Gemeindehälfte die „ried“-Orte. Zur Zeit der Gründung der Siedlung Fratersdorf,
also schon im 9. Jahrhundert, entstand östlich von Fraterdorf die Siedlung
Eckersberg = Ekkirisbuch, urkundlich im Jahre 882, südlich von Fratersdorf die
Siedlung Wolfsberg = Wolfichosperg und westlich von Fratersdorf Vorder- und
Hinterdietzberg = die Siedlung eines Tirolds. Zwischen Fratersdorf und Dietzberg
entstand, allerdings viel später, die Siedlung Poitmannsgrub. Im östlichen Teil
der Gemeindeflur Zachenberg entstand neben der Siedlung Gottlesried die Siedlung
Göttleinsberg = Gözeleinsperg und im südlichsten Teil der Gemeinde die Siedlung
Zachenberg = Zaccoperg und als weitere Siedlungen in dieser Gegend Bruckberg =
Bruckhofperg, Ochsenberg = Weideberg, ebenso Gaißberg.

Zu den ältesten Siedlungen zählen auch die im Tal gelegenen „bach“- und
“hof“-Orte, nämlich Zierbach (Cyrwa), Auerbach und Brumbach. „hof“-Orte sind
hier Wandlhof, Bruckhof, Auhof. Im Zusammenhang mit den Hofgründungen in dem
ganzen hiesigen Waldgebiete stand in jener Zeit die starke Vermehrung der
Mühlen. Fast jeder große Hof, der an einem fließenden Wasser oder in dessen Nähe
war, erhielt seine eigene Mühle, z. B. Bruckhof = Bruckmühle, Wandlhof =
Wandlmühle. Der große Wasserreichtum in den früheren Jahrhunderten hat zur
Anlage der zahlreichen Mühlen geführt. In den alten Zachenberger Urkunden ist
vielfach die Rede von den „Viermühlen“ des Wandlbachtales. Diese waren die
Hausermühle, die Reisachmühle, die Stömmermühle und die Wandlmühle. Die
letzteren 2 Mühlen sind die älteren von den 4 genannten Mühlen. Die späteren
Mühlen hatten wie die Tavernen nur einen kleinen Hoffuß, z. B. die Reisachmühle
(Reiser, Reisig, Busch) und die Hausermühle (Hauser = ein Familienname).

Eine alte Siedlung ganz im Osten der Gemeindeflur Zachenberg am Fuße des
Bartenstein ist Furth, eine seichte Stelle des Habischriederbaches, der von
Habischried herunter Zeuserbach, dann von Hausermühle weg Hausermühlbach und von
der Pommetsauermühle weg Friedbach heißt, und der bei der Raithmühle bei Regen
in die Ohe mündet. Zu den späteren Gründungen in dieser Gegend gehören noch
Gaisruck, Leuthen (Leitten) und Kirchweg (Kirchenweg).

Vom Wandlhof aus rodete eine Sippe unter Führung eines Hadubet (abgekürzt: Haw =
Hawenried oder Hafenried, auch Hawa = Hawaleuthen = Haberleuthen). Um das Jahr
1100 herum entstanden die beiden Siedlungen Viechtach und Regen. Viechtach wurde
im Jahre 1272 von dem bayerischen Herzog Ludwig dem Strengen verkauft und bekam
um das 1360 herum die Marktgerechtigkeit. Viechtach war Sitz eines Gerichtes.
Die ganze Gemeinde Zachenberg gehörte damals zu diesem Landgericht. Die
Ortschaft Regen dürfte etwa um das Jahr 1070 vom Kloster Rinchnach/Niederalteich
aus gegründet worden sein und besaß wie ja auch Böbrach das Schergenamt.

\section{Zachenberg unter der Herrschaft des Kloster Gotteszell}

1242 starb das mächtige Grafengeschlecht der Grafen von Bogen aus. Ludmilla, die
junge Witwe des Grafen Adalbert von Bogen, heiratete nach dessen Tod den
bayerischen Herzog Ludwig I. den Kelheimer, der 1231 ermordet wurde. Dieser
erbte nach Ableben des letzten Bogener Grafen den ganzen Bogener Besitz und
damit auch den Wald- und Siedlungsbesitz im Donaugau und auch im Nordgau. Die
Wittelsbacher gaben die Grafschaften, von denen sie außer Bogen noch mehrere
bekamen, nicht mehr durch Belehnung in den Erbbesitz anderer Adeliger, sondern
organisierten ihren viel größer gewordenen Staat durch Einteilung in 4
Viztumämter und diese wieder in kleinere Ämter. An der Spitze dieser Ämter stand
ein Beamter. Viechtach stand damals unter dem Viztumamte Straubing.

Eine Rodungs- und Siedlungstätigkeit war hier für die Herzöge nicht mehr
veranlasst, da ja die Besiedlung ohnehin schon sehr dicht war. Wohl entstanden
mit zunehmender Bevölkerungsdichte in den einzelnen Siedlungen neue
Siedlungshäuser, aber neue Siedlungsorte wurden nicht gegründet. Daran änderte
auch die Entstehung des Klosters Gotteszell nichts. 1205 hatte Graf Heinrich von
Pfelling dem Zisterzienserkloster Aldersbach seine Villa Droßlach geschenkt mit
der Bedingung, diesen Ort mit 2 Priestern zu besetzen, bis die Einkünfte sich
gemehrt hätten.

Nachdem 1260 auch Pfelling und andere Güter, sowie der Zehent in Geiersthal
hinzugekommen sind, begann man 1285viii das Kloster Gotteszell zu bauen. 1297
waren in diesem Kloster schon 13 Mönche. An Georgi 1320 wurde das Kloster
Gotteszell zur Prälatur erhoben und erlebte einen immer größeren Aufschwung. Die
Äbte der folgenden Zeit waren ständig bemüht durch Gütererwerbungen und
Zehentankäufe, durch Privilegien und Jahrtagsstiftungen der Nussberger und
Degenberger und anderer die Einkünfte des Klosters Gotteszell zu mehren.

Zum Roden gab es nicht mehr viel. Die Rodungstätigkeit der Zisterzienser in
Gotteszell unterschied sich nicht von der Rodungstätigkeit der privaten
Grundbesitzer dieser Zeit. Sie erstreckte sich auf verhältnismäßig kleine
Waldteile und diente in erster Linie der Gewinnung von neuen Fluren in nächster
Nähe des Klosters, nicht aber von Siedlungsland. Dabei wurden zunächst die
Wiesen verbessert und die Waldböden für den Weidebetrieb aufgeschlossen. Daraus
ist ersichtlich, dass im Droßlacherhof hauptsächlich Viehzucht betrieben wurde
und dass der Getreidebau damals noch wenig entwickelt war. Das kann man auch
daraus entnehmen, dass mit wenigen Ausnahmen in dieser Zeit eine Rede wäre von
Getreidegülten, höchstens von größeren Gütern in fruchtbarer Lage, z. B.
Zuckenried und da auch nur von der Ablieferung von Hafer.

Im Übrigen beschränkt sich die landwirtschaftliche Kulturarbeit des Klosters
Gotteszell auf die nächste Umgebung des Klosterbesitzes. Nicht einmal der an das
Klosterhofgut unmittelbar angrenzende, im Jahre 1385 erworbene Auhof wurde mit
der Eigenwirtschaft vereinigt, sondern einem zinspflichtigen Nutznießer
überlassen. Um die Ausdehnung des selbst bewirtschafteten Eigenbesitzes handelte
es sich beim Kloster Gotteszell nicht, wohl aber um die Festigung und
Erweiterung der finanziellen Unterlage des Klosterbestandes durch Erwerbung
gült- und zinspflichtiger Güter und Erwerbung verbindlicher Rechte. Reichere
Schenkungen und Stiftungen erhielt das Kloster Gotteszell nicht. Nur die mit
fast sehr bescheidenem Besitz begüterten Bewohner der ganzen hiesigen Gegend
bewiesen vorzugsweise ihre Liebe und Anhänglichkeit zu dem neuen Kloster
Gotteszell und seinem Gotteshaus durch kleinere Schenkungen und Stiftungen zu
religiösen Zwecken:

1430 stiftete ein Peter Chatzfeller, Pfründner zu Gotteszell einen Jahrtag, in
dem er für das Kloster ein Gut zu Chekleinsried (Köckersried), worauf die
Gredlerin saß, kaufte. 1423 stiftete ein Deggendorfer Bürger Pranstetter
(Brandstetter) eine Wochenmesse durch Schenkung eines Hofes zu Chekleinsried.
1441 wendete ein Deggendorfer Bürger Girgl der Thäneschl (Deuschl) dem Kloster 2
Lehen in Gnänried zur Stiftung eines Jahrtages zu. In der Mitte des 15.
Jahrhunderts erfolgte die Aufnahme weltlicher Personen in das Kloster Gotteszell
als sogenannte Pfründner. Als erste solche Pfründner werden uns genannt: Ulrich
der Murr und seine Ehefrau, die laut Urkunde dem Abt Andreas und dem Kloster
Gotteszell ihre Sölde zu Chrötzenried (Köckersried) gegen lebenslänglichen
Unterhalt von Seiten des Klosters verschrieben haben.

Zu den im 14. und 15. Jahrhundert vom Kloster Gotteszell vorgenommenen Guts-,
und Zehentankäufen gehörten vor allem der Kauf des Auhofes im Jahre 1388 und der
Kauf eines Hofes in Frättersdorf (Fraterdorf). Zehenterwerbe erfolgten in
Zachenberg und im Jahre 1395 in Furth. Das Kloster Gotteszeller Salbuch von 1400
führt in der Gemeindeflur Zachenberg folgenden Klosterbesitz auf:



I. Höfe:



Der Auhof und 1 Hof zu Chreklensried (Köckersried)

2 Höfe in Clepsing (Klessing)

2 Höfe in Cirberg (Zierbach)

3 Sölden in Fratersdorf

2 Sölden in Ginkenried (Giggenried)

1 Hof in Götzleinsperg (Göttleinsberg)

1 Hof in Hasmannsried

6 Höfe und Sölden in Lembertsdorf (Lämmersdorf)

1 Hof in Leithen (Leuthen) und

1 Hof in Labensried (Lobertsried)



II. Zehentrechte:



Auerbach

Auhof

Bruckhof

Chrekliensried

Ekkartzberg

Fratersdorf

Furth

Gukenried

Götzlensperg

Godensried (Gottlesried)

Hausenried (Hafenried)

Hasmannsried

Hausermühle

Reisachmühle

Wandelmühle

Chlessling

Pertelsleuthen (Haberleuthen),

Lämmersdorf

Leyten

Labensried

Muschenried

Peuchtmannsgrub

Driefenried

Wolfsberg

Weytenried (Weichselsried),

Czachenperg

Cyrbachix





Die vorgenannten Güter in der Gemeindeflur Zachenberg waren mir Scharwerkdienst,
mit Geldleistungen (Stiftspfennigen) und mit Abgabe von Erzeugnissen aus der
Landwirtschaft (Viehzucht) belastet (Eier, Käse, Hühner). Dazu kam noch die
Abgabe von Läutkorn oder Haber an den Mesner der Kirche in Ruhmannsfelden und in
March.

Das Kloster Gotteszell war seines Fortbestandes wegen angewiesen auf die
Einnahmen aus Kirche, aus Besitz und aus Zehent. Dadurch hat es aber auch seinen
Machtbereich ausgedehnt von Köckersried bis hinunter zur Hausermühle und war
zugleich auch der gesamte geistige Mittelpunkt für dieses ganze Gebiet.

Nach dem Güterverzeichnisse aus dem Hauptstaatsarchiv München von den Jahren
1668 und 1752/60 umfasste damals die eigentliche Hofmark Gotteszell 11 ganze, 4
halbe, 2 Viertel-, 4 Achtel- und 4 Sechszehntelhöfe, nämlich:



Gotteszell

Georg Tax der Jüngere daselbst auf dem Dengler-Hof (1752 noch Michael
Kramhöller)

ganzer Hof



Johann Tax der Ältere, daselbst auf dem Lippl-Gut

halber Hof



Wolfgang Lippl daselbst auf dem Ensten-Gut

halber Hof



Christoph Dax zu Labersried (Lobetsried) auf dem Kuchler-Gut. Derselbe besitzt
daselbst auch noch die Taxensölde.

ganzer Hof



Georg Zeitlhöfer, daselbst auf dem Ketterl-Hof

ganzer Hof



Jakob Grozer auf dem Oberhof zu Zierbach

ganzer Hof



Georg Zellner auf dem Steinbauern-Hof zu Leithen (1752 Georg Steinbauer)
derselbe hat daselbst auch das Fiedler-Gut.

ganzer Hof halber Hof



Jakob Pichelmayer auf dem Wandlhof zu Hafenried (1752 Johann Geiß)

ganzer Hof



Andre Paumgartner auf der Reisermühle (1752 Georg Groz)

Achtelhof



Michel Limpeck auf dem Prinkhof (Bruckhof)

ganzer Hof



Johann Limpeck auf dem Prunner-Hof zu Lemmersdorf

ganzer Hof



Michel Edenhofer auf der Kappenberger Sölde zu Burggrafenried

Viertelhof



Sämtliche 15 Güter gehörten unmittelbar zum Kloster Gotteszell. Auch der ganze
Markt Ruhmannsfelden gehörte mit der Grundherrschaft zur Klosterhofmark
Gotteszell.



Zum Pfarrgotteshaus Ruhmannsfelden und Hofmark Gotteszell gehörten damals:



Muschenried

Johann Greill auf der Greil-Söldex

Achtelhof

Zachenberg

Georg Schlögl auf dem Rosenstingl-Gut

ganzer Hof



Paul Gierster auf dem Ernsten-Gut

halber Hof



Georg Pfeffer auf dem Steiniger-Haus, Tagwercher

Sechzehntelhof



Georg Prunnbauer auf dem Fliendl-Haus, Tagwercher

Sechzehntelhof

Weichselsried

Johann Fink auf dem Finken-Hof

ganzer Hof



Zur Bruderschaft „Corporis Christi“ in Ruhmannsfelden und Hofmark Gotteszell:



Zachenberg

Georg Gierster auf dem Schwarzenbergerhaus zu Zachenberg

Sechzehntelhof



Wieder direkt zum Kloster Gotteszell gehörten:



Zachenberg

Adam Krämfl auf dem Krämpfl-Haus zu Zachenberg, Tagwercher

Sechzehntelhof



Adam Hof zu Sohl, auf der Hofsölden

Achtelhof

Taffertsried

Jakob Kauschinger zu Taffertsried auf dem Achaz-Häusl

Sechzehntelhof



Diese Güter gehörten also mit der Grundherrschaft und Gerichtsbarkeit zum
Kloster Gotteszell. Außerdem gehörten noch weitere 13 ganze, 41 halbe, 16
Viertel- und 5 Achtelhöfe nur mit der Grundherrschaft zum Kloster Gotteszell mit
der Gerichtsherrschaft aber zum Gericht Viechtach bzw. Linden. Davon in der
jetzigen Gemeinde Zachenberg (1752):



Gickenried

Thomas Zislsperger

Viertelhof

Lämmersdorf

Joseph Stadler

ganzer Hof



Michael Augustin

halber Hof



Christoph Pfeffer

halber Hof



Michael Zislsperger

halber Hof



Hans Stadler

halber und Viertelhof

Fratersdorf

Jakob Stadler

halber Hof

Poitmannsgrub

Paulus Achatz

halber Hof



Joseph Paur

halber Hof

Göttleinsberg

Hans Cramhöller

halber Hof, hat noch einen ganzen Hof, mit der Grundherrschaft zur Pfarrei
Geiersthal

Klessing

Hans Plötz

ganzer Hof

Zierbach

Michael Lorenz

ganzer Hof

Leuthen

Christoph Achatz

halber Hof

Hasmannsried

Christoph Achatz

halber Hof

Hinterleuthen (bei Hampermühl):

Hans Härtl

ganzer Hof

Zachenberg

Andre Prunner

ganzer Hof



Michael Achatz

ganzer Hof



Hans Aman

ganzer Hof



Michael Aman

ganzer Hof



Andre Pauer

Halber Hof und Sölde zum Kastenamt Viechtach



Christoph Löffler

halber Hof



Hans Pfeffer

halber Hof und einen Viertelhof



Andre Müller

halber Hof



Michael Kasparbauer

halber Hof



Georg Reithmeier

halber Hof

Köckersried

Georg Hacker

halber Hof



Wolf Hacker

halber Hof



Georg Kraus

halber Hof

Gnadenried

Michael Riedler

halber Hof

Auhof

Georg Kraus

ganzer Hof



Zum Kastenamt Viechtach gehörten 1752:



 Auerbach

Hans Sailer

ganzer Hof



Martin Schollenrieder

ganzer Hof



Gregor Henningen

halber Hof/Müller



Michael Trembl

halber Hof



Christoph Achatz

halber Hof

Gickenried

Hans Griendl

ganzer Hof



Hans Prunner

halber Hof



Hans Hueber

halber Hof

Haberleuthen

Christoph Steinbauer

ganzer Hof und Viertelhof zum Pfarrhof Geirsthal



Andre Trembl

ganzer Hof

Muschenried

Georg Riedler

halber Hof



Jakob Achatz

Viertelhof/Weber

Dietzberg

Paulus Peter

halber Hof



Paulus Kraus

halber Hof



Georg Limpeck

halber Hof

Frattersdorf

Adam Göstl

halber Hof, und Viertelhof zur Hofmark Geltolfing gehörig

Triefenried

Andre Koller

2 halbe Höfe und Wirtstaferne



Hans Ernst

halber Hof und halber Hof zum Gotteshaus

Regen



Hans Schreiner der Jüngere

halber Hof



Christoph Kramhöller

halber Hof



Jakob Hof

halber Hof /Weber



Hans Schreiner der Ältere

halber Hof

Köckersried

Michael Koller

ganzer Hof

Zachenberg

Andre Pauer

Viertelhof und halber Hof zum Kloster Gotteszell



Zum Kloster Oberalteich gehörte (mit der Grundherrschaft):



Gnadenried

Joseph Kraus

Viertelhof



Zum Kloster Windberg gehörte:



Köckersried

Andre Sailler

halber Hof



Zum Pfarrhof Viechtach gehörte:



Zachenberg

Georg Muhr

ganzer Hof



Mathias Sigl

ganzer Hof



Zum Pfarrhof Geiersthal gehörte:



Haberleuthen

Christoph Steinbauer

Viertelhof und 1 ganzer Hof zum Kasten Viechtach

Göttleinsberg

Hans Kramhöller

ganzer Hof und halber Hof zum Kloster Gotteszell

Triefenried

Sebastian Aichinger

halber Hof



Zur Kirche St. Johann in Regensburg gehörte:



Triefenried

Wolfgang Reister

halber Hof/Weber



Zum Pfarrhof Regen gehörte:



Muschenried

Michael Stadler

Viertelhof/Weber



Zur Pfarrkirche Regen gehörte:



Triefenried

Hans Ernst

halber Hof und halber Hof zum Kasten Viechtach



Zur Hofmark Geltolfing gehörte:



Frattersdorf

Adam Göstl

Viertelhof und halber Hof zum Kasten Viechtach



Urban Tremml

halber Hof



Zum Pflegegericht Linden gehörten 1752xi von der Hauptmannschaft Zuckenried
(Grundherrschaftliche Zugehörigkeit):



Gickenried

Joseph Mock, auf dem Meinzinger-Gütl, 1666 war auf dem Hof Wolf Mock, ebenso
1689

halber Hof

Eggersberg

Georg Probst, auf dem Penzkofer-Gut, 1666 war dort Jakob Treml, ebenso 1689

halber Hof

Wolfsberg

Michael Cramhöller auf dem Widenbauern-Hof, 1666 war dort Michael Achatz, ebenso
1689

ganzer Hof

\section{Die Aufhebung des Klosters Gotteszell}

1803 wurden die Bayerischen Klöster säkularisiert (aufgehoben). Am 24. März
1803, am Vorabend des Festes Maria Verkündigung, an dem denkwürdigen Tag des
großen Brandes des Klosters Gotteszell von 1629 17:30xii Uhr erschien der von
der in München gebildeten Spezialkommission für die Klösteraufhebung zur
Durchführung der Aufhebung des Klosters Gotteszell bestimmte Landrichter von
Viechtach, Ignaz von Schmidbauer, in den Klosterräumen, wo er dem eilig
versammelten Konvent von der landesherrlich beschlossenen Aufhebung des
Klosters, dessen ganzer Besitz vom 1. April ab an die kurfürstliche
Landesregierung überzugehen habe, Eröffnung gemacht.

.8 Die Pest in der Gemeindeflur Zachenberg*

Die Jahrhunderte die nun seit der Gründung des Klosters Gotteszell (1295) bis zu
seiner Aufhebung (1803) über das Land Bayern dahin gezogen waren, haben diesem
Land und seiner Bevölkerung mehrmals schlechte Zeiten gebracht.

Mord und Krieg, Hunger und Pest. So lesen wir, dass in der Mitte des 14.
Jahrhunderts (1354 bis 57) in der Gemeindeflur Zachenberg die Pest gehaust
hatte. Es war dies die indische Beulenpest, die von Italien hierherxiii
eingeschleppt wurde und von der kein Haus und keine Familie verschont wurden.
Ganze Familien wurden von dieser Seuche hinweggerafft. In dieser Zeit kamen von
auswärts böse Menschen, welche die Not und das Unglück, das die Pest über die
Bewohner der Gemeindeflur Zachenberg gebracht hatte, auszunützen suchten. Sie
raubten und plünderten in den menschenleeren Wohnungen und brannten die Häuser
der hilflosen Siedlungsleute nieder.

\section{Zachenberg zur Reformationszeit}

Im Jahre 1519 nach Beginn der Reformationszeit ging es auch hier drunten und
drüber. Bei diesen Meinungsverschiedenheiten kam es zu Tätlichkeiten und bei
einer in Ruhmannsfelden ausgebrochenen Unruhe ging das dem Kloster Gotteszell
gehörige Prälatenhaus in Ruhmannsfelden in Flammen auf. In der 2. Hälfte des 16.
Jahrhunderts brannte das Pfarrgotteshaus St. Laurentius zum wiederholten Male
ab. Da werden die Bewohner der Gemeindeflur Zachenberg sicher ihr Möglichstes
zum Wiederaufbau der Kirche beigetragen haben.

\section{Der 30-jährige Krieg in der Gemeinde Zachenberg}

Anfang des 17. Jahrhunderts (1618) begann der unselige 30-jährige Krieg. Da
hatten die Zachenberger Hofbesitzer Gespanne, Wägen, Stroh, Heu, Brot usw. zu
fiedern gehabt und werden von 4 den aus Regen und Cham zurückflutenden
schwedischen Horden genau so grausam und schimpflich behandelt worden sein
(Schwedentrunk), wie die Bewohner des ganzen übrigen Bezirkes Viechtach und weit
darüber hinaus (Schwedenloch bei Achslach). Nach dieser unglücklichen Kriegszeit
mag es auch in der Gemeinde Zachenberg furchtbar traurig ausgesehen haben, wenn
man in Betracht zieht, dass unmittelbar vor dem 30-jährigen Krieg (1613) zum
zweiten Mal in der ganzen Gemeindeflur Zachenberg die indische Beulenpest
gehaust hatte, die, wie das erste Mal, ungemein viele Opfer gefordert hatte,
wenn man außerdem bedenkt, dass das Geld und die Wertsachen gestohlen waren, die
Ställe und die Stadeln vollständig leer waren und Felder und Wiesen 30 Jahre
lang nicht mehr bewirtschaftet werden konnten. War schon die Kriegszeit selber
eine schreckliche Zeit für die Zachenberger, so war die Nachkriegszeit für sie
noch härter, denn da galt es nun erst wieder aufzubauen, die Arbeit wieder von
vorne anzufangen, Verdienst und Absatz zu suchen.

\section{Umschwung auf religiösem Gebiete}

In dieser seelischen Not und dem wirtschaftlichen Elend, in dem sich die Leute
in jener Zeit befanden, trat ein großer Umschwung auf religiösem Gebiete ein. An
Stelle der Auflehnung und der Unruhe, an Stelle von Unglauben und Aberglauben,
von Zauberei und Hexerei trat jetzt wieder Ruhe, echte Frömmigkeit und wahrer
Gottesglaube. Niedere, steinerne Pestkreuze wurden längs der Straße und Wege
aufgestellt und hölzerne Kapellen zu Ehren der 14 Nothelfer, der hl. Brigitta,
des hl. Florian und des hl. Sebastian errichtet, entweder im Walde an einsamer
Stelle (14 Nothelfer bei Eckersberg) oder an vielbefahrenen Wegen (Kapelle zum
sitzenden Herrgott bei Fratersdorf) oder in Mitten der Ortschaften (Kapelle in
Zachenberg). Die religiösen und wirtschaftlichen Verhältnisse haben sich
gänzlich geändert.

\section{Kriege und Seuchen im 18. Jahrhundert}

Schon nach einigen Jahrzehnten folgten dieser friedlichen und arbeitsreichen
Zeit wieder kriegerische Jahre, während derselben auch die Gemeindeflur
Zachenberg nicht verschont blieb von Quartierlasten, Plünderungen,
Brandschatzungen usw. Schwere Zeiten brachte der österreichische Erbfolgekrieg
für die Gemeindeflur Zachenberg in den Jahren 1742/45. Ganz abgesehen von der
großen Steuerlast als Folge des für Bayern unglücklich verlaufenen, spanischen
Erbfolgekrieges 1705/06 kamen die rücksichtslosen Kontributionen der Panduren im
österreichischen Erbfolgekrieg 1742/45. So heißt es in einem veröffentlichten
Auszug aus dem Tagebuch des Abtes Marian von Busch von Niederalteich, dass am
13. Oktober 1742 ein Streifkorps von 400 ungarischen Husaren in Ruhmannsfelden
eingerückt und von dort über Gotteszell bis Grafling vorgerückt sei und schwere
Kontributionsgelder eingetrieben habe. Ein Teil des Zachenberger Gebietes im
bayerischen Walde wurde durch einen am 15. und 16. Oktober 1742 stattgefundenen
Vormarsch von 4000 Soldaten durch das Wandlbachtal über Furth nach Regen hart
mitgenommen.

1762 herrschte auch in der Gemeindeflur Zachenberg zweimal hintereinander die
Viehseuche und im gleichen Jahre vernichtete ein starker Reif alles auf Feld und
Wiese. Wenn schon in vier aufeinander folgenden Jahren Hagelschäden alles total
zugrunde richteten, dann können wir leicht verstehen, wenn uns die Geschichte
von einer Hungersnot in ganz Bayern im Jahre 1770/71 berichtet, nach welcher der
bayerische Staat vom Ausland Getreide aufkaufen und zu verbilligten Preisen
verteilen musste.

\section{Abbruch der Kapelle in Folge der Säkularisation}

Nach der Aufhebung der Klöster in Bayern wurde der Abbruch der Kapellen
angeordnet, was in Gotteszell restlos gelang. Nur die Köckersrieder, sind dem
ebenfalls beabsichtigten Abbruch der dortigen Ortskapelle durch gewaltsames
Vertreiben der zum Abbruch eingetroffenen Arbeiter mutig entgegen getreten.
Köckersried, das zur Pfarrei Ruhmannsfelden gehörte, wurde 1806 in die Pfarrei
Gotteszell eingepfarrt.

\section{Zachenberg wird selbständige Gemeinde}

In früheren Jahrhunderten hatte das Kloster Gotteszell das amtliche Schriftwesen
mit der Siegelführung, die standesamtlichen Einträge und die Jurisdiktion
(niedere Gerichtsbarkeit) der Gemeinde Gotteszell, Ruhmannsfelden und
Zachenberg. Am 26. Mai 1818 gab der edle König Max I. dem Lande Bayern eine neue
Verfassung, dass damit das einzige größere Land war, das sich einer Verfassung
erfreuen konnte. Damit hat er den Staat ganz neu geordnet. Bayern bekam eine
neue Landeseinteilung in Kreise, die nach den bayerischen Flüssen benannt
wurden. So gehörte die Gemeindeflur Zachenberg damals zum Unterdonaukreis. Es
wurde die Rechtspflege neu geordnet, die Grundherrschaft aufgehoben, Hof und
Flur wurden das Eigentum des Besitzers. Es wurden große und kleine
Steuergemeinden geschaffen und den Gemeinden die Selbstverwaltung zugesprochen
(1. Gemeindeordnung). Somit entstand auch die selbständige Gemeinde Zachenberg
im Landkreis Viechtach. Wie für die Landwirtschaft, so wurden auch durch die
neue Verfassung 1818 das Gewerbe, die Ämter, die Wehrpflicht, Kirche und Schule
neu geordnet. Was die gemeindliche Selbstverwaltungsarbeit in damaliger Zeit
betrifft, so war sie allerdings anfänglich noch sehr primitiv. Die damaligen
Landbürgermeister haben das gemeindliche Schriftwesen entweder selbst in ihrer
eigenen Wohnung gefertigt oder dort machen lassen. Einen Akten - und
Bücherschrank gab es bei diesen Bürgermeistern noch nicht. Das scheinbar
unnötige Papierzeug verschwand im Ofen und auf die Erhaltung alter Akten wurde
keine Rücksicht genommen. So sind auch in der Gemeinde Zachenberg die alten
Akten gänzlich verschwunden. Bei der Säkularisation des Klosters, Gotteszell hat
man die alten und für die heutige Zeit recht wichtigen und notwendigen
Schriftstücke schubkarrenweise als Brennmaterial verkauft oder verschenkt.

\section{Ereignisse von 1845 bis 1873}

1845 wurde in Ruhmannsfelden das erste Schulhaus gebaut.

1844 wurde der alte Pfarrhof in Ruhmannsfelden abgebrochen und ein neuer gebaut.
Dabei kam es zu lang andauernden Streitigkeiten zwischen Pfarramt und
Pfarrangehörigen wegen der Hand- und Spanndienste.

1855 beklagten sich die Wirte wiederholt wegen des schlechten Bierabsatzes.
Joseph Schauer, lediger Dienstknecht von Poitmannsgrub wurde am 4.3.1858 im Wald
durch den Umsturz eines Baumes erschlagen.

Leopold Huber, lediger Geschirrhändler aus der Gemeinde Zachenberg, ist am 10.
Oktober 1861 früh morgens um 6 Uhr auf der Straße in Ruhmannsfelden gestorben.

Am 13.9.1868 wurde nach dem Anton Milter, Hilfslehrer in Ruhmannsfelden, lautxiv
Beschluss der königlichenxv Aufschlagseinnehmer August Wimmer von Ruhmannsfelden
als Gemeindeschreiber für die Gemeinde Zachenberg aufgestellt. Früher
unterzeichnete sich der Bürgermeister als Gemeindevorstand. Am 11. Juli 1869
unterschrieb sich zum ersten Male Schreiner als Bürgermeister der Gemeinde
Zachenberg. Die Gemeinderatssitzungen und Gemeindeversammlungen der Gemeinde
Zachenberg fanden zuerst beim Schusterbräu in Ruhmannsfelden und dann beim
Weißbräu in Ruhmannsfelden statt.

Vom 26. auf 27. Oktober 1870 war ein so gewaltiger Sturm, dass ganze Wälder
umgelegt wurden und kein Baum mehr davon stand.

Bei Kriegsausbruch 1870 sind auch aus der Gemeinde Zachenberg sehr viele Männer
zum Kriegsdienst einberufen worden, um für ihr Vaterland zu kämpfen. Es mussten
aber auch viele auf französischem Boden ihr Leben lassen. Am 11.3.1871 war in
der Pfarrkirche Ruhmannsfelden ein Trauergottesdienst für die Gefallenen im
deutsch-französischem Kriege 1870/71 und anschließend fand eine große
Friedensfeier statt.

Die Brücke über die Teisnach bei der Bruckmühle wurde laut Beschluss vom 12.
April 1871 neu hergestellt mit einem Kostenaufwand von 32 Gulden.

In einem Protokoll über eine Viktualien-Visitation, abgehalten in einem
Wirtshaus zu Zachenberg, heißt es unter anderem, dass die Glocke zum
“Polizeistundenläuten“ vorhanden ist.

Da der Gemeindeschreiber August Wimmer mit Tod abgegangen ist, wurde am 16.
Oktober 1873 der Schullehrer Raimund Schinagl von Ruhmannsfelden als
Gemeindeschreiber für die Gemeinde Zachenberg aufgestellt.

\section{Bau der Bahnlinie Plattling Eisenstein}

1871 wurde auch mit der Vermessung der Eisenbahnlinie Plattling-Deggendorf
begonnen und 1872 der Bau dieser Bahn über Deggendorf nach
Gotteszell-Zwiesel-Eisenstein beschlossen und genehmigt. Schon im September 1872
hatten die Ingeneure der königlich bayerischenxvi Ostbahngesellschaft mit der
Aussteckung der Bahnlinie Plattling-Eisenstein begonnen. Die Strecke wurde in 22
Lose eingeteilt. Am 25. Januar 1875 wurden die Lose zum Bau vergeben. Im Herbst
1877 war der Bahnbau bis Zwiesel vollendet. Am 16.9.1877 wurde die neue
Eisenbahnlinie Deggendorf-Zwiesel eröffnet. Bei diesem Bahnbau mit seinen hohen
Dämmen, Felsdurchbrüchen, Serpentinen, Kurven und Tunnels waren neben deutschen
Arbeitern auch sehr viele ausländische Arbeiter beschäftigt. Dass es bei einem
solch großen Bauunternehmen nicht ohne Unfälle abgehen konnte, ist erklärlich.
Bei einem Sprengschuss im Tunnel bei Ulrichsberg wurden 3 Arbeiter getötet. Bei
einer Dynamitexplosion bei Zwiesel verloren gleich 5 Arbeiter ihr Leben.



Im alten Ruhmannsfeldener Friedhof liegen begraben:



Tödlich verunglückt:



Martinus Melampe, 43 Jahre alt, aus Belano in Italien, 1876

Aloisius Todt, 30 Jahre alt aus Pilsen in Böhmen, 1876

Thomas Wacek, 26 Jahre alt aus Jolonic in Böhmen, 1877

Stefane Palaora, 32 Jahre alt aus Novaledo in Südtirol, 1877

Dominikus Cailoni aus Vattaro, Tirol, 1876



Gestorben sind:



Kaspar Deitl, Eisenbahnarbeiter aus Neubarsan in Böhmen, 1876

Franz Koptka, Eisenbahnarbeiter aus Zambeka in Böhmen, 1875

Pietro Scabzeri, Eisenbahnarbeiter aus Pedemonte in Tirol, 1875

Joseph Rodina, Kind einer Eisenbahnarbeiterin aus Böhmen, 1875

Scrivan Matthias, Eisenbahnarbeiter aus Wolenik in Böhmen, 1875

Christian Andreath, Eisenbahnarbeiter aus Piarzadi di Pini (Italien)1875

Giovanni Degasperi, Eisenbahnarbeiter aus Montan Italien, l875

Martinus Preghenella, Eisenbahnarbeiter aus Pregherr, Italien, Typhus, 1876, 45
Jahre alt

Johann Riedmannsberger, Kind aus Lotsche in Böhmen



Im Marcher Friedhof liegt begraben:



Peter Del Fabere, Erdarbeiterskind von Triefenried.



Im Gotteszeller Friedhof liegen begraben:



Tödlich verunglückt:



Joseph Steindecker, 28 Jahre alt, aus Schwanndorf, 1875

Franz Stock, 37 Jahre alt, aus Sessano, Österreich, 1876



Gestorben sind:



Johann Andreas Grosetto, 4 Tage alt, aus Camschake (Turin),1875

Florianus Javendoni, 23 Tage alt, aus Codroipo (Udine), 1876

Maria Zusolini, 2 Tage alt, aus Sesto (Udine),1876

Anna Meindl, 1 Monat alt, aus Tarvis, Gubern/Österreich, 1876

Lucia Magdalena Deguisti, 6 Wochen, alt aus Aizone die San, Italien, l876

Santa Javedoni, 3 Jahre alt, aus Codroipo (Udine) 1877

Javedoni/Söhnchen der Vorstehenden, 2 Stunden alt



Freilich mussten die Grundbesitzer an dieser Bahnlinie anlässlich dieses
Bahnbaues Grund und Boden abtreten. Aber sie wurden für ihre abgetretenen
Grundstücke gut bezahlt. So wurde in Brumbach für 1 Tagwerk Wiesen der
Grundeigentümer mit 1 000 Gulden entschädigt.

Die Arbeiter, die bei dem Bahnbau beschäftigt waren, haben viel verdient. So
wird erzählt, dass die Arbeiter am Zahltage ganze Hüte voll Goldgeld
heimgetragen haben und dass etliche Bauern, die Fuhrwerksdienste bei dem Bahnbau
geleistet haben, ganze Kisten voll Goldgeld daheim gehabt hätten, dass sie aber
trotzdem genau so schnell wieder arm wurden, wie sie damals reich geworden sind.
Dazu haben aber auch schon die ungemein vielen Marketendereien, Kantinen,
Schankwirtschaften und Gastwirtschaften mit ihren Kegelbahnen und Spielhöhlen
beigetragen, die es längs der Baustellen bei diesem Bahnbau in Hülle und Fülle
gab, so viele, dass man sie hier gar nicht alle namentlich aufführen kann und
die vielen Hausierer dazu.

\section{Ereignisse von 1874 bis 1902}

Am 28. Januar 1874 wurde vom Gemeinderat Ruhmannsfelden beschlossen im Verein
mit der Gemeinde Zachenberg eine Feuerspritze anzuschaffen. Diese Feuerspritze
sollte gemeinschaftlich bezahlt werden und der Bürgermeister Lukas von
Ruhmannsfelden sollte mit der Beschaffung derselben beauftragt und
bevollmächtigt werden. Erst zwei Jahre später ging man an den Kauf einer neuen
gemeinschaftlichen Feuerspritze heran. Diese Feuerspritze bildete die
Veranlassung zu vielen Streitigkeiten zwischen den Ruhmannsfeldnern und
Zachenbergern, bis dann im Jahre 1906 die Gemeinde Zachenberg nach Entrichtung
einer Entschädigung von 250 Mark in den alleinigen Besitz dieser Feuerspritze
kam.

1874 brach im Bruckbauernhof ein Brand aus.

Im Dezember 1875 trat der Gastwirt Franz Vogl von Zachenberg zwei Zimmer in
seinem Wohnhaus zu ebener Erde zur Unterbringung von Armen und Kranken auf
unbestimmte Zeit ab, ebenso ein Krankenlokal im 1. Stock und erhielt für
Beheizung, und Reinigung der Lokale und für die Pflege und Verköstigung der
Kranken jährlich 200 Gulden.

Dass man sich schon vor mehr als 70 Jahren mit der Frage einer Erbauung eines
Leichenhauses in der Pfarrei Ruhmannsfelden beschäftigt hat, geht aus einem
Gemeinderatsbeschluss der Gemeinde Zachenberg vom 23. April 1876 hervor, in dem
es heißt, „daß dem bezirksamtlichen Auftrag, die Errichtung eines Leichenhauses
in Ruhmannsfelden betr. von der Gemeinde Zachenberg nicht beigepflichtet werden
kann und es hat sich die Gemeinde Zachenberg in den mehrmaligen Beschlüssen der
Sepulturgemeinde Ruhmannsfelden noch nie einverstanden erklärt und ist auch
direkt solches von der Gemeinde Zachenberg als gänzlich unnötig verlangt
worden.“

1876 brannte in Muschenried das Hofmann-Anwesen ab.

Am 27. Dezember 1876 erfolgte die Umstellung der Bezüge der Gemeindeangestellten
von Gulden (fl.) auf Mark.

Am 13. Mai 1877 wurde die freiwillige Feuerwehr Zachenberg gegründet und
ausgerüstet. Damals gab es im Bezirke Viechtach folgende Feuerwehren: Viechtach,
Ruhmannsfelden (1867), Arnbruck (1869), Wiesing (1873), Schönau (1874),
Blossersberg (1874), Kollnburg (1874), Kirchaitnach (1874), Gotteszell (1875),
Allersdorf, (1875), Schlatzendorf (1875), Drachselsried (1875), Prackenbach
(1876), Patersdorf (1876), Achslach (1876), Moosbach(1876xvii), Geiersthal
(1876). Am 12.9.1880 wurde die Pflichtfeuerwehr Zachenberg eingeführt.

Da die Schülerzahl in der Volksschule Ruhmannsfelden von Jahr zu Jahr stieg und
die Schulräumlichkeiten im alten Schulhaus in Ruhmannsfelden, gebaut 1835, den
Anforderungen nicht mehr entsprachen, wurde am 5. Januar 1881 von der
Schulsprengelverwaltung Ruhmannsfelden folgender Ratsbeschluss gefasst: „In
Anbetracht, daß in der Gemeinde Zachenberg Unterschriften von Haus zu Haus
gesammelt werden, welche den Neubau des Schulhauses nach Zachenberg, ev. nach
Auerbach verlangen und die beiden Projekte in den nächsten Tagen der kgl.
Distriktsschulinspektion und dem kgl. Bezirksamt vorgelegt werden, ist die
Schulsprengelverwaltung außer Standes gesetzt worden, einen definitiven Beschluß
über die Erweiterung des hiesigen Schulhauses durch Adaptierung eines 4.
Schulzimmers zu fassen.“ Darauf hin wurde am 18.3.1883 beschlossen, noch im
selben Jahre mit der Verakkordierung des zu erbauenden Mädchenschulhauses und
mit den Erd- und Fundamentarbeiten zu beginnen und der übrige Bau sollte im
nächsten Jahre (1884) vollendet werden. Die Bauakkordanten waren: Johann
Aulinger, Joseph Amberger und Benedikt Ebner. Der Kostenaufwand für dieses
Mädchenschulhaus war 18 000 Mark. So wurde es mit einem Schulhausbau in der
Gemeindeflur Zachenberg wieder nichts.

Das Jahr 1882 war für die Gemeinde Zachenberg infolge der Spätfröste, des
beständig anhaltenden Regens und eines heftigen Hagelwetters ein Missjahr.
Getreide- und Kartoffelmangel herrschte überall.

In einem Beschluss des Gemeinderates Zachenberg vom 26. Oktober 1883
betreffenxviii Aufstellung eines 2. Arztes in Viechtach heißt es: „Unter
Hinweisung auf die Zuschrift des kgl. Bezirksamtes Viechtach vom 12. Okt. lf.
Jahres wird in der Weise Beschluß gefaßt, daß von Seite der Gemeinde Zachenberg,
kein Zuschuß geleistet werden kann, da voraussichtlich kein Gemeindeangehöriger
der Gemeinde Zachenberg den Arzt in Viechtach zu Hilfe ruft, denn die Gemeinde
Zachenberg hat die günstige Lage, daß die Bahn nach Regen und Deggendorf fährt
und sohin dort die schnellere Hilfe angerufen wird“ Im Übrigen war ja auch in
Ruhmannsfelden damals schon ein Arzt, nämlich Hr. Dr. Rötzer, der 1867 zum
Krankenhausarzt in Ruhmannsfelden ernannt wurde, Hr. Dr. Valentin Dick, der 1881
in Ruhmannsfelden starb. Nach demselben war Hr. Dr. Burger Arzt in
Ruhmannsfelden, der mit seiner Familie anfänglich im früheren Krankenhaus
(jetziger Kinderbewahranstalt) wohnte. 1894 folgte dann Hr. Dr. Danziger, der
eine Handapotheke führte. Dann folgten Hr. Dr. Beck und Hr. Dr. Grundner, der in
Brasilien verstorben ist. Jetzt üben in Ruhmannsfelden die ärztliche Praxis aus:
Hr. Dr. Wiegmann, Hr. Dr. Stern, Hr. Dr. Naunin und Frl. Dr. L. Wiegmann.

Die Herbeischaffung von Arzneien im Ernstfalle mag wohl früher manchmal auf
große Schwierigkeiten gestoßen sein, zumal für die Bewohner der weiter entfernt
liegenden Ortschaften der Gemeinde Zachenberg. Um diesem Übel abzuhelfen wurde
zunächst aus Apothekerkreisen die Errichtung einer Apotheke in Ruhmannsfelden in
Erwägung gezogen. So hat schon im Jahre 1868 Hr. Johann Stünner, Apotheker von
Kelheim um die Apothekenkonzession in Ruhmannsfelden nachgesucht, auch ein
Pharmazent Grabmeier Franz Paul von Rotthalmünster, ein Pharmazent Leonhard
Nußhard von Ingolstadt, ein Apothekenverwalter Franz Laubender von Ebeen in
Unterfranken bis endlich nach langjährigen Verhandlungen dem Herrn Vitus Voit,
Apotheker von München, die Errichtung einer Apotheke in Ruhmannsfelden genehmigt
wurde und zwar am 18.3.1910. Hr. Apotheker Voit hat sofort mit dem Neubau der
Apotheke begonnen, sodass sie 1911 schon eröffnet werden konnte. Seitdem
versorgt die Apotheke in Ruhmannsfelden die Bewohnerschaft der weiten Umgebung
von Ruhmannsfelden mit Arzneien aufs Beste. Seit 1953 ist diese Apotheke im
Besitze von Hr. Apotheker Otto Voit jr.

Am 20. Nov. 1890 fuhr der erste Zug von Viechtach nach Ruhmannsfelden. Dabei gab
es einen Dammrutsch bei Mariental, sodass wieder 8 Tage lang die Postkutsche die
Personen und die Post von Viechtach nach Ruhmannsfelden und Bahnhof Gotteszell
fahren musste.

1892 herrschte in der Gemeindeflur Zachenberg ganz arg die Influenza.

Am 4.12.1894 wurde dem Bauerssohn Joseph Achatz von Auerbach von einem großen
Stein der Kopf zerquetscht.

Am 17.2.1895 beschloss der Gemeinderat Ruhmannsfelden die Ausführung einer
märkischen Wasserleitung nach dem vom Wasserversorgungsbüro in München
ausgearbeiteten Detailprojekt. Das Ouellengebiet liegt in der Gemeindeflur
Zachenberg, in der Nähe von Muschenried. Ein Jahr darauf musste die Gemeinde
Ruhmannsfelden das Exproprioratiosverfahren gegen die Grundstückseigentümer im
Quellengebiet bei Muschenried stellen, da eine vernünftige Abgleichung sich mit
der Mehrzahl der Beteiligten nicht erzielen ließ. Eine Münchner- und eine
Ludwigshafner Firma führten die Bauarbeiten aus.

\section{Ereignisse von 1902 bis 1945}

Am 14.9.1902 fand in Auerbach das 25-jährige Jubiläumsfest der freiwilligen
Feuerwehr Zachenbergxix, verbunden mit Fahnenweihe, statt.

Als Gemeindeschreiber von Zachenberg waren auch die Lehrer … und Hochstraßer
tätig. Nach dem plötzlichen Ableben des Hr. Hochstraßers übernahm die
Gemeindeschreiberei von Zachenberg Hr. Spegele. Dieser verzog 1913 nach
Arnstorf. Daraufhin besorgte die Arbeiten in der Gemeindekanzlei Zachenberg
Lehrer Högn. Dieser trat ein Zimmer seiner Dienstwohnung im alten Schulhaus ab
und verlegte die Gemeindekanzlei Zachenberg, die sich bis dorthin im 2. Stock
der Brauerei Rankl (jetzt Aschenbrenner) befand, in dieses Zimmer. Lehrer Högn
führte die Gemeindeschreiberei für die Gemeinde Zachenberg bis 1920. Dann
übernahm sie Herr Inspektor Härtl bis 1.10.1945. An seine Stelle trat ab
1.5.1947 der Verwaltungssekretär Freisinger.

Bei Kriegsausbruch 1914 mussten auch sehr viele Zachenberger einrücken. Während
der ganzen Dauer des 1. Weltkrieges waren viele russische und französische
Kriegsgefangene bei den Bauern und Landwirten der Gemeinde Zachenberg
untergebracht, die hauptsächlich in der Landwirtschaft beschäftigt wurden.

Am 1.9.1919 starb H. Hr. Kammerer Mühlbauer von Ruhmannsfelden.

1921 brannte das Anwesen des Johann Weinberger von Köckersried ab. 22.1.1922
verunglückte ein Bub des Ludwig Wühr von Bahnhof Gotteszell beim Schlittenfahren
dadurch, dass er beim Bergabfahren die Herrschaft über den Schlitten verlor und
unter das Fuhrwerk des Johann Zellner von Auhof geriet. Dem Buben musste das
verletzte Bein amputiert werden.

Bis 1922 war das Standesamt Zachenberg an das Standesamt Ruhmannsfelden
angeschlossen. Seit 1922 hat nun die Gemeinde Zachenberg ihr eigenes Standesamt.

1925 bekamen Johann und Katharina Wilhelm die Wirtskonzession in Auerbach und
Handlos und später dann Hartl die Wirtskonzession in der Steinbruchkantine
Kleinried.

1926 erhielt Johann Wagner die Wirtskonzession auf dem Wirtshaus in Zachenberg.

1928 wurde der Bau einer Wasserleitung für die Ortschaft Triefenried geplant,
der aber nicht ausgeführt wurde wegen der zu hohen Baukosten und der zu geringen
Ausnützung, da ja damals nur 7 Häuser in Frage kamen.

1928 brannte das Anwesen der Frau Hof in Triefenried nieder.

Im Juli 1929 zog ein furchtbares Hagelwetter über die Gemeindefluren von
Zachenberg und Ruhmannsfelden, das überall großen Schaden anrichtete.

Am 27. Juli 1930 wurde die Bedürfnisfrage für den Betrieb einer Arbeiterkantine
im Steinbruch Muschenried bejaht und die Erteilung einer Erlaubnis zur Ausübung
eines Schankbetriebes in dieser Arbeiterkantine durch Johann Wagner befürwortet.

1930 erhielt Köckersried eine öffentliche Fernsprechstelle im Anwesen der
Gastwirtschaft Schnitzberger.

1931 brannte der Bauernhof des Peter Kraus in Wolfsberg ab im Frühjahr. 1932 der
Hof des Anton Weber in Gottlesried und im Juli 1932 der Hof des Alois Treml in
Triefenried.

1933 hätte man die Ortschaften Köckersried, Bahnhof Gotteszell und Auhof vom
Gemeindebezirk Zachenberg abtrennen und dem Gemeindebezirk Gotteszell
einverleiben wollen.

Im Mai 1933 baute die Gemeinde Zachenberg einen massiven Osterbrünnlsteg mit
Eisengeländer und einer Fußsteigbetondecke.

Im Juni 1933 erhielt die Gemeindekanzlei Zachenberg Anschluss an die
Fernsprechleitung.

1935 hat H. Hr. Kammerer Fahrmeier in Ruhmannsfelden freiwillig resigniert und
ist nach Deggendorf, seinem früheren Wirkungsort verzogen. Am 13.11. des
gleichen Jahres hat der frühere langjährige 1. Stadtpfarrkooperator von
Deggendorf H. Hr. Pfarrer Bauer die Pfarrei übernommen. Er starb am 17.2.1953 im
65. Lebensjahr und 40. Priesterjahr. H. Hr. Pfarrer Reicheneder von
Niedermotzing wurde als Pfarrer von Ruhmannsfelden im Mai 1953 installiert.

Der 1. Weltkrieg 1914 - 18 mit seiner Revolution und Inflation ging auch an der
Gemeindeflur Zachenberg nicht spurlos vorüber. In allen Häusern und Höfen
beklagte man den schweren Verlust eines auf dem Schlachtfelde gefallenen Vaters,
eines Sohnes oder Bruders und die Umstellung von einer Billion auf eine
Reichsmark brachte allen, auch auf dem Lande die erste Enttäuschung, die jede
Geldentwertung mit sich bringen muss. Auch die große Arbeitslosenzeit ging Hand
in Hand mit ihr.

Die 1932/33 hereinbrechende Hitlerzeit hat zwar auf der einen Seite dieser
schlimmen Arbeitslosigkeit ein Ende gesetzt, aber auf der anderen Seite den
unheilvollen zweiten Weltkrieg heraufbeschworen, der das größte Unglück, das je
über ein Land und seine Bevölkerung kommen konnte, in übervollem Maße über
Deutschland ausgeschüttet hat. Wie sich das auch für die Gemeindeflur Zachenberg
ausgewirkt hat, wird später die Heimatgeschichte Zachenberg schreiben.



\chapter{Geschichtliches über die einzelnen Ortschaften der Gemeindeflur Zachenberg}

Die Gemeinde Zachenberg besteht aus 38 Ortschaften und zwar aus

15 Dörfern, aus 13 Weilern und 10 Einöden, also aus 38 Ortschaften.



Zu den 15 Dörfern gehören:

Auerbach, Fratersdorf, Furth, Giggenried, Gottlesried, Gotteszell/Bahnhof,
Kleinried, Kirchweg, Köckersried, Lämmersdorf, Muschenried, Triefenried,
Vorderdietzberg, Zachenberg und Zierbach.



Zu den 13 Weilern gehören:

Auhof, Bruckberg, Eckersberg, Göttleinsberg, Haberleuthen, Hasmannsried,
Hinterdietzberg, Hochau, Leuthen, Lobetsried, Ochsenberg, Poitmannsgrub und
Weichselsried.



Zu den 10 Einöden gehören:

Bruckhof, Brumbach, Gausruck, Hafenried, Hausermühle, Klessing, Reisachmühle,
Wandlhof, Wandlmühle und Wolfsberg.

\section{Das Dorf Auerbach}

Vor langer, langer Zeit waren die Berge in hiesiger Gegend noch viel höher als
heute. Mächtige Gewässer stürzten sich von diesen Bergen herab in das Tal. Die
Täler waren ausgefüllt mit Seen. So war auch die Mulde von Auerbach und Wandlhof
bis hinunter zum heutigen Prünster Steinbruch ausgefüllt mit Wasser, ein
einziger See, so lange bis sich dieses Wasser nach jahrtausendlanger Arbeit
durch die Erd- und Gesteinsmassen beim Prünster Steinbruch hindurch gefressen
und sich dort einen Abfluss ermöglicht hat. Dadurch wurde auch die Mulde von
Auerbach und Wandlhof bis Ruhmannsfelden trocken und aus dem einstigen See
entstand hier eine Aue, bewachsen mit Bäumen, Gebüsch und Sträuchern, bevölkert
von Auwild der verschiedensten Art und Größe, durchflossen von dem
Muschenrieder- und Wandlbach. Von dem Bach, der hier die Aue durchfließt, hat
das Dorf Auerbach seinen Namen.

Die Vergangenheit der Ortschaft Auerbach muss weit zurückreichen, da der Name
Auerbach schon frühzeitig urkundlich erscheint. 1295 verlieh Herzog Otto seinem
Vetter Albrecht, seiner 2. Frau Elisabeth und ihren Kindern unter anderem auch
den Ort Auerbach. 1395 war hier Eberhard der Nussberger in Kollnburg begütert.
Das Salbuch von 1400 führt als Gotteszeller Klosterbesitz das „Zehentrecht“ in
der Ortschaft Auerbach an. Im Salbuch des Kastenamtes Viechtach vom Jahre 1577
werden die folgenden Anwesensbesitzer genannt:



Ein Jakob Altmann, auf einer Hube, auf der später die Hauser und Steinbauer
wirtschafteten.

Ein Hans Altmann. Er konnte einen Erbbrief von Konrad dem Nussberger vom Jahre
1422 auf Konrad Prechler und seine Hausfrau Kunigunde vorweisen.

Ein Wolfgang Loibl auf einem Lehen, auch mit demselben Erbbrief

Ein Georg Zistelberger auf dem Schollenrieder-Gutxx

Ein Christoph Kopp, auch mit Erbbrief

Ein Wolfgang Ebner auf dem Riedler-Gut mit Erbbrief von den Nussbergern Eberhard
und Konrade vom Jahre 1398



Im 17. und 18.Jahrhundert erscheinen in den alten Büchern und Urkunden immer
dieselben Haus-, Hof- oder Gutsnamen für die landwirtschaftlichen Anwesen in
Auerbach. Sogar 1823 noch bestanden nach dem Gefällebuch des Rentamtes Viechtach
die Jahrhunderte alten Hofnamen: Blanken-Gut, Hauser-Gut, Schollenrieder-Gut,
Riedler-Gut, Treml-Gut.

Blanken-Gut: Das Blanken-Gut war der heutige „Glasl-Hof“.1779 hieß der Besitzer
dieses Hofes Georg Sailler und 1823 Michael Sailler. Dieser Hof wurde 1851 neu
gebaut. Seit 1888 heißt es auf diesem Hofe „Achatz“. Die früheren Besitzer waren
Prechler, Sailler, Blank und Kasperbauer. Das frühere alte, gezimmerte Wohnhaus,
das nebenan an der Straße stand, wurde 1824 gebaut und diente als Wirtshaus. Der
Besitzer Georg Steinbauer hat aber auf diesem Hause abgewirtschaftet und ist
darauf hin nach Amerika ausgewandert. An Stelle dieses alten, gezimmerten Hauses
wurde 1952 ein schönes Wohnhaus errichtet von der derzeitigen Besitzerin des
Glasl-Hofes Fr. Maria Achatz. 2 Söhne von ihr haben im 2. Weltkrieg ihr junges
Leben dem Vaterland geopfert. 1943 Franz Achatz in Holland und 1945 Johann
Achatz in Ostpreußen.

Hauser-Gut: Das Hauser-Gut an Stelle des heutigen Wirtshauses, ein halber Hof,
war 1779 im Besitz eines Johann Steinbauer und war geschätzt auf 600 Gulden. Es
war ein baufälliges Anwesen. Der Stall stand an der Straße und das Wohnhaus war
rückwärts. Der Gastwirt Vogl, der dieses Anwesen von Irlmeier gekauft hatte,
ließ den Stall niederreißen und baute an dessen Stelle das heutige
Gastwirtschaftsanwesen. Das übernahm dann dessen Sohn Sebastian Vogl. 1912
erhielten Mathias und Katharina Aman die Wirtschaftskonzession, l925 Johann und
Katharina Wilhelm und heute führt diese Gastwirtschaft Frl. Katharina Wilhelm.

Schinröader-Gut: Das Schollenrieder-Gut, das heutige Glasl-Alois-Anwesen, hieß
im Volksmund das „Schinröader-Gut“. Im Salbuch des Kastenamtes Viechtach vom
Jahre 1577 wird genannt ein Georg Zistleinsberger auf dem Schollenrieder-Gut und
in späteren Zeiten treffen wir hier einen Martin Schollenrieder und einen
Christoph Schollenrieder. Nach einer Urkunde vom 16. Juli 1836 erhielt aus der
Verlassenschaft seines Großvaters Michl Schollenrieder ein Joseph Schollenrieder
das Anwesen. Später hieß es auf diesem Anwesen „Fritz“. Das Anwesen wurde dann
vertrümmert. Der Stadel wurde abgebrochen und auf die Wiese außerhalb des Dorfes
gestellt. Lenz Michl (Treml) baute ein Haus dazu, das später dem Treml Heinrich
gehörte, und heute verpachtet ist.

Treml-Anwesen: Das Treml-Anwesen auf dem Wege zum Wandlhof hat gebaut Treml
Georg, genannt Lenz Girgl. Der derzeitige Besitzer Treml Joseph ist seit 1944
vermisst. Das Anwesen ist verpachtet.

Schollenrieder-Gut: Das Schollenrieder-Gut kaufte dann ein Achatz Alois von
Kaickenried, ein Brüder vom Besitzer des Glasl-Hofes. Dessen Sohn Alois Achatz
übernahm das Anwesen 1938 und 1948 übernahm es der derzeitige Besitzer Obermeier
Joseph. Im 1. Weltkrieg fiel der als Ulane dienende Alois Achatz in Lothringen
und im 2. Weltkrieg fiel ein Sohn der Fr. Obermeier, nämlich Graf Joseph bei
Aachen, 1944. Auch Plötz Johann von Ruhmannsfelden, genannt Schinröader, stammte
von dem Schollenrieder-Gut in Auerbach ab.

Riedler-Gut: Das Riedler-Gut war das jetzige Anwesen des Sägewerkbesitzers
Joseph Obermeier. Im Salbuch des Kastenamtes Viechtach vom Jahre 1577 wird uns
genannt ein Wolfgang Ebner, Besitzer eines Lehens, des sog. Riedler-Gutes. Dazu
gehörte auch eine Mahlmühle mit Säge. Er konnte einen Erbbrief von Konrad und
Eberhard den Nussbergern vom Jahre 1398 vorweisen. Haus, Stadel und Stall und
die Säge waren baufällig, die Mühle dagegen befand sich in ziemlich gutem
Zustande. 1631 wurde Wolfgang Kramheller von Auerbach um 2 Gulden l7 Kreuzer und
1 Heller bestraft, weil er am Feste des hl. Erzengels Michael 1 Stunde lang in
der Mühle gemahlen hatte.1779 wird als Besitzer des Riedler-Gutes genannt ein
Johann Achatz und 1823 ein Georg Achatz, der das Anwesen um 1500 Gulden durch
Übergabe von seinem Vater Johann Achatz erhalten hatte. Später hieß es dann auf
diesem Anwesen Plötz und Pfeffer und jetzt ist Besitzer Joseph Obermeier. 1930
wurde von Obermeier in dieses Anwesen ein Elektrizitätswerk eingebaut und dieses
an die Mühle angehängt und damit das Anwesen und etliche Nachbarsanwesen mit
elektrischem Licht versorgt. Seit 4.3.1953 ist ganz Auerbach an das Überlandwerk
angeschlossen.

Treml-Gut: Das Treml-Gut war das jetzige Kappenberger-Anwesen. Am 18. Juli 1835
übernahm Michl Treml von seinen Eltern Lorenz und Anna Maria Treml das Anwesen
um 1200 Gulden. Seit 1906 heißt es auf diesem Treml-Anwesen Kappenberger Alois.
Dieser hat von Lämmersdorf auf den Hof in Auerbach geheiratet.1907 wurde der
Stall neu gebaut, 1922 die Scheune und 1952 das Wohnhaus. Der jetzige Besitzer
des Hofes, nämlich Kappenberger Alois jr. ist seit Stalingrad vermisst.

Hartl-Hof: Als Besitzer des Hartl-Hofes in Auerbach wird 1779 ein Achaz Gregor
genannt. Dieses Anwesen war damals auf 334 Gulden geschätzt. Der vormalige
Besitzer Xaver Hartl hat das Anwesen 1911 übernommen von Maria Augustin, nachdem
es zuvor auf diesem Anwesen Hofdeck geheißen hat. Xaver Hartl hat das Anwesen
seinem Sohne Joseph Hartl übergeben, welcher der jetzige Besitzer des Anwesens
ist. Gefallen ist der Sohn Xaver Hartl 1944 auf der Insel Krim.

Angl-Anwesen: Auf dem Angl-Anwesen hieß es früher Lorenz Schlögl und ab 1877
Joseph Treml. Angl hat das Anwesen 1909 gekauft. Der jetzige Besitzer Karl Angl
hat das Anwesen seit 1951. Angl Joseph kam 1945 von Frankreich in die Tschechei
und ist seitdem vermisst.

Jungbeck-Anwesen: Das Jungbeck-Anwesen gehörte früher zur Auerbacher Mühle und
war lange Zeit das Ausnahmshaus der Mühle und wurde dann das Eigentum des
sogenannte Mühl-Pfeffer. Nach den Pfeffern hieß es auf diesem Anwesen Graßl
Ludwig. Nach den Graßl war es sieben Jahre unbewohnt. Dann kaufte es Jungbeck
Johann. Dieser baute 1947 einen neuen Stall, 1949 einen neuen Stadel und baute
1951 den 1. Stock des Anwesens neu. 1953 wurde im ganzen Anwesen das elektrische
Licht installiert.

Kopp-Anwesen: Auf dem heutigen Kopp-Anwesen hieß es früher Englmeier. Ein
Nikolaus Egner hat auf das Anwesen geheiratet. Später besaß es ein Thurnbauer,
der 1884 den Steinbruch bei diesem Anwesen anfing und der nach Amerika
auswanderte. Dann Anwesen wurde dann 1918 von Joseph Kopp gekauft von Sebastian
Vogl. Zwei Söhne der jetzigen Besitzerin Fr. Kopp sind in Russland gefallen,
Kellermeier Fridolin 1937 und Kellermeier Georg 1943.

Geiger-Anwesen: Das Anwesen von Geiger Franz wurde vor ungefähr 100 Jahren von
Nikolaus Egner gebaut. Nach ihm folgte ein Achatz von Zuckenried und ein
Zitzelsberger.1896 folgten dann die Geiger. Der jetzige Besitzer Xaver Geiger
übernahm das Anwesen 1921. Er hat 1922 den Stadel und die Holzschupfe neu
gebaut. Zuerst wurde die Straße nach Muschenried 1924 verbessert und 1951 neu
gebaut.

Wurzer-Anwesen: Das heutige Wurzer-Anwesen wurde 1886 von Georg und Walburga
Schlögl von Zachenberg gebaut.1898 wurde das Anwesen von Jakob und Maria
Pfeffer, geb. Schlögl übernommen. 1905 ist auf der Eisenbahn von Triefenried
nach Gotteszell der Anwesensbesitzer Jakob Pfeffer tödlich verunglückt. Die
Besitzerin des Anwesens übergab ihr Anwesen dem Sohne Jakob Pfeffer 1913. Im
gleichen Jahr heiratete er die Müllerstochter Maria Kramhöller von Wühnried. Vom
ersten Weltkrieg kam Jakob Pfeffer nicht mehr zurück. Er gilt als vermisst. 1920
wurde das Anwesen dem jetzigen Besitzer Joseph Wurzer und seiner Ehefrau Maria
Wurzer, verwitwete Pfeffer als Eigentum zugeschrieben. 1937 wurden Wohnhaus,
Scheune und Stall neu gebaut. Im 2. Weltkrieg sind zwei Söhne gefallen, Wurzer
Ludwig 1943 in Russland und Wurzer Joseph 1941 auch in Russland. Am 26. April
1945 war ein feindlicher Flieger nachts in nächster Nähe des Wurzer-Anwesens
Bomben ab. Die Splitter flogen durch die Fenster in die Zimmer des
Wurzer-Hauses. Die Tochter, Katharina Linsmeier, geb. Pfeffer stand aus dem
Bette auf‚ um die schlafenden Eltern zu wecken. Dabei wurde sie von einem
Splitter so schwer verletzt, dass auch ärztliche Kunst ihr junges Frauenleben
nicht mehr retten konnte. R.I.P.



Die neu gebauten Anwesen in Auerbach sind:



Brem Johann, 1928

Niedermeier Joseph, 1930

Peter Xaver, 1930

Weißhäupl Wolfgang, 1936

Preis Siegfried, 1936

Albert Fischl von Zachenberg baute 1936 am Ortseingang nach Auerbach ein neues
Anwesen mit einer gut gehenden Krämerei und einer mit modernsten Maschinen
ausgerüsteten Wagnerei.

1937 baute Kilger Leopold

Bielmeier Xaver, 1937

Kappenberger Johann jr., 1939 (1944 in der Normandie gefallen), heißt jetzt
Muhr.

Düft Jakob, 1950

Obermeier Otto, 1951

Bielmeier Karl, 1951

Kilger Johann, 1952



1830 zählte Auerbach 12 Häuser mit 72 Einwohnern. Heute wohnen in den 22 Häusern
der Ortschaft Auerbach 162 Einwohner.

Nach der Straubinger Landschreiberrechnung vom Jahre 1650 wurde Paulus Sturm von
Auerbach 5 Tage und 5 Nächte bei geringer Nahrung eingesperrt, weil er seinen
Inmann Georg Göbl an einem Sonntag zu Mitternacht ins Haus gelaufen war, Fenster
und Türen zertrümmerte, die Bettstätte zerschlagen hatte und den Inmann samt
seinem Weibe nackt aus dem Hause gejagt hatte. Dieser Paul Sturm wurde im
genannten Jahre nochmals in Strafe genommen, weil er eine Kuh geschlachtet und
das Fleisch pro Pfund über den gewöhnlichen Preis von 5 Kreuzer verkauft hatte
Dieses Mal kam er mit einer Geldstrafe im Betrage von 2 Gulden,17 Kreuzer und 1
Heller weg.

\subsection{Der Teufel auf dem brennenden Strohhaufen}

Die Glasl-Buam waren auf einer Hochzeit in Kaickenried. Es ging schon auf die
Mitternachtsstunde hin, als sie auf ihrem Rückmarsche auf der Lämmersdorfer Höhe
ankamen. Plötzlich erhob sich ein gewaltiger Sturm, der so heftig war, dass sie
sich auf den Erdboden hinlegen mussten, um nicht von dem Sturm mitgerissen zu
werden. Da kam ein mächtiger Feuerschein. Ein brennender Strohbauschen, auf dem
der Teufel wahrhaftig saß, flog über sie hinweg. Erst oben auf dem Breitenstein
stürzte die feurige Garbe zu Boden und mit einem furchtbaren Krach fuhr der
Teufel in die Tiefe des Berges. Nachdem wieder alles ruhig war, liefen die Glasl
Buam, so schnell als sie konnten, dem Elternhaus in Auerbach zu und erzählten am
anderen Tage, was ihnen auf dem Heimwege von Kaickenried passiert sei.

\subsection{Von der „alten Wirtin“ in Auerbach}

Eine weit und breit bekannte Persönlichkeit war die alte Wirtin von Auerbach.
Sie wusste immer was zu erzählen, war zu jedem Spaß bereit und alles kehrte bei
der alten Wirtin zu Auerbach gerne ein. Auch die alten Jäger gingen am Wirtshaus
in Auerbach nicht vorüber, denn da ließ sich mancher Schabernack treiben. Einmal
wollte der alten Wirtin das Butterausrühren gar nicht gelingen. Da wussten doch
die alten Jäger zu helfen. Der alte Scheuböck erklärte der alten Wirtin, dass
der Teufel schuld sei daran, der im Butterfasse sitze. Dieser müsse, wenn sie
Butter wolle, erst erschossen werden. Gut! Die alte Wirtin war daher schon
dabei, wenn es dem Teufel an den Kragen gehen sollte. Das Butterfass wurde auf
die Wiese hinausgetragen. Die Jäger postierten sich in Schützenlinie und mit
einem Krach war das Butterfass in Trümmer zerschossen. Jetzt begann das Jammern
der alten Wirtin um ihr gutes, altes Butterfass, aber auch gleich ein
furchtbares Schimpfen über die alten Jäger. Die alte Wirtin von Auerbach hatte
das Zeug schon, mit den Jägern nicht bloß fertig zu werden, sondern auch auf
ihre Rechnung für das zerschossene Butterfass zu kommen.

\subsection{Beim „Kalten Röhrl”}

 Die alte Wirtin von Auerbach war auch Haarabschneiderin. Die Haare brachte sie
 selbst nach Deggendorf. Da es damals noch keine Eisenbahn gab, musste sie den
 Weg nach Deggendorf zu Fuß zurücklegen. Da ging sie von Auerbach schon um
 Mitternacht weg. Einmal kam sie auch zum gefürchteten „Kalten Röhrl.“ Auf
 einmal fing ein Sturm an, ein Heulen, ein Gebell, ein Gewinsel und Jammern. Ein
 feuriger Wagen mit Funkensprühenden Rädern, ohne Deichsel und ohne Gespann
 sauste über die auf dem Boden liegende alte Wirtin hinweg. Auf dem Wagen saß
 vorne der Teufel und hinten der Weber von Gotteszell. Der alten Wirtin, die
 sonst nicht furchtsam war, wurde schon gewaltig angst, obwohl sie mit Revolver
 ausgerüstet war. Aber da ertönten die Morgenglocken im Graflinger Tal und
 vorüber war das „Gejaide“. Der „Teufelsweber“ von Gotteszell war im Besitze des
 Schwarzbuches und wer das hatte, der stand in Verbindung mit dem Teufel. Wer in
 diesem Schwarzbuch rückwärts lesen konnte, der konnte sich auch jederzeit aus
 der Gewalt des Teufels losmachen. Und das hat der „Teufelsweber“ von Gotteszell
 getan, wenn es der Teufel zu bunt mit ihm trieb.

\section{Der Weiler Auhof}

Der ganze Talkessel vom Hochbühl bis zur Leuthenmühle war früher mit Wasser
überflutet, solange, bis endlich diese Wassermassen sich einen Durchfluss durch
die Leuthe selbst erzwungen haben und das Wasser ab fließen konnte. Dadurch
entstand in diesem Talkessel eine Aue. Bei der ersten Besiedlung dieses Gebietes
wurde dann hier ein Hof errichtet, der von seiner Umgebung, das war damals die
Aue, seinen Namen Auhof erhielt.

Au-Hof: Dieser Hof ist schon sehr alt. Seine Entstehung reicht womöglich zurück
bis in das 12. oder 13. Jahrhundert. 1385 hat das Kloster Gotteszell diesen Hof
erworben. Die Bewirtschaftung dieses Hofes würde einem zinspflichtigen
Nutznießer übertragen. Der Name „Auhof“ erscheint auch um das Jahr 1400
urkundlich. 1612 wird als Auhof-Besitzer ein Michl Achatz genannt. 1658 musste
der damalige Besitzer Maniz Müller 14 Pfund Schmalz abliefern. 1721
bewirtschaftete den Auhof ein Lenz Kraus. Er war Grunduntertan des Klosters
Gotteszell. Seinen Zehent hatte er zu je 1/3 dem Pfarrer zu Viechtach, dem
Kloster Oberalteich und dem Stift St. Johann in Regen zu entrichten. Nach dem
Stifts- und Salbuche des Klosters Gotteszell von 1790/99 hauste zu jener Zeit
auf dem Auhof ein Michl Krauß. 1821 treffen wir auf diesem Hofe, das ein ganzer
Hof war, einen Anton Krauß. Dieser hatte den Hof am 29.9.1818 von seinen Eltern
Andreas und Therese Krauß um 3 000 Gulden übernommen. 1843 wird dieser Anton
Krauß auf diesem Hofe noch genannt. Der Grundbesitz des Hofes war in damaliger
Zeit 104,85 Tagwerk. Zum Hof gehörte, wie es damals üblich war, auch ein eigenes
Brechhaus, das auf dem Anger stand. Nach Aufhebung des Klosters Gotteszell
(1803) war der Auhof dem Staate Erbrechtsweise grundbar und zehentbar. Später
wurde dann der Auhof von Brunner und Doll vertrümmert. Den kleiner gewordenen
Hof haben dann die Rädlinger erworben 1899. 1914 wurde der Besitzer des Auhofes
Johann Zellner sen. und 1939 dessen Schwiegersohn Trellinger Johann und 1947
dessen Stiefsohn Johann Zellner jr., welcher der derzeitige Besitzer des Auhofes
ist.

1945 haben die feindlichen Flieger auf die Anwiese in unmittelbarer Nähe des
Auhofes eine 20 kg schwere Bombe abgeworfen, die aber nicht explodierte. Das
Vieh weidete über dieser Stelle, es wurde darüber gemäht und das Heu geerntet,
bis man 1948 zufällig über diese Bombe kam. Die gefährliche Stelle wurde
polizeilich abgesperrt und die Bombe 1949 entschärft und abtransportiert.

Peter-Anwesen: Das Leithumhaus des Auhofes war das heutige Peter-Anwesen. Dieses
wurde 1903 von Peter Joseph, der von Dörfl war, eingesteigert und 1926 von
dessen Sohn Johann Peter übernommen. Auf diesem Anwesen wurden früher einmal die
sogenannten „Auhof-Schneideruhren“ gefertigt, die aus Holz waren. Noch
vorhandene Exemplare von solchen Uhren werden im Museum zu München aufbewahrt
und gezeigt.1896 ist das Anwesen abgebrannt. In der Bahnbauzeit wurde in diesem
Hause Bier ausgeschenkt.

Hacker-Anwesen: Das Hacker-Anwesen hat 1925 Peter Joseph sen. gebaut. Dieses
Anwesen wurde 1927 von dem Schwiegersohne Peter, von Hacker Johann,
pensionierter Ladeschaffner, übernommen.

Kerschl-Anwesen: Das Kerschl-Anwesen wurde 1910 auf damaligem Auhof-Grund gebaut
von Lorenz Pfeffer, „Bretter-Pfeffer“ genannt.

Der Weiler Auhof zählt in seinen 5 Anwesen zurzeit 35 Einwohner.

\section{Der Weiler Bruckberg}

Auf der Karte heißt dieser Weiler der „Bruckbauernberg“, wurde also von Bruckhof
aus benannt. Auf der Ost- und Nordseite dieses Berges ist Wald und auf der
Westseite sind Wiesen und Felder und zwei Anwesen.

Plötz-Anwesen: Das heutige Plötz-Anwesen gehörte früher einem Johann Pritzl und
die zu dem Anwesen eines Georg Pritzl von Bruckberg gehörigen Grundstücke
stammten aus der Gantmasse des Michl Reitmayer von Zachenberg. Die Pritzl waren
ein tüchtiges Bindergeschlecht. Auf dem Pritzl-Binderanwesen schenkte während
der Bahnbauzeit 1875/77 ein Simon Lauden Bier aus.1911 erwarb dieses Anwesen
Joseph Plötz käuflich und dieser hat 1950/51 das ganze Anwesen neu gebaut. Sein
Sohn Alois Plötz ist 1944 gefallen.

Rosenlehner-Anwesen: Das zweite Anwesen in Bruckberg ist das heutige
Rosenlehner-Anwesen. Dieses wurde 1900 gebaut. Seine früheren Besitzer hießen:
Huber, Pinzl Alois, Muhr Max, Geiger Max. Seit 1936 ist Rosenlehner Franz der
Besitzer dieses Anwesens mit 13 Tagwerk Grund.

In den zwei Anwesen in Bruckberg wohnen 17 Einwohner.

\section{Die Einöde Bruckhof}

Der „Hof an der Pruck“xxi wird schon 1295 in einer Verkaufsurkunde der Herzoge
Otto und Stephan genannt. Später wechselten öfters die Namen der Besitzer dieses
Hofes.1671 übergaben Mathias Edenhofer von Pruckhof und Walburga, seine Hausfrau
ihren Erbrechtshof zu Pruck um 620 Gulden ihrem Sohn Hans Edenhofer. Ein Georg
Koller, der diesen Hof als Erbrechtler des Klosters Gotteszell innehatte, musste
hierfür zu Georgi und Micheli je 2 Gulden, 30 Kreuzer per Stift zahlen und als
Stiftrecht nochmals 2 Kreuzer, 6 Heller entrichten.

Bruck-Hof: Der Name Fenzl erscheint zum ersten Mal auf dem Bruckhof, als der
ledige Hans Venzl am 7.2.1729 in Ruhmannsfelden die ledige Barbara Edenhofer von
Bruckhof heiratete. Ihm folgten sein Sohn Michael Fenzl und dessen Ehefrau Anna
Maria geb. Muhr von Zachenberg. Johann Fenzl, der Sohn des Michael Fenzl
heiratete am 19.8.1800 in Ruhmannsfelden die ledige Anna Feldner, Tochter des
Stephan Feldner, Bauer in Rettenbach. Nach Johann Fenzl kam ein Joseph Fenzl
angehender Bauer in Unterzuckenried und darauf folgend wieder ein Joseph Fenzl
und dessen Ehefrau Anna geb. Heigl von Hofen, Landgericht Mitterfels, die am
20.11.1860 in Ruhmannsfelden geheiratet haben. Am Fronleichnamsfest, 4. Juni
1874, brannte während der Fronleichnamsprozession der Bruckbauernhof ab. Ein
Sohn von diesem Joseph Fenzl war H. Hr. Peter Fenzl, Geistlicherxxii Rat und
Oberpfarrer an der Strafanstalt Straubing, geb. 28.6.1866. Dieser besuchte die
Volksschule in Ruhmannsfelden, studierte in Metten und Regensburg und feierte
sein 1. hl. Messopfer am 8.5.1892 in der Laurentius-Pfarrkirche, in
Ruhmannsfelden. Er wirkte dann als Priester in Kallmünz, Neukirchen-Balbini,
Schwandorf und Straubing und wurde bei der Eröffnung der Strafanstalt Straubing
1902 zum Oberpfarrer dieser Anstalt ernannt, an der er bis zu seiner
Pensionierung im Jahre 1927 wirkte. Am 17.5.1932 feierte er sein 40-jähriges
Priesterjubiläum und am 8.5.1942 sein 50-tes. Er starb am 23.1.1945 in
Straubing. R.I.P. Nach Joseph Fenzl übernahm den Bruckbauernhof der Bruder des
H. Hr. Geistlichen Rates Peter Fenzl, nämlich Georg Fenzl, der schon 1914 starb.
Nach dessen Tode führte die Wirtschaft auf dem Bruckbauernhof die Witwe des
verstorbenen Georg Fenzl, Anna geborene Hacker von Köckersried bis 1926 weiter.
Da passierte am 19.8.1919 ein großes Unglück. Dem Sohne Alois Fenzl wurde bei
der Dreschmaschinenarbeit der Arm ausgerissen, an dessen Folgen er in ganz
jungen Jahren sein Leben lassen musste. 1927 übernahm der Bruder des
verunglückten Alois Fenzl, nämlich Georg Fenzl den Bruckbauernhof, der im Juni
1955 als Witwer starb. Am ganz hinteren Acker, entlang der Teisnach, stand das
Brechhaus und beim Stauwehr stand ein altes Haus, die Schleif genannt.

Die Einöde Bruckhof zählt zurzeit 15 Einwohner.

\subsection{Teufelsstein und Hexenstein}

Die Hexe von der nahen Hexenblöße war es, die dem Teufel ins Ohr flüsterte, dass
er hier auf dem Leuthenberge die beste Gelegenheit habe, mit den lästigen
Wallfahrern zum Osterbrünnl radikal abrechnen zu können. Er trug sich haushohe
Felsblöcke herbei, vom Kandlerriegel, vom Bumsenberg, vom Rehberg und vom
Teufelstisch. Diese mächtigen Steine wollte er auf die nichts ahnenden unten an
der Teisnach am Fuße des Leuthenberges vorüber ziehenden Wallfahrer von der Höhe
des Leuthenberges hinunterrollen lassen, damit diesen betenden Störenfrieden der
Garaus gemacht werde. Es war der zweite Juli, Maria Heimsuchung, der Tag, an dem
alle Jahre schon seit langer Zeit die Bischofsmaiser zur Mutter Gottes im
Osterbrünnl wallfahrten. Sie waren schon auf dem Weg von Muschenried nach
Auerbach. Schwarze Wolken zogen vom Westen her über den Hirschenstein. Die
Bischofsmaiser beschleunigten ihre Schritte, um doch noch trocken ins
Osterbrünnl zu kommen. Aber auf halben Wege zwischen Auerbach und Bruckhof fing
ein furchtbarer Sturm zu wehen an, sodass ein Weiterkommen fast unmöglich war.
Nur mit größter Anstrengung erreichten die Bischofsmaiser den Bruckhof. Dann kam
ein wolkenbruchartiger Regen. Blitz und Donner wechselten pausenlos ab, solange
bis endlich Teufel und Hexe unter entsetzlichem Krachen und Zittern des ganzen
Leuthenberges in der Tiefe des Berges verschwinden mussten. Dann verzogen sich
die Wolken und es war wieder ein herrlicher Julitag. Der Bruckvater, so wurde
der alte Bruckbauernvater genannt, rief seine Leute herbei. Mit Brettern wurde
über die Wühr (Stauwehr) schnell ein Behelfssteg gemacht, sodass die
Bischofsmaiser singend und betend dem Osterbrünnl zueilen konnten. Sie dankten
der Osterbrünnl-Gottesmutter und eine Votivtafel mit der Inschrift: „Maria hat
in großer Not geholfen“ kündete noch viele Jahrzehnte in der Osterbrünnlkapelle
von dieser wunderbaren Rettung. Zwei solche Riesensteine liegen noch am Abhang
des Leuthenberges. Diese wurden früher von den Buben und Mädchen der Schule
Ruhmannsfelden gerne bestiegen und dann riefen sie sich gegenseitig zu:
„Deifös-stoa! Hexn-stoa!“

\section{Die Einöde Brumbach}

Nach Schmidt weist der Name Brumbach auf Brunnbach, Brunnenbach oder Quellbach.

Brumbach-Anwesen: Das Anwesen, genannt Brumbach, hat Hans Löffler, Bauer von
Zachenberg 1866 gebaut. Dann war ein Georg Löffler darauf und 1899 wurde ein
Hans Zitzelsberger Besitzer von Brumbach. Auch der jetzige Besitzer desselben
ist ein Johann Zitzelsberger.

Zitzelsberger-Anwesen: Alois Zitzelsberger von Zachenberg baute 1934 ein Anwesen
auf Brumbacher Grund.xxiv

Von 1875 bis 1878 wurde hier durch Brumbacher Grund die Eisenbahn mit dem hohen
Damm und dem 30 m hohen tiefen Einschnitt bei Brumbach gebaut. An das Anwesen
war damals eine Kantine angebaut, in der ein großer Betrieb herrschte. Auf der
gegenüberliegenden Seite des Bahndammes war auch eine Kantine, dessen Pächter
Schäfer hieß. Darum spricht man heute noch von dem dortigen Schäferloch. Und auf
dem heutigen Kothmühl-Acker neben dem Walde stand ein Backofen, die
Klopferhütte. Wo heute das Bahnwärterhaus Posten 6 steht, stand eine Schmiede
und dieser gegenüber war ein Tanzboden. An den zwei Einschnitten bei Brumbach
wurde volle zwei Jahre Tag und Nacht gearbeitet. Es ist dabei viel Unglück
geschehen über den 30 m hohen Einschnitt führt eine eiserne Brücke.

Die Einöde Brumbach zählt 10 Einwohner.

\section{Der Weiler Eckersberg}

Dieser Ort heißt um das Jahr 882 herum „Ekkirichesbuch“, also der Berg oder der
Buch (soviel als Wald) des Ekkirich. Daraus klingt die Verbindung mit einem
Personennamen heraus, wie er zur Zeit der Herrschaft des Klosters Metten über
den Nordwald üblich war. Um 1300 herum heißt dieser Ort Ekkhartzberg = der Berg
des Ekkhart oder Eckart. Eckersberg wird wieder erwähnt in einer Verkaufsurkunde
vom 5.2.1431 von Heimenan dem Nussberger zu Neueglofsheim. Auch in einem
Degenberg'schen Salbuch aus dem Ende des 16. Jahrhunderts erscheint es wieder.
Ein Georg Philipp war damals hier sesshaft.

Irlbauern-Hof: 1612 und 1625 bemaierte Georg Prunner den Irlbauern-Hof
(Erlen-Hof). Im Jahre 1612 wurden auf dem Hofe 2 Pferde, 3 Kühe und 5 Jungrinder
gehalten. 1721 hielt sich Georg Probst als Leihrechtler des Gerichtes Linden 1
Pferd, 2 Ochsen, 3 Kühe und 6 Jungrinder. Seinen Zehent hatte er dem Kloster
Gotteszell zu reichen. 1772 war der Hof im Besitze eines Andreas Probst, der ihn
am 17.10.1761 übernommen hatte. Nach Probst folgte als Besitzer dieses Hofes
Andreas Treml und nach diesem 1864 Joseph Reisinger, nachdem zuvor (1841) das
Haus und (1842) die Wagenschupfe neu gebaut wurden. 1876 wurde die erste Ehefrau
des Joseph Reisinger auf dem Heimweg von der Kirche von einem Räuber überfallen,
gewürgt und ausgeraubt. An den Folgen dieses Überfalles ist sie nach 3 Wochen
gestorben. 1902 hat dann den Hof Ignatz Vogl übernommen der 1920 starb. Sein
Nachfolger war Karl Bielmeier. Der jetzige Besitzer Vogl Karl hat den Hof 1941
übernommen. Zwei Brüder von ihm sind auf dem Schlachtfelde gefallen: Vogl Ignatz
am 10.5.1940 in Frankreich und Vogl Ludwig am 8.1.1942 in Russland. 1951 wurde
der ganze Irlbauern-Hof elektrifiziert.

Lenz-Bauernhofe: In Eckersberg waren schon immer 2 Höfe. Nach dem
Grundsteuerkataster vom Jahre 1843 hauste auf dem Lenz-Bauernhofe ein Joseph
Kley, der diesen Hof am 23.6.1815 von der Witwe Katharina Treml um 2 500 Gulden
übernommen hatte. Diese beiden Höfe waren 1830 dem Staatsärar Erbrechtsweise
grundbar und zehentbar. Auf dem Lenz-Bauernhof war ein Joseph Plötz, der den Hof
1909 verkaufte und als Privatier in Ruhmannsfelden starb. Nach Joseph Plötz
folgte auf dem Lenz-Bauernhof ein Xaver Pinzl, dann ein Xaver Kraus. 1924
heiratete ein Wolfgang Pfeffer auf dieses Anwesen. Dessen Schwiegervater Xaver
Kraus baute sich auf seinen Grund 1925 selbst ein Haus.

Der Weiler Eckersberg zählt 21 Einwohner in 3 Anwesen.

\subsection{Die Vierzehn-Nothelfer-Kapelle in Eckersberg}

Auf dem Weg zwischen Eckersberg und Vorderdietzberg steht im Walde die Kapelle
zu den 14 hl. Nothelfern. Sie ist aus Holz gezimmert und vor der Eingangstüre in
das Innere der Kapelle sind etliche überdachte Betstühle angebracht für die
müden Pilger und die stillen Beter, die früher in großer Zahl von weit und breit
zur Vierzehn-Nothelfer-Kapelle in Eckersberg wallfahrteten. Diese Kapelle soll
ursprünglich weiter oberhalb gestanden haben. Die Unterhaltspflicht dieser
Kapelle ruht auf dem Irlbauern-Anwesen. Die Vierzehn-Nothelfer-Kapelle wurden in
früheren Jahrhunderten errichtet ganz abseits im Walde, an Waldwegen, an
Waldwegkreuzungen, an Wildwechseln und waren nicht nur Ziel für die Wallfahrer
zu den Vierzehn-Nothelfer-Kapelle, sondern bildeten auch für die Jäger die
sichersten Treffpunkte und die schönsten Ruheplätze. Die Verehrung der
Vierzehn-Nothelfer geht zurück bis in das 11. Jahrhundert, blüht aber
besondersxxv auf im 14. Jahrhundert in der Zeit, wo die Pest und Viehseuchen
wüteten und in der Zeit nach dem 30-jährigen Krieg, wo ja wieder solche
Krankheiten auch in dieser Gegend überaus große Ausmaße annahmen.

\subsection{Das unterirdische Geheimnis von Eckersberg}

Tief unter dem Lenzbauern-Anwesen soll die Erdleutlschlucht durchziehen. Darum
klingt alles im ganzen Anwesen so hell und manchmal vernimmt man von unten
herauf ein leises Singen. Die Sage erzählt, dass von Eckersberg nach Zinkenried
ein unterirdischer Erdgang bestanden habe. Er sei von Zwergen bewohnt gewesen.
Der Eingang zu diesem Gange war in einem Keller des Irlbauern-Hofes. Dieser
Einganz zu diesem Gang wurde aber 1910, als der Keller eingeschüttet wurde,
unzugänglich gemacht. Unter der Wagenschupfe, wo früher das Wohnhaus des
Irbauern-Hofes stand, soll ein großer Schatz vergraben sein. Wer den Schatz
finden und heben will, braucht dazu drei Tannenzweiglein, das eine mit rotem,
das andere mit blauem und das dritte mit schwarzen beerenähnlichen Früchten. Die
drei Zweiglein zu einem Sträußchen zusammengebunden werden dem Schatzsucher den
rechten Weg ins Ohr flüstern.

\section{Das Dorf Fratersdorf}

Der Name Fratersdorf wird in den Urkunden verschiedentlich geschrieben, z. B.
Frahatsdorf oder Frättersdorf usw. Aus diesen Schreibweisen darf man aber keine
Schlüsse ziehen. Bei Fratersdorf hat sich der 2. Laut nämlich das „r“ erst
später eingeschlichen, sodass der anfängliche und ursprüngliche Name Fatosdorf
geheißen hat. (P. N. Fato - Fater) R.V.116 a, 37.

Die Siedlung Fratersdorf ist entstanden als Karl der Große den Nordgau dem
Kloster Metten schenkte. Dieses hat dann an den ebenen und sonnigem Plätzen und
Hängen Meierhöfe errichten lassen und von diesen Musterhöfen ging die
Christianisierung und Kultivierung aus. So entstand wohl im 9. Jahrhundert auch
der Meierhof Fratersdorf. Die ersten Urkunden, die uns Aufschluss über
Fratersdorf geben, erscheinen erstmalig im 15. Jahrhundert. Am 1. September 1466
verschaffte Kathrey, des Nickel zu Fratersdorfs Witwe ihrem Eidam, Michl Sinkl
zu Frattersdorf, Barbara, seiner Tochter und all ihren Erben 1 1/2 Tagwerk
Wiesenmahd, genannt die Prunnwies, zur Haltung einer jährlichen Seelenmesse auf
dem Steinbühl (im Zellertal) zu St. Nikolaus. Das Salbuch des Kastenamtes
Viechtach vom Jahre 1577 nennt uns hier folgende Anwesensbesitzer:

Urban Paumgartner, so mit seinem Gut und der davon dienenden Gilt dem Krapflein
zu Grafenried gehört; darüber der Landrichter von Viechtach von Amtswegen Grund-
und Vogtherr war.

Sigmund Hackl auf einem Lehen, so den Herren von Perlaching gehörte, darüber
ebenfalls der Landrichter von Viechtach Vogtherr war.

Der genannte Sigmund Hackl hatte noch ein 2. Gut in Frattersdorf. Später saß auf
diesem Gute eine Familie Loibl.


1703 übergaben Sebastian Altmann von Frattersdorf und seine Hausfrau Katharina
ihren hiesigen Erbrechtshof ihrem Stiefsohn Paul Altmann um 600 Gulden. Die
Hauptsteuerbeschreibung des Gerichtes Viechtach vom Jahre 1721 nennt die
folgenden bäuerlichen Familien in Frattersdorf:



Adam Göstl auf seinem Dreiviertelhofxxvi

Jakob Stadler auf einem halben Hof

Michl Altmann auf einem halben Hof sitzend



Nach dem Sal- und Stiftsbuch des Klosters Gotteszell aus den Jahren 1790/99 war
der Nachfolger des Bauern Georg Stadler ein Georg Pfeffer. Das Gefällebuch des
Rentamtes Viechtach vom Jahre 1823 nennt uns die folgenden Anwesensbesitzer in
Frattersdorf:



Joseph Pfeffer auf dem Altmann-Hofxxvii

Michal Treml auf dem Weber-Hof

Joseph Loibl (Göstl) auf dem Loibls-Hof



1823 werden im Grundsteuerkataster folgende Hofbesitzer genannt:



Joseph Loibl beim Göstl. Zu diesem Hofe gehörte noch Viertelhof, der früher zum
Schlosse Geltolfing lehenbar war.

Joseph Pfeffer (Goglbauer) übernommen am 22. Juli 1801 von der Mutter Magdalena
Pfeffer um 1200 Gulden.

Michl Artmann beim Urban, am 18. Februar 1833 von den Eltern Michl und Walburga
Artmann um 1600 Gulden übernommen. Früher war dieser Hof auch zum Schlosse
Geltolfing lehenbar.



Goglbauern-Anwesen: Nach dem Joseph Pfeffer übernahm das Goglbauern-Anwesen ein
Lorenz Pfeffer. Ihm verdankte Fratersdorf die Erbauung einer Kapelle im Jahre
1856. Das Altarbild, Mutter Anna darstellend, wurde von Leopold Baumann in
Ruhmannsfelden gemalt. Der Voranschlag für das kleine Gotteshaus betrug damals
227 Gulden, 26 Kreuzer, eine ganz schöne Summe in damaliger Zeit. Leider weiß
niemand, wo diese Mutter-Anna-Kapelle in Fratersdorf ihren Platz hatte. Auch
über das verschwundene Bild kann niemand Auskunft erteilen. Das
Goglbauern-Anwesen brannte unter dem Besitzer Alois Pfeffer 1932 ab, der es in
dem heutigen Zustand aufbauen ließ.1934 übernahm der Sohn Stephan Pfeffer das
elterliche Anwesen. Das ehemalige Pfeffer-Häusl, zum Goglbauern-Anwesen
gehörend, ist weggerissen.xxviii

Pfeffer-Anwesen: Der frühere Besitzer Alois Pfeffer baute auf seinem Grund
zwischen Fratersdorf und Eckersberg ein weiteres Anwesen, das der Sohn Max
Pfeffer bekommen hatte. Dort stand auch früher das Brechhaus. Von Pfeffer Alois
sind zwei Söhne gefallen: Pfeffer Michael in Frankreich bei St. Michel und
Pfeffer Anton in Russland.

Göstl-Haus: Auf dem Adam Göstl Haus heißt es jetzt Zitzelsberger Michael, seit
1913. Seine Vorgänger auf diesem Anwesen waren der Deuschl Joseph und ein Lorenz
Geiger, genannt „Seppn Lenz“. Der Sohn Joseph Zitzelsberger ist vermisst in
Russland. Der frühere Besitzer des Schneider-Andrel-Anwesens war auch ein
Altmann. Kleingütl Andreas besitzt es seit 1867.

Weinberger-Anwesen: Auf dem heutigen Weinberger-Anwesen hieß es früher Pongratz.

Das Dorf Fratersdorf zählt zurzeit 36 Einwohner in 5 Häusern.

\subsection{Die Kapelle zum „sitzenden Herrgott“ in Fratersdorf}

Nahe bei Fratersdorf, an der alten Fahrstraße nach Regen, ist eine kleine
Kapelle mit dem sitzenden Herrgott. Es wird erzählt, dass einmal ein Reisender
mit seinem Einspännerwägelchen diese Stelle passierte. Auf einmal hielt das
Pferd inne und ein dumpfes Pumpern ertönte aus der Tiefe. Das Pferd zitterte am
ganzen Körper, der Schweiß rann ihm vom ganzen Körper, das ganze Gefährt konnte
nicht mehr von der Stelle. Da hat der Reisende die Erbauung einer kleinen
Kapelle mit einem sitzenden Herrgott ausgeheißen, wenn er wieder von diesem
unheimlichem Orte wegkomme. Im selben Augenblicke hörte das unterirdische
Grollen auf und der Reisende konnte mit seinem Pferdlein weiterfahren. Er hat
sein Versprechen wahr gemacht und so entstand die Kapelle zum sitzenden
Herrgott. Der sitzende Herrgott in Fratersdorf wurde weit und breit geehrt und
heute noch wird diese Kapelle von den Fratersdorfern sauber instand gehalten. Ja
sogar Deggendorfer haben schon einmal eine neue Bekleidung des sitzenden
Herrgottes in Fratersdorf gestiftet.

.8 Das Dorf Furth

Der Name Furth weist hin auf eine seichte Wasserstelle, die als Bachübergang und
Bachdurchfahrt gedient hatte. Der Name Furth kommt auch in Zusammensetzungen
vor, z. B. Erfurt, Frankfurt usw.

Urkundlich lässt sich die Ortschaft Furth bis 1386 zurückverfolgen. 1612 besaß
hier Wolf Stadler ein Gut, das er von einem Wolf Hanemann gekauft hatte. Damals
wurden auf diesem Hofe 3 Pferde, 2 Stiere, 4 Kühe und 5 Jungrinder gehalten.
1658 war der hiesige Hofbesitzer ein Lorenz Fink. 1721 treffen wir hier auf
diesem ganzen Hof einen Sebastian Steinbauer. Er war Leibrechter des Kastenamtes
Linden. Ehedem war in Furth nur eine Zündholzstöß. Der alte Achatz hat 1915
diese alte Stöß modernisiert, Turbinen eingebaut und ein Sägewerk daraus
gemacht. Der spätere Besitzer Achatz ist 1945 gefallen. Heute ist der Besitzer
des Sägewerkes Furth Johann Kollmer.

Anetzberger-Gut: Ein altes Anwesen in Furth ist das Heinrich Anetzberger-Gut,
das schon 100 Jahre alt sein dürfte und seinerzeit von Johann Herrnböck gebaut
wurde.

In letzter Zeit war die Bautätigkeit in Furth sehr groß. Neubauten dort sind:



Hof Theres, 1932

Fritz Johann, 1945

Schwarz Hermann, Gasthaus zur Waldesrast 1945

Fritz Joseph, 1947

Achatz Joseph, 1948

Annetsberger Heinrich jr., 1949

Fritz Johann, Gastwirt, 1953



In der Nähe des Anwesens Annetsberger Heinrich sen. steht „der große Herrgott“,
der einmal bei einem Unglück ausgeheißen wurde.

Das Dorf Furth zählt in 9 Häusern 46 Einwohner.

.9 Die Einöde Gaisruck

Der Name „Ruck“ sagt, dass es sich dabei um einen aus dem nahem Bergzug
herausgerückten Berg handelt, der entweder in keiner Verbindung mehr steht mit
dem anderen Bergmassiv, z. B. Gaisruck, oder wenigstens als Einzelberg in einem
Bergzug erscheint, z. B. Bocksruck. Handelt es sich aber nur um einen
vorgeschobenen Berg, der noch mit dem übrigen Bergzug im Zusammenhang ist, so
spricht man von einem „Buckel“, z. B. Kälberbuckel. Sind es aber Berge, die sich
quer in den Bergzug hineinziehen, so spricht man von „Riegeln“, z. B.
Breitenauerriegel, Dreitannenriegel, die sich beide riegelartig in das
Donaugebirge hineinschieben. Der Vorname „Gais“ weist darauf hin, dass dieses
Gebiet früher Weideland war.

Kramhöller-Gut: Das frühere Kramhöller-Gut in Gaisruck dürfte schon sehr alt
sein, da das Geschlecht der Kramhöller von Gaisruck in der ganzen dortigen
Umgebung reich verzweigt war. Dieses alte Kramhöller-Anwesen besitzt jetzt ein
Alois Gruber und die frühere Besitzerin Maria Kramhöller hat sich dort selbst
ein neues Haus gebaut im Jahre 1906.

Im Gaisrucker Steinbruch hat sich 1936 ein Unglück ereignet, wobei zwei
Steinhauer, Schwarz und König, tödlich verunglückten.

Die Einöde Gaisruck zählt 8 Einwohner in zwei Häusern.

.10 Das Dorf Giggenried

Die ältesten Ortsnamen sind die „ing“-Namen. Die Gründung dieser Siedlungen geht
auf die Agilolfinger-Zeit zurück. In der Gemeindeflur Zachenberg finden wir
“ing“-Ortsnamen, weil ja das ganze Gebiet hier bis um 800 herum nicht besiedelt
war. Den „ing“-Namen folgen die „dorf“-Ortsnamen und in unmittelbarer Nähe der
“dorf“-Orte entstanden dann 1 oder 2 Siedlungen mit dem „ried“ Namen. Die echten
“dorf“- und „ried“-Namen waren mit einem Rufnamen zusammengesetzt. So ist es
auch bei dem Namen Giggenried, das die Siedlung, das Rodungsgebiet eines Guche
oder Gneke war (Guckenried oder Gneckenried). Der Name Giggenried wird im
Volksmund „Gickaroid“ ausgesprochen.

Der Ort Giggenried erscheint bereits im herzoglichen Salbuche 1280 und in einem
Windberger Salbuche in der Zeit von 1399 bis 1406. Das Salbuch des Kastenamtes
Viechtach vom Jahre 1577 verzeichnet in Giggenried folgende Anwesen:



Johann Plümbl auf dem sogenannten Huber-Gut

Sebastian Khreuzer auf dem Grundel-Gut

Georg Prunner auf dem Peter-Urban-Gut

Georg Rapper, Grunduntertan des Klosters Gotteszell



1772 berichten Urkunden von einem Meinzinger-Hof. 1763 war Hans Fuchs auf der
Loibl-Sölde. Am 5. September 1777 verlieh die kurfürstliche Hofkammer dem Michl
Mock den zum Gerichte Linden gehörigen halben Hof zu Leibgeding. 1625 hatte Hans
Prunner, Bauer zu Giggenried, seine Leihgeding-Sölde um 180 Gulden an Gilg
Stadler von Ditzberg verkauft. Das Sal- und Stiftsbuch des Klosters Gotteszell
aus dem Jahre 1790/99 verzeichnet in Giggenried die folgenden bäuerlichen
Anwesen:



Johann und dann Michl Krampfl

Thomas Fuchs, später Wolf Ziselsberger.



1823 finden wir in Giggenried die folgenden Hofbesitzer:



Michl Raster auf dem Mock-Hof

Joseph Baumann

Jakob Grundl

Michl Krampfl

Joseph Ziselsberger

Fuchs auf dem Deinel-Hof



1843 sind im Grundsteuerkataster folgende Anwesensbesitzer vorgetragen:



Joseph Gründl, Art-Bauer

Joseph Brunner, Peter-Bauer

Michl Raster

Johann Biller

Michl Stadler

Lorenz Krampfl, Weber

Anna Maria Bernauerin



Mock-Hof: Das Anwesen von Michl Raster auf dem Mock-Hof besaß von 1884 bis 1889
ein Alois Kauschinger. Ab 1889 bewirtschaftete dieses Anwesen Alois Kilger, der
Vater des jetzigen Besitzers Joseph Kilger, der den Hof 1919 übernahm. Die Eser
baute 1926 das Wohnhaus und 1936 den Backofen und den Stall. Alois Kilger, ein
Bruder von Joseph Kilger, ist im Mai 1916 in den Vogesen gefallen. Von den 2
Söhnen des Joseph Kilger ist der Alois Kilger im August 1943 in Russland
gefallen und der andere Sohn Joseph Kilger wird seit Oktober 1943 in Russland
vermisst.1949 wurde im ganzen Kilger-Anwesen das Elektrische eingerichtet.

Spirken-Hof: Das heutige Bernauer-Anwesen war der frühere Spirken-Hof. Seine
früheren Besitzer waren ein Michl Stadler und ein Hell. 1849 hat den Hof ein
Lorenz Bernauer übernommen. Der frühere Spirken-Hof existiert aber nicht mehr.
An der Stelle des heutigen Bernauer-Anwesens in Giggenried stand das Inhaus des
Spirken-Hofes. Der jetzige Besitzer, Joseph Bernauer, bewirtschaftet dieses
Anwesen seit 1915 und hat es 1925 vollständig umgestaltet. Seit 1948 ist im
ganzen Anwesen das Elektrische eingerichtet. Neben diesem Anwesen steht seit
1952 das neu gebaute Wohnhaus des Alois Treml, dem Schwiegersohne des Joseph
Bernauer.

Auf dem Wege von Lämmersdorf nach Vorderdietzberg steht ein Marterl zur
Erinnerung, dass in der Kripp der Bauer Iglhaut von Lämmersdorf (früher
Schierer) tot aufgefunden wurde.

Bielmeier-Anwesen: Auf dem heutigen Bielmeier-Anwesen hieß es früher einmal
Heidenberger, dann Stadler, Hell und Bockshorn und Deuschl Joseph, bis es dann
im Februar 1912 Joseph Bielmeier erwarb. Die Besitzerin des Bielmeier Anwesens
ist heute Anna Bielmeier. Ein Sohn ihrer Tochter, Alois Fendl, ist 1943 in
Russland gefallen.

Biller-Anwesen: 1843 erscheint in den Urkunden ein Johann Biller, dem später ein
Jakob Biller folgte. Dieser hat dann 1937 das Haus neu gebaut. 1952 übernahm
durch Einheirat ein Joseph Steinbauer von Haberleuthen dieses Anwesen. Am
7.5.1944 ist ein Sohn der bereits verstorbenen Biller-Eheleute in Russland
gefallen.

Krampfl-Anwesen: Auf dem Krampfl-Anwesen hauste 1790 ein Johann Krampfl, 1823
ein Michl Krampfl, 1843 ein Lorenz Krampfl. Die Krampfl waren früher ein
tüchtiges Webergeschlecht und die Weberstube beim Krampfl in Giggenried wurde
von weit und breit her aufgesucht.1919 übernahm das Krampfl-Anwesen ein Joseph
Krampfl, 1929 ein Joseph Bauer und 1951 ein Michl Kellermeier. Lorenz Krampfl
ist am 27.5.1943 im Kaukasus gefallen.

Peterbauern-Hof: Ein sehr alter Hof in Giggenried ist der Peterbauern-Hof. Schon
1577 erscheint im damaligen Salbuche des Kastenamtes Viechtach ein Georg Prunner
auf dem Peter-Urban-Gute. Vielleicht kommt von Peter Urban der heutige Name
“Peterbauern-Hof“ her. Der jetzige Besitzer Michl Brunner hat das Anwesen von
seinem Vater Anton Brunner übernommen, der es seit ungefähr 1862 in Besitz
hatte. Michl Brunner hat 1908 den Stadel, 1910 den Getreidekasten und die
Schweinestallungen, 1912 das Haus und 1949 die Stallung neu bauen lassen. 3
Söhne musste der jetzige Besitzer des Peternbauern-Hofes in Giggenried im 2.
Weltkrieg verlieren, nämlich den Alois Brunner, der am 7.3.1942 im
Reserve-Lazarettxxix in Deggendorf verstarb, den Michael Brunner, der am
21.9.1942 in Russland sein Leben lassen musste und Joseph Brunner, der am
18.8.1946 an den erlittenen Kriegsstrapazen im Lager Babenhausen starb.

Kauschinger-Anwesen: Ebenso alt wie der Peterbauern-Hof ist auch das heutige
Kauschinger-Anwesen. Es ist auch im Salbuche des Kastenamtes Viechtach vom Jahre
1577 schon aufgeführt als „Khreuzer auf dem Grundlgute“. 1823 findet man auf
diesem Anwesen einen Jakob Grundl und 1843 einen Joseph Gründl. Von diesem
erwarb den Hof ein Art-Bauer und von diesem Art-Bauer kaufte diesen Hof der
Alois Kauschinger. 1904 baute sich der alte Alois Kauschinger einen Ausnahmehof
außerhalb der Ortschaft Giggenried auf eigenem Grund. Diesen Hof besitzt jetzt
die Schrankenwärterswitwe Fr. Brunner von Haberleuthen. 1905 übernahm den Hof in
Giggenried der Sohn des Alois Kauschinger, nämlich Xaver Kauschinger. Dieser
baute sich auf eigenem Grund außerhalb der Ortschaft Giggenried ein
landwirtschaftliches Anwesen, auf das Karl Ziselsberger von Ruhmannsfelden
geheiratet hat, und übergab seinen Hof in Giggenried seinem Sohne Georg
Kauschinger, der ihn heute bewirtschaftet. 1948 wurde dieser Hof elektrifiziert.
Auf dem Grunde des Kauschinger-Anwesens neben dem Stadel steht eine riesige
Eiche mit einem Durchmesser von zirka 3 m und einem Alter von ungefähr 400
Jahren. Daneben steht eine mächtige Linde mit einem Durchmesser von ca. 2 m und
auch einem Alter von mehreren Jahrhunderten. Beide Bäume sind Eigentum des
jeweiligen Besitzers des Kauschinger-Hofes und stehen unter behördlichem
Naturschutz.

.)a Die Hunger-Eiche von Giggenriedxxx

Die Eiche führt im Volksmunde die Bezeichnung „Hunger-Eiche“ und soll in
Pestzeiten gepflanzt worden sein von dem einzigen Überlebenden des Ortes
Giggenried. Am 3. Januar 1957 brach der morsche, hohle Stamm dieser Steineiche
über dem Erdboden ab und der ganze Baum stürzte krachend zusammen. Am 10.1.1957
erschien in der „Neuen Passauer Zeitung“ mit einem Bild der Eiche nachstehender
Bericht:

“Die Hungereiche stürzte krachend um, nach der Überlieferung wurde sie 600 Jahre
alt, die Linde neben ihr blieb stehen. Giggenried: Am Donnerstag (3.1.1957) in
den Mittagsstunden stürzte plötzlich die große unter Naturschutz stehende
“Hungereiche“ ohne jegliche Einwirkung durch Wind und Wetter unter großem Krach
um. Sie versperrte für Stunden die Ortszufuhr.

Hauptlehrer a. D. Hermann Steiner teilt uns unter anderem folgendes mit: „Die
Eiche stand auf dem Wege von Ruhmannsfelden nach Giggenried auf dem Grundstück
des Bauern Kauschinger. Sie hatte einen Umfang von 11,20 m am Fuß und in
Brusthöhe von 4,60 m.

Der verstorbene Heimatkundler Anton Trellinger schrieb über die Hungereiche in
Giggenried folgendes nieder: „In Giggenried erregt eine riesige Eiche mit einem
Durchmesser von drei Metern und eine mächtige Linde mit einem Durchmesser von
etwa zwei Metern das Erstaunen des vorübergehenden Wanderers. Beide Bäume sind
Eigentum des Bauern Kauschinger aus Giggenried und stehen unter behördlichen
Naturschutz. Die Eiche führt im Volksmund die Bezeichnung Hungereiche und soll
in Pestzeiten gepflanzt worden sein von dem einzigen Überlebenden des Ortes.“
Nach geschichtlichen Aufzeichnungen herrschte die Pest in Ruhmannsfelden und
Umgebung. Nach Geheimrat Eberl von 1354 bis 1357, nach Riedel     von 1347 bis
1349. Es handelt sich um die aus Italien eingeschleppte Beulenpest, die kein
Haus und keine Familie verschonte. Näheres kann der „Geschichte von
Ruhmannsfelden“, bearbeitet und herausgegeben von Rektor i. R. A. Högn,
entnommen werden. Ruhmannsfelden war Kreuzungspunkt von zwei rege benützten
Handelstraßen von Süden nach Norden und von Ost nach West. Es liegt daher nahe,
dass die Pest hier von durchziehenden Kaufmannsleuten eingeschleppt wurde.

Die gewaltigen Ausmaße der Eiche, deren Hauptäste die Dicke hundertjähriger
Bäume übertrifft und das bekannt sehr lange Wachstum der Steineichen, um eine
solche handelt es sich hier, bestärken die Richtigkeit der mündlichen
Überlieferung. Das Alter dieser gewaltigen Eiche kann demnach auf etwa 600 Jahre
angesetzt werden. Setzt man den äußersten Termin der geschichtlichen
Aufzeichnung, das Jahr 1357 an, so ist die Hungereiche genau 600 Jahre alt
geworden. Die Beschreibung der Eiche er folgte wahrscheinlich im 19. Jahrhundert
und wurde mit dem gleich lautenden Text in die Naturschutzakten des 20.
Jahrhunderts übernommen. Die große Höhle im Stamm der Eiche, in der ein
Erwachsener eine bequeme Lagerstatt finden konnte, sowie einzelne in den letzten
Jahren abgestorbene in den Himmel ragende Äste ließen erkennen, dass der Zahn
der Zeit auch an diesem knorrigen Riesen genagt hat. Noch im vergangenen Jahr
spendete ein dichtes Blätterdach wohligen Schatten. An wenigen Stellen des
Stammes konnte der innen morsche Baum den Leben spendenden Saft dem Boden
entnehmen. Menschliche Kunst hätte den fortschreitenden Verfall des Riesen kaum
aufhalten können.

Nach Schildrungen des Grundstückbesitzers Kauschinger vernahm man am Vormittag
des 3. Januar am Baum ein eigentümliches Knacken und Knarren, das wie
Gewehrschüsse klang, als ob der sterbende diese seinen Fall ankündigen und alle
warnen wollte. Gegen 13.30 Uhr brach der morsche Stamm über dem Erdboden.
Krachend stürzte die alte Hungereiche in entgegen gesetzter Richtung der neben
ihr stehenden Linde nieder, als obxxxi der Riese noch im Tode seine Jahrhunderte
alte treue Begleiterin schonen wollte. Nun steht die alte, mächtige Linde allein
aufrecht und renkt ihre kahlen Äste, wie in stiller Trauer in den graufarbigen
Winterhimmel. Vielleicht findet die Hungereiche in einem jungen Eichenbaum einen
Nachfolger, der die Tradition des dahin gegangenen Riesen fortführt.“ (Soweit
der Zeitungsbericht.)

Das Dorf Giggenried zählt in seinen 19 Häusern 132 Einwohner.

.)b Der Goldschatz auf der Giggenrieder Blöß

Es war seit langer, langer Zeit schon bekannt, dass auf der Giggenrieder Blöß
ein wertvoller Goldschatz vergraben liegt. Ihn zu heben war noch niemals und
noch niemanden gelungen. Es fehlte hierzu stets das Glück. An einem schönen
Sommermorgen trieb der Peterbauern Hütbub von Giggenried das Vieh auf die
Giggenrieder Blöß. Es war schon vormittags recht heiß. Da wird bekanntlich das
Vieh auf der Weide wegen der stechenden Mücken recht unruhig und da gibt es für
den Hütbuben viel zu laufen. So war es auch an diesem Vormittag. Er sprang über
Granitblöcke und über Bodenlöcher, zwischen Birkenbäumen und Brombeergestrüpp
hindurch. Das ganze Gelände auf der Hütweide war ihm ja bekannt wie seine immer
leere Hosentasche. Aber siehe! Da war plötzlich vor ihm eine Grube, die zuvor
niemals da war. In dieser Grube lagen schneeweiße frisch geschälte Nusskerne. O
wie gerne hätte er darnach mit beiden Händen greifen wollen! Wie gerne hätte er
seinen Hosensack voll solcher herrlicher Kerne haben wollen. Aber das Vieh hatte
sich inzwischen soweit von ihm entfernt, dass es höchste Zeit für den Hütbuben
war, die Viehherde wieder auf den Weideplatz zurückzubringen bei den Nusskernen
nochmals nachsehen zu können. Da ertönte von Ruhmannsfelden herüber das
Mittagläuten. Der Hütbube musste das Vieh Heimtreiben. Als er im die Stube trat,
schimpfte der Peterbauer, weil der Hütbube nicht, wie gewöhnlich um 11 Uhr,
sondern erst um 12 Uhr das Vieh heimgetrieben hatte. Jetzt erzählte der Hütbube
sein Erlebnis an der Grube mit den schönen Nusskernen. Als, alle
Peterbauern-Leute, die in der Stube waren, beisammen saßen, schlugen sie die
Hände zusammen und riefen voll Erstaunen: „Nusskerne? Lauteres Gold!“ Sie
stürzten aus der Stube hinaus und mit Schwingen und Körben mit Tüchern und
Säcken eilten sie auf die Hütweide hinaus, der Hütbube voraus. Nur Ahnl blieb
auf der Ofenbank sitzen und strickte ruhig weiter. Und den ganzen Nachmittag und
die ganze Giggenrieder Blöß wurden aufs Genaueste abgesucht, aber die Grube war
nicht mehr zu finden und mit den Nusskernen waren Gold und Reichtum und Glück
wieder einmal entschwunden. Der Hütbube hätte bloß etwas Geweihtes in die Grube
legen dürfen, dann wäre das Gold gebannt gewesen. Am Abend setzte sich der
traurige Hütbube auf die Ofenbank zur Ahnl hin, weckte sie aus ihrem Halbschlaf
und sagte zu ihr: „Ahnl, was sagst denn du dazu?“ Die Ahnl erwiderte darauf:
“Bua, mirk dir dös! S'Glück muass ma beim Schopf packa! Wenn mas hat, muass mas
mit den zwoa Händ haltn und nimma auslassn! Wenn oana's Glück mit den Füassn
wegstösst, dem hängt's Lampö seina Lebta ausser!“

.)c Zwei Burschen auf der Suche nach dem Goldschatz

Es waren zwei richtige Waldlerburschen, der eine der Stil (Abkürzung für
Augustin) und der andere der Ferri (Abkürzung fürs Xaver). Um der Langweile des
Lebens aus dem Weg zu gehen zu können, befassten sie sich mit Fischstechen in
den Gewässern, mit Vogelfang auf den Auen und mit Schlingen- und Faltenstellen
in den Wäldern. Hierzu gab es damals reichlich Gelegenheit. Da die beiden auch
von einem vergrabenen Goldschatz auf der Giggenrieder Blöße gehört hatten,
trieben sich die beiden in letzter Zeit viel in der Giggenrieder Umgebung herum.
Eines Tage durchstreiften sie wieder das Mühlholz von hinten bis vorne und von
oben bis unten. Da sah der Stil vor sich am Rande der Blöße ein kleines Licht am
hellen Tage. Es war so scharf, dass es die Augen des Stil auszustechen drohte
und die Farben des Lichtes waren die Regenbogenfarben. Stil sagte zum Ferri:
“Schau dort das herrliche Licht!“ Ferri erwiderte: „Komm, Stil, das gehört uns
beiden!“ Aber das Licht war nicht zu erreichen. Es entfernte sich von ihnen umso
mehr als sie glaubten sich dem ungewöhnlichem Lichte auf dem Erdboden nähern zu
können. Plötzlich standen zwei Kapitalhirsche, Sechzehnender, vor ihnen, ruhig
und regungslos. Da trat Ferri auf einen dürren Ast. Sofort sprangen die beiden
Hirsche weg. Aber es eilte ihnen nicht. Es schien, als hätten sie schon einen
weiten Weg zurückgelegt. Stil und Ferri erkannten das und eilten den Hirschen
nach. Stil sagte: „Der eine gehört mir!“ Ferri schrie, was er herausbrachte:
“Der andere gehört mir!“ Aber es gelang ihnen halt doch nicht die Hirsche zu
fangen, trotzdem sie den beiden Hirschen ganz nahe waren und trotzdem sie die
Hirsche schon verfolgt hatten bis nach Zachenberg und Kleinried. Und da
verschwanden plötzlich die beiden Hirsche im Bocksrucker Hochwald. Jetzt
erinnerten sich Stil und Ferri wieder an das seltsame Licht auf der Giggenrieder
Blöße. Sie traten eilig den Rückweg an. Aber da senkte sich ein
undurchdringlicher Nebel herab über Berg und Tal. Keine zwei Schritte weit war
etwas zu sehen. Sie gingen immer weiter und weiter, bergauf, bergab, über Feld
und durch Wald. Am andern Morgen standen Stil und Ferri vor der Klosterpforte in
Rinchnach. So haben die beiden, Stil und Ferri, den Goldschatz vertauscht mit
einem Teller warmer Klostersuppe. Stil sagte im Heimgehen ruhig vor sich hin:
“Wer Geld und Gut zu gleicha Zeit daglanga ko, der muaß scho a narrisch Glück
habn!“ Sein Begleiter erinnerte sich bei diesen Worten an das, was einmal der
Mesner zu seinen Nachbarn sagte: „von den Wildlan und Kartenspielern hats no
koana weida bracht, als zu da Bettlsuppn“ und schwieg.

.11 Das Dorf Gotteszell/Bahnhof

Früher war das heutige Pfarrdorf Gotteszell nur ein einziger Hof, der Droßlach
hieß. Heinrich von Pfelling war bestrebt, aus dem Herrschaftshof Droßlach eine
kleine, fromme Niederlassung zu machen und diese dann zu einem selbständigen
Kloster auszugestalten. Dazu besaß er auch das Einverständnis seines Schwagers,
des Bischofs Heinrich II. Dieser genehmigte mit Konfirmationsbrief vom 8.5.1286
die neue Niederlassung von Klostermönchen und verlieh dieser den Namen „Cella
Dei“ (Zelle Gottes – Gotteszell). Die Bahnstation, die hier auf Köckersrieder
Grund bei dem Bau der Bahnlinie Plattling-Eisenstein 1876/77 gebaut wurde,
erhielt auch den Namen Bahnstation Gotteszell. Alle die vielen Anwesen, die seit
Errichtung der Bahnstation Gotteszell im Laufe der Jahrzehnte erbaut wurden,
führen die Ortsbezeichnung Bahnhof Gotteszell bzw. Gotteszell/Bahnhof.

Vor dem Bahnbau stand wohl das Schober-Anwesen, zuvor Joseph Treml, jetzt
Achatz. Das Peterbauer-Anwesen, jetzt Kaiser Joseph, steht seit 1875.

Bindl-Marktetenderei: Im Jahre 1878 wird in den Lebensmittelvisitationsberichten
der Gemeinde Zachenberg ein August Bindl, Marketender in Bahnhof Gotteszell
genannt. Anton Hirtreiter hat dann diese Marketenderei in eine weit und breit
bekannte Gastwirtschaft und Metzgerei umbauen lassen, deren Besitzer später
Fink, dann Schuder und seit 1920 Hafeneder heißen. 1892 baute Hirtreiter auch
das Anwesen des Johann Stern, Krämerei, Bäckerei und Gastwirtschaft.

Seidl-Anwesen: Im gleichen Jahre entstand auch das Anwesen der Fr. Seidl, früher
Zitzelsberger und Peter Hilde.

Späth-Anwesen: Aus der gleichen Zeit um 1890 stammt auch das Anwesen, das jetzt
der Frau Späth gehört und dessen frühere Besitzer Kandler, Geiger, Frank und
Schnitzbergen waren.

Kilger-Restaurant: 1896/97 ließ der Brauereibesitzer Leopold Kilger von
Gotteszell eine große Bahnhofrestaurant gegenüber dem Bahnhof errichten.

Kraus-Anwesen: 1901 erbaute der Schreinermeister Ludwig Kraus sein Anwesen.

In dieser Zeit entstanden auch die Anwesen von Aigner, Kilger, Strohmeier und
Kasperbauer Johann.

Egner-Haus: 1903 wurde das Haus von Rebl, jetzt Egner, erbaut, das 1930
abbrannte.

Pointinger-Haus: 1907 erbaute Kestl das sogen. Pointinger-Haus, das jetzt Hr.
Treiber gehört.

Holzstiftefabrik: 1919 kaufte Rödl und Co. (Rödl, Danhof, Gerstner und
Hirtreiter) die kleine Eisendreherei von Laschinger und baute sie aus in eine
Holzstiftefabrik, die zwar damals und während des 2. Weltkrieges ein gutes
Geschäft machte, heute aber aus wirtschaftlichen Verhältnissen still gelegt ist.
Die Holzstiftefabrik Gotteszell/Bahnhof, die jetzt Frl. Busl gehört und an Hr.
Hödl verpachtet ist, soll in einen Holzbetrieb umgestellt werden. Das alte
Laschinger-Haus erwarb Hirtreiter Anton, der es später umbaute. Die Ortschaft
Bahnhof Gotteszell vergrößerte sich durch die Neubauten von Joseph Schuster,
Fritz Schosser, Bruno Ellert, Deschinger Xaver, Zitzelsberger Joseph, Altmann
Johann, Altmann Xaver.

1939 wurde das große Gebäude der Bahnmeisterei aufgeführt mit Büros und
Wohnungen. Da gleichzeitig die Erweiterung des Bahnkörpers notwendig war, wurde
das frühere Oswald-Anwesen weggerissen.

In der Ortschaft Bahnhof Gotteszell sind 2 Restaurationen und eine
Gastwirtschaft, eine Metzgerei und eine Bäckerei, sowie die Lagerhäuser
Schwannberger (1921), Doll (1922, jetzt Treml Wolfgang), Treml Wolfgang (1939)
und Zadler (1947). Auf dem Gelände des Bahnhofes Gotteszell befindet sich der
Holzlagerplatz der Firma Hr. Bohnekamp, GmbH Drevenack und der Bretterlagerplatz
von Ebner, Sägmühle.

1930 verunglückte im Bahnhof Gotteszell Hr. Gerstner tödlich, indem er bei der
Einfahrt des Eisensteiner Zuges zu nahe am Gleis stand, von der Lokomotive
erfasst wurde und unter die Räder zu liegen kam. Am 7. Juli 1927 fuhr ein
Rangierzug aus der Station Gotteszell in Richtung Triefenried. Gleichzeitig kam
in entgegen gesetzter Richtung ein Güterzug. Bei diesem versagten die Bremsen.
Ein Zusammenstoß der beiden Züge war unvermeidlich. Der Lockführer und Heizer
des Güterzuges sprangen ab und die beiden Lokomotiven und einige Wägen des
Güterzuges entgleisten und fielen auf den hohen Bahndamm herab.

1945 kamen aus nördlicher Richtung amerikanische Flieger, die in Richtung
Viechtach-Deggendorf flogen. Zwei Tiefflieger überflogen in ganz geringer Höhe
den Markt Ruhmannsfelden und wandten sich dann dem Bahnhof Gotteszell zu, der
von ihnen auch beschossen wurde. Getroffen wurden die Güterhalle und das
niedrige Brennstoffhäuschen, in welchem sich Benzin und Petroleum befand.
Getroffen wurden auch Danhof sen. am Oberschenkel und Weber Karl jr., der mit
seinem Kinde am Fenster stand, am Arm.

Von der Ortschaft Bahnhof Gotteszell sind im zweiten Weltkrieg gefallen Albert
Danhof in Russland, Karl Frank in Russland, Heinrich Hafeneder in Russland,
Eduard Kraus in Russland, Adolf Rödl in Russland. Vermisst: Joseph Deschinger in
Russland, Heinrich Höferer in Russland, Franz Kaiser in Rumänien, Alois Kappl in
Russland, Franz Seidl in Russland. Anton Doll verunglückt und gestorben in
Deutschland. Ernst Eidenschink für tot erklärt in Mitteldeutschland. Die
Ortschaft Bahnhof Gotteszell zählt in 29 Häusern 310 Einwohner.

.12 Der Weiler Göttleinsberg

Der Name dieses Weilers wurde ehedem geschrieben: Gösleins-, Götzleins-,
Gösseleins-, Gozleins- und Gozeleinspergxxxii. Dieser Name wird sicher von einem
Siedler Gozlo oder Gözlo hergeleitet. Man findet diesen Ortsnamen urkundlich
1536 als Gözleinsberg und 1577 als Gozleinsberg. Göttleinsberg dürfte die
hochgelegene Siedlung eines Gozelo oder Gössele sein.

1574 saß in Göttleinsberg auf einem zur Pfarrei Geiersthal gehörigen Hofe Hans
Müller. 1637 besaß Paul Simpekh den hier bestehenden Hof und hatte darüber einen
Kaufbrief vom Jahre 1592 in Händen für Elisabeth, die hinterlassene Witwe von
Hans Müller und ihrem Sohn. Der Hof gehörte damals zur Pfarrei Geiersthal. 1703
hauste hier auf einem Hofe Jakob Fink, der als Kloster Gotteszeller Untertan
seinen hiesigen Erbrechtshof um 450 Gulden seinem Tochtermann Hans Kramhöller
und seiner ehelichen Tochter Maria übergab. Nach dem Sal- und Stiftsbuch des
Klosters Gotteszell aus der Zeit von 1790/99 bewirtschaftete hier Joseph
Kramheller ein Lehen und musste von diesem Stift und Zehent abführen, entweder
als Naturalgabe oder in Geld. 1823 bestanden hier ein ganzer und ein halber Hof,
die dem Staate grund- und zehentbar waren und beide dem Kramheller gehörten.

Hanslbauern-Hof: Nach dem Grundsteuerkataster vom Jahre 1843 bestanden hier zwei
bäuerliche Anwesen, von denen das eine in den Händen eines Georg Schreiner, das
andere, der sogenannten Hanslbauern-Hof, in Händen des Johann Kramheller war.
Letzterer hatte sein Besitztum am 31.3.1800 von den Eltern Joseph und Anna
Kramheller um 900 Gulden übernommen. Nach dem Johann Kramheller waren zwei
Lorenz Kramheller hintereinander die Hofbesitzer und dann ein Joseph Kramheller.
Seit 1919 bewirtschaftet diesen Hof Joseph Klimmer. Ein Lorenz Kramheller ist
1914 in Frankreich gefallen. Zwei Söhne von Klimmer sind gefallen: Joseph
Klimmer 1943 in Russland, und Alois Klimmer 1944 in Mazedonien.

Schreiner-Hof: Der Hof, der im Besitze eines Georg Schreiner war, hieß kurz:
„der Göttleinsberg.“ Nach Georg Schreiner bewirtschaftete diesen Hof ein Anton
Eidenschink. Von diesem erwarb diesen Hof die Siedlung. Von der Siedlung kaufte
das Ausnahmshaus dieses Hofes Gruber Joseph sen., der es für seinen Sohn Joseph
Gruber jun., der seit 1943 in Rumänien vermisst ist, in ein Wohnhaus umbauen
ließ (1920). Den Hof selber erwarb käuflich ein Xaver Fischer (1936/37), der ihn
jetzt bewirtschaftet. Zwei Söhne des Xaver Fischer sind gefallen: der
Unteroffizierxxxiii Xaver Fischer 1942 in Russland und der Unteroffizier Joseph
Fischer 1944 ebenfalls in Russland. Der Sohn Alois Fischer hat den rechten Arm
verloren und der Sohn Karl Fischer ist schwer verwundet worden am 16.2.1944 in
Russland.

1937 baute Peter Pöhn ein Anwesen auf Eggbauern-Grund (von der Siedlung
erworben) und 1952 Ludwig Hartl daneben ebenfalls ein Anwesen.

Gruber-Anwesen: Auf Göttleinsberger Grund steht noch das Anwesen des Joseph
Gruber Steinbruchbesitzers von Göttleinsberg, in dem sich auch eine Kantine
befindet.

Als einmal auf dem Eggl-Bauernhof eine Viehseuche ausgebrochen war, soll der
damalige Besitzer dieses Hofes die Errichtung einer Kapelle versprochen haben.
Seit dieser Zeit steht oben im Egglbauern-Holz eine Muttergotteskapelle, die
heute noch in Ehren gehalten wird.

Der Weiler Göttleinsberg zählt in seinen 7 Häusern zurzeit 42 Einwohner.

.)a Die Kapelle im singenden Stein

Der einzige Sohn des Ritters Rumar von Ruhmannsfelden, Gebar genannt, hatte sich
auf der Rückreise vom Turnier in Zürich mit einem Grafensohn von Offenberg
entzweit. Es musste ein Zweikampf ausgetragen werden, bei dem Gebar den Tod
fand. Seine Geliebte, Theorina, das Burgfräulein von Weißenstein, die in Gebar
verliebt war, hat schon bei Antritt seiner Reise nach Zürich schweren Herzens
und mit größten Befürchtungen Abschied genommen. Während seiner Abwesenheit ging
sie jeden Tag von der Burg Weißenstein halben Weges Ruhmannsfelden zu, weil sie
hoffte, Gebar werde sicher wieder zurückkommen. Da kam sie immer zu dem großen
Stein, der oberhalb Göttleinsberg mitten im Walde steht, ein würfelförmiger
Granitblock ungefähr 2,5 m hoch und ebenso breit und lang. Eines Tages, nach
langem Warten, saß sie wieder bei diesem großen Stein. Die Nachtigall sang ein
ganz trauriges Lied. Theorina weinte so heftig, dass sie erblindete und Gefahr
lief, nicht mehr heimzufinden und im Walde verhungern zu müssen. Da kam die
Zauberin Akanita von der Hexenblöße. Bei dem Anblick des hilflosen Burgfräuleins
schrie sie: „So, nun hab ich dich endlich, du verliebtes Burgfräulein, du
verfluchte Christin. Du sollst für alle Zeit in diesen großen Felsblock hinein
verzaubert sein.“ Theornina war von da ab unauffindbar und verschollen. Sie war
verzaubert. Wenn die Förster und Jäger von Burggrafenried oder die Steinhauer
und Waldarbeiter von Göttleinsberg an diesem großen Stein vorübergingen, hörten
sie aus diesem großen Stein heraus ein leises Rufen und Singen. Niemand konnte
sich das erklären. Dem Steinhauer Bertl ließ das keine Ruhe. Sein
Gewohnheitsspruch war: „Der Sach muss man auf den Grund gehen!“

Von da ab nahm er nun jeden Tag abends nach getaner Arbeit im Steinbruch seine
Werkzeugkraxn, ging damit zum großen Stein hinauf und meißelte mit Meißel und
Hammer eine Höhlung in den großen Stein hinein , … cm breit und ebenso hoch und
tief. Dann stellte er ein schönes Bild von der Ortschaft Göttleinsberg, über der
die Gottesmutter auf den Wolken schwebte, hinein und zierte das Ganze mit
Waldblumen. So hat der Steinhauer Bertl ganz unbewusst das kleine Kapellchen im
singenden Stein geschaffen. Und siehe! Von nun an war das Rufen und Singen im
großen Stein verstummt. Die verzauberte Seele war erlöst. Maria hat geholfen.
Die Begebenheit wurde bald überall bekannt und von weit und breit her kamen die
Wallfahrer, und Pilgerzüge zur Kapelle im singenden Stein. Heute wird sie nur
mehr kurz die Göttleinsberger Kapelle genannt, wird aber von der Bewohnerschaft
der dortigen Gegend immer noch in hohen Ehren gehalten.

.13 Das Dorf Gottlesried

Wie bereits schon geschrieben wurde, haben die „ried“-Orte mit Klöstern oder
Mönchen nichts zu tun, da die „ried“-Orte nach den „dorf“-Orten, die
Klostersiedlungen waren, entstanden sind. Die „ried“-Orte: Hafenried,
Lobetsried, Triefenried, Gottlesried, Habischried sind also nach Patersdorf,
Fratersdorf, Lämmersdorf gekommen. Der Name Gottlesried erscheint 1351 als
Godersried, 1577 als Gottelsried und auf einer Landkarte von Appian im 18.
Jahrhundert als Köhlersried. Im Volksmund heißt die Ortschaft Gollersroid. Es
dürfte das dortige Gebiet das Rodungsgebiet eines Gothars oder Godilos gewesen
sein. Gottlesried ist sehr alten Ursprungs und erscheint erstmals urkundlich am
26. Januar 1351, als die Stiftungsurkunde des Heiligen-Geistspitals in Viechtach
ausgefertigt wurde.

Nach dieser Urkunde vermachte Konrad der Nussberger zu Neunussberg dem
Gotteshause Schönau die Einkünfte von einem Gute zu Gottlesried mit der
Bedingung, dass man in Schönau seinen Jahrtag begehen und an demselben jedem
Menschen 3 weizene Päugl (Brote) und einen Hering am Charfreitag aber jedermann
3 weizene Brote und 1 Seidel Bier verabreichen solle. 1577 saß auf diesem Gute
Georg Wiedenpaur, der mit seinem Gute mit Gilt und aller Herrlichkeit der Frau
Rosina von Stauff zu Neunussberg gehörte. Er musste zu Georgi und Micheli
Landsteuer an das Kastenamt Viechtach zahlen. Später saß auf diesem Gute die
Familie Stadler. Auf einem anderen, hiesigen Gute hauste 1677 Philipp Rodler.
Der Hof gehörte mit aller Herrlichkeit Hans Christoph Pfaller zu Au. Nach dem
Sal- und Stiftsbuch des Klosters Gotteszell aus der Zeit vom 1790/99 saß zu
jener Zeit hier Joseph Kronfellner als Erbrechter auf einem Lehen. 1823
bestanden in Gottlesried ein ganzer Hof und eine Sölde, die dem Staatsärar
Erbrechtsweise grundbar waren, früher aber zur Hofmark Au, March und Zell
gehörten. Nach dem Grundsteuerkataster vom Jahre 1843 waren hier die folgenden
Familien ansässig:



Katharina Pfeffer: Siexxxiv hatte das Anwesen durch Einheiratung mit 500 Gulden
Vermögen nach dem 1835 erfolgten Tode ihres 1. Mannes Anton König als
Alleineigentümer erworben.

Joseph Brunner: Er gelangte in den Besitz des Hofes am 4.11.1830 um 2580 Gulden
aus der Gantmasse des Vaters Michl Brunner. Der Hof war früher Erbrechtsweise
grundbar zur Gutsherrschaft March und Zell.

Peter Dachs: Er erhielt das Anwesen um 700 Gulden von Joseph Brunner in
Gottlesried am 14.11.1830 durch Kauf. Auch dieser Hof war früher der
Gutsherrschaft Au und Zell unterworfen.

Paul Saller: Er hatte seinen Hof am 15.9.1840 vom Vater Michl Saller um 2800
Gulden übernommen. Dieser Hof war mit 79 1/2 Tagwerk seiner Zeit der größte unter
diesen vieren. Weber Anton kaufte diesen Hof, den heute dessen Sohn Wolfgang
Weber bewirtschaftet. Ein Weber Joseph wurde vom Eisenbahnzug erfasst und
getötet.



Aigner-Anwesen: Auf dem Nachbarhof, auf dem früher ein Weber Johann, ein Kopp
und ein Waschinger hauste, wirtschaftet jetzt ein Hermann Aigner. Auf dem
Anwesen des Anton Dachs war später ein Michl Weber und dann seit 1907 ein Johann
Mader. Das Anwesen des Johann Mader brannte 1915 ab. Seit 1924 gehört das wieder
aufgebaute Anwesen wieder einem Johann Mader. Auf dem Herrnböck-Anwesen heißt es
jetzt Billich.

Weißhäupl-Anwesen: Das frühere Weißhäupl-Anwesen gehört jetzt der Frau Monika
Löffler. Daneben hat ein Hr. Schöpp einen Neubau aufgeführt.

Ein Weißhäupl wurde auch vom Eisenbahnzug erfasst und getötet.

Geiger-Anwesen: An der Grenze der Ortsflur von Gottlesried steht das, 1952 neu
gebaute Anwesen des Xaver Geiger.

Das Dorf Gottlesried zählt in seinen 7 Häusern 72 Einwohner.

Bemerkung: Auf einer Landkarte aus dem 18. Jahrhundert steht an Stelle von
Gottlesried der Name Köhlersried.

.)a Die böse Hexe von Köhlersried

In Köhlersried hauste die böse Hexe, die von Menschen und Tieren, besonders aber
von den kleinen Zwergen gleich gefürchtet wurde. Besonders auf die Zwerge war
die böse Hexe schlecht gelaunt. Und das wussten auch die Zwerge. Wenn die Hexe
einen von den Zwergen schnappen konnte, so tat sie das auch mit besonderem
Wohlgefallen. Die Zwerge kamen in den Bereich der Hexe, weil ihr unterirdischer
Gang von Zinkenried herüber nach Wolfsberg und von da über Köhlersried auf den
Barthenstein führte. Da war es natürlich nicht ausgeschlossen, dass Zwerge und
Hexe in Berührung kamen. Dabei mussten beide sehr vorsichtig sein, denn sie
waren ja Todfeinde. Plötzlich horchten die Zwerge auf. Zankerl, der größte und
der stärkste unter den Zwergen, pfiff einmal durch die Finger. Das hieß
aufhorchen. Jetzt pfiff er zweimal durch die Finger so kräftig, dass man es
weitab hören konnte. Das hieß mit Schaufel und Werkzeug kommen. Wie ein Schwarm
Bienen kamen sie daher gerannt. Sie gruben ein tiefes Loch und deckten es oben
mit dürren Ästen zu und legten darüber grüne Tannenzweige. Das war die Falle für
die böse Hexe. Da musste sie hineinfallen. Und am anderen Tag war sie auch schon
in der Grube und bat bitterlich um Hilfe. Zankerl kannte aber keine Gnade, weil
sie den kleinen und schönen Zwerg, den Pinkerl, entführt hatte. Die Grube wurde
zugeschaufelt. Dann stellten sie sich im Kreise um die Grube, um ihren
Freudentanz über den Tod der bösen Hexe zu feiern.

Und wie sie gerade im Tanzen und Singen waren, kam der Pinkerl angesprungen. Die
Zipfelmütze schwingend machte er Sprünge so hoch er nur konnte. Und dann
erzählte er: „Der Freund Trapperl hatte Bauchweh. Da musste Pinkerl Moosbirl
holen. Und wie er bereits die Zipfelmütze voll hatte, packte ihn die Hexe beim
Hosenboden und steckte ihn in die Kürbn. Im Hause der bösen Hexe kam er in den
Hühnerstall zu dem bösen Gickerl, der so bös war, dass öfters die Hexe zum
stumpfen Besen greifen musste, wenn der Gickerl zu frech wurde. Pinkerl hatte
ein kleines Stückchen Marzipan in der Hosentasche und das teilte er mit dem
bösen Gickerl, der das Hühnerloch selber auf- und zumachen konnte. Am zweiten
Tage schon vor Sonnenaufgang zog er mit seinem ganzen Hühnervolke hinab in das
Wiesental. Er vergaß das Hühnerloch wieder zu schließen. Da die Hexe noch
schlief, schlüpfte Pinkerl durch das Hühnerloch hinaus ins Freie und rannte wie
ein Wiesel zu Euch herauf. Pinkerl ist gerettet! Und die Hexe?“ Da schrieen sie
im Chor: „Hexlein, Hexlein in der Grube drin!“ Dann nahmen sie den Pinkerl in
den Kreis und tanzten wieder um die Grube herum. Leider kam auch das Ende der
Zwerge bald. Die Menschen mauerten die Zugänge zu den unterirdischen Gängen der
Zwerge zu, sodass sie sich den Menschen nicht mehr zeigen und helfen können. Sie
haben über Nacht Häuser aufgebaut oder Burgen weggerissen, den Kindern, bis
diese in der Frühe erwachten, die Hausaufgabe geschrieben, den Armen Schüsseln
voll Gold in den Kasten gelegt, sodass sie über Nacht reich wurden, und den
Hungrigen dampfende Speisen auf den Tisch gestellt, ohne dass sie nur ein
kleines Steckerl Holz verbrennen brauchten. Undank ist der Welt Lohn.

.14 Der Weiler Haberleuthen

Dieser Ort verdankt seine Entstehung dem gleichen Hadubert wie der Ort
Hafenried. Hadubert wurde abgekürzt „Hawa“ genannt. Darum wird auch Haberleuthen
im Volksmund ausgesprochen: Hawaleuthen. Der Name Haberleuthen erscheint das
erste Mal in einer Urkunde vom 19.11.1403. Damals verkauften Heinrich der
Leitner zu Haberleitn 60 Pfennige jährlicher Gilt auf einen hier bestehenden Hof
an Konrad den Nussberger zu Kollnburg gegen Erbrecht auf einen anderen Hof an
der Haberleiten und auf einer Sölde, genannt Vjechleinzöd.

1577 saß hier auf einem Gute, das mit Gilt zur Pfarrei Geiersthal gehörte, ein
Hans Stiglbauer. Der Pfleger von Viechtach war über diesen Hof Grundherr und der
dortige Landrichter war der Vogtherr. Seit 1749 wirtschaftet auf diesem Anwesen
die Familie Steinbauer:1736 war auf diesem Anwesen ein Hans Steinbauer und 1770
ein Joseph Steinbauer. In dem Gefällebuch des Rentamtes Viechtach vom Jahre 1823
treffen wir hier, zwei Höfe verzeichnet, den des Franz Plötz, ein ganzer Hof,
und den des Joseph Steinbauer, ein 4/5-Hof. Die beiden Höfe waren dem Staatsärar
grund- und zehentbar.

Bräu-Anwesen: Nach dem Grundsteuerkataster vom Jahre 1843 saß auf einem Gute in
Haberleuthen ein Franz Plötz, der dasselbe durch seine Ehefrau Anna Maria, eine
geborene Bauer um 1600 Gulden übernommen hatte. Der letzte Besitzer dieses
Anwesens, ein Franz Plötz, der Bürgermeister der Gemeinde Zachenberg war, starb
1889 auf dem Wege nach Bahnhof/Gotteszell an Schlaganfall. Das Marterl steht
oberhalb Peter, Auerbach. Seine Ehe war kinderlos. Das Anwesen, im Volksmund
Bräu-Anwesen genannt, wurde später aufgeteilt und verschwand, wenn es auch
früher eines der größten Anwesen der Gemeinde Zachenberg war, vollständig von
Erdboden.

Auf dem ehemaligen Grund des „Bräu-Anwesens“ entstanden dann später die 2
Feldmeier-Anwesen, das Niedermeier-Anwesen und das Kilger-Anwesen.

Kilger-Anwesen: Das Kilger-Anwesen kaufte 1885 ein Joseph Kilger. 1919 übernahm
es dessen Sohn Wolfgang Kilger. Sein Sohn Joseph Kilger ist seit 1944 vermisst.
Den Grund, auf dem das Feldmeier-Anwesen steht, erwarb zuerst Rauch Franz.
Dieser baute eine ganz primitive Hütte hin. Diese wurde von Feldmeier gekauft
und umgebaut.1929 haben die Feldmeierleute einen Neubau aufgeführt und 1933 das
alte Feldmeieranwesen neu gebaut.

Niedermeier-Anwesen: Das Niedermeier-Anwesen hat 1890 ein Georg Plötz gebaut.
Von dem hat es Lorenz Niedermeier gekauft und dessen Sohn Johann Niedermeier hat
es 1926 übernommen. Der Sohn von ihm, Unteroffizier Franz Niedermeier, ist am 8.
August 1942 in Russland gefallen.

Steinbauer-Hof: Den anderen Hof in Haberleuthen bemaierte 1843 ein Georg
Steinbauer. Er hatte das Anwesen von seiner Mutter Anna Steinbauer um 2 200
Gulden am 21. April 1804 erhalten. Ein weiterer diesem Steinbauer gehöriger Hof
war Erbrechtsweise grundbar zur Pfarrkirche Geiersthal. Auf diesem
4/5-Steinbauern-Hof hat 1763 ein Michl Steinbauer gehaust. 1852 wurde der
Getreidekasten dieses Hofes geschrauft und fiel dabei zusammen. Am 8.9.1899
brannte die Scheune mit samt den Erntevorräten infolge Blitzschlages ab.
Wohnhaus und Stadel wurden wegen des starken Regens und des günstigen Ostwindes
vor größerem Schaden bewahrt.

Ein kleines Elektrizitätswerk für Haberleuthen und Wandlhof wurde 1923 von dem
damaligen Hofbesitzer Johann Steinbauer gebaut, das aber dann 1936 in den
Wandlhof verlegt wurde.1910 starb dessen Sohn Joseph Steinbauer im 18
Lebensjahre. Infolgedessen musste der andere Sohn Johann Steinbauer, der schon 4
Jahre lang im Kloster Metten studierte, heim und musste nach dem Tode des Vaters
1936 das väterliche Anwesen übernehmen. Er starb aber schon 1952. 1928 wurde die
Scheune umgebaut, wodurch eine Hocheinfahrt gewonnen wurde. 1942 musste wegen
Einsturzgefahr der Stall neu gebaut werden. Das Anwesen bewirtschaftet zurzeit
die Witwe Therese Steinbauer mit ihren Kindern.

Der Weiler Haberleuthen zählt 42 Einwohner in 5 Anwesen.

.)a Der Hirsch auf dem Scheunendach

Lange vor der Bahnzeit, als es in hiesiger Gegend noch viele Hirsche gab und der
Bahneinschnitt bei dem Steinbauern-Hof in Haberleuthen noch nicht existierte,
soll im Winter bei viel Schnee ein Prachthirsch auf das Scheunendach des
Steinbauernhofes gekommen sein. Er fiel auf der anderen Seite des Daches
herunter auf einen Obstbaum. Da kamen die Steinbauern-Buben und erschlugen den
Hirsch mit dem schweren Haustürriegel.

.15 Die Einöde Hafenried

Wie unter der Herrschaft des Klosters Metten neben dem Maierhofe Lämmersdorf die
Siedlung „Giggenried“ entstand, genau so gründete hier neben dem Maierhofe
Wandldorf, später Wandlhof, ein Ministerialer mit Namen Hadubert, abgekürzt
Hawe, die Siedlung Hawenried, aus dem später die Ortsbezeichnung Hafenried
würde. Das Hafenried ist also erst nach Wandlhof entstanden. Es waren die
Gebäulichkeiten dieses kleinen Besitztums ein alter Bau, aus Findelsteinen
aufgeführt. Die alten Stallgebäude, die dabei waren, sagten uns mit Gewissheit,
dass das Hafenried ursprünglich eine selbständige Siedlung war. Diese
Selbständigkeit muss aber von nicht langer Dauer gewesen sein, da der Wandlhof
samt Hafenried im Jahre 1344 von einem Lorenz Geiß um 4 800 Gulden übernommen
wurde. Bis in die jüngste Zeit herein stand das alte Hafenried. 1951 wurden
diese 1 000-jährigen alten Gebäulichkeiten weggerissen und an ihre Stelle ein
modernes einstöckiges Wohnhaus gebaut von dem jetzigen Wandlhof-Besitzer Alois
Kraus jun.

Die Einöde Hafenried zählt … Einwohner.

.16 Die Einöde Hasmannsried

Der Name „Hasmannsried“ wird im Volksmund „Hammersröid“ ausgesprochen. In einer
Urkunde vom Jahre 1527 erscheint der Name „Hezmannsried“, später „Hazmannshried“
und „Haßmannsried“, jetzt „Hasmannsried.“ Es handelt sich sicher um eine
Siedlung eines Ministerialen namens Hosnod.

1397 vermachten Andreas von Haßmannsried und seine Ehefrau Elisabeth dem Kloster
Gotteszell ihren Teilzehent zu Poitmannsgrub unter der Bedingung, dass ihnen,
ihren Vorfahren und Nachkommen jährlich am Sonntag nach Ulrich ein ewiger
Jahrtag gehalten werde.

Oischinger-Gut: 1668 übergaben Simon Oischinger zu Haßmannsried und Barbara,
seine Hausfrau, ihr Erbrechtslehen zu Haßmannsriedsamt den völlig angebauten
Feldern unter Bareingabe von 1 Ross und 12 Stück Vieh, dann aller Wägen,
Geschirr und Baumannsfahrnis ihrem Sohn Simon Oischinger um 250 Gulden. 1763
treffen wir auf diesem Gute einen Michl Augustin, der nach dem Stifts- und
Salbuch des Klosters Gotteszell von 1790/99 zu Georgi und Micheli außer Geld
auch noch Naturalien als Gilt zu entrichten hatte. Dieser Hof war nach Aufhebung
des Klosters Gotteszell Erbrechtsweise grund- und zehentbar dem Staate. 1823
bewirtschaftete das Gut Lorenz Augustin. Er hatte das Gut von seiner Mutter
Magdalena Augustin durch Übergabe um 1 000 Gulden erhalten. 1843 befand sich bei
diesem Hof ein Grundbesitz von 97,24 Tagwerk. Nach Augustin war auf diesem Hof
eine Eder. Dieser verkaufte 1809 das Anwesen an Sperl aus Straubing. Nach 2
Jahren verkaufte Sperl die Waldung an die Papierfabrik Teisnach. Den Hof, die
Wiesen und die Felder kaufte Sagmeister Michl von Ruhmannsfelden von Sperl und
die Waldung tauschte er sich von der Papierfabrik Teisnach wieder zurück, und
gab dafür die Loderhard. Somit bekam Sagmeister den ganzen Hasmannsrieder-Hof
zusammen. Das Haus steht noch am alten Platze. Nur hat Sagmeister von dem von
dem Baumeister Gegenfurtner darauf bauen lassen (1913). Als Bewirtschafter des
Anwesens und als Waldaufseher war ehedem ein Joseph Hakker aufgestellt und jetzt
fungiert als solcher Kilger Johann. Übrigens ist der ganze Sagmeister-Michl-Wald
(früher Hasmannsrieder-Wald) heute im Besitze des Staates, sodass hier der Staat
einen Waldbesitz (mit dem Weichselsrieder- und Muschenriederwald) von 150
Tagwerk besitzt. 1944 wurden von amerikanischen Fliegern 2 Bomben im Staatsforst
und 1 Bombe im Klimmerwalde abgeworfen. Mächtige Trichter im Waldboden zeigen
uns heute noch diese Abwurfstellen.

Mock-Anwesen: Ob das zweite Anwesen in Hasmannsried auch zum Oischinger-Gut
seiner Zeit gehört hatte, ist nicht bekannt. Früher war auf diesem Anwesen ein
Joseph Hartl. Dieser verkaufte es 1882 an Jakob Dobusch von Muschenried und von
diesem kaufte es 1886 Johann Mock, der Großvater des jetzigen Besitzers Johann
Mock.

Die schöne Kapelle von Weichselsried gehört jetzt durch den Kauf des
Weichselsriederholzes durch den Staat zu Hasmannsried und wird auch vom Staat
und von den Bewohnern der zwei Anwesen in Hasmannsried unterhalten und gepflegt.
Es finden dort im Monat Mai Andachten statt zu Ehren der Maienkönigin.

Die Einöde Hasmannsried zählt in 2 Häusern 12 Einwohner.

.)a Der Teufelsbrunn

“Muatta, Muatta“, so kam die kleine Katherl hereingestürzt in die Wohnstube beim
Waldaufseher Hans in Hasmannsried. „Was ist dir den passiert?“, fragte besorgt
die Mutter. „I hob den Deifö hänga sehgn“, schrie die Kleine aus Leibeskräften.
“Ja, wo denn?“, fragte neugierig die Mutter. „Oben, bei unserm Brunn“ (Quelle
des laufenden Hauswassers), antwortete Katherl. Nun machte sich Hans auf dem Weg
ausgerüstet mit Mistgabel und langem Messer. Wirklich! 30 Schritte oberhalb des
Hauses, beim Brunn, da hing an einem jungen, roten Baum in seiner Pflanzung
etwas, als wäre es wirklich der leibhaftige Teufel, der sich hier aufgehängt
hätte. Hans machte nochmals einen festen Schnauferer und dann ging er mit
aufgepflanzter Mistgabel an den Teufel näher heran. Da sah er aber deutlich,
dass das nicht der Teufel, sondern ein Rehbock war, mit seinem Schädel, mit
seinem schönen Gwichtl, der knallroten Sommerdecke und den vier Läufen. Das
Essbare vom Rehbock war verschwunden. Später hat sich herausgestellt, dass es
sich dabei um einen von Wilderern erlegten und ausgeweideten Rehbock gehandelt
hatte. Was die Wilderer vom Rehbock nicht brauchen konnten oder sie verraten
hätte können, haben sie an den Baum gehängt als Spott für die Jäger und als
Schreck für alle, die zum Brunnen kommen um Wasser, um den ersten Brunnkress
oder um den schönen Brunnsalat, Hans aber hat diesem Schreck gleich abgeholfen.
Die Rottanne steht nicht mehr, der vermeintliche Teufel ist an Ort und Stelle
vergraben und das schöne Gwichtl ziert die Wohnstube des Waldaufsehers in
Hasmannsried.

.17 Die Einöde Hausermühle

Die Hausermühle hatte am 21.3.1800 ein Franz Freisinger von seinen Eltern Michl
und Maria Freisinger um 1 200 Gulden übernommen. Die Hausermühle gehörte früher
mit der niederen Gerichtsbarkeit zur Hofmark March und Zell. Später war hier die
in hiesiger Gegend reich verzweigte Familie Kramheller ansässig. Heute
bewirtschaftet die Hausermühle wieder ein Freisinger.

In nächster Nähe der Hausermühle steht das Ausnahmshaus. Es wird seit 1930
bewohnt und bewirtschaftet von Freisinger Kreszenz. Während der Bahnbauzeit war
in diesem Anwesen ein Büro der Bahnbauleitung (1875 bis 1877). 1938 ist das
Anwesen abgebrannt.

.)a Die „Streit“ und der „Friedbach“ bei March (M. Waltinger)

Der Hausermühlbach heißt von seinem Ursprung weg bis zur Hausermühle Zeuserbach,
von da ab bis zur Pometsauermühle Hausermühlbach, von da weg bis zu seiner
Mündung in die Ohe bei der Reithmühle „Friedbach.“

Am Jakobi-Tag 1742 rückte der Pandurenoberst Trenk mit seinem gefürchteten
Soldaten im Markte Regen ein. Schon am folgenden Tage kamen ein paar Dutzend
seiner Leute nach March und verlangten Lebensmittel. Sie wollten sich aber mit
dem, was ihnen angeboten wurde nicht begnügen und schickten sich an, das Dorf zu
plündern. Den mutigen Ortsbewohnern gelang es jedoch sie davon abzuhalten und
aus March zu verdrängen. Aber auf einem freien Platze vor dem Dorfe kam es noch
einmal zum Streite. Doch die Panduren sahen ein, dass sie in der Minderheit
waren und gingen einen friedlichen Ausgleich ein. Der Platz, an dem der Streit
stattgefunden hat, heißt heute noch „Streit“ und der Bach in der Nähe, an dem
sich die Panduren zurückgezogen haben und von dem aus die dann ihre friedlichen
Anträge machten, wird „Friedbach“ genannt.

.18 Der Weiler Hinterdietzberg

Um das Jahr 1300 wird dieser Ort Tieroltsperg genannt. Es war dies die Siedlung
eines Tirold. Der Name Tiroldsberg ist also ein echter Bergname, weil er mit
einem Personennamen zusammengesetzt ist. Freilich wurde im Laufe der Zeit auch
dieser Name anders geschrieben. Wahrscheinlich wurde zuerst daraus ein
“Interditzberg“xxxv. Im Volksmund bedeutet inter soviel als unter. Also war
Hinterditzberg das untere oder tiefer gelegene Ditzberg gewesen und das
„Vorderditzberg“xxxvi wäre dann das obere oder höher gelegene Ditzberg gewesen.
Später wurde dann Hinterdietzberg geschrieben. In der Hauptsteuerbeschreibung
des Gerichtes Viechtach von 1721 werden in Hinterdietzberg 2 Höfe genannt:

Peternhansl-Hof: Ein Peter Sailer (Saller), auf einem ganzen Hofe sitzend und
Erbrechtsweise zum Kloster Niederalteich gehörig. 1823 war ein Georg Sailer auf
diesem Hofe. In diesem Jahre ist der Hof abgebrannt. Später bewirtschafteten den
Sailer-Hof die Geschwister Peter, Hansl, (darum heute noch beim Peternhansl)
Girgl und Annamirl (Sailer) dieses große Anwesen. Ein weiterer Bruder der
genannten Geschwister war Franz Thadäus Sailer, der Benediktinerpaterrat in
Niederalteich war. H. Hr. P. Damian Merk OSB (Archivar) im Kloster Niederalteich
teilte hierüber Folgendes mit: „In einem Professbuch 111/364 steht Folgendes:



P. Thadäus (Franz) Sailler (Saller) geb.18.11.1774 zu Dietlsperg (oder
Dietzberg).

Profess in Niederalteich 23. Juni 1799,

Ordinariusxxxvii in  Passau 6. Juli 1800,

Primiz in Niederalteich 10. August 1800,

Kooperator in Gravenau 1803 bis 1837,

Kommorant in Gravenau 1837 bis 1848,

Gestorben in Gravenau 2. September 1848.“



H. Hr. Pfr. Rankl von Grafenau teilte Folgendes mit:



“Franz Thaddä Sailler (wohl verschrieben für Saller), Exkonventual des Klosters
Niederalteich, Benediktinerordens und von 1804 bis 1840 Kooperator in Grafenau,
verstorben in Grafenau Nr. 47 (dem heutigen Mesnerhaus, das der Kirche gehört),
an Brand am 2. September 1848 um 8 1/2 Uhr morgens, beerdigt am 4.Sept.durch Pfr.
Stephaner. Er war 74 Jahre alt und wurde von Pfarrer Stephaner versehen.“



In der hiesigen Allerseelenkapelle ist in der Nordwand die Gedenktafel dieses
Priesters eingelassen. Sie trägt die Inschrift:



„Denkmal

des Hochwürdigen Herrn

Franz Thaddä S a i l e r

(merkwürdig heißt es auch hier wieder Sailer) Conventual des
Benediktinerklosters Niederalteich und nach Auflösung desselben Cooperator in
Grafenau bis zum Jahre 1840. Er war ein Muster der Demut und ein Freund der
Armen. Geb. 18. November 1774, zum Priester geweiht am 6. Juli 1800, gest. am 2.
September 1848.

R. I. P.“



Nachträglich heißt es in der Mitteilung von Niederalteich:



“Im Taufbuch, wo der taufende Priester eingetragen ist, heißt es ständig Saller
und nicht Sailer bis zum Jahre 1830, dann auf einmal Sailer mit Beginn eines
neuen Bandes.“



Von diesem Benediktinerpater wird erzählt, dass er 1823 beim Brande des
Peternhansl-Hofes in Hinterditzberg zufällig anwesend war, dass er etwas
Geweihtes in den Brand geworfen habe und dass darauf hin das Löid (= Licht,
Feuer) sofort kleiner wurde. Diesen Peternhansl-Hof haben die alten
Sailergeschwister einem Steinbauer Michl von Haberleuthen vermacht,
unentgeltlich und ohne jede Verbindlichkeit. Nach diesem übernahm diesen Hof
dessen Sohn Michael Steinbauer und heute bewirtschaftet ihn wieder dessen Sohn
Peter Steinbauer. Auf diesem Hof sind viele Umbauten und Neubauten im Laufe der
Zeit vorgenommen worden so hat auch der jetzige Besitzer seine eigene
Elektrizitätsanlage. Das Ausnahmshaus wurde 1938/39 vollständig neu
hergerichtet. Ein Bruder des jetzigen Besitzers des Peternhansl-Hofes Xaver
Steinbauer musste 1944 bei einem Fliegerangriff in Italien sein junges Leben
lassen. Er baute sich 1937/38 ein landwirtschaftliches Anwesen in
Hinterdietzberg, das außerhalb der Ortschaft Hinterdietzberg steht.

Carlbauer-Hof: Der Carlbauer-Hof, der 1721 von einem Georg Carl bewirtschaftet
wurde.1823 finden wir einen Johann Steinbauer auf diesem Carlhof und später
einen Michl und die beiden Brüder Joseph und Franz Brunner, welche Häuserhändler
waren und die auch das heutige Glasschröder-Anwesen in Ruhmannsfelden und später
ein Anwesen in Schaching bei Deggendorf aufkauften.1899 übernahm Michl Brunner
das Carlbauern-Anwesen. Nach dessen Tode 1921 bewirtschaftete über 25 Jahre lang
die Witwe Maria Brunner den Carlbauern-Hof, bis dann 1947 deren Sohn Joseph
Brunner das Anwesen übernahm. Ein Sohn der Carlbäuerin Alois Brunner, ist am 20.
Juli 1941 gefallen. Neben dem 1902 neu gebauten Wohnhause des großen Hofes steht
das 1947 renovierte Ausnahmshaus.1949 wurde der Stall neu gebaut. In ganz
geringer Entfernung vom Hofe, etwas höher gelegen, befindet sich eine Kapelle.

Der Weiler Hinterdietzberg zählt in seinen 5 Häusern 35 Bewohner.

.)a Die „Stoarieglkapelln“

Beim Carlbauern in Hinterdietzberg gab es eine ganze Stubn voll Kinder. Der
älteste, der Peterl, sollte aus der Feiertagsschule kommen und die jüngste, die
Liserl (Elisabeth), ein liebes, nettes, aufgewecktes Waldlermädel. Diese war der
Liebling des Vaters und durfte sich deshalb auch etwas mehr erlauben als die
anderen Geschwister. Wenn die Liserl fehlte, so war sie halt beim Vater auf der
Wiese oder auf dem Felde oder im Wald. Eines Abends kam der Vater von der Arbeit
zurück ohne Liserl. Sofort begann das Rufen und Schreien nach allen Richtungen,
das Suchen und Fahnden und erst um Mitternacht kehrte alles wieder in die Stube
zurück aber ohne Erfolg. Liserl wurde nicht mehr gefunden. Der Vater sagte: Da
beten wir einen Vater unser, wenn wir 's Liserl gefunden haben. „--“ Einen
Rosenkranz beten wir“, rief die Mutter dazwischen. Und der Peterl, der ja auch
die kleine Liserl recht gern hatte und dem das Fehlen des Schwesterchens schon
sehr zu Herzen ging, meinte: „Gell, Vater! Da bauen wir eine Kapelle an der
Stelle, wo wir die Liserl wieder finden!“ Der Vater sagte: „Peter, das wird
gemacht!“ Am anderen Tag kam schon frühzeitig die Brottragerwabn. Diese hat die
Sache von der Liserl in den Nachbarhäusern schon gehört. Sie erzählte, dass sie
eine Zigeunerin gesehen habe mit einem Kind. Kurz darauf kam der Postbote.
Dieser hat aus der Zeitung gelesen, dass in Rinchnach ein Wolf gesehen worden
sei und dass die Eltern recht auf ihre Kinder Obacht geben sollten. Zum,größten
Glück kamen an diesem Tage doch nicht noch ein paar Hausierer, wie an sonstigen
Tagen. Aber nur ein paar Minuten! Da steckte der Kaminkehrer-Schorsch seinen
schwarzen Kopf zur Türe herein und fragte ob die Bäuerin doch keine Wäsche auf
dem Boden droben habe. Die Carlbäuerin gab ihm keine Antwort. Da fragte er die
Carlbäuerin, ob sie heute morgens vielleicht mit dem linken Fuß zuerst aus dem
Bette heraus gestiegen sei. Die Carlbäuerin erzählte dem Kaminkehrer-Schorsch
ihr Leid. Wie sie fertig war, gestand der Schorsch der Carl-Bäuerin, dass er vor
wenigen Minuten, wie er durch den Steinriegel ganz in der Nähe des
Carlbauern-Hofes ging, ganz leise das Weinen eines Kindes gehört habe. Da schrie
die Carlbäuerin dem Vater und dem Petern und nun rannten sie, der Vater, der
Peterl, der Kaminkehrer-Schorsch, der Knecht und die Dirn hinüber in den
Steinriegel oberhalb des Hofes. Und wirklich! Da saß die Liserl, unversehrt, die
gepflückten Blumen neben sich. Sie muss am späten Nachmittag des Vortages auf
der von der Sonne erwärmten Steinplatte eingeschlafen sein und hat die ganze
Nacht durchgeschlafen bis sie die hellen Sonnenstrahlen des späten Vormittags
des anderen Tages aufweckten. Und dabei ist ihr gar nichts passiert. Der Vater
nahm sie auf seine Arme und trug sie heim. Tränen gerührt voll Freude sagte die
Mutter: „Ja, die Kinder haben wirklich einen Schutzengel!“ Der Vater hielt sein
Versprechen, ließ oben im Steinriegel in Hinterdietzberg eine Kapelle bauen, die
heute noch steht und von dem Carlbauern instand gehalten wird.

.19 Die Einöde Hochau

Es ist ein großer Talkessel, der umsäumt wird im Süden von einem Bergzug, im
Norden von welligen Wiesen und Ackerland und jetzt von einem Bahndamm durchzogen
wird. Der Talkessel war früher ein See, bis sich das Wasser einen Abfluss nach
Westen bahnte. Das ganze Tal wurde im Laufe der Jahrhunderte eine Aue. Der
östlichste Teil davon liegt höher, daher der Name Hochau. Auf der östlichen
Seite der Hochau fließt das Wasser, das von der Habischrieder Hochfläche
herabfließt, am Gaißruck vorbei, nach Osten dem Regenflusse zu, sodass man hier
von einer Wasserscheide sprechen kann.

In Hochau sind drei Anwesen:

Michael Peter, jetzt Peter Schiefeneder, Joseph Dachs und das erst 1951 neu
gebaute Anwesen des Max Brumbauer. Von Joseph Dachs ist der Sohn Michael Dachs
1916 gefallen.

Die Einöde Hochau zählt in seinen 3 Anwesen zurzeit 23 Einwohner.

.20 Das Dorf Kirchweg

Der Name Kirchweg (schreibt Schmidt) bedarf keiner Erklärung. Im Volksmund heißt
das Dorf „Kirweg“. Das Auffallende dabei ist nur, dass durch Kirchweg überhaupt
kein Weg oder eine Straße führt. Der Weg von Zachenberg nach Ruhmannsfelden ist
oberhalb der Ortschaft durch den Bruckbauernberg. Kirchweg hat keine grosse
Vergangenheit, weil es erst unmittelbar vor der Bahnbauzeit (1870/77) oder
gleich nach derselben entstanden ist.

Treml-Anwesen: So war z. B. das schöne Besitztum des Wolfgang Treml bei der
Bahnbauzeit eine Bierhütte. Dort hieß es von 1875 bis 1932 Maria Klimmer. 1932
hat dieses Klimmer-Anwesen Wolfgang Treml erworben. 1933 hat Treml den Backofen
gebaut, 1935/36 den Stall, 1946 die Autohalle und 1947 das Haus neu gebaut. 1939
hat Treml in Gotteszell/Bahnhof ein Lagerhaus gebaut, 1946 dasselbe vergrößert.
1948 hat er das Lagerhaus Doll in Gotteszelle/Bahnhof käuflich erworben und 1951
das Lagerhaus Triefenried gebaut. Der Sohn Joseph Treml ist seit 1942 bei
Stalingrad vermisst. Ab 1950 wurde das Besitztum des Wolfgang Treml mit
elektrischem Strom von Auerbach aus versorgt und seit 1952 ist das Anwesen an
das Überlandwerk angeschlossen.

Hagengruber-Anwesen: Das Anwesen des Michael Hagengruber dürfte das älteste
Anwesen von Kirchweg sein, da seine Entstehung schon auf 1863/64 zurückgeht,
also in die Zeit vor der Bahnbauzeit. Früher hieß es auf dem Anwesen Pfeffer
Lorenz. Alois Hagengruber erwarb es 1893 und Michael Hagengruber heißt es seit
1920. 1925 wurde der Stadel neu gebaut und 1931 Wohnhaus und Stall. Seit 1953
ist das Anwesen elektrifiziert. Hagengruber Alois, ein Bruder des
Anwesensbesitzers ist 1917 in den Karpaten gefallen.

Löffler-Anwesen: Auf dem Löffler-Anwesen hieß es ab 1903 Joseph Löffler und seit
1932 Alois Löffler. Ein Bruder der Frau Löffler, Karl Marchl ist seit 1944 in
Italien vermisst.

Brem-Anwesen: Auf dem Brem-Anwesen hieß es 1888 Michael Brem. Später war dann
Holzapfel Ludwig, Schneidermeister darauf und seit 1949 ist Ludwig Petersamer
der Besitzer dieses Anwesens. 1931 wurde die Schupfe gebaut und 1952 das
laufende Wasser eingerichtet. Im 1. Weltkrieg ist Brem Michl 1917 gefallen und
im 2. Weltkrieg kam der Anwesensbesitzer Max Brem 1945, nachdem er 5 Jahre lang
Wehrmachtsdienst geleistet hatte, krank heim und ist am 29. Juni 1946 an den
Folgen der im Felde sich zugezogenen Erkrankung gestorben.

Kraus-Florl-Anwesen: Auf dem Kraus-Florl-Anwesen in Kirchweg heißt es heute seit
1931 Marchl Joseph. Ein Stiefbruder der Frau Marchl, Joseph Kraus, erlernte die
Brauerei, wurde zum Militär eingezogen und ist dann am 25.10.1917 in Flandern
gefallen. Das Inhaus wird wahrscheinlich der Vater des Kraus Florl gebaut haben.

Steiglbauern-Gutes: Der jetzige Besitzer des Steiglbauern-Gutes ist seit 1945
Joseph Steiglbauer. Dessen Vater hat 1902 auf dieses Anwesen hingeheiratet.
Zuvor hieß es beim Weber Wolfgang und vor diesem Brückl.

Seit Februar 1953 hat Kirchweg das Elektrische.

Kirchweg zählt in seinen 6 Häusern 55 Einwohner.

.21 Das Dorf Kleinried

Im herzoglichen Salbuch vom Jahre 1280 erscheint bereits dieser Ort unter dem
Namen „Gnänried „. Gnän war ein althochdeutsches Wort und bedeutete „klein“.
Also ist dieser „ried“-Name mit keinem Personennamen zusammengesetzt ist demnach
kein echter „ried“ -Name. Im Laufe der Zeit wurden diese urkundlichen Namen so
unleserlich geschrieben, dass aus den ursprünglichen Namen ganz andere Namen
daraus gemacht wurden, wie z. B. hier aus Gnänried ein Gnadenried und aus
Gnänhof ein Knabenhof. Lange Zeit wurde in den gemeindlichen Akten der Gemeinde
Zachenberg der Ortsname Gnadenried geführt, bis dann sich doch der wirkliche und
echte Name Kleinried durchsetzte.

1335 verkauften Heinrich von Heygestorf und seine Frau Mechtildis an den
Schwager Cunrad von Gnänried und seine Hausfrau Margaret 2 Äcker und eine Wiese
um einen Schilling Regensburger Pfennig. 1577 saß hier auf einem Gute Leonhard
Krampfl. Später hauste hier ein Georg Riedler. Auf einem anderen Anwesen
wirtschaftete 1577 ein Michl Donhauser, der Gilt an das Kloster Oberalteich zu
entrichten hatte. Später treffen wir hier einen Joseph Krauß. Am 8. Juli 1586
verlieh Hans Sigmund Freiherr von Degenberg sein Gut zu Gnänried gegen Geldgilt
dem Michl Kandl, seiner Hausfrau Katharina und etwaigen Erben. 1658 mussten
Georg Riedler und Georg Muhr, beide von Gnänried, ihren Zehent an das Kastenamt
Viechtach abliefern. 1692 heiratete die Witwe Ursula Rauscher von Gnadenried den
Christoph Kraus von Auhof. 1650 verkaufte ein Khräpl oder Khräußl von Gnadenried
seine hier besitzende und nach Oberalteich grundbare Sölde mit Zubehör dem
Kloster Gotteszeller Untertan Stephan Prunner. Nach der Hauptsteuerbeschreibung
des Gerichtes Viechtach vom Jahre 1721 bestanden hier zwei halbe Höfe, von denen
der eine in den Händen von Michl Riedler und der andere aber in denen von Joseph
Krauß waren. Nach dem Sal- und Stiftsbuch des Klosters Gotteszell aus der Zeit
von 1790/99 saß damals die Familie Riedler auf einem hiesigen Erbrechtshof. Im
Gefällebuch des Rentamtes Viechtach vom Jahre 1823 sind hier die folgenden
Anwesensbesitzer vorgetragen:



Michl Riedler

Johann Krauß

Johann Brunner

Michl Hackl, Zimmermann auf einer Neusiedlung



Nach dem Grundsteuerkataster vom Jahre 1843 waren in Kleinried die folgenden
Familien ansässig:



Wolfgang Riedler vom Vater Pichl Riedler um 1 950 Gulden übernommen, seit 1848
Walburga Riedler.

Baltasar Brunner vom Vater Johann Brunner 1825 um 1 800 Gulden übernommen.

Georg Krauss von der Mutter Katharina Krauß 1828 um 1 100 Gulden übernommen.

Johann Hackl, seit 1846 dessen Sohn Michl Hackel



Riedler-Anwesen: Das alte Riedler-Anwesen wurde von 1893 bis 1929 von Albert
Handlos bewirtschaftet und 1929 hat es dessen Sohn Albert Handlos übernommen.
Der Sohn Johann Handlos ist am 17.4.1942 gefallen. Von diesem ehemaligen
Riedler-Anwesen wurde 1925 der Bumsenberg an Dr. Ziegler in Regensburg
wegverkauft, der dort einen großen Steinbruch eröffnete. Hr. Dr. Ziegler wollte
auch von diesem Steinbruch zum Lagerplatz Bruckmühle eine Schmalspurbahn
errichten. Aber die Ungunst der damaligen Zeit hat diesen Plan zunichte gemacht,
umso mehr, als auch die Arbeit in diesem Steinbruch lahm gelegt wurde. Das alte
Riedler-Anwesen zeigt sich nicht mehr in seinem ursprünglichen Aussehen, da es
umgebaut wurde.

Kraus-Anwesen: Das alte Kraus-Anwesen dürfte aber heute noch das alte sein, das
1721 einem Joseph Kraus gehörte. 1823 gehörte es den 2 Brüdern Georg Kraus und
Anton Kraus. Georg Kraus hat dann an seinen Bruder Anton Kraus Grund abgetreten
und dieser hat sich dann selbst ein landwirtschaftliches Anwesen gebaut, das
heute von Jakob Riedl bewirtschaftet wird. Nach den Gebrüdern Kraus gehörte das
Anwesen einem Achatz Franz, der am 29. Juli 1933 bei der Einfahrt des Zuges in
die Station Gotteszell vorzeitig vom Zuge absprang und dabei tödlich
verunglückte. Bis 1937 bewirtschaftete diesen Hof dessen Witwe Theodolinde
Achatz. Dann übernahm Michael Brumbauer das Anwesen, der aber am 2. Mai 1945
sein Leben für das Vaterland opfern musste. Heute heißt es auf diesem Anwesen
Sixt Wolfgang.

Brunner-Anwesen: Auch das Brunner-Anwesen ist schon ein altes Anwesen in
Kleinried. Von 1899 bis 1921 war ein Brunner Joseph der Besitzer dieses Hofes.
Bis 1933 hauste dann die Witwe weiter und diese übergab 1933 den Hof an ihren
Sohn Jakob Brunner.

Ernst-Anwesen: Das Ernst-Anwesen soll 1840 gebaut worden sein. Die früheren
Besitzer dieses Anwesens waren Menacher Wolfgang, Michl Leidl und seit 1909
Ernst Johann. Heute befinden sich in diesem Anwesen eine Krämerei und eine
Gastwirtschaft, geführt von Georg Ertl. Der Sohn Xaver Ernst wurde in Russland
schwer verwundet und ist dann in einem Lazarett in Sachsen am 29. August 1944
gestorben.

Riedler-Anwesen: Das Anwesen des Xaver Riedler, das jetzt seit 1921 Franz Achatz
bewirtschaftet, wurde 1859 gebaut. 1924 hat Franz Achatz den Stadel neu gebaut.
Ein Sohn, Joseph Achatz, ist bei Riga vermisst und ein Sohn, Franz Achatz, ist
in russischer Gefangenschaft.

Achatz-Anwesen: Wo früher ein Brechhaus stand, da steht heute das Anwesen des
Joseph Achatz, das bis 1916 von Albert Pfeffer bewirtschaftet wurde. Dann wurde
dieses Anwesen von Joseph und Maria Brunner gekauft und seit 1925 ist Joseph
Achatz der Eigentümer dieses Hofes. Von diesem Achatz waren 4 Brüder im 1.
Weltkriege im Felde und alle vier sind sie glücklich heimgekehrt.

Pritzl-Anwesen: Auf dem heutigen Pritzl-Anwesen hieß es früher Hartl Jakob, dann
Hartl Lorenz, genannt Oisch-Jagl-Lenz, und später war auf diesem Anwesen der
Schuhmachermeister Joseph Schrötter, genannt „Oisch-Schuster“. 1919 wurde das
Anwesen gekauft von Xaver und Therese Pritzl. 2 Söhne sind vermisst, Alois
Pritzl in Elbing und Michl Pritzl in Russland. Der Sohn Xaver Pritzl kam aus
Russland krank zurück und ist daheim gestorben.

Hofmann-Anwesen: Der Hausname auf dem Anwesen des Alois Hofmann ist heute noch
Kopp. Die Eheleute Joseph und Katharina Hofmann haben das Anwesen vom Großvater
Kopp übernommen 1902 und jetzt ist deren Sohn Alois Hofmann der Besitzer des
ehemaligen Kopp-Anwesens. Stall und Stadel wurden 1948/49 neu gebaut.

Greger-Anwesen: Das frühere Greger-Anwesen wurde später von einem Peter
Kellermeier bewirtschaftet und 1911 von Wolfgang und Katharina Sixt erworben.
1947 ist Wolfgang Sixt gestorben und seit dieser Zeit hausen auf diesem Hofe die
Witwe mit ihren Kindern. Der Sohn Joseph Sixt ist am 23. März 1942 gefallen bei
Leningrad und Martin Sixt wird vermisst bei Königsberg. Zu Kleinried gehören
auch die beiden Holzapfelhäuser von denen das eine von der früheren Besitzerin
Walburga Ernst eingesteigert wurde.

Das Dorf Kleinried zählt in seinen 13 Häusern 116 Einwohner.

.)a Dö guade, altö Zeit

Der alte Riedler von Kleinried und der noch ältere Hiaslbauer von Habischried
waren keine Todfeinde. Saßen sie aber auf der Bierbank nach der 10. Maß noch
beisammen, so konnte man mit Gewissheit sagen, dass einer von den Beiden
bestimmt der Ausgeschmierte sein würde. Da der Hiaslbauer dem Riedler einen
Baum, den 5 Männer mit ausgestreckten Armen nicht umfassen hätten können, an der
Besitzgrenze zwischen den Beiden um geschnitten hatte, suchte auch der Riedler
dem Hiaslbauer etwas anzutun, sobald das sein könnte. Debattiert über den
“roten“ Baum und über den Verlauf der Waldgrenze an fraglicher Stelle hatten sie
ja ohnehin schon tage- und nächtelang und sie hatten wegen dieses Baumes schon
das Vielfache von dem vertrunken, was dieser Baum zur damaligen Zeit wert
gewesen war. Heute musste Riedler in die Schmiede. Er spannte seinen Bräunt ein
und fuhr nach Ruhmannsfelden. Da kaufte er sich beim Almerbräu eine Maß. Da der
Hiaslbauer anlässlich einer Beerdigung am gleichen Tage auch in Ruhmannsfelden
war, trafen sich die Beiden, der Riedler und der Hiaslbauer, und sie kamen nach
kurzem Gespräch auf den „roten“ Baum. Der Riedler sagte nicht viel, da er noch
nicht in Stimmung war. Außerdem wollte er doch mit seinem Bräunt gut nach Hause
kommen. Sie diskutierten solange, bis die Nacht hereinbrach und es dabei recht
schön finster wurde. Riedler lud den Hiaslbauern ein zum Heimfahren. Der
Hiaslbauer willigte ein und als dann die Beiden auf dem Wägelchen saßen, ging
die Fahrt los, aber in entgegen gesetzter Richtung. Riedler wollte den
Hiaslbauer diesmal richtig ausschmieren. Er war ja noch der Nüchternste von den
Beiden. Der Hiaslbauer schloss seine Augen, weil es zum Sehen ja ohnehin nichts
mehr gab und schlief ein. Der Riedler fuhr dem Mooshof zu. Wie sie dort waren,
half der Riedler dem Hiaslbauern vom Wagen heruntersteigen, schob ihn beim
Moosbauern zur Haustüre hinein und dann fuhr Riedler wieder über Ruhmannsfelden
Kleinried zu. Am anderen Nachmittag schrie der Hiaslbauer beim Vorbeigehen am
Riedler-Anwesen in Kleinried dem Riedler zum Fenster hinein: „Du lumpata
Riedler! Dö kos dö nochmal an Baum!“ Der Riedler lachte herzlich und erwiderte:
„Auf an Baum gehts net drauf zam! Ausgschmiert hab i di do!“

.22 Die Einöde Klessing

Der Name Klessing erscheint in den früheren Urkunden verschiedentlich,
geschrieben als: Chlessling, Clebsing, Klepsing, Clössing, auch als Stemmer-Hof.

Der Name Klessing ist kein echter „ing“-Name, da er nicht mit einem
Personennamen zusammengesetzt ist. Der Name Klessing besteht aus dem Grundwort
“sing“, was heißt brennen, sengen und aus dem Bestimmungswort „Klee“, was heißt
Klippe. Das Wort Klessing bedeutet also: Brandrodungsstelle an einer
vorspringenden Höhe. Der Name Stemmer-Hof wird hinweisen auf einen Familiennamen
entweder Stemmer oder Stömmer. Der Name Klessing kommt urkundlich zuerst am 25.
August 1343 vor. Nach dieser Urkunde gestattete Schloss Linden, dass Cunet
Eimichl von Klessing seine Alteicher Lehen seiner Frau zur Morgengab verleihe,
ebenso Wilhelm von Cierlberg das Lehen in Hasmannsried.

Klessing wird auch im herzoglichen Salbuche aus der Zeit um 1280 genannt. 1660
schuldete ein Adam Wirsingen zu Klessing seinem Eidam (= Schwiegersohn) Adam
Oischinger auf der Reisachmühle 100 Gulden. 1670 übergaben Adam Wirsingen zu
Klessing und seine Ehefrau Eva in Einvernahme mit ihrer Grundherrschaft
Gotteszell ihren Erbrechtshof Klessing mit 22 Stück Vieh, aller Haus- und
Baumannsfahrnisse ihrem lieben Eidam und ihrer Tochter Katharina um 1 000
Gulden.

Plöz-Hof: 1721 saß in Klessing ein Hans Plöz. Sein Viehstand war damals 2
Pferde, 4 Menochsen, 4 Mastochsen, 7 Kühe, 4 Jungrinder und 4 Schafe. Nach dem
Sal- und Stiftsbuche des Klosters Gotteszell aus der Zeit um 1790/99
bewirtschaftete eine Georg Plöz diesen Hof. 1823 saß ein Georg Plötz auf diesem
Hofe. 1843 war ein Franz Plöz Besitzer dieses Hofes, den er von den Eltern Georg
und Anna Maria Plöz um 2 800 Gulden übernommen hatte. Vor 1900 bewirtschafteten
diesen Hof Joseph und Walburga Plötz und von 1900 bis 1904 Martin und Therese
Plötz. 1904 wurde dann das Anwesen an Xaver Brem, Stegmühle, verkauft, der es an
Kauschinger, Hochstadt, verpachtete. 1905 kaufte den Hof in Klessing ein Michl
Muhr aus Masselsried, der den Hof 1907 an Ludwig Kandler, Rautenstock bei
Kostenz verkaufte. 1908 wurde aber der Hof versteigert. Im November 1908 bezog
Michl Artmann aus Bärwinkel dieses Anwesen, der es 1914 an seinen Alois und
dessen Ehefrau Maria Artmann geb. Steinbauer von Haberleuthen übergab. 1932
verunglückte genannter Alois Artmann beim Getreideeinfahren, an dessen Folgen er
auch am 29. August 1932 starb. Bis 1952 bewirtschaftete die Witwe Maria Artmann
den Stömmer-Hof. Im Oktober 1952 übergab sie den Hof ihrem Sohn Alois Artmann
und dessen Ehefrau Mathilde geb. Fritz von Asbach und baute sich ein schönes
Ausnahmshaus. Im 2. Weltkrieg standen 5 Söhne der Frau Artmann von Klessing an
der Front. Zwei davon, Max und Michl Artmann, sind im Osten gefallen. Früher war
auch ein Brechhaus beim Stömmer-Hof und am Wandlbach stand die Stömmermühle.

Die Einöde Klessing zählt in zwei Häusern 18 Einwohner.

.23 Das Dorf Köckersried

In den ältesten Urkunden erscheint der Name Chkokleinsried, Chokleinsried,
Chroklensried, in einer Urkunde vom 23.4.1383 Chrötzleinsried, in einer Urkunde
vom Jahre 1577 Khockhesriedt und 1333 Khögleinsriedt. Dieser Ort war der
Rodungsplatz eines Cotger, eines Cote, eines Coteschalk, heißt also richtig in
der Zusammensetzung mit dem Personennamen „Coteschalksried“ Dieser Ort ist also
ein echter „ried“-Ort, weil er zusammengesetzt ist mit seinem Personennamen.

Nach der Urkunde vom 23.4.1383 gaben Ulrich der Murr und seine Hausfrau
Kunigunde dem Kloster Gotteszell ihre freieigene Sölde zu Chrötzleinsried und
ihren Zehent zu Pergern, Weihmannsried, Grub, Weinried, Niederried und Oberried
unter der Bedingung, dass das genannte Kloster ihnen am Klosterhof Gotteszell
ein Gemach zimmern lassen und jedem täglich 3 Herrenbrote und 2 Kopf Herrenbier,
an den Tagen aber, an denen die Klosterherren ein Ader haben, Ulrich dem Murr 4
Kopf Wein verabreichen und außerdem im Winter noch für Murr 2 Stück Rind und 1
Schwein unterhalten sollen. 1404 waren hier die Nussberger begütert. Nach dem
herzoglichen Salbuch des Gerichtes Viechtach vom Jahre 1577 hatte die
Schlossherrin von Neu-Nussberg, Frau Rosina von Stauff, in Köckersried 2 Güter
inne, welche dem Schlosskaplan von Neunussberg giltpflichtig waren. Wir sehen
daraus, dass in Neunussberg ein eigener Schlosskaplan war, wie es ja bei allen
Schlössern und Burgen von irgendeiner Bedeutung in jener Zeit Sitte war. Auf dem
einen dieser Neunussberg'schen Güter in Köckersried saß 1577 ein König
xxxviiiund auf dem anderen ein Wolf Klein. Als Besitzer von anderen bäuerlichen
Gütern finden wir 1577 in Köckersried:



Georg Poschinger, später die Familie Sailer

Georg Ortmayer

Albrecht Kramheller auf dem sog. Edenhofer-Gute,später Familie Kraiberger

Wolfgang Klein, später die Familie Brunner

Christoph Schuster, später die Familie Loibl

Leonhard Mock, später die Familie Ziselsberger

Andreas Mock, später die Familie Keller.



1646 verkaufte Barbara, die Hausfrau des Wolf Schreindorfer zu Auerbach ihr mit
Grund- und Baurecht dem Kloster Windberg gehöriges Lehen zu Köckersried um 500
Gulden an Andreas Zoller zu Zachenberg. 1670 wird ein Georg Sailler als
Hofbesitzer in Köckersried genannt. 1671 verkauften Thomas Traiber zu
Köckersried und seine Hausfrau Margarete ihr Erbrechtslehen an Christoph Sailler
zu Köckersried um 292 Gulden und 3 Taler Leykauf. 1699 verkauften Hans Tremel zu
Köckersried und seine Hausfrau Barbara ihre hiesige Erbrechtssölde an Michl
Hacker zu Köckersried. 1714 übergaben Georg Loibl zu Köckersried und seine
Hausfrau Elisabeth ihr Erblehen in unserem Dorfe um 514 Gulden ihrem Sohne Jakob
Loibl. Nach der Hauptsteuerbeschreibung des Gerichtes Viechtach vom Jahre 1721
bestanden zu jener Zeit in Köckersried die folgenden Anwesen:



Hans Keller

Georg Hackher

Wolfgang Hackher

Jakob Loibl

Andre Sailler

Wolf Krampfl



Um 1830 bestanden in Köckersried 2 ganze Höfe, 5 halbe Höfe, 1 Sölde und eine
Ansiedlung. Der Grundsteuerkataster von Jahre 1843 verzeichnet in Köckersried
die folgenden bäuerlichen Anwesen:



Michl Saller

Andreas Kandler, später Joseph Kandler

Johann Kopp

Lorenz Brunner, früher Jakob Brunner

Michl Koller

Jakob Hacker, früher Joseph Hacker

Georg Hacker, früher Johann Hacker

Anton Schedlbauer, er kaufte sein Anwesen am 13. Juli 1829 von Georg Hacker.



Schwaighofer-Anwesen: Das Anwesen, das früher der Franziska Schwaighofer
gehörte, war ursprünglich ein Leitthumshaus und wurde erst von Franziska
Schwaighofer ausgebaut. Seit 1931 heißt es auf diesem Anwesen Johann Kandler.

Kraus-Anwesen: Auf dem Anwesen des früheren Joseph Kraus heißt es jetzt Johann
Plenk, seit 1919. Dieser hat 1933 den Stadel und 1949 den Stall gebaut und 1943
das Haus ausgebaut. Der Sohn Joseph Plenk ist seit 1944 in Rumänien vermisst.

Brunner-Anwesen: Anton Gierl ist der Besitzer des Anwesens Nr. 137 seit 1910 und
des Anwesens Nr. 134. Dieses war das frühere Brunner-Anwesen.1929 und 1948 hat
Hr. Gierl an beiden Anwesen bedeutende Bauarbeiten vornehmen lassen. Der Sohn
Xaver Gierl ist 1943 in Afrika gefallen. Der Sohn Anton Gierl ist in einem
Donau-Altwasser bei Straubing ertrunken.

Löffler-Häuschen: Ein Häuschen im alten Waldlerstil, etwas rückwärts gelegen,
ist das Haus der Maria Löffler, das früher dem Pensionisten Pledl gehörte.

Hinkofer-Gastwirtschaft: Die Gastwirtschaft in Köckersried, früher Theodor
Hinkofer, dann Kopp, dann Schmitzberger, gehört jetzt Hr. Karl Raith seit 1938.
Das alte Wirtshaus in Köckersried ist vollständig umgebaut. Es enthält einen
geräumigen Saal, ein Gastzimmer und 1 Nebenzimmer. Im Obergeschoss sind lauter
Mietwohnungen und die Unterkellerungen dienen als Garagen.

Saller-Anwesen: Auf dem Saller-Anwesen hieß es früher immer Michl Saller bis
1949. In diesem Jahre übernahm Franz Saller den Hof. Franz Saller kam als Knabe
in die Häckselmaschine (Handbetrieb) mit der rechten Hand, wobei er die vorderen
Glieder der beiden Mittelfinger der rechten Hand verlor. Ein Bruder von ihm,
Johann Saller, wird seit 1942 im Osten vermisst.

Kandler-Anwesens: 1843 war der Besitzer des Kandler-Anwesens in Köckersried ein
Andreas Kandler. Ab Mai 1846 hat dann ein Joseph Kandler um 1 500 Gulden
übernommen. Dessen Sohn Joseph Kandler bewirtschaftete diesen Hof bis 1937. In
diesem Jahre erwarb dieses Besitztum ein Joseph Steer, jetzt Michael Steer.

Hacker-Anwesen: Auf dem früheren Hacker-Anwesen, später Johann Brem, heißt es
jetzt Joseph Oischinger. Das Haus wurde 1951 ausgebaut.

Krampfl-Anwesen: Das Anwesen des Georg Krampfl und des späteren Joseph Krampfl
bewirtschaftet jetzt seit 1919 Maria Zitzelsberger. 1923 wurde Wohnhaus, Stadel
und Stall umgebaut.

Schwarz-Heigl-Anwesen: Das frühere Schwarz-Heigl-Anwesen, später beim Rager, kam
1914 in den Besitz der Familie Kilger, die es von Kern Pauli käuflich erworben
hatte. Da der Ehemann als Holzbauer tödlich verunglückte, führt die Wirtschaft
auf diesem Anwesen die Witwe Maria Kilger. Der Sohn Johann Roth ist am 18.
November 1944 in Ungarn gefallen. 1952 wurde in diesem Anwesen das Elektrische
eingerichtet.

Krampfl-Anwesen: Der Grund, auf dem heute das Krampfl-Anwesen steht, war früher
noch Waldung, die wahrscheinlich zum Kandler-Hofe gehörte. Nachdem hier gerodet
war, baute ein Kauer ein Anwesen in diesen Grund. Sein Nachfolger Jakob Krampfl
hat das Anwesen ausgebaut. Ihm folgte 1911 der Michael Krampfl und seit 2.
November 1950 bewirtschaftet diesen Hof Leopold Krampfl. Dieser hat auch schon
auf das Wohnhaus drauf gebaut. Früher hat einmal ein mächtiger Sturm das
Hausdach fortgerissen. Bei einem heftigen Gewitter 1893 wurde beim Heuen die
Ehefrau des Jakob Krampfs vom Blitze erschlagen.

Penzkofer-Anwesen: Auf dem früheren Penzkofer-Anwesen heißt es seit 1928 Georg
Kilger. Georg Spielbauer bewirtschaftet seit 1936 das Anwesen, das früher der
Balbina Kilger gehörte und das zuvor das Hacker-Hansl-Gut genannt wurde. Der
Sohn der Balbina Kilger, Johann Kilger, ist im 1. Weltkrieg in Frankreich
gefallen. Ein Sohn des Georg Spielbauer, Xaver Spielbauer, wurde in Russland
verwundet und ist an den Folgen dieser Verwundung 1943 in der Heimat gestorben
und liegt im Friedhof zu Gotteszell beerdigt.

Pöhn-Anwesen: In unmittelbarer Nähe des Spielbauern-Anwesens hat Michl Pöhn jun.
ein Wohnhaus gebaut, von dem aus man eine sehr schöne und weite Fernsicht hat.
Michl Pöhn sen. besitzt seit 1926 das Anwesen, das 1912 Hinkofer Theodor gebaut
hatte und das 1922 abgebrannt ist. Der Sohn Karl Pöhn ist im März 1945 in
Russland gefallen.

Greil-Anwesen: Das Anwesen des Sebastian Greil wird von Lorenz Greil
bewirtschaftet und das frühere Brunner Girgl-Gut von Joseph Strohmeier.

Wie bereits im 1.Teil der Heimatgeschichte von Zachenberg berichtet wurde,
sollte bei der Säkularisation des Klosters Gotteszell die Kapelle in Köckersried
niedergerissen werden, was aber die Köckersrieder zu verhindern suchten. Nach
den urkundlichen Aufzeichnungen muss das später dann doch erfolgt sein, weil die
Urkunden dann berichten, dass an Stelle der abgebrochenen Kapelle 1830 bis 1833
ein Kirchlein erstanden ist, das dem hl. Erzengel Michael geweiht ist. In diesem
Kirchlein wird weder Gottesdienst noch Andacht gehalten, wird nicht gepflegt.

Das Dorf Köckersried zählt in seinen 17 Häusern 194 Einwohner.

.24 Das Dorf Lämmersdorf

Dieses Dorf dankt seine Entstehung der seinerzeitigen Herrschaft des Klosters
Metten über das Gebiet des Nordgaues. Die Rodungsorte der damaligen Zeit führen
alle, sowohl im bayerischen Wald als auch auf dem Gäuboden, das Grundwort „dorf“
und das Bestimmungswort ist regelmäßig ein Taufname. So ist es auch bei dem
Siedlungsnamen Lämmersdorf, der ursprünglich in den alten Urkunden geschrieben
wurde: Lemmerstorff, Lempferstorf (bei Apian, Oberbayerisches Archiv 39,369),
dann Lampferstorf, Lambertsdorf. Aus diesem Ortsnamen für Lämmersdorf klingt der
Taufname Lampert heraus, sodass der betreffende Rodungsvorarbeiter
(Ministrieller) hier an dieser Stelle ein Lampert gewesen sein wird. Der Name
Lämmersdorf ist also ein echter „dorf“-Name, weil er mit einem Personenamen
zusammengesetzt ist. Die Gründung dieses Ortes muss noch vor Arnulf dem Bösen
(911 bis 937) erfolgt sein, da ja dieser Herzog bekanntlich dem Kloster Metten
den Besitz im Nordgau wieder genommen hatte. Die Orte, welche nach dieser
Säkularisation, gegründet wurden sind die „ried“-Orte. Patersdorf, Fratersdorf,
Lämmersdorf, Wandldorf sind also schon sehr alte Ortschaften. Bei Apian finden
wir ein Lempferstorf. Laut einer Urkunde vom Jahre 1295 verkauften der Herzog
Otto und Stephan an das Kloster Aldersbach unter anderem auch den Hof von
Lempferstorf um 400 Pfundxxxix.

Nach einem Gerichtsurteil vom Jahre 1650 wurden Andre Achatz von Lemmersdorf,
Hans Six von Wolfersried und Georg Schaffer von Arnbruck jeder zu einem Tag und
einer Nacht Gefängnis verurteilt, weil sie sich dem Verbote des Tabakreuchens
schuldig gemacht hatten. 1656 übergaben Michael Schauer zu Lemmersdorf und seine
Hausfrau Katharina ihre Erbgerechtigkeit auf dem Hofe zu Lemmersdorf mit allem
Zubehör, samt den angebauten Winterfeld und dem Samen zum Sommergetreide ihrer
Tochter Maria Schauer und ihrem künftigen Ehemanne Peter Limpeck von Arnetsried
um 600 Gulden. 1663 übergab Hans Zieselsberger von Lemmersdorf seinem
gleichnamigen Sohn Hans seinen mit Grund und Boden zum Kloster Gotteszell
gehörigen Hof um 250 Gulden, wovon 78 Gulden als Heiratsgut abzurechnen waren.
Sie übrigen 172 Gulden gehörten seinen 3 Geschwistern Georg, Michael und
Barbara. 1669 übergaben Michl Kraus zu Lemmersdorf und seine Hausfrau Katharina
ihr mit Grund und Boden zum Gerichte Linden gehörige Lehen samt dem angebauten
Wintergetreide und 15 Stück Vieh ihrem Sohne Hans Kraus um 400 Gulden. 1679
verkauften Hans Ziselsberger von Lembertsdorf und seine Hausfrau Katharina ihr
Erbrechtslehen, genannt das Scheybeck-Lehen, mit Ein- und Zubehör um 361 Gulden
und 3 Taller Leykauf an Michl Ziselsberger. 1692 wird uns ein Jakob Limpeck als
Besitzer eines Erbrechtshofes in Lemmersdorf genannt. Die
Hauptsteuerbeschreibung des Gerichts Viechtach vom Jahre 1721 verzeichnet in
unserem Dorfe die folgenden Anwesensbesitzer:



Joseph Stadler

Michael Augustin

Christoph Pfeffer

Michael Ziselsberger

Hans Stadler



1763 waren in Lemmersdorf die folgenden Anwesensbesitzer:



Joseph Stadler auf dem Limpeck-Hof

Michl Augustin auf dem Franzen-Gut

Michl Ziselsberger auf dem Scheybeck-Hof

Christoph Pfeffer

Andreas Stadler auf dem Krausen-Hof



1790/99 waren in Lemmersdorf nach einem Auszuge aus dem Sal- und Stiftsbuche des
Klosters Gotteszell die folgenden Anwesensbesitzer:



Michl Augustin

Georg Stadler

Michl Ziselsberger

Joseph Stadler

Johann, dann Michael Pfeffer



1823 bestanden in Lemmersdorf folgende Anwesen:



Jakob Stadler auf dem Liebl-Hof

Michl Pfeffer auf dem Pfeffer-Hof

Michl Augustin auf dem Panzer-Hof

Georg Stadler auf dem Krauss-Hof

Joseph Ziselsberger auf dem Scheybeck-Hof

Michl Limpeck auf dem Hofbauern-Hof



Sämtliche Bauern von Lemmersdorf waren dem Staatsärar Erbrechtsweise grundbar
und hatten ihren Zehent ganz an den Staat zu entrichten. Im Grundsteuerkataster
vom Jahre 1843 finden wir die folgenden Anwesensbesitzer:



Michl Limpeck

Joseph Geiger, durch die Ehefrau Anna, eine geborene Stadler um 1 800 Gulden
erworben

Wolfgang Pfeffer, vom Vater Michl Pfeffer übernommen

Michl Augustin, von der Mutter Anna, nachher verehelichte König um 1 914 Gulden
übernommen.

Andre Pfeffer, durch Übernahme, bzw. Heiratsvertrag durch die Ehefrau Katharina,
geb. Stadler um 1 450 Gulden erhalten.

Joseph Achatz, von Joseph und Ursula Ziselsberger am 27. Juli 1838 um 1 600
Gulden übernommen.



Limbeck-Hof: Auf dem Limbeck-Hof hieß es später Iglhaut.1894 kaufte Joseph
Schierer diesen Hof, den heute dessen Sohn Karl Schierer bewirtschaftet.1945 hat
den Pferde- und Schweinestall neu gebaut. Ein Bruder von dem Hofbesitzer, Johann
Schierer, ist 1918 in Frankreich gefallen.

Pfeffer-Hof: Auf dem ehemaligen Pfeffer-Hof hieß es später Kronner und nach
diesem Kronner kam dann ein Ludwig Graßl und nach diesem dessen Sohn Ludwig
Graßl, der am 12.12.1944 gefallen ist. Seit 1949 heißt es auf diesem Anwesen
Kopp Johann. Das Leithumhaus von diesem Pfeffergut hat seinerzeit Plötz Johann
von Wenzleinsgrub erworben. Seit 1914 bewirtschaftet dieses Anwesen Michael
Plötz, von dem der Sohn, Michael Plötz, gefallen ist am 13.7.1943 und Sohn,
Johann Plötz, seit 1943 vermisst ist in Russland.

Zellner-Hof: Den anderen Pfeffer-Hof hat Joseph Zellner 1882 gekauft. 1909
heiratete Johann Dachs von Lobetsried auf diesen Hof und 1938 hat dessen Sohn
Johann Dachs diesen Hof übernommen. Dieser hat 1948 den alten Hof vollständig
niedergerissen und alles neu aufgebaut. Ein Bruder von ihm, Joseph Dachs, ist
1941 in Russland gefallen.

Franzen-Hof: Auf dem Franzen-Hof hieß es früher einmal Michl Augustin. Dieser
hatte am 8. April 1814 den Franzen-Hof von seiner Mutter Anna, nachher
verehelichte König, um 1 914 Gulden übernommen. Der Grundbesitz war damals 72,59
Tagwerk. Auf diesen Franzen-Hof hat ein Wolfgang Bielmeier von Busmannsried
geheiratet und am 10.7.1899 hat dann dessen Sohn Alois Bielmeier diesen Hof
übernommen, denn seit 10.7.1929 wieder dessen Sohn Alois Bielmeier
bewirtschaftet. Ein 19-jähriger Sohn eines früheren Franzhof-Bauern in
Lämmersdorf ist vom Stadel heruntergestürzt und ist an den Folgen der erlittenen
Verletzungen gestorben.

Achatz-Anwesen: Auf dem Anwesen des Michael Achatz hieß es früher Bernhard
Steindl. Es soll 1752 erbaut worden sein. 1892 hat es Achatz Michael käuflich
erworben und 1920 dessen Sohn Michael Achatz übernommen, der das Anwesen 1936
vollständig umbauen ließ. Ein Sohn des Michael Achatz, nämlich Franz Achatz, ist
auf der Insel Krim vermisst.

Steind-Hof: Das frühere Inhaus vom Steind-Hof hat ein Kappenberger erworben, der
es 1906 an Joseph Graf verkaufte und der es heute noch im Besitze hat.

Scheybeck-Anwesen: Das Scheybeck-Anwesen ist schon sehr alt. Von 1679 weg wird
das Scheybeck-Gut bewirtschaftet immer von einem nachfolgenden Michl
Ziselsberger. Erst 1823 folgt ein Joseph Ziselsberger. Am 27.7.1838 hat ein
Joseph Achatz dem Scheybeck-Hof um 1 600 Gulden von Joseph und Ursula
Ziselsberger übernommen. Vor 1895 bewirtschaftete diesen Hof ein Johann Wilhelm.
Nach diesem kam wieder ein Joseph Achatz, der den Scheybeck-Hof hatte von 1895
bis 1930. Dann übernahm diesen Hof ein Johann Achatz, der in russischer
Gefangenschaft am 14.7.1946 starb. Seitdem bewirtschaftet die Witwe Franziska
Achatz den Scheybeck-Hof. Auf dem Leitthumshaus des Scheybeck-Gutes war früher
ein Achatz Joseph, dann ein Johann Wilhelm und später wieder ein Joseph Achatz.
Dieser hat dann dieses Leitthumshaus neu gebaut. Seit 1948 ist dieses kein
Leitthumshaus mehr, sondern Achatz Alois ist der alleinige Besitzer desselben.
1881 hat ein Alois Achatz ein landwirtschaftliches Anwesen neu gebaut. Dessen
Sohn Anton Achatz hat 1922 das Anwesen übernommen , 19.. eine Schreinerwerkstatt
gebaut und diese 1928 und 1935 vergrößert, dann das Wohnhaus aufgebaut, 1951
eine eigene elektrische Anlage errichtet und 1953 im ganzen Anwesen eine eigene
Wasserleitung installiert. Im ersten Weltkrieg sind 3 Brüder von Anton Achatz in
Frankreich gefallen Alois, August und Florian Achatz.

Achatz-Anwesen: 1947 hat Michal Achatz ein landwirtschaftliches Anwesen neu
gebaut.

1906 wurde in Lämmersdorf das Feuerhaus errichtet.1952 feierte die freiwillige
Feuerwehr Lämmersdorf die Fahnenweihe und 1953 bekam das Feuerhaus einen
Schlauchturm. Von besonderer Bedeutung für Lämmersdorf ist die 1953 erfolgte
Erbauung einer neuen, breiten Fahrtstrasse von der Giggenrieder Höhe über
Ortschaft Giggenried nach Lämmersdorf, die auch mit Pkw und Lkw befahrbar ist.
1953 bekam Lämmersdorf den elektrischen Strom vom Überlandwerk.

Das Dorf Lämmersdorf zählt in seinen 10 Häusern 95 Einwohner.

.25 Der Weiler Leuthen

Urkundlich findet man diesen Ortsnamen erstmals 1527. Hergeleitet wird dieser
Name ohne Rücksicht auf die verschiedenen Schreibweisen desselben von Leite =
Abhang. 1527 erscheint der Name Leitten. Auch Leytten wurde früher geschrieben.

Nach dem Salbuch des Kastenamtes Viechtach vom Jahre 1577 war zu jener Zeit
Mathias Prem hier sesshaft. 1657 übergaben Sebastian Steinbauer von Leuthen,
Gotteszeller Untertan, und seine Hausfrau Walburga ihren hiesigen Hof ihrem
Sohne Lorenz Steinbauer mit allen Baumannsfahrnissen, Wägen, Pflügen, Eggen, 3
gerichteten Betten und 30 Stück Vieh um 800 Gulden. Die Übergeber durften in
Häusl neben dem Hof wohnen. 1763 war Christoph Steinbauer hier. Nach dem Sal-
und Zinsbuche des Klosters Gotteszell aus der Zeit von 1790/99 saß damals die
Familie eines Lorenz Zellner auf einem hiesigen Hofe. Dieser hatte auch das
Erbrecht auf einem hiesigen Lehen. Ein anderes hiesiges Lehen besaß damals besaß
damals Joseph Achaz. 1823 hauste hier Johann Plötz auf dem Steinbauern-Hof und
Lorenz Zellner auf einem anderen Hof. Nach dem Grundsteuerkataster vom Jahre
1843 wurden die hiesigen Anwesen von den Familien Kronschnabel, Kramheller und
Lippl bewirtschaftet. Hans Kronschnabel heiratete 1839 mit Walburga Plötz und
beide übernahmen den elterlichen Hof um 3 500 Gulden. Joseph Kramheller kam in
den Besitz eines Anwesens durch Heirat der Magdalena Steinbauer. Wolfgang Lippl,
beim Weber, hatte seinen Hof um 800 Gulden von der Magdalena Steinbauer 1830
gekauft. Dieses alte, sogenannte Weber-Gütl steht nicht mehr. Joseph Kramheller
sen. hat dasselbe weggerissen und dafür 1947 einen Neubau aufgeführt.

Kronschnabel-Gut: Auf dem Kronschnabel-Gut hieß es später Raster Peter, dann
Bauer Alois (1903) und seit 1925 Georg Süß.1940 haben die kolossalen
Schneemassen den Stadel des Süß-Hofes zusammengedrückt. Süß hat aber den Stadel
im gleichen Jahre wieder aufgebaut. Der Sohn Georg Süß ist 1944 in Russland
gefallen. Ein Sohn des früheren Besitzers, Alois Bauer ist im 1. Weltkrieg
gefallen.

Steinbauern-Hofes: Anfang des 19. Jahrhunderts heiratete ein Joseph Kramheller
die Magdalena Steinbauer und wurde damit Besitzer des Steinbauern-Hofes in
Leuthen. 1869 übernahm ein Georg Steinbauer diesen Hof und 1899 ein Joseph
Kramheller der diesen Hof samt dem Inhaus 1948 an seinen Sohn Georg Kramheller
übergab.

Außerhalb der Ortschaft Leuthen steht eine alte Kapelle, dem hl. Lorenzi
(Laurentius) geweiht. Diese Kapelle spielte schon seit vielen Jahrhunderten in
Jägerkreisen eine große Rolle, weil an dieser Kapelle der Wechsel des ganzen
Wildes der dortigen Gegend vorüberging. Wer auf der Treibjagd auf diesem Stande
stand und abwartete, der kam sicher zum Schuss. 1878 wurde Leuthen von der
Pfarrei Ruhmannsfelden in die Pfarrei March umgepfarrt. Die Straße
Auerbach-Triefenried, die auch durch die Ortschaft Leuthen führt, wurde 1952 mit
einem Kostenaufwand von 125 000 DM neu gebaut.

Der Weiler Leuthen zählt in seinen 4 Häusern 22 Einwohner.

.)a Die Entstehung der Laurentius-Kapelle in Leuthen

Katherl und Anderl, die Eheleut vom Steinbauern-Hof in Leuthen hausten nicht gut
miteinander. Da es schon lange zwischen den Beiden nicht mehr stimmte, nahm
eines Tages der Anderl Abschied von der Katherl. Er ging nach Niederalteich und
fuhr auf einem Klosterfloß des Kloster Niederalteich auf der Donau hinab bis ins
Ungarn, wo ja das Kloster Niederalteich große Weinbergbesitzungen hatte. Von
Ungarn weg ging Anderl zu Fuß donauabwärts und kam hinunter bis ins Palästina.
Aber die Leute dort, die Gebräuche, die Sprache, das Essen dort, alles das
veranlasste Anderl wieder den Heimweg anzutreten. Auf dem Wege durch das
unwirtliche Banater-Gebirge traf er, einen herrenlosen, sehr großen und
kräftigen Geißbock, der ihm als Gepäckträger dienen musste. So traf er auch
eines Tages wieder in Leuthen ein. „Grüß dö Gott, Anderl“, rief die Katherl aus,
als sie den Anderl auf dem Geißbock sitzend daher reiten sah. „Wo kommst denn
her?“, fragte Katherl. „Von Palästina“, erwiderte Anderl . „Bleibst jetzt da?“,
fragte Katherl, kleinlaut. „Kann da nix Bestimmtes sagn“, war die grobe Antwort
des Anderl. Dann traten beide in die große Bauernstube des Steinbauern-Hofes.
“Bleibst jetzt da“, so fragte Katherl alle Tage den Anderl. Dieser antwortete
jedes Mal dasselbe: „Kann da nix Bestimmtes sagn. „Aber von Tag zu Tag wurde das
Verhältnis zwischen den Beiden besser. Eines Tages, als die Beiden die
Gewissheit hatten, dass jetzt wieder alles gut ginge, versprachen sie eine
Kapelle zu bauen. Sie hielten ihr versprechen und bauten eine hölzerne
Laurentiuskapelle in Leuthen. Bei der Christenverfolgungxl wurde die hölzerne
Kapelle in Leuthen im Innern mit Stroh gefüllt das Stroh angezündet und die alte
Kapelle niedergebrannt. 1839 wurde sie massiv, aus Stein, wieder aufgebaut. Es
werden in derselben Maiandachten abgehalten und für die Verstorbenen von Leuthen
oder der nächsten Umgebung Rosenkranzandachten.

.26 Der Weiler Lobetsried

Im Volksmund wird dieser Ortsname auch „Labertsried“ ausgesprochen. In den
Urkunden erscheinen die Namen: Lawandried, Labensried, Lobhardsried und
Lobersried. Auf die Schreibweise eines Ortsnamens darf man aber nicht achten.
Für jeden Fall ist dieser Ortsname Lobetsried ein echter „ried“-Name, weil er
mit einem Personennamen zusammengesetzt ist. Der dortige, damalige
Rodungsverarbeiter dürfte ein Labhart gewesen sein.

Lobetsried kam im Jahre 1295 durch Kauf von den bayerischen Herzögen an das
Kloster Aldersbach. 1657 heiratete Martin Zeidlhöfer von Labertsried die Barbara
Schreiner von Metten. Martin war Witwer. Er verheiratete ihr seinen
Erbrechtshof. 1671 übergaben beide ihren Hof um 620 Gulden. 1679 übergab Paul
Kramheller von Labertsried seinen hiesigen Erbrechtshof seinem Sohn Lorenz
Kramheller um 700 Gulden. Nach dem Sal- und Stiftsbuch des Klosters Gotteszell
von 1790/99 waren zu jener Zeit hier ansässig:



Christoph, später Georg Dax. Außerdem besaß dieser Dax noch eine Sölde.

Joseph Zeidelhofer



1823 hießen die beiden hiesigen Hofbesitzer Georg Zeidlhofer auf dem ganzen
Ketterl-Hofe und Andreas Dax auf dem Kufner-Hofe, welcher aus einem ganzen und
einem Viertelhof bestand. Beide Höfe waren grund- und zehentbar dem Staatsärar.
Nach dem Grundsteuerkataster vom Jahre 1843 treffen wir hier die Katharina Dax,
die den Hof, der um 2 400 Gulden übergeben wurde, durch Heirat mit Andreas Dax
erwarb und Georg Zeidelhofer, beim Godel-Bauern, von den Eltern Joseph und
Katharina Zeidelhofer 1820 um 900 Gulden übernommen.

Ketterl-Hof: Von diesem Georg Zeidelhofer erwarb den Ketterl-Hof ein Engelbert
Geiger, später ein Alois Wurzer, Brauereibesitzer von March, dann ein Johann
Rankl, von dem 1919 Ebner Ludwig den Hof käuflich erwarb und der jetzt diesen
Hof seinem Sohne Michael Ebner übergab. Dessen Sohn Friedrich Ebner ist 1949 bei
einer Fuhrwerksleistung tödlich verunglückt. 3 Söhne von ihm waren im 2.
Weltkriege fort. Einer war in russischer, einer in französischer Gefangenschaft
und einer wurde nach seiner Verwundung in bei Stalingrad in die Heimat
transportiert. Alle drei sind glücklich heimgekommen.

Kufner-Hof: Den Kufner-Hof bewirtschafteten lange Zeit immer die Dax. Seit 1946
ist der Besitzer dieses Hofes Wolfgang Pöhn. Er war in amerikanischer
Gefangenschaft.

Außerhalb der Ortschaft Lobetsried ist ein Marterl, das uns berichtet, dass dort
ein Knecht tödlich verunglückt ist. Ebenso die Woferl-Kapelle, die aber nicht
mehr auf Zachenberger-Gemeindegrund steht. Da blieb früher einmal ein
Holzfuhrwerk stecken und es schien als wolle es auf dem feuchten Waldesgrund
versinken. Die mit dem Holzfuhrwerk Beschäftigten versprachen eine Kapelle zu
bauen, wenn man sie wie der glücklich von dieser Stelle wegkommen. Sofort kamen
sie auch von dieser Stelle weg. Sie hielten ihr Versprechen und so entstand die
Woferl-Kapelle, die früher viel besucht war, obwohl sie recht abseits im Walde
steht.

Der Weiler Lobetsried zählt in seinen 2 Höfen und Nebengebäuden 24 Einwohner.

.27 Das Dorf Muschenried

Muschenried wird die Rodungsstelle eines „Musco“xli gewesen sein und wird von
diesem Rodungsvorarbeiter den Namen Musceried oder Muskenried bekommen haben.
Der Namen Muschenried ist also ein echter „ried“-Name, weil er mit einem
Personennamen in Verbindung steht.

Der Ort Muschenried lässt sich schon in einer Urkunde vom Jahre 1357 nachweisen.
Damals kaufte das Kloster Gotteszell den Zehent zu Muschenried und Zuckenried.
Am 23. Mai 1393 veräußerten Stephan der Degenberger zu Altnussberg, sein Sohn
Albrecht unterxlii anderem auch Güter zu Muschenried an ihren Vetter Konrad und
Eberhard die Nussberger zu Kollnburg. Nach dem Salbuch des Kastenamtes Viechtach
vom Jahre 1577 bestanden damals hier die folgenden Anwesen:



Jakob Küffner auf dem späteren Schollenrieder-Gut, später saß hier Georg
Riedler. In seinen Händen befand sich ein Leibgedingsbrief der Herzöge Wilhelm
und Ludwig vom Jahre 1543 auf Achaz Schiederauer und seine Hausfrau Walburga und
ihrer beider eheliche Kinder.

Hans Prumbauer auf der sogenannten Kaufmanns-Sölde, später hauste hier die
Familie Achatz. Prumbauer hatte in seinen Händen einen fürstlichen Erbbrief der
Herzöge Wilhelm und Ludwig auf Wolfgang Fenzl, seine Hausfrau und Erben vom
Jahre 1543.

Auf einem anderen Gute saß 1577 Georg Riedl. Das Anwesen war mit Gut und Gilt
dem Gotteshaus Ruhmannsfelden unterworfen. Später wirtschaftete hier die Familie
Stadler.

Leonhard Schweiber gehörte mit einem weiteren Gute dem Pfleger zu Regen. 1676
übergaben Georg Crambheller zu Muschenried und Magdalena, seine Hausfrau ihren
Leibrechtshof zu Muschenried mit Ein- und Zubehör ihrem Sohne Wolf Crambheller
um 200 Gulden.



1714 verkaufte Jakob Jungpeck von Muschenried und seine Hausfrau Rosina,
Vogtuntertan des Klosters Gotteszell, ihre Erbrechtssölde zu Muschenried an
Georg Oischinger zu Auerbach um 230 Gulden. 1770 saß auf der Kaufmanns-Sölde zu
Muschenried Jakob Achatz. 1823 bestanden hier die folgenden Anwesen:



Georg Stadler, später Adam Hofmann

Joseph Bernauer auf dem Piller-Hofe

Joseph Crambheller

Johann Hofmann, dann Georg Wurzer auf der Kaufmanns-Sölde

Georg Treml auf dem Kiesel-Hofe

Wolfgang Riedler



1830 bezog von hier den ganzen Zehent, bis auf den von einem Klein-Gütl, der
Staatsärar. Ebenso waren diese Anwesen dem Staatsärar Erbrechtsweise grundbar.
Das Klein-Gütl gehörte der Kirche zu Ruhmannsfelden. Der Grundkataster vom Jahre
1843 nennt uns hier die folgenden Anwesen:



Adam Hofmann, übernommen 1808 vom Stiefvater Andreas und der Mutter Magdalena
Stadler

Adam Schlegl, Weber, übernommen 1817

Joseph Kramheller, 1810 übernommen von den Eltern Michl und Eva Kramheller

Lorenz Hofmann, 1839 von der Mutter, der Witwe Anna Maria Wurzer übernommen

Georg und Anna Maria Treml, nach dem Tode der Mutter, der Witwe Anna Maria Treml
aus der Verlassenschaft 1820 erworben

Georg Kauschinger, beim Riedl, durch die Ehefrau Walburga, geborene Riedl, um 1
000 Gulden durch Heirat erworben, 1823



Crambhöller-Hof: Auf dem ehemaligen Crambhöller-Hof hieß es früher immer Michael
Crambhöller. Erst Anfang des 19. Jahrhunderts taucht als Besitzer dieses Hofes
ein Martin Kramhöller auf. 1894 hat dessen Sohn Martin Kramhöller den Hof
übernommen und 1933 wieder dessen Sohn Joseph Kramhöller. 1948 hat dieser Joseph
Kramhöller alles vollständig neu gebaut, Wohnhaus, Stall und Stadel. Das Inhaus
ist bei dem Brande 1902, bei welchem das Hofmann-Gut abbrannte, zu Schaden
gekommen. Ein Bruder des Joseph Kramhöller, nämlich Johann Kramhöller, ist im 1.
Weltkrieg gefallen. Sein jüngster Bruder, Alois Kramhöller, Studienprofessor,
ist 1945 in Würzburg gestorben. Zum Michlbauern-Hof gehört auch ein eigener
Steinbruch.

Schierer-Gut: Auf dem heutigen Schierer-Gut hieß es früher Winterer, dann
Kauschinger, dann Niedermeier, dann Leopold Kilger. Vom Jahre 1893 ab
bewirtschaftete ein Alois Bielmeier dieses Anwesen, dann dessen Sohn Alois
Bielmeier und seit 1919 ist der Besitzer dieses Hofes ein Joseph Schierer.
Dieser hat 1928 den Stadel und 1937 den Stall neu gebaut. Von dem früheren
Besitzer Alois Bielmeier ist der Sohn August Baumgartner im 1. Weltkrieg
gefallen. Das alte Weberhaus steht nicht mehr. Das Brechhaus stand früher am
Südwestausgang des Dorfes.

Hofmann-Anwesen: Das Hofmann-Anwesen in Muschenried ist auch schon ein sehr
altes Gut. 1808 hat ein Adam Hofmann den Hof übernommen von seinem Stiefvater
Andreas Stadler. 1902 hat ein Jakob Hofmann den Hof übernommen und 1939 ein
Joseph Hofmann. Jakob Hofmann, der Bruder des Joseph Hofmann, hat sich 1934 ein
Anwesen gebaut außerhalb der Ortschaft Muschenried. Das alte Hofmann-Anwesen ist
1876 das erste Mal abgebrannt.

Kramhöller-Anwesen: Das Anwesen des Xaver Kramhöller war 1860 die Werkstätte des
Beni. 1876 hat dann Ernst Johann an dieser Stelle ein landwirtschaftliches
Anwesen gebaut, das 1919 durch Einheirat Xaver Kramhöller übernommen hatte. 1919
hat dieser einen Steinbruch eröffnet und 1946 das Wohnhaus, den Stall, den
Stadel umgebaut. Im Anwesen des Xaver Kramhöller befindet sich jetzt auch eine
Kantine.

Dobusch-Anwesen: Das heutige Dobusch-Anwesen ist durch eine Köhlerhütte
entstanden. Als frühere Besitzer dieses Gutes sind noch bekannt: ein Egner Adam,
Oswald Michl, Winterer Max. Unter Oswald ist die Säge 1824 errichtet worden.
Jakob und Therese Dobusch haben 1885 dieses Anwesen gekauft und zur Säge auch
eine Mühle gebaut, welche der derzeitige Besitzer Anton Dobusch 1932 aufgelassen
hatte. Dieser hat 1920 das Anwesen übernommen und zugleich auch den Steinbruch,
der 1902 von dem Vater Jakob Dobusch eröffnet wurde. Neu gebaut hat der jetzige
Besitzer 1921 Stallung und Stadel,1922 die Säge und 1935 das Wohnhaus. Der Vater
des jetzigen Sägewerkbesitzers, Jakob Dobusch, ist 1915 durch einen Unfall ums
Leben gekommen. Joseph Dobusch ist 1917 gefallen. In nächster Nähe der
Dobusch-Säge waren ein Brechhaus und eine Kapelle.

Steinbauer-Anwesen: Der Erbauer des heutigen Steinbauer-Anwesens, früher
Rosenlehner, war Achatz von Muschenried um das Jahr 1883 herum. Joseph
Rosenlehner hat dieses Anwesen von Achatz käuflich erworben im Jahre 1885.
Joseph Rosenlehner, der 1870/71 als Ulane den Krieg mitgemacht hatte, ist 1892
beim Holzziehen tödlich verunglückt. Der Sohn des Joseph Rosenlehner, auch ein
Joseph Rosenlehner, hat 1910 das Rosenlehner-Gut von seiner Mutter übernommen.
Seit 24. Februar 1946 ist Johann Steinbauer Inhaber dieses Anwesens. Dieser hat
1947 den Stadel neu gebaut und 1948 auf das Haus einen neuen Dachstuhl
aufgebaut. Durch das obere Feld des Rosenlehner-Gutes führt die Wasserleitung
zum Markte Ruhmannsfelden. Das Anwesen des Lorenz Rosenlehner wurde 1928 neu
gebaut.

Steinbruch Wagner: Das Anwesen im Steinbruch Wagner in Muschenried wurde 1911
gebaut und der dortige Steinbruch 1922 eröffnet. Im Juli 1934 passierte dort ein
Unglück, indem der Kippwagen über den Abraum stürzte.

Jungbeck-Anwesen: Das heutige Jungbeck-Anwesen ist in den achtziger Jahren
gebaut werden von Joseph Handlos und Michl Hartl sen. Die früheren Besitzer
dieses landwirtschaftlichen Anwesens waren Michl und Anna Hartl. Seit 1930 ist
August Jungbeck der Besitzer dieses Hauses. Dieser ist aber seit 6. Februar 1943
bei V. Charkow in Russland vermisst. Die Ehefrau Anna Jungbeck bewirtschaftet
das Anwesen.

Die Strasse Auerbach-Muschenried wurde 1952 mit einem Kostenaufwand von 30 000
DM so ausgebaut, dass sie auch für Pkw und Lkw befahrbar ist.

Das Dorf Muschenried zählt in seinen 11 Häusern 75 Einwohner.

.)a Was von der „Muschenrieder Kapelle“ erzählt wird

Es war Herbstzeit. Da war der Burgherr von der Burg Bayreck bei Neuern in Böhmen
auf dem Wege zu dem Grafen von Ortenburg in Niederbayern. Ein schönes, flinkes
Rösslein diente ihm als Reittier und eine große, furchtlose Dogge bewachte und
beschützte ihn, Tag und Nacht. Den böhmischen Flussläufen entlang reitend kam er
zum Fuße des Arbers. Von hier aus ritt er dem Ufer des schwarzen Regen entlang
und kam nach Regen. Von da weg musste er eine andere Richtung einschlagen. Einem
Wasserlauf folgen konnte er von da ab nicht mehr. Er ritt bergauf, hergab, durch
Täler und über Berge. Einmal war er wieder auf einer Anhöhe, mitten im Urwald.
Es brach die Nacht herein. Eine riesengroße Eiche stand vor ihm. Die dürren,
abgefallenen Eichenblätter boten für Reiter, Pferd und Hund eine gute
Liegestatt. Und alle drei schliefen auch bald ein. Doch die Dogge schlug heftig
an. Als der Ritter die Augen öffnete, sah er, dass Himmel und Erde in ein
feuriges Rot getaucht waren. Es schien, als gäbe es einen riesigen Weltenbrand.
Der Ritter, dem ja das Fluchen und Schelten zur zweiten Gewohnheit wurde, rief
aus: „Erde, Welt, jetzt hol dich der Teufel!“ Die Bewohner der Ortschaft
Muschenried wurden auch wach. Sie glaubten es brenne der ganze Muschenrieder
Wald. Sie eilten gemeinsam auf die Anhöhe und riefen unausgesetzt: „Herr, Herr,
lass uns nicht untergehen! Herr, Herr, rette uns!“ Oben auf der Waldeshöhe
trafen sie den Ritter mit dem Rösslein und dem großen Hunde. Anfänglich nahmen
die Muschenrieder eine bedrohliche Haltung gegen den Ritter ein. Aber dieser hob
auch seine Hände zum Himmel und rief mit den Muschenriedern: „Herr, wende das
Unheil von uns ab! Herr, erweise uns deine Barmherzigkeit!“ Und siehe! Das Rot
des Firmamentes verwandelte sich in ein Gelb, dann in ein Grün und plötzlich war
es wieder Nacht. Die Muschenrieder dankten Gott und versprachen auf dieser
Stelle eine Kapelle zu errichten. Der Ritter kam auf seinem Rückwege nochmals
nach Muschenried und spendete zur Erbauung einer Kapelle etliche Golddukaten. In
der Hitlerzeit wurden die Innen- und Außenwände der Kapelle mit den unflätigsten
Schriften beschmiert und in der Zeit des 2. Weltkrieges und der schlimmen
Folgezeit wurde die Muschenrieder Kapelle bestialisch ausgeplündert. Der Altar
wurde seines Schmuckes beraubt. Holz und Nägel, Türschloss und Dachziegeln
wurden der Raub von Beeren- und Pilzsuchern, von Erwachsenen und Kindern. Da
diese Kapelle zu abseits liegt, wurde versprochen, eine neue Kapelle, innerhalb
der Ortschaft Muschenried zu erbauen.

.28 Der Weiler Ochsenberg

Der Ochsenberg war der Weideplatz für das Vieh des Auhofes. Seine Ansiedlung
beginnt erst mit der Zeit des Bahnbaues. Allerdings gab es um diese Zeit am
Bahndamm entlang nur Baracken und Kantinen. So lesen wir in einen Protokollbuch
der Gemeinde Zachenberg von einer Bierschenke in Ochsenberg, die von einem
Heinrich Zimmermann geführt wurde im Jahre 1875.

Kopp-Anwesen: Das Anwesen des Michl Kopp in Ochsenberg muss um diese Zeit schon
bestanden haben, da es von einem Franz Xaver Vogl, Gastwirt auf dem heutigen
Fleischmann-Hof in Zachenberg, um 1860 herum gebaut wurde. Die früheren Besitzer
dieses Anwesens waren ein gewisser Bauer, der Mühlrichter war, später ein Alois
Obermeier von der Wandlmühle. Dieser gab seiner Pflegetochter über, die einen
Hilarius Pfeffer von Wolfertsried heiratete. Nachdem dieser bald starb,
heiratete sie einen Mühlbauer, der auch von der Achslacher Gegend war. Dieser
vertauschte das Anwesen und dann erwarb es der jetzige Besitzer Michl Kopp 1909.
Von den 10 Kindern des Michl Kopp waren 6 Söhne im 2. Weltkrieg im Felde. Von
diesen 6 Söhnen ist der Sohn Hans Kopp am 5. Dezember 1944 bei Aachen gefallen.

Friedrich-Anwesen: Auf dem Platze einer früheren Kantine wurde von den Eheleuten
Alois und Magdalena Friedrich das Anwesen gebaut, das 1895 Joseph Kraus erwarb
und auf das dann 1939 Kappl hin heiratete. Die jetzige Besitzerin ist Frau
Kappl.

Krampl-Anwesen: Zum Ochsenberg gehören noch die Anwesen der Fr. Fritz, das
Krampfl-Anwesen, das der alte Wirt von Auerbach gebaut hatte und auf dem es
später Kopp und Höferer geheißen hat, und das von Karl Schrötter 1928 gebaute
Anwesen, auf dem es heute Joseph Vogl heißt.

Plötz-Anwesen: Seit 1921 ist der Besitzer des früheren Plötz-Anwesens Xaver
Muhr, dessen Sohn Max Muhr in Gendorf tödlich verunglück ist.

Der Weiler Ochsenberg zählt zurzeit 59 Einwohner in 6 Häusern.

.29 Der Weiler Poitmannsgrub

Man findet in den alten Urkunden diesen Ortsnamen geschrieben: Pewentsgrub,
Peuchtmannsgrub, Paitmannsgrub, Peuntmannsgrub. Der Name Poitmann kommt her von
dem Flurnamen Peunt oder Beunt, auch Point oder Beind und bezeichnet im
Allgemeinen ein durch Einfriedung von seiner Umgebung losgelöstes Grundstück.
Das Poitmannsgrub ist an einer Bodensenke (Gruabn = Grubn) liegende Siedlung
eines Poitmann. Poitmann ist die Bezeichnung für einen Mann der eine Point, also
ein umhegtes Grundstück, bewirtschaftet oder an einem solchen wohnt.

1394 vermachten Andreas von Hasmannsried und seine Hausfrau Elisabeth dem
Kloster Gotteszell ihren Teilzehent zu Poitmannsgrub zur Begehung eines
Familienjahrtages, der alljährlich am Sonntag nach Ulrich gefeiert werden
musste. 1658 lesen wir einen Adam Paur in Poitmannsgrub. 1688 übergab Adam
Prunner von Poitmannsgrub Witwe ihren Erbrechtshof allhier um 400 Gulden ihrem
Sohn Michl Prunner. 1721 besaß hier Paulus Achaz einen halben Hof. Den anderen
1721 hier bestehenden Hof bewirtschaftete zu jener Zeit Hans Paur. Beide Höfe
waren damals Erbrechtsweise grundbar dem Kloster Gotteszell.

Nach dem Sal- und Stiftsbuch des Klosters Gotteszell von 1790/99 bestanden zu
jener Zeit hier zwei Anwesen, eine Erbrechtssölde, die einem Michl Achaz gehörte
und ein Lehen, das Michl Paur bewirtschaftete. Um 1830 bestanden hier zwei Höfe,
die dem Staat Staatsärar Erbrechtsweise grund- und zehentbar waren. Die Besitzer
hießen Georg Paur und Johann Achaz. Der Grundsteuerkataster von Jahre 1843 weist
noch dieselben zwei Besitzer auf. Lorenz Paur hatte von seinem Vater Michl Paur
um 1600 Gulden übernommen, während Achaz von seinen Eltern Michl und Maria Achaz
bei der Übergabe 1813 2 900 Gulden bezahlen musste. Der Grundbesitz des Ersteren
war damals 92 Tagwerk und der des Zweiten 80 Tagwerk.

Lorenz-Paur-Anwesen: Auf dem Lorenz-Paur-Anwesen hat es später geheißen Zitzler,
dann Bartholomäus Plötz, ab 1894 Joseph Plötz sen. und seit 1926 bewirtschaftet
diesen Hof Johann Plötz. Von den ehemaligen Pauern soll einer 1812 in Russland
gefallen sein. Johann Plötz hat 1946 die Stallung neu gebaut.

Achaz-Anwesen: Das Achaz-Anwesen, genannt das Boais-Anwesen, hat 1906 ein Johann
Sailler von Obermitterndorf gehabt. Von diesem Johann Sailler haben es die
Häuserjuden bekommen und von diesen hat es Anna Sailler gekauft. Deren Sohn
Xaver Sailler hat es 1937 übernommen.

Das Marterl oberhalb Poitmannsgrub wurde errichtet aus Anlass einer großen
Viehseuche, die hier geherrscht hatte. In der ganzen Umgebung von Poitmannsgrub
herrschte ein großes Viehsterben. Nur die zwei Höfe in Poitmannsgrub blieben
verschont. Aus Dankbarkeit setzten die beiden Hofbesitzer von dort das eiserne
Kreuz auf Steinsockel auf die Anhöhe. Jedes Jahr am Gründonnerstag wird bei
diesem Kreuz von den Poitmannsgrubern gemeinsam gebetet.

Der Weiler Poitmannsgrub zählt in seinen 3 Häusern 19 Einwohner.

\section{Die Einöde Reisachmühle}

Der Name Reisachmühle wird schon 1295 genannt. Das Wort Reisach wird abgeleitet
von Ris, Reis, Reiser = Busch. Die Reisachmühle ist also die Mühle im oder am
Buschwald.

Reisach-Mühle: 1660 war der Besitzer dieser Mühle ein Adam Oischinger. 1692
finden wir hier die Familie Prumbauer. In diesem Jahre verkauften Georg
Prumbauer und seine Hausfrau Katharina ihre Erbgerechtigkeit auf der Mühle zu
Reisach samt der Säggerechtigkeit um 175 Gulden einem Georg Lippl zu Achslach.
1790 gehörte die Mühle einem Georg Baumgartner. 1823 und 1843 hauste hier die
Familie Hanningerxliii. Nach Hanninger bewirtschaftete die Reisachmühle ein
gewisser Wirth. Nach diesem kam ein Martin Plötz und diesem kaufte sein Bruder
Franz Plötz 1903 die Reisachmühle ab. Dieser Franz Plötz starb 1946 und seit
1947 führt dessen Sohn Joseph Plötz die Säge und die Mühle in der Reisachmühle.

Dort steht auch eine gut erhaltene Kapelle. Andachten werden dort nicht
abgehalten, weil der Innenraum zu wenige Personen fasst. Reisachmühle gehört zur
Pfarrei Ruhmannsfelden, obwohl Leuthen und Lobetsried schon zur Pfarrei March
gehören. Zur Schule gehen die Kinder von Reisachmühle nach March, da dorthin
viel näher ist als nach Ruhmannsfelden.

Die Einöde Reisachmühle zählt 6 Einwohner.

.31 Das Dorf Triefenried

Triefenried ist ein echter „ried“-Name, weil er zusammengesetzt ist mit dem
Taufnamen Trube, Trune, Trunolf, also Trunolfsried. Im Volksmund heißt diese
Ortschaft Triefaried, weil die Waldler die Silbe „en“ in der Mitte des Wortes
wie „a“ aussprechen, z. B. Zachenberg = Zachaberg, Muschenried = Muscharied.
Genau so macht es die hiesige Bevölkerung mit der Silbe „er“ in der Mitte des
Wortes, z. B. Auerbach = Auabach, Köckersried = Köckasried, Lämmersdorf =
Lämmasdorf, Haberleuthen = Habaleuthen. In alten Urkunden erscheint auch der
Name Driefenried.

Am 1. August 1379 begaben sich Friedrich der Auer zu Brennberg und sein Sohn
Wilhelm der Robauer aller Ansprüche bezüglich des Dorfes Triefenried ledig,
welche ihr Schwager Sweikker der Tuschel zu Söldenau zu Lebzeiten Weindlein dem
Nussberger eingeräumt hatte. Früher war hier auch das Kloster Oberalteich
begütert, das hier einen Hof und einige Lehen besaß. Am 9. März 1500 verkaufte
Sebastian Diebler seine Erbgerechtigkeit auf dem herzoglichen Lehen zu
Triefenried an einen gewissen Hans Vidler. Zeugen des Verkaufs waren Stefan
Püchler und Georg Petzinger zu Triefenried. 1577 waren hier die folgenden
Anwesensbesitzer:



Christoph Sieß oder Rieß. Er hatte über sein Besitztum einen 1458 von Konrad dem
Nussberger ausgefertigten Rezess.

Hans Koller auf einem Lehen. In seinen Händen befand sich ein Kaufbrief vom
Jahre 1500, welcher uns sagt, dass damals Georg Pockhinger den Hof an Sebastian
Diebler und allen seinen Erben verkaufte.

Peter Wierth auf der Taverne. Laut Kaufbrief vom Jahre 1511 hatte damals Georg
Pockhinger dieses Anwesen dem Matheus Freindorfer seiner Hausfrau und etwaigen
Kindern verkauft.

Jakob Lindler auf dem sogenannten Schreiner-Gut. Er konnte über den Besitz des
Anwesens einen fürstlichen Leibgedingsbrief der Herzöge Wilhelm und Ludwig auf
Pobber Wühr, seine Hausfrau Barbara und etwaige Kinder vom Jahre 1543 vorlegen.

Georg Ebenmayer, Lehensbesitzer. Er besaß einen Leibgedingsbrief des Herzogs
Albrecht auf Ebenmayer, seine Hausfrau und ihren Sohn Stefan. Später
bewirtschafteten diesen Hof die Familien Stadler, Kramheller und Schreiner. 1603
saß auf diesem Anwesen Mathias Gansl.

Wolfgang Stadler, auch Lehensbesitzer. In seinen Händen befand sich ein
fürstlicher Leibgedingsbrief von Herzog Ludwig vom Jahre 1518 auf Anton Lang,
seine Hausfrau Margarete und ihre Erben.

Christoph Kufner, ebenfalls Lehensbesitzer. Das Gut war Eigentum der Kirche in
Regen. Er konnte über sein Besitztum einem Erbbrief vom Jahre 1483 vorweisen.
Haus und die dabei befindliche Schmiedewerkstätte waren gut gebaut.

Gilg Kuchler auf einem Lehen. Später wirtschaftete hier die Familie Altmann.
Kuchler hatte in seinen Händen einen Leibgedingsbrief der Herzöge Wilhelm und
Ludwig vom Jahre 1545 auf Andreas Kuchler, seine Hausfrau Elisabeth und ihren
Sohn Anton.



Auf drei anderen Gütern hausten 1577 die Familien Puggl, Schütz und Achaz. Nach
dem Stiftsregister der Schlossherrschaft Neunussberg vom Jahre 1578 saßen damals
hier Georg Kurz und Michael Achaz, jeder auf einem Hofe. 1594 war in Triefenried
ein gewisser Georg Ebeneier Lehensbesitzer. 1577 besaß ferner hier noch Georg
Riss ein Lehen, das später Hans Ernst innehatte.

Einige gerichtliche Urteile aus der damaligen Zeit sind ganz interessant: 1666
wurde Blasy Schwarz von Triefenried wegen des im öffentlichen Gerichtssitzung
getanen Schwurs, dass ihm der Teufel die Zunge zum „Gnakh“ (=Nacken)
herausreißen solle, 2 Stunden lang an die Schandsäule gebunden. Im selben Jahre
wurde Andreas Stadler von Grub um 24 Kreuzer und 2 Heller gestraft, weil er den
Kaspar Aigner zu Triefenried bei einer Rauferei den Ärmel vom Hemd gerissen und
ihn überdies auch noch einen Fretter geheißen hatte. 1733 wurde Hans Riedl von
Triefenried, weil er zur österlichen Zeit Beicht und Kommunion nicht abgelegt,
vormittags und nachmittags 2 Stunden lang in Stock gespannt und er überdies die
Beichte noch nachholen musste.

1674 verkauften Michl Lanzinger zu Triefenried und seine Frau Maria ihren,
bisher mit Grund und Boden inne gehabten Leibrechtshof, worauf laut eines
Briefes vom Jahre 1650 Andre Raith von Triefenried und Andre Raith aus der
Breitenau Leihgerechtigkeit hatten, mit aller Ein- und Zubehör um 160 Gulden an
Georg Gschweimb von Eckersberg und seiner Frau Eva. 1687 besaß hier Michl
Schreiner einen Hof, den er um 90 Gulden gekauft hatte. Die
Hauptsteuerbeschreibung des Gerichts Viechtach vom Jahre 1721 nennt in
Triefenried die folgenden Anwesensbesitzer:



Andreas Koller, Wirt, auf einem ganzen Hof

Mathias Ernst auf einem ganzen Hof

Andreas Schreiner, auf einem halben Hof

Christoph Kramheller, auf einem halben Hof

Peter Reister, auf einem halben Hof

Paul Reister, Viertelhof-Besitzer

Hans Schreiner, auf einem halben Hof

Sebastian Oischinger, auf einem halben Hof



1789 saß auf einem der hiesigen Höfe Hans Schwarz und auf einem anderen Hans
Lippl. Beide waren Grunduntertanen des Freiherrn von Hafenbrädl auf Eisenstein.
Am 8. Juni 1790 verkaufte der Halbbauer Joseph Schreiner von Triefenried den im
Jahre 1762 durch Übernahme an sich gebrachten zum kurfürstlichen Kastenamte
Viechtach gehörigen Hof in Triefenried um 1 600 Gulden an den Bauern Joseph
Achaz von Leuthen. Das Gefällebuch des Rentamtes Viechtach vom Jahre 1823 nennt
uns hier die folgenden Hausbesitzer:



Michl Reister (Raster)

Lorenz Schreiner (Aigner-Bauer)

Michl Schreiner, auf dem Brunner-Hof

Joseph Schreiner

Lorenz Schwarz

Michl Koller, Wirt

Johann Bapt. Wurzer, auf dem Gansl-Hof

Johann Tremmel

Michl Weber, auf dem Oischinger-Hof

Franz Hof, Gager-Bauer



Im Grundsteuerkataster vom Jahre 1843 waren in Triefenried die folgenden
bäuerlichen Anwesen:



Lorenz Schreiner, durch die erste Ehefrau Walburga, geb. Achatz von der Mutter
Maria, nachher verehelichte Schreiner, übernommen

Georg Schreiner, vom Vater Michl Schreiner übernommen

Johann Schreiner, von den Eltern Andreas und Anna Maria Schreiner übernommen

Michl Koller, Wirt, von der Mutter Walburga Koller übernommen

Franz Hof, vom Vater Michl Hof übernommen

Lorenz Schwarz, von der Mutter Barbara Schwarz übernommen

Johann Wurzer, von den Eltern Georg und Therese Wurzer durch Übernahme erhalten

Johann Treml, Weber, laut Kaufbrief vom Jahre 1816 von Joseph und Barbara Lippl
von Triefenried um 504 Gulden erworben

Michl Weber, Wastl-Bauer, übernommen von den Eltern Michl und Magdalena Weber

Joseph König von den Schwiegereltern Michl und Anna Maria Raster 1837 um 2 000
Gulden erhalten. Dieses Gut war ehedem Erbrechtsweise grundbar zur Kirche St.
Johann in Regen.



Koller-Gastwirtschaft: Die Gastwirtschaft in Triefenried wird schon seit einigen
Jahrhunderten von dem gleichen Familiengeschlechte der Koller bewirtschaftet.
Allerdings sind die früheren hölzernen Gebäulichkeiten dieses großen Besitztums
1902 niedergebrannt. Dafür wurde aber im Laufe der letzten Jahrzehnte
Gastwirtschaft und Ökonomie zeitgemäß gebaut, sodass der jetzige Besitzer
Wolfgang Koller eine modern eingerichtete Gastwirtschaft und eine vorbildliche
Landwirtschaft besitzt.

Schwarzhansl-Hof: Der frühere Schwarzhansl-Hof wurde vertrümmert und wurde von
einem Kagel erworben. Später kam dann auf diesen Hof Joseph Klimmer. Nach dessen
Tod bewirtschaftete die Witwe den Hof von 1903 bis 1923. In diesem Jahre
übernahm den Hof der Sohn Peter Klimmer. Der Klimmer-Hof ist 1921 abgebrannt.
Dabei verbrannte der schlafende Hütbube.

Treml-Anwesen: Auf dem Treml-Anwesen hieß es früher Georg, dann Alois Treml und
seit 1919 Lorenz Treml. Dieses Anwesen ist 1932 abgebrannt. Mit dem Viehstand
hatten die Treml-Leute viel Unglück. Während ihrer 33-jährigen Verheiratung
mussten sie 27 Stück Vieh einbüßen.

Gagerbauern-Hof: Ein ganzes altes Besitztum ist der Gagerbauern-Hof, der schon
mehr als hundert Jahre von dem Familiengeschlechte der Hof bewirtschaftet wird.
Nach dem Tode des Georg Hof bewirtschaftete die Witwe Anna Hof seit 1922 dieses
Bauern-Anwesen und seit 1940 ist der Besitzer dieses Hofes Wolfgang Hof.1928
brannte das Anwesen nieder und wurde im gleichen Jahre wieder aufgebaut.

Mühlbauer-Anwesen: Das Wohnhaus, das seit 1900 Joseph Mühlbauer besitzt, stammt
noch vom Wastlbauern-Hof. 2 Söhne von ihm sind vermisst: Michael Mühlbauer seit
1943 in Russland und Joseph Mühlbauer seit 1944 in Russland.

Ernst-Anwesen: Auf dem früheren Ernst-Anwesen heißt es jetzt Georg Stadler. Der
Sohn Georg Stadler ist seit 1944 in Russland vermisst.

Joseph-König-Hof: Der jetzige Besitzer des früheren Joseph-König-Hofes ist seit
1900 Josef Pöhn. Dessen Vater Joseph Pöhn hat im Jahre 1900 die Wasserleitung
für den Hof gebaut und 1920 den Stadel. Der jetzige Besitzer dieses Hofes,
Joseph Pöhn, hat 1951 einen ganz neuzeitlichen, schönen Stall gebaut. Dieser Hof
ist 1852 abgebrannt, sodass also von den sämtlichen Anwesen in Triefenried
keines von dem Brande verschont blieb.

Kappl-Anwesen: Auf dem Kappl-Anwesen heißt es seit 1951 Georg Baumgartner.

Freundorfer-Maria-Krämerei: In Triefenried sind zwei Krämereien. Freundorfer
Maria hat 1929 ein Wohnhaus gebaut und darin eine Krämerei errichtet. Diese
Krämerei ist verpachtet.

Weber-Maria-Krämerei: Die andere Krämerei wurde von Maria Weber 1928 gebaut und
ist ebenfalls verpachtet.

Freisinger-Anwesen: Neben dieser Krämerei hat 1952 Alois Freisinger ein Wohnhaus
gebaut.

Wurzer-Lagerhaus: Das von Wurzer, March, am Bahnhof Triefenried errichtete
Lagerhaus wurde 1952 von Treml, Kirchweg gekauft und dieses Lagerhaus wird von
dessen Sohn Hans Treml geführt.

Früher bezogen die Triefenrieder Anwesensbesitzer das elektrische Licht von der
sogenannten Stöß in Furth. Seit 1950 ist Triefenried an das Überlandwerk
angeschlossen. 1910 wurde die Triefenrieder Feuerwehr gegründet, dann ein
Feuerwehrhaus gebaut und 1953 bekam sie eine ganz moderne Motorfeuerspritze. Der
1893 gegründete Krieger- und Veteranenverein, der in der Hitlerzeit und
Nachkriegszeit nicht mehr bestehen durfte, ist 1953 wieder neu gegründet worden.

Das Dorf Triefenried zählt in seinen 14 Häusern 110 Einwohner.

.32 Das Dorf Vorderdietzberg

Schon im 1.Teil der Geschichte der Gemeindeflur Zachenberg wurde bei dem Bericht
über den Weiler Hinterdietzberg darauf hingewiesen, dass der Name Dietzberg
herkommt von dem Personennamen Tirold, dass also Ditzberg ein echter „Berg“-Name
ist. Es darf mit Recht angenommen werden, dass anfänglich von Fratersdorf,
Lämmersdorf und Patersdorf her dieses Gebiet gerodet wurde, weil ja diese Orte
viel älter sind als Ruhmannsfelden oder Giggenried. Darum wird auch
Vorderdietzberg erst nach Hinterdietzberg entstanden sein.

1577 finden wir in dem Salbuch des Kastenamtes Viechtach einen Wolfgang Nickl
auf einer Hube, dem späteren Hacker-Gut (Hackö). In seinem Besitz hatte er einen
Erbbrief der Herzöge Wilhelm und Ludwig vom Jahre 1543 auf Georg Erl, seine
Hausfrau Walburga und ihre Erben. In der Hauptsteuerbeschreibung des Gerichts
Viechtach vom Jahre 1721 werden die folgenden Hofbesitzer in Ditzberg genannt:



Paulus Peter, Besitzer eines halben Hofes

Paulus Kraus, auch Besitzer eines halben Hofes

Georg Limpeck, Besitzer eines halben Hofes



1823 waren es in Vorderdietzberg folgende Hofbesitzer:



Lorenz Kraus, von den Eltern Joseph und Magdalena Kraus im Jahre 1832 um 2 000
Gulden übernommen

Georg Fritz laut Heiratsbrief im Jahre 1832 durch die Ehe der Frau Therese
geborene Limpeck, von der Mutter Anna Limpeck um 1 888 Gulden übernommen

Joseph Loibl, beim Göstl, seit 1844 Schwiegersohn Georg Achatz, laut Kaufvertrag
vom Jahre 1822 von Andreas Peter von Ditzberg um 3 008 Gulden aus freier Hand
gekauft. Auf diesem Hacker-Gut (Hackö-Hof) hieß es dann vom Jahre 1889 ab Kufner
Joseph und ab 1906 Kufner Xaver. Dieser hat 1914 den Stadel und 1925 den Stall
neu gebaut. Sein Sohn Alois Kufner ist 1941 bei der Insel Kreta um sein junges
Leben gekommen.



Limpeck-Anwesen: Das Limpeck-Anwesen existiert nicht mehr. Es stand neben dem
Hacker-Gut. Seine Besitzer waren Limpeck, dann Fritz Hirtreiter, unter dem das
Anwesen vertrümmert wurde. Dann waren die Besitzer Schwarz Alois und hernach
Karl Franz.

Fritz-Anwesen: 1842 hat die alte Häusl-Mutter Therese Fritz das heutige
Fritz-Anwesen neben der Straße durch Vorderdietzberg gebaut und sie und ihre
Schwester Katharina hatten das Ausnahms-Recht auf diesem Haus. Seit 1902 gehörte
es dem Franz Fritz, der es 1951 an Alois Loth übergeben hattexliv.

Steinbauern-Anwesen: Auch kamen ehemalige Limpeck-Gründe zum jetzigen
Steinbauern-Anwesen, das 1890 gebaut wurde. Auf diesem Anwesen wirtschafteten
Reitmeier Karl, dann Muhrhauser Joseph und seit 1927 Michl Steinbauer. Dieser
hatte 1927 eine Werkstätte gebaut und 1952 das Wohnhaus aufgestockt.1953 wurde
von der Firma Joseph Eckart, Landshut, ein Windrad aufmontiert, das Wasserpumpe
und elektrisches Werk betreibt. Der kriegsbeschädigte Sohn Xaver Steinbauer kam
1948 aus der Gefangenschaft zurück.

Krauß-Anwesen: Das frühere Krauß-Anwesen wurde vertrümmert. Es entstand dabei
neben dem Krauß-Hof das heutige Bielmeier-Anwesen. Die zu diesem Anwesen
gehörigen Grundstücke wurden vom Krauß-Anwesen abgetrennt. Das Anwesen
bewirtschaftete früher schon ein Bielmeier, aber nur drei Jahre lang, da er
starb. Die Witwe heiratete 1903 den Xaver Fenzl von Bruckhof. Aus dieser Ehe
ging eine Tochter hervor, welche Hr. Ludwig Bielmeier, den jetzigen Besitzer
dieses Anwesens 1934 heiratete, der schon in der 2. Wahlperiode der
Bürgermeister der Gemeinde Zachenberg ist.

Krauß-Hof: Von den früheren Krauß auf dem Krauß-Hof in Vorderdietzberg ist
Johann Krauß 1915 in Frankreich gefallen.1931 übernahm Rankl Joseph von
Gotteszell diesen Krauß-Hof. Dieser Joseph Rankl starb aber schon 1943. Die
Witwe wirtschaftete bis 1952 weiter. 2 Söhne von ihr sind gefallen. Rankl Karl
1943 und Rankl Georg 1945. Seit 1953 ist der Besitzer dieses ehemaligen
Krauß-Hofes der Flüchtlingsbauer Tax Wolfgang.

Pinzl-Anwesen: An der Grenze der Ortsflur von Vorderdietzberg, ganz oben am
Waldesrand, ist das Pinzl-Anwesen, das von Joseph Kufner, Bauer in
Vorderdietzberg im Jahre 1888 gebaut wurde und auf das dann Alois Pinzl sen.
getauscht hatte (1907). Seit 1951 bewirtschaftet dieses Anwesen Michael Pinzl.
Dessen Bruder Karl Pinzl ist 1943 in Belgien gefallen und seit Mai 1945 ist sein
Bruder Johann Pinzl vermisst in Russland. In unmittelbarer Nähe dieses Anwesens
steht das Wohnhaus des Alois Pinzl, das er sich 1934 dorthin gebaut hatte.

Das Dorf Vorderdietzberg zählt in seinen 9 Häusern 57 Einwohner.

.33 Die Einöde Wandlhof

Die Ortschaft Wandlhof hat mit großer Gewissheit früher einmal Wandldorf
geheißen und gehörte der Zeit der Vorsiedlung an, jener Zeit, in welcher unter
der Herrschaft des Klosters Metten in hiesiger Gegend die Maierhöfe Patersdorf,
Fratersdorf und Lämmersdorf entstanden sind. Diese „dorf“-Orte hatten eine
sommerseitige Lage und in ihrer nächsten Nähe waren ein oder zwei „ried“-Orte,
die sich aber mit einer weniger günstigen Lage begnügten. Die echten „Dorf“-Orte
sind mit einem Personennamen zusammengesetzt. So ist es auch mit dem Namen
Wandlhof, weil man bei diesem Ortsnamen unwillkürlich an den alten Namen Wantila
erinnert wird. Also hieß der Ort ursprünglich Wantilendorf. Hat sich diese
Ortschaft zu einem wirklichen Dorf im heutigen Sinn entwickelt, so blieb die
Ortsbezeichnung „dorf“. Blieb es aber nur ein einziger Hof, so wurde die
Ortsbezeichnung „dorf“ im 14. Jahrhundert umgewandelt in „hof“, sodass aus dem
ursprünglichen Wandldorf das Wandlhof wurde. Das war nicht bloß hier, sondern
derartige Umbenennungen gab es im Landkreis Viechtach mehrere.

Wandl-Hof: Nach dem Sal- und Stiftsbuch des Klosters Gotteszell 1790/99
bewirtschaftete ein Joseph Geiß diesen Hof. 1809 haben dann Joseph Geiß und
seine Hausfrau Maria die Wandlmühle um 2 700 Gulden übergeben, sodass dann 1843
als Besitzer von Wandlhof und Wandlmühle der Lorenz Geiß erscheint. Der Hof samt
Mühle war ehedem Erbrechtsweise grundbar dem Kloster Gotteszell. 1844 besaß ein
Johann Geiß den Wandlhof, der ihn um 4 800 Gulden übernommen hatte. Ein
Schollenrieder Achatz hat dann den Wandlhof käuflich erworben und Johann Geiß
verzog in den Markt Ruhmannsfelden und wurde der Marktwandler genannt. Eine
Schwester des Johann Geiß heiratete 1892 den Auhof-Bauern Michl Kraus, der dann
den Wandlhof bekam. Dieser Michl Kraus übergab dann später diesen Hof seinem
Sohne Alois Kraus, der ihn im Jahre 1939 seinem Sohne Alois Kraus übergab.

In der Nähe des Wandlhofes stand auch ein Brechhaus und in der Bahnhauzeit 1875
bis 1877 eine Marketenderei.

.34 Die Einöde Wandlmühle

Stand an einem Bauernhof an einem fließenden Gewässer, so wurde bei diesem Hof
auch eine Mühle errichtet. So entstanden bei dem Bruckhof die Bruckmühle, beim
Wandlhof die Wandlmühle, beim Stömmerhof die Stömmermühle beim Reisachhof die
Reisachmühle.

Wandl-Mühle: Von der Wandlmühle wissen wir, dass ein Wolfgang Steinbauer im
Jahre 1790 Müller auf der Wandlmühle und Witwer war und dass dieser Wolfgang
Steinbauer seine am 28.2.1776 an sich gebrachte Erbrechtsmühle nebst Haus und
Baumannsfahrnis, samt lebendem und totem Inventar um 800 Gulden an seine Base
Katharina Schürzinger, ledige Söldners-Tochter von Prünst übergeben hatte. 1809
besaß die Wandlmühle ein Lorenz Geiß, der sie von seinen Eltern Joseph und Maria
Geiß um 2 700 Gulden in diesem Jahre übernommen hatte. 1823 saß hier Joseph Geiß
auf dem Wandlhof, Haseneder-Hof, und ein Joseph Bauer auf der Wandlmühle. 1843
war Sebastian Obermeier Besitzer der Wandlmühle, die er am 4. April 1829 vom
Müller Thadäus Krauß um 1 850 Gulden erworben hatte. 1903 wurde der Besitzer der
Wandlmühle Peter Plötz sen. und Peter Plötz jun. bewirtschaftet die Wandlmühle
seit 5. Oktober 1937.

Der 8. Mai 1882 war für die Wandlmühle ein großer Unglückstag. Von Osten her,
über Regen und March, zog ein Ungewitter herauf. Es kam ein furchtbares
Hagelwetter, verbunden mit einem schweren Wolkenbruch. Der sonst so friedliche
Wandlbach wurde zu einem reißenden und zerstörenden Gewässer. Das Wasser nahm
die in der Reisachmühle aufgestapelten Blöcher mit, die mit solcher Wucht die
äußere Hausmauer in der Wandlmühle eindrückten, dass in der Wohnstube das Wasser
bis auf einige cm zur Zimmerdecke reichte und die Hagelkörner tischhoch in der
Stube lagen. Das Wasser nahm auch die Bettstatt samt dem Bett aus der Kammer mit
fort. Als nach einiger Zeit das Wasser des Wandlbaches wieder zurückging, lagen
auf den Bachwiesen neben dem Wandlbache so viele Fische, dass die Lämmersdorfer
und Giggenrieder diese Fische Eimer- und Körbe weise Heim tragen konnten. 150 m
hinter dem Inhaus stand früher das Brechhaus und in der Bahnbauzeit auch eine
Marketenderei.

Die Wandlmühle zählt 5 Einwohner.

.35 Der Weiler Weichselsried

Der Name „Weichselsried“ kommt urkundlich schon 1295 vor. 1577 erscheint der
Name „Weixleinsried“, 1633 Wäslansriedt und später der Name „Weytenried“.
Weichselsried war die Siedlung eines Weigiles oder Weigil, oder auch eines
Wigile oder Wigold.

Diese Siedlung wurde damals nebst einigen umliegenden Ortschaften von den
bayerischen Herzögen an das Kloster Aldersbach verkauft. Nach dem
Degenberg'schen Salbuche vom Jahre 1596 gehörte ein Wald bei Weichselsried zum
Degenberg'schen Waldbaue.

Mock-Anwesen: 1672 werden als Anwesensbesitzer von Weichselsried die Familie
Mock und später die Familie Hartmannsgruber genannt. 1843 hauste in
Weichselsried ein Georg Fink. Nach dessen Tod bewirtschaftete diesen Hof dessen
Ehefrau Anna Maria Fink, die mit ihrem Anwesen der Pfarrkirche Ruhmannsfelden
Erbrechtsweise grundbar war. Später war der Weichselsrieder-Hof im Besitze der
Wurzer, der sogenannte Hofbauer von March. 1905 ging der Hof dann über in den
Besitz des Joseph Hartl und 1919 auf dessen Sohn Joseph Hartl. Dieser hat sich
für den Weichselsrieder-Hof ein Anwesen in Hirschbach eingetauscht. Heute ist
der Hof in Weichselsried Eigentum des Franz Frisch.

Der Weiler Weichselsried mit dem Bauernhof, dem Ausnahmshaus und zwei Neubauten
zählt in den 4 Häusern 10 Einwohner.

.)a D`Weihwasserschwemm

Den alten Weichselsroider plagte das Zipperl zur rechten Zeit. Und da war ihm
halt gar nichts recht zu machen und da kam er bei seinem Sinnieren und Studieren
den ganzen lieben Tag schon über alles. Es war der Johannes-Tag. Da griff er
nach der Weihwasserflasche. In dieser war aber nur mehr ein Noagerl (Neigerl)
Weihwasser. Da sagte er zu seiner Tochter: „Lena, heut holst no a Weichwasser in
unserer Pfarrkirch! Am Johannes-Tag darf auf koan Bauernhof s'Weichwasser
ausgeh!“ Und der alte Weichselsroider sagte seine Befehle nicht zweimal.
Nachmittags suchte sich die Lena ein leeres Apothekerfläschlein verstaute es in
ihrem Handtäschchen und machte sich auf den Weg nach Ruhmannsfelden. Auf halbem
Wege dorthin hörte sie ihr wohl bekannten Töne einer Tanzmusik. Sie ging wieder
eine kurze Strecke und schon wieder klang die Musik an ihr Ohr. Ja! Es war ja
heute Johannes-Tag und an diesem Tage war das herkömmliche Gartenfest mit Musik
und Tanz im nahen Auerbach. Schnell war die Lena dort und der Tanz riss nicht
ab, bis der Hansenbauer Sepperl sagte: „Lena! Zum Hoamgehn is höchste Zeit.
D'Stallarbeit tu uns neamad!“ Auf dem Heimweg fiel der Lena plötzlich eint dass
sie Weihwasser heimbringen müsse. Und da waren beide gerade auf dem Wege
zwischen Hasmannsried und Weichselsried. Da ist ganz nahe am Wege eine Quelle,
aus der das ganze Jahr schönstes Wasser fließt und eine kleine Schwemme ist auch
dort. Schnell griff Lena nach dem Fläschchen und röhrlte es voll Wasser aus
dieser Schwemme. Daheim stellte sie es auf den Tisch und sagte ein klein wenig
schalkhaft: „Vata, an schöna Gruß vom Herrn Pfarra!“ Seit dieser Zeit heißt
diese Schwemme „d'Weihwasserschwemm!“

.36 Die Einöde Wolfsberg

Wolfsburg zählt zu den ältesten Siedlungen der Gemeinde Zachenberg. Es dürfte
wohl Ende des 9. oder Anfang des 10. Jahrhunderts entstanden sein, in der Zeit,
in welcher das Kloster Metten das hiesige Gebiet zu eigen hatte und in der neben
den „dorf“-Orten auch schon die „berg“-Orte entstanden. Allerdings müssen diese
“berg“-Namen mit einem Personennamen in Verbindung stehen, wie es ja auch bei
den damaligen „dorf“-Namen der Fall sein musste, wenn es echte „dorf“- und
“berg“-Namen sein sollten. So entstanden gleich nach Lämmersdorf die Siedlung
des Eckart = Eckersberg und die Siedlung des Wolfiches = Wolfsberg.

Widenpaur-Hof: 1596 hauste ein Stefan Widenpaur als Hofbesitzer in Wolfsberg. Am
21. November 1616 verlieh Herzog Max den hiesigen Hof an Lorenz Widenpaur, seine
Hausfrau Walburga und ihren Sohn Sebastian gegen Geldgilt auf den, Kasten
Linden. 1672 übernahm ein Stefan Widenpaurxlv den hiesigen Hof von seinem Vater
Mathes Widenpaur um 175 Gulden. 1721 bewirtschaftete den Wolfsberg ein Niklas
Achaz. Am 22. Februar 1723 wurden getraut ein Michael Kramhöller, ledigen
Standes, von Burggrafenried mit Maria Achaz, Witwexlvi von Wolfsberg. Am 26.
November 1755 heiratete ein Georg Kraus von Oberried die Walburga Kramhöller von
Wolfsberg. 1772 finden wir in Wolfsberg zwei bäuerliche Familien ansässig,
nämlich die eines Wolf Kramhöller und die eines Georg Kraus, der zwei Jahre
vorher übernommen hatte. 1779 bewirtschaftete das Widenbauern-Gut oder den
ganzen Hof von Wolfsberg ein Georg Kraus. Aus dieser Zeit ist noch eine Urkunde
vorhanden, die in den Händen des Ausnahmsbauers Peter Kraus sich befindet. Sie
ist 1780 ausgestellt und lautet:

“Wir, Karl Theodor von Gottes Gnaden Pfalzgraf bey Rhein, Herzog .... usw.,
bekennen als einzig regierender Landesfürst für Uns, Unsere Erben und
nachkommend regierenden Fürsten, daß wir aus Gnaden, dem Georg Kraus auf dem
Wiedenbauerngut oder ganzen Hof in Wolfsberg, Unseres Pfleggerichts und
Kastenamtes zu Linden, das von ihm und seinen Vorältern auf Leibrecht bisher
inne gehabte ganze Hofsgut zu eben benannten Wolfsberg nach Meynung Unseres
gnädigsten General-Mandats vom 3. May ad 1779 dergestalten auf Freirecht
verliehen haben, und in Kraft dieß dergestalten verleihen und verlassen, daß er
und alle seine Erben und rechtmäßgen Nachkommen besagten ganzen Hof zu
Wolfsberg, wie Erbrecht ist, hinfür innehaben, benutzen und genießen und
gebrauchen sollen und mögen.“

Dann folgen die Giltleistungen für ihn und seine Erben. Ein Sohn von diesem
Georg Kraus, nämlich Joseph Kraus, heiratete am 27. Oktober 1795 die Therese
Würr, Tochter des Georg Würr, Bauers in Langdorf und dessen Ehefrau Walburga,
geborene Weiß von Schwarzen. 1834 übernahmen den Widenbauern-Hof ein Johann
Kraus, dessen Ehefrau eine Anna Kreuzer, Bräuerstochter von Regen war und 1866
wieder ein Johann Kraus, dessen Ehefrau eine Magdalena Geiger, Bauerstochter von
Lämmersdorf war. Von 1901 bis 1946 bewirtschaftete den Wolfsberger-Hof der Bauer
Peter Kraus. Am Himmelfahrtstag 1922 schlug ein Blitz im Hofe ein. Der Blitz
fuhr an der Dachrinne entlang, am Viehstall vorbei, suchte sich die eichene
Planken im Rossstall, tötete eines von den 3 Pferden und fuhr dann im Boden an
den Wasserleitungsröhren entlang, die vollständig zusammen geschmolzen wurden,
sodass kein Tropfen Wasser mehr durchfließen konnte. Am 15. November 1932
brannte der Wolfsberger-Hof ab. Peter Kraus ließ den Hof sofort wieder aufbauen.
Seit 1946 bewirtschaftet den Hof dessen Sohn Johann Kraus.

Die Einöde Wolfsberg zählt in Hof und Ausnahmshaus 22 Einwohner.

.37 Das Dorf Zachenberg

Im Volksmund wird dieser Name „Zachaberg“ ausgesprochen. In früheren Urkunden
erscheint der Name Czachenperg und später Zachenperg. Das Dorf Zachenberg
verdankt seine Entstehung einem gewissen Zacco oder Zakko, der sich mit den
Seinen hier niederließ und rodete.

Einen Pabo von Zachenberg treffen wir in einer Niederalteicher Urkunde vom Jahre
1273. Ferner wird Zachenberg in der Bulle genannt, mit welcher Papst Gregor IX.
dem Kloster Oberalteich im Jahre 1274 seine Güter und Rechte bestätigt. Nach dem
herzoglichen Salbuche (Steuerbuch) aus der Zeit um 1280 waren zu jener Zeit hier
die Nussberger begütert. Drei Güter nannten sie in Zachenberg ihr Eigentum. Auch
das mächtige Rittergeschlecht der Degenberger hatte dereinst in Zachenberg
Besitzungen. Am 23. Mai 1393 verkaufte Stefan der Degenberger zu Altnussberg und
sein Albrecht etliche Güter zu Zachenberg, Muschenried, Gnänried, Auerbach,
Cschorleinzried, Ekkarzperg und weitere an ihre Vettern Konrad und Eberhard die
Nussberger zu Kollnburg. Der Kaufpreis betrug 180 Pfund Pfennige. Einige Jahre
später veräußerte der erstgenannte Degenberger seinen Zehent zu Zachenberg an
den Abt Andreas und den Konvent zu Gotteszell um 9 Pfund Pfennige. Mitsiegler
der hierüber ausgestellten Urkunde war des Verkäufers Sohn Albrecht. 1550 wurde
ein Öls Michael von Zachenberg um 3 Gulden gestraft, weil er seinen Nachbarn
Peter Schober im Rücken blutig geschlagen hatte. Im Salbuch des Kastenamtes
Viechtach vom Jahre 1577 werden uns folgende Anwesensbesitzer in Zachenberg
genannt:



Jakob Schwarzenberger, später saß auf diesem Hofe die Familie Pfeffer
(Pfeffer-Gut).

Michael Schwarzenberger, später die Familie Schlegl



Nach einer Gerichtsrechnung vom Jahre 1600 wurde Georg Prunner von Zachenberg um
2 Gulden gestraft, weil er zu seiner Hochzeit in Ruhmannsfelden 8 Personen zu
viel geladen hatte. 1631 wurde Balthasar Schurr wieder um einen Gulden, 8
Kreuzer, 4 Heller gestraft, weil er am Feste Maria Himmelfahrt vor der Sigzeit
Getreide in die Mühle brachte. Auch Adam Oischinger von Zachenberg wurde wegen
derselben Übertretung in Strafe genommen. Er erhielt in Anbetracht seiner
schlechten Vermögensverhältnisse ein Tag Gefängnis mit wenig Nahrung. 1656 waren
in Zachenberg die Hausbesitzer Christoph Loibl und Jakob Kilger. 1657 verkauften
Michl Amann von Zachenberg und seine Hausfrau Walburga ihre Sölde an den
dortigen Leinweber Georg Krampfl. 1659 veräußerte Magdalena, die Witwe des
Paulus Stattenbauer von Zachenberg mit Genehmigung der Vogtherrschaft Gotteszell
ihre Sölde zu Zachenberg um 50 Gulden, 45 Kreuzer an Hans Vest zu Zachenberg.
1660 heiratete ein Georg Krampft von Zachenberg eine Maria Härtl von Reinbeck.
1668 übergaben Adam Prunner von Zachenberg und seine Frau Rosina ihren
Erbrechtshof mit allem Zubehör ihrem Sohne Georg Prunner, noch ledig, um 400
Gulden. 1670 verkauften Lorenz Schlegl und seine Hausfrau ihr hiesiges
Erbrechtslehen um 335 Gulden an Thoman Reitmäyer und im selben Jahre überließen
Adam Kandler von Zachenberg und seine Frau Maria ihr Erbrechtslehen mit allem
Zubehör um 350 Gulden ihrem Sohn Paul Kandler. 1676 ging die Erbrechtssölde der
Familie Lehner um 162 Gulden und 1 Reichstaler Leykauf an einen Georg
Schwarzenberger von Grub über. 1691 übernahm Wolfgang Schlegl das Anwesen seiner
Eltern um 630 Gulden. 1694 starb Thoman Reitmäyer, Weber von Zachenberg und
hinterließ 5 Kinder. Michl, der Jüngere übernahm diesen Hof um 550 Gulden. In
der Hauptsteuerbeschreibung des Gerichts Viechtach vom Jahre 1721 wird
Zachenberg als der Sitz einer Hauptmannschaft genannt. Damals bestanden die
folgenden Anwesen in Zachenberg:



Michl Prunner

Michl Muhr

Michl Achaz

Michl Sigl

Hans Amann

Michl Amann

Wolf Prunner

Georg Löffler

Georg Pfeffer

Andreas Müller

Georg Kasperhauer

Michl Reitmäyer

Georg Pfeffer, der Ältere



Auch in der ersten Hälfte des 18. Jahrhunderts gab es eine große Anzahl von
Übergabeverhandlungen, z. B. vom Paulus Kandler, von Ambrosius Müller, von Wolf
Pfeffer usw. Ohne gerichtliche Verurteilungen ging es in derselbigen Zeit für
die Zachenberger auch nicht ab: 1726 hat man Mathias Sigl und George Marchl,
gewester Inwohner daselbst wegen hitzigen Wortstreits und darauf folgende
Rauferei jeden zu 17 Kreuzer, einen Heller bestraft.1740 wurde der Bauer Michl
Prunner von Zachenberg, weil er in dem kurfürstlichen Hochwald ohne Erlaubnis
und ordentliche Anzeige Holz geschlagen hatte, nebst einem erhaltenen Verweis um
34 Kreuzer, 2 Heller gestraft. 1740 erhielt der Halbhöfler Andreas Müller in
Anbetracht seiner Bedürftigkeit eine Geldstrafe von 8 Kreuzer, 4 Heller, weil
seine Küche unterhalb des Rauchfangs und der Backröhre unsauber und mit
angebranntem Holze befunden worden war. 1740 wurde Christoph Löffler um 17
Kreuzer, einen Heller bestraft, weil er sich unterstanden hatte, die Walburga
Schleglin, ein lediges, armes Mensch aus der Kloster Gotteszeller Hofmark ohne
obrigkeitliche Bewilligung in Herberg zu nehmen. Zugleich erhielt er den
Auftrag, dieselbe sofort aus der Herberge zu tun. 1760 wurde der Halbbauer Michl
Kasberbauer um 2 Gulden, 17 Kreuzer in Strafe genommen, weil er noch nicht
genügend ausgekühlte Asche unter das Dach hinauf trug, wo dieselbe dann zu
schmelzen anfing und im Boden ein Loch hinein brannte. In dem Salbuche des
Klosters Gotteszell vom Jahre 1790 finden wir in Zachenberg folgende
Anwesensbesitzer:



Johann, dann Michl Amann

Jakob Brunner

Joseph, dann Andreas Pfeffer

Georg Kronschnabel

Georg Loibl

Georg Reitmayer

Christoph Löffler

Michl Edenhofer

der Söldner Georg Pfeffer

der Lehensbesitzer Georg Pfeffer

Joseph Treimer

Andreas, dann Michl Brunner



1801 heiratete Joseph Schlegl, Bauer von Zachenberg, die Bauerstochter Anna Geiß
von Wandlhof, die 600 Gulden und eine standesgemäße Ausstattung erhielt. 1823
bestanden in Zachenberg 19 Anwesen, nämlich:



Michl Sigl auf dem Leibl-Hof

Joseph Schlögl auf dem Rosenstängl-Hof

Georg Loibl auf dem Müller-Hof

Andreas Pfeffer auf dem Achaz-Hof

Michl Artmann auf dem Ebnerhansl-Hof

Michl Reitmayer

Paul Pfeffer auf dem Ernst-Hof

Georg Pfeffer auf dem Maier-Hof

Georg Kerschl

Georg Krombauer auf dem Steffel-Hof

Jakob Brunner

Christoph Löffler

Georg Pfeffer auf dem Seppen-Gut

Lorenz Edenhofer

Auf dem Kasperbauern-Hof

Andreas Krampft

Georg Treiber

Anton Brumbauer, Schneider

Johann Gierster



Der Grundsteuerkataster vom Jahre 1843 nennt uns in Zachenberg die folgenden
Anwesensbesitzer:



Franz Vogl

Johann Kufner, beim Loibl

Georg Pfeffer, beim Waschberger

Johann Wanninger, beim Ebnerhansl

Joseph Kranschnabel bei Reitmayer

Georg Pfeffer, beim Mayer

Joseph Angl, von Georg Pfeffer gekauft

Joseph Schlögl aus der Verlassenschaft der Anna Mauer, nachher verehelichte Hof,
übernommen

Anton Pfeffer

Lorenz, dann Joseph Edenhofer

Georg Kronschnabel

Johann Pfeffer, beim Brunner

Christoph Löffler

Jakob Kerschl, vom Vater Georg Kerschl

Johann Gürster, Weher

Johann Härtl, beim Brumbauer

Andreas Treiber

Andreas Krampfl, Weber

Lorenz Steinbauer, Wagner



Die Ortschaft Zachenberg hatte ein eigenes Brechhaus mit einem eigenen Hirten.

Pritzl-Anwesen: Die zu dem Anwesen des Georg Pritzl, Binder in Bruckberg
gehörigen Grundstücke stammen aus der Gantmasse des Michl Reitmauer von
Zachenberg. 1875 verkaufte Franz Vogl, Bauer in Zachenberg den Brumbachacker an
den Bauern Georg Kronschnabel von Zachenberg. Von diesen genannten Höfen stehen
aber heute nicht mehr alle, sind entweder vertrümmert und weggerissen worden
(Ebnerhansl-Hof), sind abgebrannt und neu aufgebaut worden (Kerschl-Hof), oder
die alten Gebäulichkeiten sind durch Neubauten ersetzt worden (Weber-Gut).

Ebnerhansl-Hof: Der Ebnerhansl-Hof steht nicht mehr. Das Ebnerhansl-Häusl, auf
dem es früher Hanninger geheißen hat, kaufte 1914 Lorenz Fischl. Dieses Anwesen
bewirtschaftet seit 1947 die Witwe Rosa Fischl, deren Sohn Heinrich Fischl 1943
im Kaukasus gefallen ist.

Wagner-Haus: Das gegenüber liegende frühere Wagner-Haus des Franz Meier besitzt
seit 1931 Michael Fischl. Das Marterl, das in dessen Garten steht, wurde
ausgeheißen für eine glückliche Heimkehr aus dem Kriege. Auf dem früheren
Reitmeier-Hof beim Kronschnabel hieß es später beim Oisch, seit 1909 beim Hartl
Joseph und seit 1939 beim Geiger Michl.

Meier-Hof: Auf dem Meier-Hof, der noch in seiner ursprünglichen Bauart erhalten
sein dürfte, hieß es früher beim Georg und Joseph Pfeffer, später beim Georg
Kilger. 1913 heiratete auf diesen Hof Xaver Ertl. Dessen Sohn Joseph Ertl
übernahm diesen Hof. Dieser ist aber seit 1945 vermisst. Die Frau Anna Ertl die
diesen Meier-Hof (Moar-Hof) bis 1953 bewirtschaftet hat, verpachtete 1953 den
Hof an Alfons Geiger.

Kilger-Krämerei: An diesen Meierhof reiht sich die von Joseph Kilger 1928 neu
erbaute Krämerei, auf der es jetzt seit 1949 Joseph Kramhöller heißt.

Ernst-Karl-Anwesen: An die mitten im Dorfe stehende Kapelle schließen sich die
Anwesen des Schlögl Girgl an, welches seit 1891 in Händen des Michl
Schnelldorfer ist und das Anwesen des Ernst Karl, das dann von dem Schmied
Ludwig Schiendlmeier käuflich erworben wurde, der in diesem Anwesen eine
Schmiede errichtete. Unter dem Besitzer Kellermeier wurde in diesem Haus eine
Krämerei betrieben. Dieses Anwesen, das früher zum Schlögl-Hof gehörte,
bewirtschaftet seit 1948 August Bielmeier. Der Stiefbruder der Frau Bielmeier,
Joseph Schwarzensteiner ist seit 1944 in Polen vermisst. Den Hof des Joseph
Achaz kaufte 1887 Pfeffer Anton, ein gebürtiger Zachenberger. 1921 hat diesen
Hof der Sohn Heinrich Pfeffer übernommen. Dessen Bruder Anton Pfeffer ist 1915
im Elsass gefallen.

Zitzelberger-Ludwig-Anwesen: Das alte Anwesen des Joseph Zitzelsberger wurde
weggerissen und neu aufgebaut. Seit 1948 besitzt dieses Anwesen Ludwig
Zitzelsberger. Sein Bruder Xaver Zitzelsberger ist 1916 in Frankreich gefallen.

Waschberger-Anwesen: Beim Waschberger hieß es einmal Georg Pfeffer. Der Hausname
war beim Goglbauern. Auf diesen Hof heiratete ein Joseph Edenhofer. 1898 erwarb
ihn Michl Kandler, der ihn 1921 seinem Schwiegersohne Alois Kronner von
Lämmersdorf übergab. Dieser starb 1948. Seit dieser Zeit bewirtschaftet die
Witwe Anna Kronner diesen Hof. 1903 ist die große Schupfe abgebrannt, die 1904
wieder aufgebaut wurde. 1953 führte Frau Kronner einen Umbau in ihrem Anwesen
aus.

Kerschl-Hof: Der alte Kerschl-Hof ist 1939 abgebrannt und es wurde an dessen
Stelle ein neuzeitlicher Bau aufgeführt. Der jetzige Besitzer dieses Hofes ist
seit 1945 Anton Kerschl jun., der im Wohnhaus des Hofes eine Gaststätte und
daneben eine Kegelbahn eingerichtet hatte. Den alten Kerschl-Hof
bewirtschafteten die Geschwister Anna und Maria Kerschl, Anton Kerschl, dessen
Vater Michl Kerschl war, früher ein Jakob und auch ein Georg Kerschl.

Kellermeier-Anwesen: Auf dem Anwesen des Sebastian Kellermeier hieß es früher
Blüml, Kandler, Friedrich, Edenhofer usw. Es wurde dieses Haus 1935 renoviert
und 1953 vollständig neu gebaut. Der Ausbau des 1. Stockes erfolgte
nachträglich. Ein Joseph Edenhofer stürzte in den Keller dieses Hauses an dessen
Folgen er starb. Joseph Kellermeier ist 1940 in Polen gefallen.

Niedermeier-Anwesen: Das Niedermeier-Anwesen war das frühere Bitzenbauern-Gut,
das früher ein großer Bauernhof gewesen sein muss. Zuletzt aber war nur mehr
wenig Grund und nur mehr zwei Kühe vorhanden. Niedermeier sen. hat noch das
Feil-Anwesen dazugekauft. Dieses Besitztum hat er 1929 seinem Sohne Alois
Niedermeier übergeben, der seit 1945 in der Ostsee als vermisst gilt. Ludwig
Niedermeier kam aus dem 2. Weltkrieg krank zurück und starb 1948 in der Heimat.
1930 wurde der Stadel und 1947 das Haus neu gebaut. Das Anwesen, das früher
einem Kilger und einem Meier gehörte, erwarb 1924 Alois Schnelldorfer und seit
1937 gehört es einem Johann Lippl.

Brunngell-Anwesen: Das alte Brunngell-Anwesen steht nicht mehr.

Pfeffer-Anwesen: Das frühere Pfeffer-Anwesen, auf dem ein Johann Pfeffer später
ein Kopp wirtschaftete, brannte 1905 nieder. Auf diesen neu gebauten Hof zogen
1910 die Brunner von Köckersried und 1920 hat dann Johann Brunner übernommen.
Seit 1939 ist Anton Dobusch der Besitzer dieses Hofes.

Zitzelsberger-Anwesen: Auf dem Anwesen des Johann Zitzelsberger, früher Mock,
heißt es jetzt Hermann Ertl, der 1946 an Stelle des alten, hölzernen Gebäudes
einen gemauerten Bau aufführte und 1953 einen neuen Dachstuhl aufsetzte, sodass
dieses Einfamilienhaus einen freundlichen Eindruck macht.

Weber-Anwesen: Das daneben stehende Weber-Anwesen hat seinen Namen von seinem
früheren Besitzer Krampfl, der das Weberhandwerk ausübte. Das Weber-Haus kam in
den Besitz des alten Brandl, dem es sein Schwiegersohn Joseph Treml abkaufte,
niederriss und neu aufbaute. Von diesem Treml sind zwei Söhne 1947 in Russland
gefallen, Xaver und Johann Treml. Seit 1947 bewirtschaftet das Anwesen der
Schwiegersohn Xaver Rossberger.

Brunner-Wirtshaus: Das Wirtshaus, von Brunner in Lindenau erbaut, dessen
Besitzer Niedermeier Joseph, Heller, Wagner, Schönberger, Brunner, Holzfurtner,
Schrötter waren, gehört seit 1949 dem Xaver Kufner.1953 wurde die Schupfe des
Wirtshauses neu gebaut.

Fleischmann-Hof: Der Fleischmann-Hof ist schon seit vielen Jahrzehnten in den
Händen der Fleischmann. Ganz früher bewirtschafteten ihn die Klimmer, die Vogl,
die Schuster. Dann kamen die Fleischmann. 1919 heiratete ein Hanninger auf
diesen Hof. Dieser starb aber bald, sodass 1921 Fleischmann Joseph diesen Hof
übernehmen konnte. 1905 brannte das Leithumshaus dieses Hofes ab, das aber
sofort wieder aufgebaut wurde. 1937 wurden die Wagenremise und 1950 der Backofen
und der Schweinestall neu gebaut. Fleischmann Michael ist 1916 in Frankreich
gefallen.

Schlögl-Alois-Anwesen: Auf dem Wege nach Kleinried steht das frühere
Schlögl-Alois-Anwesen, das seit 1935 dem Joseph Schnelldorfer gehört.

Häuslbauern-Hof: Dann folgt der Häuslbauern-Hof, der 1874 einem Joseph Pfeffer
gehörte. Dieser vertauschte diesen Hof mit Brumbauer von Habischried. 1883 kam
der alte Joseph Blüml auf diesen Hof. Dieser starb 1903 und 1906 übernahm der
Sohn Joseph Blüml den Häuslbauern-Hof. 1951 verkaufte dieser den Hof an die
Tierzuchtgenossenschaft des Landkreises Viechtach. Ein Stiefsohn des Joseph
Blüml, Ludwig Baumgartner, ist 1941 in Russland gefallen.

Ganz abseits und allein auf dem sogenannten Geiger-Acker stehend befindet sich
das Weißhäupl-Anwesen, das 1883 von Katharina Haimerl erbaut wurde. 1903 erwarb
es Alois Weißhäupl und seit 1932 gehört es Johann Weißhäupl. 1949 wurden Stall
und die Scheune neu gebaut. Der Sohn Alois Weißhäupl ist 1914 in Frankreich
gefallen.

Baumgartner-Anwesen: An den Häuslbauern-Hof anschließend folgt das Anwesen des
Johann Baumgartner.

Löffler-Anwesen: Das landwirtschaftliche Anwesen des Löffler Uil (Ulrich) wurde
von einer Geiger Franziska erbaut. 1886 erwarb es Löffler Uil. Dieser starb 1946
und das Anwesen bekam der Sohn Xaver Löffler. Heute ist das Anwesen verpachtet
an den sudetendeutschen Meier.

Holzapfel-Haus: Auf dem Holzapfel-Haus hieß es früher Ernst Lorenz, der im
Volksmund „Gschloß Lenz“ genannt wurde, weil er auf dem Schloss-Gut Linden lange
Zeit Dienstknecht war. Nach dem Ernst hat es dann Holzapfel Alois bekommen.

Ernst-Anwesen: In der Nähe ist dort das Ernst-Anwesen, das durch Einheirat 1935
ein Joseph Gierl bekam. Seit 1951 heißt es auf diesem Haus Alois Blüml, während
auf dem daneben stehenden Ernst-Joseph-Anwesen ein Otto Kilger heiratete.

Kappl-Anwesen: Auf dem Wege von Zachenberg nach Köckersried ist das frühere
Kappl-Anwesen, das Kappl Johann sen. von einem Rau Michl gekauft hatte und auf
dem es früher Brunner Michl geheißen hat. Seit 1939 gehört dieses Anwesen dem
Albert Zitzelsberger. Früher war in diesem Hause eine Schmiede.

Thurnbauern-Hof: Daneben steht der alte Thurnbauern-Hof, den der alte, bereits
verstorbene Thurnbauer von der Frath, Gemeinde Drachselsried abstammend, kaufte.
Diesen Hof übernahm 1944 der Sohn Alois Thurnbauer. Dessen Bruder Johann
Thurnbauer ist 1939 in Polen gefallen. Der andere Bruder Joseph Thurnbauer
führte in unmittelbarer Nähe des Thurnbauern-Hofes einen Neubau 1942 auf.

Englmeier-Anwesen: Das Englmeier-Anwesen am Fuße des Kühberges wurde 1870 von
Xaver Schmied gebaut, und von diesem bis 1893 bewirtschaftet. Von 1893 bis 1898
gehörte es einem Jakob Pfeffer, der es dann 1898 verkaufte an Johann Englmeier.
1935 übernahm das väterliche Gut der Sohn Xaver Englmeier. Ein Bruder von ihm,
Johann Englmeier, ist seit 1915 vermisst in Russland.

Brandl-Anwesen: Auf dem Wege von Zachenberg zum Bahnhof/Gotteszell steht das
Brandl-Anwesen. Der alte Brandl hat das Weber-Anwesen verkauft an seinen
Schwiegersohn Joseph Treml und hat dann den Ebner-Hansl-Kasten in der Hachö
gekauft und ein landwirtschaftliches Anwesen daraus errichtet, das jetzt der
Sohn Xaver Brandl bewirtschaftet. Gegenüber steht das Einfamilienhaus des Joseph
Brandl, das er 1945 gebaut hatte. Dessen Bruder Xaver Brandl ist in Frankreich
vermisst.

Wurzer-Hof: Dann folgt der Wurzer-Hof auf der südlichen Seite des Gaisberges.
Früher hieß es auf diesem Hofe beim Alois Friedrich. 1904 hat ihn Wurzer Johann
gekauft. Am 3. März 1910 ist dieser Hof abgebrannt und im gleichen Jahre wieder
aufgebaut worden. 1928 hat ihn der Sohn Otto Wurzer übernommen, und Waschhaus
und Futterkammer neu gebaut. Einmal hatte auch der Blitzschlag den Stadel
beschädigt, was aber sofort wieder repariert wurde.

Ebner-Anwesen: Auf dieser Seite des Gaisberges steht auch das Anwesen des Xaver
Ebner, das früher dem alten Brandl von Zachenberg gehörte und wahrscheinlich in
der Bahnbauzeit errichtet wurde. Dieses Anwesen wurde nach dem Besitzer
Baumgartner 1931 von Georg Ebner erworben und 1950 von Xaver Ebner jun.
übernommen. Dieser hat 1949 auf den Stall eine Wohnung darauf gebaut. Ebner
Georg ist 1944 in Russland gefallen. Baumann Franziska, geb. Ebner, hat auf der
Westseite des Gaisberges ein Einfamilienhaus gebaut.

Dr.-Sell-Grundstück: In der Nähe dieses Hauses befindet sich das große
Grundstück des Herrn Dr. Sell, Deggendorf, das in nächster Zeit bebaut wird.

Kraus-Anwesen: Auf der Nordseite des Gaisberges steht das landwirtschaftliche
Anwesen der Franziska Kraus, deren Ehemann 1934 im Steinbruch Wagner,
Muschenried, tödlich verunglückt ist. Dieser Joseph Kraus hat 1904 auf dieses
Anwesen geheiratet, das damals dem Johann Wanninger gehörte. Zuvor hießen die
Besitzer Huber, Kraus usw. 1912 wurde der Stall, 1919 der Stadel und 1921 das
Wohnhaus gebaut.

Ertl-Haus: Auf dem Wege Zachenberg/Osterbrünnl steht das 1949 neu gebaute
Einfamilienhaus des Georg Ertl, das nunmehr im Besitze des Ernst Karl von
Kleinried ist.

Mock-Anwesen: Auf dem Wege von Zachenberg/Kirchweg ist das Anwesen, das 1907
Joseph Mock gebaut hatte und das jetzt von Max Mock bewirtschaftet wird. Dessen
Sohn Joseph Mock ist 1916 in Galizien gefallen und der Sohn Albert Mock wird
vermisst. Das daneben stehende Haus hat 1929 Johann Mock gebaut.

Inmitten des Dorfes Zachenberg steht die Dorfkapelle auf ganz freiem Platz. Das
ist die einzige Kapelle in der ganzen Gemeindeflur Zachenberg, in der noch
Gottesdienst gehalten wird. Es wird erzählt, dass auf dem Vorplatz dieser
Kapelle in der Pestzeit ein Wagen stand. Schaute die Deichsel dieses Wagens zur
Kapelle hin, so gab es keinen Pesteten in Zachenberg oder nächster Umgebung.
Schaute aber die Deichsel von der Kapelle weg, so war im Dorf oder in dessen
Nähe eine Pesteter zu suchen, zu begraben, was mit großem Schwierigkeiten
verbunden war, da sich diese Personen vollständig vermummen mussten, auch
Gesicht und Hände. Es wird behauptet, dass in der Pestzeit die gesamte
Bevölkerung der Ortschaft Zachenberg an der Pest gestorben ist. Eine einzige
Weibsperson soll noch übrig geblieben sein, die zu Lebzeiten eine Linde pflanzte
zur Erinnerung an diese schreckliche Zeit und zum Dank für ihre glückliche
Rettung. Diese Linde stand beim Pfefferhof (jetzt Dobusch) und hat beim Brande
dieses Hofes 1905 großen Schaden gelitten. Außer der Pest brachte auch eine
entsetzliche Viehseuche großes Unglück über die Zachenberger. Nicht ein Stück
Vieh der ganzen Ortschaft Zachenberg blieb von dieser Seuche verschont. Seit
dieser Zeit, es mögen wohl 140 Jahre her sein, wallfahren die Zachenberger aus
Anlass dieser Viehseuche jedes Jahr nach Loh (Stephansposching) im Juni am St.
Bennotag.

Das Dorf Zachenberg zählt in seinen 38 Häusern 277 Einwohner.

.)a Der letzte Bär im Bärenloch

Zwischen Zachenberg und Köckersried ist das Bärenloch, das seinen Namen davon
hat, weil dort einmal Bären hausten. Erwachsene und Kinder die sich auf dem Wege
durch das Bärenloch befanden, hatten öfters Gelegenheit einen alten Bären zu
sehen. Im Allgemeinen waren die Bären, wenn auch zur Gattung der Raubtiere
gehörig, nicht grausam und blutdürstig. Im Gegenteil! Wenn ein Bär, Menschen
witterte, wich er aus und flüchtete. Dieser alte Bär aber ließ die Menschen
unbehelligt an sich vorüberziehen, brummte höchstens leise vor sich hin und
wartete darauf, ob die Menschen nicht Brot, Eier, Obst auf dem Wege liegen
ließen. Ameiseneier und Honig, das waren allerdings seine Leckerbisse. Die
Bienen lebten zur damaligen Zeit noch nicht in Bienenkörben oder Bienekästen wie
heute, sondern in den hohlen Bäumen der Wälder. Da war es für den Bären eine
Leichtigkeit, den Bienen den Honig zu rauben. Das wussten die Bienen nur allzu
gut. Sie warteten schon längst auf die Gelegenheit, sich einmal an dem Bären
rächen zu können. Da saß der Bär an einem warmen Herbstnachmittag im
abgefallenen Laub des Buchenwaldes. Leise brummte der Bär vor sich hin, ein
Zeichen des Wohlbehagens. Die Bienen, die in nächster Nähe in einer mächtigen
hohlen Eibe ihr Heim hatten, hörten das Brummen des Bären und sie kamen alle
herbei und von allen Seiten stürzten sie auf den Bären los. Ein ganzer Schwarm
Bienen umhüllte den Kopf des Honigräubers. Sie stachen ihn in die Ohren, in die
Augen, in die Schnauze, in die Zunge, in die Pranten. Der Bär wälzte sich vor
Schmerzen und brummte so laut, dass man dies weit und breit hören konnte.
Infolge der unzähligen Bienenstiche konnte der Bär nicht mehr hören, nicht mehr
sehen, nicht mehr gehen und weil er auch nicht mehr fressen konnte, musste der
letzte Bär vom Bärenloch vor 150 Jahren jämmerlich verhungern. Die Haut diesen
Bären wurde als Schlittendecke verwendet und gab viele, viele Jahrzehnte hernach
noch davon Kunde, dass es im Bärenloch wirklich Bären gegeben hatte.

.38 Das Dorf Zierbach

Das Dorf Zierbach besteht aus dem alten Zierbach und den während der Bahnbauzeit
und nach derselben errichteten Anwesen. Das alte Zierbach steht auf einer Anhöhe
mit einem wunderschönen Ausblick. Das mag früher ein herrlicher Anblick gewesen
sein, als die Talmulde zu Füßen von Zierbach mit Wasser angefüllt war und einen
See bildete. Im Laufe der Jahrhunderte hat aber dieses Wasser sich einen Abfluss
gesucht nach Osten zum Regen und nach Westen durch das Wandlbachtal zur
Teisnach. So liegt Zierbach, im Volksmunde Zierwa genannt, auf einer mäßig hohen
Anhöhe, schmuck und schön, wie eine Zier. Nach Osten zu ist eine Talmulde, die
von der Eisenbahnlinie Landshut/Eisenstein durchzogen wird. Südlich ist eine
bewaldete Bergkette und nach Osten und Norden ist der Blick frei über ein
welliges Gelände in Richtung Triefenried, Regen und hin zur Arberkette. In den
alten Urkunden erscheint der Name Cyrbach und auch Zierwa. Zierbach wird
ebenfalls unter den Gütern genannt, welche im Jahre 1295 die bayerischen Herzöge
an das Kloster Aldersbach verkauften.

1678 übergeben Michl Lorenz von Zierbach, Gotteszeller Untertan und seine
Hausfrau Margareta ihren Erbrechtshof zu Zierbach ihrem Sohne Wolf Lorenz um 400
Gulden. 1720 wurde derselbe Hof von Andre Lorenz, Bauer zu Zierbach und seiner
Hausfrau Margareta mit allem Ein- und Zubehör, samt dem angebauten Sommer- und
Wintergetreide, 2 Pferden, 2 Ochsen, 4 Kühen, 6 Stierl, 3 Kälbern, 2
Eheschaftbetten usw. ihrem Sohn Michl Lorenz übergeben. Diesen Michl Lorenz
finden wir noch 1763 auf diesem Hof. Nach dem Sal- und Stiftsbuche des Klosters
Gotteszell von 1790/99 besaß dasselbe in jener Zeit hier 2 Höfe, von denen der
eine in den Händen von Andre Geiger, der andere aber in jenen von Georg
Zeidelhofer war. Die beiden Höfe fielen nach Aufhebung des Klosters Gotteszell
(1803) dem Staate zu.

1823 saßen hier die Familien Krotzer und Zeidelhofer. 1843 war Johann
Zeidelhofer Mühl-Bauer. Er hatte den Hof von seinen Eltern Georg und Katharina
Zeidelhofer 1841 um 3 300 Gulden übernommen. Ihm folgte im Besitze des Hofes
Peter Knott, der durch Heirat der Witwe Katharina Zeidelhofer gegen ein
eingebrachtes Vermögen von 2 500 Gulden Bauer hier wurde.

Krotzer-Hof: Auf dem anderen Hofe wirtschaftete 1843 Michl Kratzer, genannt
Krotzer. Er erhielt den Hof von seinem Vater Andre Kratzer 1810 um 2 700 Gulden.
Auf diesen Krotzer-Hof kam später ein gewisser Fink. Nach diesem hieß es auf
diesem Hof ab 1910 Joseph Freisinger, ab 1915 Johann Freisinger und jetzt
bewirtschaftet seit 1948 diesen alten, schönen und großen Hof Xaver Freisinger.
2 Söhne des Johann Freisinger sind im ersten Weltkrieg gefallen, nämlich Joseph
Freisinger 1915 in Serbien und Alois Freisinger 1915 in Frankreich. Auch ein
Sohn des früheren Hofbesitzers Fink ist im ersten Weltkrieg gefallen.

Zeidlhofer-Anwesen: Das Zeidlhofer-Anwesen hat später ein Peter Knott erworben
und nach diesem Alois Biller, der es 1943 an seinen Sohn Anton Biller übergehen
hatte. Rauch Joseph aus dem Biller-Hause ist 1913 im Steinbruch Göttleinsberg
tödlich verunglückt.

Lippl-Anwesen: Auf dem Anwesen des Wolfgang Lippl heißt es seit 1928 Alois
Kramhöller. Das frühere Geiger-Anwesen bewirtschaftete ab 1904 ein Alois Meindl,
ab 1936 ein Kufner und seit 1945 ein Max Weikl. Ein Sohn Kufners, nämlich Alois
Kufner, musste 1941 bei einem Schiffsunglück bei der Insel Kreta sei junges
Leben opfern.

Ebner-Hof: Der Ebner-Hof wurde 1909 von Ludwig Ebner gebaut und 1919 von Michl
Ebner übernommen. Dieser Michl Ebner ist 1948 gestorben. Dann hat die Witwe
weiter gewirtschaftet, bis dann Joseph Ebner 1953 den Hof übernommen hatte.

Schwarz-Anwesen: Das Hermann-Schwarz-Anwesen dürfte wohl in der Bahnbauzeit
entstanden sein. Seit 1949 bewirtschaftet dieses Anwesen Lorenz Schwarz, der
auch eine Schreinerei betreibt. Ferdinand Schwarz ist in einem Lazarett in
Göggingen gestorben. Das Schwarz-Anwesen ist 1914 abgebrannt. Neubauten in
Zierbach sind das Anwesen der Therese Schwarz 1935, das Anwesen des Johann
Zitelsberger (1949) und des Georg Fritz 1950. Seit 1948 ist Zierbach
elektrifiziert und 1952 wurde die Strasse Triefenried/Bahnhof über Zierbach nach
Göttleinsberg (1200 m) neu gebaut, die auch für Pkw und Lkw fahrbar ist.

Das Dorf Zierbach zählt in seinen 8 Häusern 68 Einwohner.xlvii

\chapter{Wirtschaftliches und Kulturelles von der Gemeindeflur Zachenberg}

\section{Größe und Bevölkerungszahl der Gemeinde Zachenberg}

Der Gesamt-Flächeninhalt der Gemeinde Zachenberg beträgt 2 731 ha = 8 015,85
Tagwerk. Davon entfallen auf Wald 1 033,74 ha, Wiesen 708,46 ha, Felder 635,20
ha. Ödland 67,25 ha, Weide 78,06 ha, Wasser 3,52 ha, Straßen 55,80 ha, Gebäude-
und Hofflächen 84,85 ha und Eisenbahngelände 44,12 ha.

Die Gemeinde Zachenberg ist demnach im Landkreise Viechtach die Drittgrößte
Gemeinde, nachdem der Gesamtflächeninhalt der Gemeinde Drachselsried 3 471 ha =
10 187,90 Tagwerk und der der Gemeinde Achsfach 3 018 ha = 8 858,25 Tagwerk
beträgt. Alle übrigen Gemeinden im Landkreis Viechtach sind nicht so groß. Die
kleinste Gemeinde im Landkreis Viechtach ist die Marktgemeinde Ruhmannsfelden
mit 580 ha = 1 702,34 Tagwerk.

Die Gemeinde Zachenberg hatte nach der Volkszählung vom Jahre 1950 1 049
männliche und 1 171 weibliche Einwohner = 2 220 Einwohner. Zachenberg stand mit
seiner Einwohnerzahl im Jahre 1950 an 4. Stelle im Landkreise Viechtach. Die
Stadt Viechtach zählte damals 4 126 Einwohner, der Fabrikort Teisnach 2 657
Einwohner, der Markt Ruhmannsfelden 2 518 Einwohner. Vergleichsweise hatte die
Gemeinde Drachselsried im Jahre 1950 weniger Einwohner als die Gemeinde
Ruhmannsfelden, obwohl Drachselsried flächenmäßig 20-mal größer ist als
Ruhmannsfelden. Die bevölkerungskleinste Gemeinde im Landkreis Viechtach ist die
Gemeinde Wiesing mit 564 Einwohnern.

Aus dem Verhältnis von Flächengröße und Einwohnerzahl einer Gemeinde ergibt sich
deren Bevölkerungsdichte. So ergibt sich bei der Stadtgemeinde Viechtach mit
ihrer geringen flächenmäßigen Ausdehnung und ihrer großen Einwohnerzahl eine
Bevölkerungsdichte von 491 Personen pro qkm, Ruhmannsfelden 429 Personen,
Gotteszell 138 Personen, Drachselsried 88 Personen, Zachenberg 81 Personen. Die
Gemeinde Achslach nimmt nach ihrer Flächenausdehnung im Landkreis Viechtach die
zweite Stelle ein, ist aber am dünnsten besiedelt, da nur 39 Personen auf 1 qkm
treffen.

\section{Ursprüngliche Lebensweise und Namensgebung}

Die ersten Ansiedler der Gemeindeflur Zachenberg mussten zunächst die Wälder weg
brennen den Boden entsteinen und anbaufähig machen, befahrbare Wege herrichten,
Moore und Sümpfe entwässern, usw. Zugleich waren diese Leute auch Jäger und
Fischer. Sie mussten sich ihre Hütten und Höfe bauen, sich ihre Wägen und
Karren, das Geschirr für die Zugtiere, alle Arbeits- und Hausgeräte, die Waffen,
ja selbst das ganzen Werkzeug selbst anfertigen. Die Frauen haben gekocht,
gemahlen, gegerbt und genäht, gesponnen und die Kleider für Männer, Frauen und
Kinder selbst angefertigt, den süßen Trank, den Met gebraut und aus den
Heilkräutern die Medizin hergestellt. Diese Leute mussten in damaliger Zeit
alles aus sich selbst heraus machen. Zu kaufen gab es noch nichts.

Ursprünglich führten die Menschen nur den Taufnamen, z. B. den Namen Peter.
Wurde dieser Peter ein guter Wagner, oder ein guter Schmied, oder ein guter
Schreiner, so wurde dieser Peter in der Folgezeit der Wagner Peter oder der
Schmied Peter oder der Schreiner Peter. So entstanden zwei Namen und zwar der
ursprüngliche Taufname und der Handwerksname wurde der Familienname.

\section{Wirtschaftlichen Verhältnisse in der Gemeinde Zachenberg}

Was die wirtschaftlichen Verhältnisse in der Gemeinde Zachenberg betrifft, so
standen von jeher Landwirtschaft und Waldwirtschaft an erster Stelle, und zwar
im nördlichen Teil der Gemeinde die Landwirtschaft und im südlichen Teil die
Waldwirtschaft. Das hängt mit der Verschiedenartigkeit der Bodenbeschaffenheit
und den klimatischen Verhältnissen innerhalb der Gemeindeflur Zachenberg
zusammen. Bei Fratersdorf, Dietzberg, Giggenried, Lämmersdorf, Wandlhof,
Leuthen, Lobetsried, Triefenried ist der Boden tiefgründiger, lehmiger, flacher,
ebener und sonniger als der Boden im südlichen Teil, dessen Boden
seichtgründiger, mehr quarzsandig und bergig und, weil auf der Schattenseite
liegend, auch wenig sonnig ist.Anfänglich wurde ja nur Korn und Haber gebaut.
Den Samen hierzu konnten die Leute vom Kloster Metten bekommen und in
Schreindorf (Schranendorf) holen. Der Anbau von Korn und Haber war nur so viel
als es der Eigenbedarf und die Zehentpflicht erforderten. In frühester Zeit
kümmerten sich die Bauern nicht viel um möglichst hohe Erträge aus ihrem
Grundbesitz. Erst in den letzten Jahrzehnten gingen vor allem die Jungbauern an
eine gründliche Bodenbearbeitung und damit Bodenverbesserung, sodass heute in
verhältnismäßig kurzer Zeitspanne weit höhere Erträge erzielt werden als früher.

\section{Flachsanbau und Weberei in der Gemeinde Zachenberg}

Neben dem Anbau von Korn und Haber gab es anfänglich in der Gemeindeflur
Zachenberg den Flachsbau. Dieser bildete neben Viehzucht die Haupteinnahmequelle
für den Bauern. Die Gemeinde Zachenberg war bekannt durch ihren starken
Flachsanbau. 1830 lieferte sie den meisten Flachs ab von allen Gemeinden des
Bezirkes Viechtach, nämlich 434 Zentner. Am nächsten stand die Gemeinde
Geiersthal mit 301 Zentner. Infolge des großen Flachsanbaues war auch die Zahl
der Brechhäuser sehr groß. Ebenso gab es in der Gemeinde Zachenberg sehr viele
Weber und zwar solche alt eingesessene Webergeschlechter, die wegen ihrer
Tüchtigkeit weit und breit bekannt waren. 1809/10 lebten im Gerichte Viechtach
326 Weber. In der Nr. 188 des Deggendorfer Donauboten vom Freitag, den 4.
Dezember 53 wird von einem 83-jährigen Weber erzählt, der aus der Viechtacher
Gegend stammt und dass sich seine Eltern 1885 in Ensbach Gemeindexlviii
Schaufling angesiedelt haben und dass er und seine Eltern und Großeltern das
Webern ausgeübt haben. Das Webern war ein hartes Brotverdienen. An einem Tag
stellte ein Weber etwa ein Ellen Leinwand her, das sind 4 Meter. Für eine Elle
erhielt der Weber 10 Pfennig. Die Bauern brachten das selbstgesponnene
Flachsgarn in Strängen zum Weber. Die Leinwand wurde gewöhnlich in Stücken von
30 Ellen gewebt. Dabei wurden auch noch schöne Muster hineingewebt. Auch das
Waschen des fertigen Stückes Leinen in Aschenlauge und das anschließende
Bleichen auf den Wiesen besorgte der Weber. (Soweit der Donaubote).

\section{Viehzucht in der Gemeinde Zachenberg}

Zur Bewirtschaftung von Feld, Wiese, Weide und Wald kam bei den Zachenberger
Bauern auch die Viehzucht. Auf allen Höfen in der Gemeindeflur Zachenberg
standen in den Ställen Mastochsen, Menochsen, Kühe, Jungrinder und bei den
größeren Bauern auch Pferde. Das Halten von Mastochsen wird uns schon durch die
großen Viehmärkte bestätigt, die in Viechtach, Regen und in Deggendorf in
früheren Jahrzehnten und Jahrhunderten abgehalten wurden. Durch die Zucht der
Mastochsen wanderte viel Geld in die Geldgürtel der Zachenberger Bauern. Bis von
Straubing, Regensburg, Neuburg a. D., ja sogar von Oberfranken kamen die Käufer
um Vieh aufzukaufen. So kaufte im Jahre 1583 am Donnerstag vor Pauli Bekehrung
(25.1.) ein Nürnberger Metzger nicht weniger als 139 Ochsen auf einem
Viechtacher Viehmarkt und im gleichen Jahre ein Augsburger Metzger am selben Ort
470 Ochsen und am 16.1.1620 erwarb auf einem Viehmarkt in Viechtach der Metzger
Lutz aus Straubing 15 Ochsen und am selben Tag ein Metzger aus Regensburg 80
Ochsen. Wenn auch der Getreidebau in der Gemeinde Zachenberg früher unbedeutend
war, so hat man doch die Viehzucht als eine gute Einnahmequelle für den Bauern
gehalten und sich darnach eingestellt.

\section{Holzwirtschaft in der Gemeinde Zachenberg}

Weniger ertragreich war in früheren Jahrhunderten der Wald, weil das Holz nur
einen geringen Preis hatte. Das Holz wurde verwendet als Bau- und Brennholz oder
es wanderte zur Säge oder in die Stöß. In dieser wurden die Zündholzstäbchen,
die Rouleaustäbchen usw. angefertigt, die zum Versand kamen. Solche
Zündholzstöße waren in Furth, in Muschenried. In den Wintermonaten beschäftigten
sich die Zachenberger mit der Herstellung von Rechen, Schaufeln, Sensenstiele,
Schwingen usw. Diese Ware brachten sie dann im Frühjahr zum Verkauf von Hof zu
Hof oder auf die Märkte oder auf den Gäuboden zu den Bauern.

\subsection{Granitwirtschaft in der Gemeinde Zachenberg}

Eine vielseitige Beschäftigung in der Gemeindeflur Zachenberg gab die
Bearbeitung des reichlich vorhandenen Granitgesteins. Zunächst fertigte man aus
den Findlingssteinen, Viehbarren, Wasserbehälter, Türstöcke, Fensterstöcke für
die Viehställe, Stufen, Platten usw. Aus der schlechten Gesteinsqualität wurden
Grenzsteine und Gartenpfosten usw. gemacht. Bestand Aussicht, dass der Stein
auch in der Tiefe des Bodens schön weiß und blau bleibt, so wurde der oben
aufliegende Abraum entfernt und ein Steinbruch von einem Unternehmer eröffnet.
Solche Steinbrüche gab es in der Gemeindeflur Zachenberg viele, von denen die
einen noch im Betriebe sind, die anderen aber zur Zeit still liegen und außer
Betrieb sind, z. B. Gaisberg (Quader Tunnelbau)‚ Kotmühle (Straßenschotter)‚
Auerbach (Vogl-Bruch), Auerbach (Kopp-Bruch), Bumsenberg (Dr. Ziegler),
Haberleuthen (Hartl-Bruch), Göttleinsberg (Egglbauern-Bruch), Bartenstein,
Wolfsberg (Straßenschotter), Im Betrieb sind: Zachenberg (Dobusch), Muschenried
(Wagner), Muschenried (Kramhöller), Muschenried (Brickelmann), Gottlesried
(Gruber), Gaisruck.

\section{Religiosität in der Gemeinde Zachenberg}

Die ersten Leute in der Gemeindeflur Zachenberg waren ihrer Religion nach
Heiden. Die Bekehrung zum Christentum dieser Leute erfolgte von Geiersthal aus.
Die geschlossene Siedlung „Giristal“ dürfte schon um das Jahr 800 herum
entstanden sein und kann deshalb auf eine 1 100-jährige Geschichte
zurückblicken. 1209 wurde der Aldersbacher Mönch Sieghart erster Pfarrer in
Geiersthal, also zu einer Zeit als die östlichen Teile der Gemeindeflur
Zachenberg (Triefenrieder Gegend) erst gerodet und besiedelt waren. Die Leute
dieser Gegend (Triefenried, March) wurden von Geiersthal aus christianisiert und
religiös und kirchlich betreut. In March stand ein kleines Rokokokirchlein.

\section{Der östliche Teil der Gemeindeflur Zachenberg bei der Kirche in March}

1496 bekam March einen eigenen Expositus. 1753 wurde der Seelsorgsprengel March
durch die Vermittlung des Abtes von Aldersbach eine eigene Pfarrei, von der
Mutterpfarrei Geiersthal abgetrennt und bis zur Säkularisation 1803 von den
Aldersbacher Zisterziensermönchen betreut. Während das Unterdorf von March zur
Pfarrei March gehörte, war das Oberdorf von March der Pfarrei Regen einverleibt,
weil dieses Oberdorf zur Schlossherrschaft Auf gehörte. 1805 kam auch das
Oberdorf von March zur Pfarrei March. Das uralte Rokokokirchlein von March war
von jeher schon zu klein und entsprach nunmehr gar nicht mehr seinen
Anforderungen. Die Erbauung einer neuen großen Pfarrkirche in March wurde zur
Notwendigkeit. 1906 wurde mit dem Neubau derselben unter Leitung des Architekten
Schott in München begonnen und 1908 wurde der Bau dieser modernen Barockkirche
vollendet. Sie wurde im selben Jahre noch von Hr. Dekan Kroiß von Arnbruck
eingeweiht und seiner Bestimmung übergeben. Leider haben Kirchturm und Dach
dieser Pfarrkirche am 25. April 1945 unter dem Beschuss der Amerikaner schweren
Schaden erlitten. Am Feste der Patrona Bavariae 1949 wurde das neue
Glockengeläute geweiht von H. Hr. Abt Korbinian Hofmeister von Metten. Die
Pfarrkirche von March bekam auch neue Kirchenfenster, neue Beichtstühle, eine
Heldengedenktafel für die 76 Kriegsopfer der Pfarrei March und einen neuen
Kreuzweg. Ein feines Andenken an das ehemalige Gotteshaus konnte gerettet
werden, nämlich das einzigartige Altarbild der Mutter Anna mit ihrer Tochter
Maria und den drei heiligen Maderln, heute ein feiner Schmuck des Presbyteriums.
Der jetzige Pfarrer der Pfarrei March, H. Hr. Pfarrer Schefbeck, hat einen
großen Verdienst um die Restaurierung der Marcher Pfarrkirche. Zuvor waren die
H. Hr. Joseph Dietl, Pfarrer Rankl, Pfarrer Poiger usw. Pfarrer in March.

\section{Der westliche Teil der Gemeindeflur Zachenberg bei der Kirche in
Ruhmannsfelden}

Die Leute der westlichen Gegend der Gemeindeflur Zachenberg (Ruhmannsfelden)
wurden auch von Geiersthal aus christianisiert und kirchlich betreut. Oberhalb
der Siedlung Ruhmannsfelden, am Bühl, stand eine kleine, hölzerne Kapelle zu
Ehren des hl. Laurentius. Die Sebastiani- und Laurentius-Kapellen standen immer
und überall außerhalb der eigentlichen Siedlung. So war es auch hier in
Ruhmannsfelden. Im und bei dieser Laurentius-Kapelle am Bühl fanden die
Andachten und Gottesdienste statt, zu denen der Aldersbacher-Mönch von
Geiersthal kommen musste. Im Jahre 1295 gelangte die Siedlung Ruhmannsfelden
durch Kauf in den Besitz des Zisterzienserklosters Aldersbach. Gleichzeitig hat
Bischof Heinrich II. von Regensburg die neue Klosterniederlassung in Gotteszell
genehmigt, hat sie selbständig gemacht und vom Pfarrverband Geiersthal gelöst.
Damit stand nun dieser westliche Teil der Gemeindeflur Zachenberg unter der
kirchlichen und religiösen Betreuung, der Aldersbacher Mönche, die in Gotteszell
ihr Kloster und in Ruhmannsfelden ihre Prälatur hatten. Von jetzt ab waren auch
die Gottesdienste in Ruhmannsfelden häufiger und regelmäßiger. Da bei der
Laurentius-Kapelle am Bühl auch eine Begräbnisstätte war, fanden da auch die
Beerdigungen statt. Es erwies sich als notwendig, eine neue, aber größere Kirche
zu bauen. Im Laufe der Zeit ist aber die Kirche in Ruhmannsfelden aus den
verschiedensten Anlässen heraus immer wieder ein Raub der Flammen geworden. Sie
wurde stets wieder aufgebaut den erforderlichen Raumverhältnissen und den
maßgebenden Baustilen entsprechend. So ist die damals gotische Pfarrkirche in
Ruhmannsfelden zuletzt 1820 abgebrannt. Bei dem Neuaufbau der Kirche wurde auf
die Vergrößerung derselben zuerst gesehen. Da aber die Kirche in ihrem Innern
auf Geheiß König Ludwig I., dessen finanzielle Unterstützung zum Wiederaufbau
benötigt wurde, nach den Plänen seines königlichen Hofbaumeisters ausgeführt
werden musste, so finden wir jetzt in Ruhmannsfelden ein Schmuckkästchen von
einer Kirche in klassizistischem Baustile. Bei Aufhebung des Klosters Gotteszell
löste sich auch der Klosterkonvent allmählich auf. So lesen wir, dass 1803 ein
P. Bernhard Kammerer gebeten hatte, nach Ruhmannsfelden verziehen zu dürfen, um
dort Beihilfe in der Seelsorge leisten zu können. Der Exprior Pater Xaver Sämer
übersiedelt 1804 nach Ruhmannsfelden, wo er bald darauf starb. 1824 starb der
von Viechtach nach Ruhmannsfelden übergesiedelte Pater Marian Triendorfer. 1805
wirkte als erster definitiver Pfarrer in Ruhmannsfelden H. Hr. Joseph
Castenauer. Ihm folgte als 2. Pfarrer H. Hr. Peter Blaim. Nach dessen Tod versah
das Provisorat Hr. Kooperator Wagner. Von 1821 folgten als Pfarrer in
Ruhmannsfelden: Dieß, Linhard, Wagner, Wandner, Hösl, Uschalt, Rötzer, Englhirt,
Neppl, Mühlbauer, Fahrmeier, Bauer und jetzt Reicheneder.

\section{Der südliche Teil der Gemeindeflur Zachenberg bei der Kirche in Gotteszell}

Die Ortschaften der Gemeinde Zachenberg südlich von Ruhmannsfelden, sowie ein
Teil der heutigen Gemeinde Gotteszell, gehörten früher auch zur Pfarrei
Geiersthal, wie March und Ruhmannsfelden. Das war aber für die religiösen
Bedürfnisse dieses Gebietes wegen der großen Entfernung von Geiersthal nicht
gut. Darum entschloss sich auch Graf Heinrich von Pfelling seinen eigenenxlix
Gutshof „Droßlach“ (heutiges Gotteszell) dem Zisterzienserkloster Aldersbach
Schenkungsweise zu übertragen. Der damalige Bischof von Regensburg, Heinrich II.
genehmigte 1286 die neue Niederlassung, löste sie vom Pfarrverbande Geiersthal
und gab ihr den Namen Cella Dei, d. h. Gotteszell. 1297 war die Zahl der
Klosterbrüder Gotteszell schon auf 13 angewachsen. 1320 wurde die Klosterfiliale
Gotteszell in eine Abtei umgewandelt. Bruder Bertholdus wurde zum 1. Abt
gewählt. Dieser Abt baute eine neue Kirche in dem einfachen frühgotischen
Zisterzienserstil, dreischiffig mit 2 Kapellen beiderseits des Chores. Nach den
Regeln des Zisterzienserordens hatte die Kirche keinen größeren Glockenturm,
sondern die Kirche war nur mit einem kleinen Turmaufbau über dem Chore
(Dachreiter) versehen. Eine schwere Heimsuchung brachte dem Kloster Gotteszell
das Jahr 1629. Am 24. März dieses Jahres wurde durch einen Föhnsturm das Fenster
der Klosterküche eingedrückt und das Herdfeuer durch den Kamin auf die
ausgetrockneten Strohdächer getrieben. Das Feuer verbreitete sich mit rasender
Schnelligkeit auf die Kirche und die angrenzenden Gebäude und verursachte einen
großen Schaden. Die alte, nicht gewölbte St. Anna-Kirche stürzte in sich
zusammen und begrub unter ihren Trümmern das Holzbild der Mutter Anna, das nach
Löschung des Brandes wunderbarerweise ganz unversehrt blieb unter dem Schutte
des Brandherdes aufgefunden wurde. Wenn auch sofort nach dem Brande ein Teil
wieder neu aufgebaut, der andere Teil ausgebessert, der übrige Teil aber
notdürftig repariert wurde, so fielen zu allem Unglücke mitten in diese
Klosterwiederaufbauarbeiten die schlimmen Ereignisse des 30-jährigen Krieges. Am
Lichtmesstag 1641 brach im Kloster Gotteszell schon wieder ein Brand aus, der
den neu erbauten Konventbau bis auf die Außenwände einäscherte. 1651 hatte Abt
Gerhard mit dem Wiederaufbau des Klosters beginnen lassen und schon 1654 konnte
nach Aldersbach berichtet werden, dass der Klosterbau nun fertig gestellt sei.
1729 war die Jahrhundertjubiläumsfeier zur Erinnerung an die wunderbare
Erhaltung des hl. Mutter-Anna-Bildes beim großen Brande 1629. Schwere Zeiten
brachten dem Kloster Gotteszell die Jahre 1742 bis 1745 mit den Bedrängnissen,
Kriegssteuern, Quartierlasten, Kontributionen usw. im österreichischen
Erbfolgekriege. Das Jahr 1803 brachte dann die Aufhebung des Klosters. Die
Durchführung der Verweltlichung des Klosters dauerte über 1 1/2 Jahre. Die frühere
freundliche Anlage der Klostergebäulichkeiten wurde im Laufe der Zeit immer mehr
verunstaltet, durch den Abbruch, durch Neubau von kleinen, ärmlichen Wohn- und
Ökonomiegebäuden und Anlage von Düngerstätten im ehemaligen Klosterhofe.

Am 22. Juli 1830 schlug der Blitz in die Kirche ein. Kirchendachstuhl und
Glockentürmchen wurden vernichtet, das Innere der Kirche, sowie Chor mit Orgel,
beschädigt. Die ganze Kirche erhielt dann die jetzige unstilgemäße äußere
Gestaltung. Im Kircheninnern hat das 19. Jahrhundert mancherlei Umänderungen
vorgenommen, die dem Innern der Kirche nicht zustatten kamen, sodass die Kirche
bis in unsere Zeit herein keinen erfreulichen Anblick bot. Erst H. Hr.
Geistlicher Rat Thurmayr, selbst ein vorzüglicher Kunstkenner und eifriger
Förderer wahrer kirchlicher Kunst, begann 1938 das Innere der Kirche, gründlich
zu restaurieren. In 14-jähriger Tätigkeit, unterstützt vom Landesamt für
Denkmalpflege, vom Staate und vom bischöflichen Ordinariat Regensburg und vom
Opfersinn der Pfarrangehörigen ist es ihm gelungen, ein Werk zu schaffen, das
von den Kunstfreunden bewundert wird und die sich immer wieder erfreuen über
diesen edlen und hellen mittelalterlichen Kirchenraum mit seinen reichen
Kunstschätzen aus früheren Jahrhunderten. Damit hat H. Hr. Geistlicher Rat
Thurmayr sich ein ehrenvolles Denkmal gesetzt. Zu den früheren Pfarrherren der
Pfarrei Gotteszell gehören: H. Hr. Pfarrer Ignaz Stauber, der zuletzt Pfarrer in
Aiterhofen war und dort gestorben ist. H. Hr. Pfarrer Mathias Hirsch, H. Hr.
Pfarrer Moritz Stern, H. Hr. Pfarrer P. Gerard Haindl mit seinem Hilfspriester
P. Nivard Sarte. 1922 wurde der neue Pfarrhof gebaut. Der ehemalige
Prälatenstock ging in Privatbesitz von Hr. Dostler über.

\section{Die schulischen Verhältnisse in der Gemeinde Zachenberg}

Wie die kirchlichen Verhältnisse in der Gemeinde Zachenberg früher waren und
heute noch sind, so ist es auch mit der Schule dort. Die Gemeinde Zachenberg
gehört zu den 3 Pfarrbezirken March, Ruhmannsfelden und Gotteszell. Ebenso
gehören auch die Schulkinder der Gemeinde Zachenberg diesen 3 Schulen an und
zwar die Schulkinder von 11 Ortschaften der Gemeinde Zachenberg besuchen die
Schule March, von 23 Ortschaften derselben Gemeinde die Schule Ruhmannsfelden
und von 4 Ortschaften derselben die Schule Gotteszell. Es kann auch vorkommen,
dass die Fratersdorfer in gewissen Zeiten Kirche und Schule in Kaickenried
besuchen, da den Fratersdorfern das nahe gelegene Kaickenried leichter zu
erreichen ist, als Ruhmannsfelden oder March. Dass in der Gemeinde Zachenberg
keine Kirche ist, liegt an der schmalen, lang gestreckten Ausdehnung der ganzen
Gemeindeflur Zachenberg und der dauernden Abhängigkeit dieser Gemeinde von der
damaligen dauernden klösterlichen Vorherrschaft von Gotteszell und Aldersbach.
Und wenn im Gemeindebezirk Zachenberg, der drittgrößten Gemeinde des Landkreises
Viechtach, heute noch kein Schulhaus ist, so sind die Zachenberger am wenigsten
schuld daran. Diese wollten ja schon öfters ein Schulhaus bauen. Aber die
verschiedensten Interessen von allen möglichen Seiten her wussten die Errichtung
eines Schulhauses in der Gemeinde Zachenberg immer wieder zu verhindern. Die
Gemeinde Zachenberg hat auch kein Gemeindehaus und keine Gemeindekanzlei
innerhalb des Gemeindebezirkes.

\section{Verkehrswege und Infrastruktur in der Gemeinde Zachenberg}

Die Bundesbahnlinie Bayerisch Eisenstein/Landshut durchzieht die Gemeinde r
Zachenberg ihrer ganzen Länge nach. Im Osten ist der Bahnhof Triefenried. Dort
wird hauptsächlich Quarzkies vom nahen Pfahl verladen. Im Süden ist der Bahnhof
Gotteszell. Hier wird hauptsächlich Langholz, Schleifholz und Brennholz
verfrachtet. Mit der Motorisierung und Elektrifizierung der Landwirtschaft auch
im Gemeindebezirk Zachenberg geht die Verwendung von Motorrädern, Traktoren und
Automobilen im öffentlichen Verkehrs- und Handelsleben Hand in Hand. Überall ist
der moderne Fortschritt wahrzunehmen, und was die früheren Geschlechter in
Zachenberg reichlich versäumt haben, das wird durch die heutige Generation
gründlich nachgeholt, insbesonders auf dem Gebiete der Wegeverbesserung und des
Straßenbaues in der ganzen Gemeindeflur Zachenberg. Da sind die
Wegeverbesserungen von Gotteszell/Bahnhof nach Köckersried, von Auerbacherstraße
über Kirchweg nach Zachenberg, von Bruckhof nach Wandlhof und die
Straßen-Neubauten von Auerbach nach Muschenried, von Bahnhof Triefenried über
Zierbach nach Göttleinsberg und vom Wandlhof über Klessing nach Lobetsried, von
Kirchweg nach Zachenberg und Kleinried. Es herrschte eine große Bautätigkeit. Es
wurden nicht bloß Wohnhäuser in der ganzen Gemeindeflur Zachenberg gebaut und
alte Bauernhäuser modernisiert, sondern es entstanden auch in den
verschiedensten Ortschaften der Gemeinde Zachenberg Krämereien,
Bierverkaufsstellen, Lebensmittelgeschäfte usw. Es herrscht im ganzen
Gemeindebezirk Zachenberg ein intensiver Verkehr, Handel und Wandel auf allen
Gebieten des öffentlichen Lebens.l

\part{Anhang}

\chapter{Quellenangaben}

H. Hr.li P. Wilhelm Fink, Kloster Metten: „Der älteste Besitz der Abtei Metten“
(Monatsschrift für Ostbayerischelii Grenzmarken)

H. Hr. Pfr. Oswald, Rinchnachmündt: „Geschichte der Stadt Regen“

H. Hr. Pfr. Schefbeck, March: „200 Jahre Pfarrei March“ (Gäu und Wald)

Hr. Anton Eberl: „Geschichte des ehemaligen Zisterzienserklosters Gotteszell.“
Jahresbericht des hist. Vereins für Straubing und Umgebung, 27. und 28. Jahrgang

Schmidt: „Die Ortsnamen des Bezirksamtes Viechtach“

Schmidt: „Besiedlungsgeschichte des oberen bayerischen Waldes mit besonderer
Berücksichtigung des Viechtacher Gebietes.“

M. Waltinger: „Niederbayerische Sagen.“

„Durch Gäu und Wald“ (Beilage zum Deggendorfer Donaubote) 1952 Nr. 8 -1953 Nr.
188

.2 Anmerkungen

\end{document}