\section{Quellenverzeichnis}

\subparagraph{Primärliteratur}

\textbf{Ackermann, Peter,} \textit{Messe,} in: Die Musik in Geschichte
und Gegenwart, Hrg.: Finscher, Ludwig, Bärenreiter, Metzler, Kassel,
Stuttgart, 1997, Band 6, S. 209 – 228

\textbf{Chrobak, Werner,}~\textit{Peter Griesbacher zum 50. Todestag,
28. Jan. – 18. Febr. 1983,} Bischöfl. Zentralbibliothek, Regensburg,
1983

\textbf{Dantl, Georg,} \textit{Vom Schullehrling zum Schulmeister –
Geschichte der Lehrerbildung im 19. Jahrhundert,} in: Oberpfälzer
Raritäten, Band 5, Verlag der Buchhandlung Taubald, Weiden, 1989

\textbf{Gärtner, Helmut,} \textit{Deggendorfer Originale – Originelles
Deggendorf,} Morsak Verlag, Grafenau, 2. Auflage, 1995

\textbf{Geyer, Otto,} \textit{Schule und Lehrer in Niederbayern,} Hrsg.
Otto Glaser, Niederbayerischer Bezirkslehrerverein im BLLV,
Neue-Presse-Verlag-GmbH, Passau, 1964, 235 Seiten

\textbf{Goller, Martina,} \textit{Die Musik in der Lehrerbildung
Niederbayerns und ihre Ausstrahlung am Beispiel niederbayerischer
Lehrerkomponisten dargestellt an der Präparandenschule Deggendorf und
am Lehrerbildungsseminar Straubing,} Zulassungsarbeit zur ersten
Staatsprüfung für das Lehramt an Hauptschulen in Bayern eingereicht bei
Prof. Dr. Eckard Nolte im Fach Musikpädagogik, Deggendorf, April 1988

\textbf{Högn, August,} \textit{Geschichte und Chronik der freiwilligen
Feuerwehr Ruhmannsfelden,} Manuskript, 1951 (Feuerwehr)

\textbf{Högn, August,} \textit{Geschichte von Ruhmannsfelden,} Michael
Laßleben, Kallmünz, 1949 (Ruhmannsfelden)

\textbf{Högn, August,} \textit{Heimat-Geschichte der Gemeinde
Zachenberg,} Manuskript, 1954 (Zachenberg)

\textbf{Kirsch, Winfried,} \textit{Caecilianismus,} in: Die Musik in
Geschichte und Gegenwart, Hrg.: Finscher, Ludwig, Bärenreiter, Metzler,
Kassel, Stuttgart, 1995, Band 2, S. 317 – 326

\textbf{Kronsteiner, Hermann,} \textit{Vinzenz Goller – Leben und Werk,}
Linz, Passau, Veritas, 1976

\textbf{Lippert, Heinrich,} \textit{Die Präparandenschule Deggendorf
(1866 – 1924) – Zur Geschichte einer niederbayerischen
Lehrerbildungsanstalt,} in: Deggendorfer Geschichtsblätter 17,
Deggendorf, 1996, S. 153 – 192

\textbf{Proft, Hans}\textbf{\textit{,}}\textit{ „Immer froh und heiter
bleibt der Kutschenreuter“ – Leben und Werk des niederbayerischen
Komponisten Erhard Kutschenreuter,} Verlag Karl Stutz, Passau, 2004

\textbf{Quoika, Rudolf,} \textit{Gruber, Josef,} in: Die Musik in
Geschichte und Gegenwart, Hrg.: Blume, Friedrich, Bärenreiter, Kassel,
1956, Band 5, S. 978 – 979

\textbf{Scharnagl, August,} \textit{Griesbacher, Peter,} in: Die Musik
in Geschichte und Gegenwart, Hrg.: Blume, Friedrich, Bärenreiter,
Kassel, 1956, Band 5, S. 910 – 911 (Griesbacher)

\textbf{Scharnagl, August,} \textit{Regensburg als zentrale Pflegestätte
des Caecilianismus,} in: Der Caecilianismus, Hrg.: Unverricht Hubert,
Verlag Hans Schneider, Tutzing, 1988, S. 279 – 292 (Pflegestätte)

\textbf{Schott, Anselm,} \textit{Das vollständige Römische Messbuch,
lateinisch und deutsch,} Verlag Herder Freiburg, Basel, Wien, 1963

\textbf{Schul-Anzeiger für Niederbayern,} 54. Jahrgang, Cl.
Attenkofersche Buch- und Kunstdruckerei, Straubing, 1938

\textbf{Schwermer, Johannes,} \textit{Der Cäcilianismus,} in: Geschichte
der katholischen Kirchenmusik, Band II, Hrsg.: Fellerer, Karl Gustav,
Bärenreiter, Kassel, 1976

\textbf{Seidel, Elmar, }\textit{Die mehrstimmige Kirchenmusik,} in:
Geschichte der katholischen Kirchenmusik, Band II, Hrg.: Fellerer, Karl
Gustav, Bärenreiter, Kassel, 1976, S. 308 – 325

\textbf{Stengel, Georg Josef,} \textit{Geschichte der
Lehrerbildungsanstalt Straubing von 1824 – 1924, }Manz, Straubing, 1925

\subparagraph{Sekundärliteratur}

\textbf{Edition MGG,} \textit{Musikalische Gattungen in
Einzeldarstellungen,} Band 2: Die Messe, Bärenreiter-Verlag, 1985

\textbf{Eisenhofer, Ludwig,} \textit{Grundriss der Katholischen
Liturgik,} Herder \& CO. GmbH Verlagsbuchhandlung, Freiburg, 1926, 2.
und 3. verbesserte Auflage

\textbf{Gurlitt, Wilibald (Hrg.), }\textit{Gruber, Josef,} Rieman Musik
Lexikon, Personenteil A – K, B. Schott´s Söhne, Mainz, 12. Auflage,
1959, S. 688

\textbf{Hofer, Achim, }\textit{Marsch,} in: Die Musik in Geschichte und
Gegenwart, Hrg.: Finscher, Ludwig, Bärenreiter, Metzler, Kassel,
Stuttgart, 1996, Band 5, S. 1675 – 1680

\textbf{Leuchtmann, Horst,} \textit{Messe und Motette,} in: Handbuch der
musikalischen Gattungen, Hrg.: Mauser, Siegfried, Laaber-Verlag,
Laaber, 1998 Band 9, S. 325 – 344

\textbf{Massenkeil, Günther}, \textbf{Messe,} in: Das Große Lexikon der
Musik, Hrg.: Honegger, Marc, Massenkeil Günther, Herder, Freiburg,
1981, Band 5, S. 291 – 297

\subparagraph{Quellen aus Archiven}

\begin{itemize}
\item Reicheneder-Chronik:

\begin{itemize}
\item Die Seelsorger der Pfarrei (Seelsorger)

\item Auszüge a. d. Protokollbüchern: Ehrenbürger von Ruhmannsfelden
(Ehrenbürger)

\item Schul- und Bildungswesen in Ruhmannsfelden (Schulwesen)

\item Der Pfarrmesner von Ruhmannsfelden (Pfarrmesner)

\item Religiöse Feiern in der Pfarrei (Feiern)

\item Die Pfarrkirche, C. Nach 1820, II. Einrichtung – Die Orgel (Orgel)
\end{itemize}

\item Chronik der Volksschule Ruhmannsfelden

\item Protokollbuch der Feuerwehr Ruhmannsfelden
\end{itemize}