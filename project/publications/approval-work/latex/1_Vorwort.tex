\section{Vorwort}

Als „Mozart von Ruhmannsfelden“
wurde August Högn in einer Ansprache an seinem 80. Geburtstag
bezeichnet. \footnote{Interview Nr. 24, Johann Glasschröder,
28.12.2004, Absatz 10} Der schmeichelnde Vergleich mit dem Salzburger
Komponisten war eine Verbeugung vor seinem jahrzehntelangen engagierten
Mitwirken am musikalischen Leben im kleinen Ort des Bayerischen Waldes
und besonders vor dem umfangreichen und in Ruhmannsfelden noch nie da
gewesenen kompositorischen Schaffen, auf das er zurückblicken konnte.
Auch ein beachtliches heimatkundliches Werk hatte er bis zu diesem
Zeitpunkt zusammengetragen und rechtfertige mehr als genug die weiteren
am letzten runden Geburtstag erteilten Ehren. Kaum größer könnte der
Unterschied zu Lebzeiten und heute sein, bezüglich der Würdigung seines
Schaffens und die Anerkennung seiner Dienste für die Allgemeinheit.
Fast 50 Jahre nach seinem Tod kennt kaum jemand der Ruhmannsfeldener
noch den Namen „August Högn“, geschweige denn sein musikalisches und
heimatkundliches Werk. Durch Zufall habe ich im Notenschrank der
Ruhmannsfeldener Pfarrkirche einige seiner Handschriften entdeckt, die
mein Interesse seinem Leben und Wirken erweckten und mich somit zu
Verfassen dieser Arbeit führten.

Ich möchte mit dieser Arbeit an das Leben und Werk des Rektors,
Heimatforschers und Komponisten August Högns erinnern, um somit seine
Person sowie Werk vor dem Vergessen zu bewahren. Da Högn sehr eng mit
meinem Heimatort Ruhmannsfelden verbunden war, wurde diese Arbeit fast
zwangsläufig zu einer „musikalischen Heimatkunde.“ Insofern kann man
die Arbeit auch als „Hommage an meine Heimat“ ansehen.

Bei der Erforschung von Högns Leben haben sich Berichte von Augenzeugen
als unverzichtbare Informationsquelle herausgestellt. Die Interviews
wurden mitgeschnitten (Daten-CD I) und transkribiert (Band II, Seite 7
- 56). Es war der letztmögliche Zeitpunkt durch Interviews mit
Zeitzeugen den hinter dem Werk stehenden Menschen aufzuzeigen. Der Tod
von zwei meiner meist hoch betagten Interviewpartner noch vor
Fertigstellung dieser Arbeit beweist eindringlich, wie schnell sich im
Laufe der Jahre die Spuren eines Menschen verwischen können, wenn sie
nicht rechtzeitig festgehalten werden.

Mehr als nur eine ergänzende Funktion zu den Interviews haben die in
eine Sammlung aufgenommen Dokumente. Ihnen ist es zu verdanken, dass
einige von Högns Lebensstationen, wie beispielsweise seine drei Phasen,
in denen er Chorregent war, zeitlich sehr genau bestimmt werden
konnten. Vor allem zum jungen Leben von Högn bildeten die Dokumente die
einzige Informationsquelle, da hier Augenzeugenberichte nicht mehr
möglich waren. Neben Dokumenten aus dem Gemeindearchiv waren besonders
die Archivalien, die das Pfarramt bereitstellen konnten, von
außerordentlichem Aufschlussreichtum in Bezug auf das Leben von August
Högn. Hier gab die „Chronik Ruhmannsfelden“ vom ehemaligen
Ruhmannsfeldener Pfarrer Franz Seraph Reicheneder nicht nur über Högns
Leben Auskunft, sondern auch über die Geschichte von Ruhmannsfelden.
Mancher Glücksfund erweitert zusätzlich das Spektrum an Schriftstücken,
zum Beispiel vier Briefe von Högn, die mir aus Australien übersandt
wurden. Sehr mühsam war die Entzifferung der in alter deutscher
Schreibschrift verfassten Dokumente. Einzige Möglichkeit, ihre Aussage
dauerhaft nachvollziehen zu können, blieb das Abschreiben. Da
handschriftliche Dokumente in der Sammlung überwiegen, habe ich mich
dazu entschlossen, alle Dokumente, auch die gut lesbaren, zu editieren.
So können alle Schriftstücke im Dokumentationsteil (Band II, Seite 57 -
90) wortwörtlich nachgelesen werden.

Der erste Teil dieser Arbeit handelt von Högns Leben. Eine Einbindung
und Bezugnahme von Högns musikalischem Werk in die Biographie war nicht
möglich, da man den Entstehungszeitpunkt der Kompositionen nur sehr
grob schätzen kann. Da sich sein Geschichtswerk hingegen sehr genau
zeitlich einordnen lässt und noch dazu viele Hintergrundinformationen
zur Entstehung bekannt sind, widmet sich ein Kapitel des ersten Teil
Högns heimatkundlichen Abhandlungen und ihrer Entstehung. Großer Wert
wurde auch auf die Abbildung der noch vorhandenen Fotos aus Högns
Privatleben gelegt, die große Aussagekraft leisten und den Text
auflockern.

Zum Erstellen des Werkverzeichnisses waren umfangreiche
Recherchearbeiten nötig. An mehreren unbekannten Plätzen lagerten Högns
Kompositionen, ehe sie ausfindig gemacht werden konnten. Die Anzahl der
Kompositionen steigerte sich im Laufe der Recherchearbeiten von den 12
Kompositionen aus dem Notenschrank in der Pfarrkirche Ruhmannsfelden,
dem ersten Fundort, auf fast 70 Werke. Um die tatsächliche Anzahl der
erhalten Werke festzustellen, musste der Kontakt zu Högns Nachfahren
hergestellt werden. Da alle Verbindungen zwischen der Ruhmannsfeldener
Bevölkerung sowie zwischen der Deggendorfer Verwandtschaft zu Högns
Enkelkindern abgerissen waren, gestaltete sich die Such nach den
Nachfahren entsprechend schwierig. Zwar besaßen die Nachfahren leider
auch keine Kompositionen mehr, doch konnte wenigstens die Suche als
beendet erklärt werden und das Werkverzeichnis bekam einen endgültigen
Charakter. Im Zuge der Durchsuchungsarbeiten nach Kompositionen am
Notenmaterial in der Pfarrkirche habe ich Listen aller Arrangements
(Seite ) und aufgeführter Kompositionen (Band
II, Seite 99 - 101) von Högn angefertigt. Die Listen liefern eine
deutliche Abbildung von Högns musikalischem Umfeld und zeigen die
dadurch mögliche stilistische Beeinflussung auf Högns Werke auf. Einige
seiner Kompositionen wurden ediert (Band III) und lieferten den Anstoß
für etliche Wiederaufführungen (Band II, Seite 102 – 103, Daten-CD I).
Eine streng quellenbezogene Ausgabe der Werke mit kritischem Bericht
macht angesichts von Högns geringer Bedeutung in der Musikgeschichte
wenig Sinn. Stattdessen sollte das entstandene Notenmaterial mehr
aufführungspraktischen Überlegungen Rechnung tragen.

Der zweite Teil der Arbeit handelt vom musikalischen Werk Högns. Der
Beginn des Abschnitts „Werk“ ist vor allem um Rekonstruktion bemüht.
Fragen, die das Werkverzeichnis aufwirft, sollen hier beantwortet
werden. Abschließen wird auf einzelne Werke eingegangen. Da sich die
vorhergehenden Kapitel überwiegend mit der kirchenmusikalischen
Schaffen beschäftigt haben, wird dort auch auf seine weltliche
Kompositionen eingegangen. Neben seiner größten und wohl besten
geistlichen Komposition, der „Josephi“-Messe, erhalten auch die
weltlichen Kompositionen, nämlich der Marsch „In Treue fest!“ und der
Weihegesang Es-Dur ihren Platz. Högns einzig erhalte
nationalsozialistische Komposition soll bewusst nicht verschwiegen
werden. Ein Vergleich mit den geistlichen Grabliedern mag exemplarisch
darstellen, wie sich die damalige Propagandamusik Elemente aus anderen
Genres bediente.

Um Interessierten meine Forschungsergebnisse zugänglich zu mache, habe
ich eine Internetseite (www.august-hoegn.de) eingerichtet, die sich dem
Leben und Werk von August Högn widmet. Weitere noch zu verfolgende
Ziele wären eine Ausdehnung der regionalen Pflege seines musikalischen
Nachlasses von Ruhmannsfelden auf die Pfarreien das Dekanat Viechtach
und die Drucklegung einer heimatkundlichen Schrift über sein Leben und
Werk.

An dieser Stelle möchte ich mich bei den zahlreichen Personen bedanken,
die mir bei der Erforschung des Lebens und Werks von August Högn
behilflich waren. Ohne sie wäre diese Arbeit nicht möglich gewesen.
Eine Liste von über 40 am Projekt beteiligten Personen ist im Anhang
(Band II, Seite 104) zu finden.

München, April 2005 \hfill Josef Friedrich