\subsubsection{Besondere Anlässe}

Für besondere kirchliche Anlässe
passende Gebrauchsmusik zu schreiben, dürfte ebenfalls ein Grund für
Högns kompositorische Tätigkeit gewesen sein. Die „Laurentius“-Messe
C-Dur op. 14 und die für das Laurentius-Patrozinium entstandenen
Offertorien D-Dur op. 26 und G-Dur op. 48 nehmen Bezug auf den Patron
der Ruhmannsfeldener Pfarrkirche. Das Offertorium C-Dur op. 30
komponierte Högn anlässlich des 1925 von Papst Pius XI. eingeführten
Christkönigfests. \footnote{Schott, Seite. 1100 – 1101} In der
Offertoriensammlung von Vinzenz Goller, die in Ruhmannsfelden verwendet
wurde, war dieses Fest noch nicht berücksichtig worden, sodass Högn
wahrscheinlich die Notwendigkeit sah, selber ein Offertorium zu
schreiben. Zum Marianischen Jahr 1954 entstand das Marienlied Nr. 12
F-Dur op. 63. In diesem Jahr beginn man die Hundertjahrfeier der
Proklamation des Dogmas der Unbefleckten Empfängnis Marias.\footnote{
http://www.fides.org/deu/news/2004/0410/12\_2928.html} Zu den alle 3
Jahre \footnote{Reicheneder-Chronik, Feiern, Blatt 1 Vorderseite –
Blatt 2 Vorderseite} stattfindenden Firmungen wurden die Stücke Ecce
sacerdos und Juravit dominus benötigt. Laut Högn gab es nur wenige
solcher Kompositionen zu kaufen. \footnote{Interview Nr. 13, Lorenz
Schlagintweit, 29.11.2003, Absatz 2} Nur jeweils ein Exemplar der
seltenen Stücke war deshalb unter dem Notenbestand in der Pfarrkirche
aufzufinden. Das Ecce sacerdos von Kempf war anscheinend für den
damaligen Chor zu schwierig und das Juravit dominus von Alt vielleicht
zu anspruchslos, sodass sich Högn für die Komposition einer passenderen
Musik entschied.
