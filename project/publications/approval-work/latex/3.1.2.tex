
\subsubsection{Entstehungszeit der Kompositionen}
\label{bkm:Ref98427747}\hypertarget{RefHeadingToc100333742}{}Nur zu ganz
wenigen Kompositionen hat August Högn deren Entstehungszeit notiert. Da
die Opuszahlen die Reihenfolge, wie die einzelnen Kompositionen
nacheinander entstanden sind, zumindest ungefähr wiedergeben (siehe
\ref{bkm:Ref99368924} „Anmerkungen zu den Opuszahlen“, Seite
\pageref{bkm:Ref98509933}), kann man mit den spärlichen Datumsangaben
zumindest zeitliche Bereiche eingrenzen. Informationen aus Högns
Biographie, zeitgeschichtliche Fakten sowie Beobachtungen an den
Handschriften lassen noch weiteren Kompositionen eine nähere Zeitangabe
zuordnen, sodass man die Gruppierung in zeitliche Bereiche
differenzieren kann.

\begin{flushleft}
\tablefirsthead{}
\tablehead{}
\tabletail{}
\tablelasttail{}
\begin{supertabular}{|m{6.242cm}|m{9.621cm}|}
\hline
{\bfseries Werk} &
{\bfseries Entstehungszeit}\\\hline
{\bfseries Veni creator Spiritus B-Dur} &
1897/98\\\hline
{\bfseries Marsch {\textquotedbl}In Treue fest!{\textquotedbl} D-Dur } &
1905\\\hline
{\bfseries Marienlied Nr. 9 G-Dur op. 34 } &
19.6.1928\\\hline
{\bfseries Marienlied Nr. 12 F-Dur op. 63 } &
1. Fastensonntag 1954\\\hline
{\bfseries Marienlied (Nr. 13) C-Dur} &
Laetare 1954 (Text), 300-jähr. Osterbrünnljubiläum 1960 (Musik)\\\hline
\end{supertabular}
\end{flushleft}
Högns op. 11 ist wahrscheinlich zu Beginn seiner ersten Chorregentenzeit
1921 entstanden. Zwischen op. 11 und op. 16 sind alle Kompositionen
vollständig vorhanden, und es handelt sich in diesem Bereich um
ausschließlich geistliche Werke. Die vielen Lücken zwischen op. 1 und
op. 10 – nur zwei Kompositionen aus diesem Bereich waren auffindbar –
lassen sich dadurch erklären, dass Högn in dieser Zeit vor allem
weltliche Musik für die Sänger- und Orchesterriege des Turnvereins
geschrieben hat, die sowie die meisten Stücke seines weltlichen
Schaffens nicht überliefert ist. Die Übernahme des Kirchenchores 1921
machte offenbar eine stark gesteigerte kompositorische Tätigkeit auf
dem Gebiet der Kirchenmusik notwendig. So entstanden in einem seinen
Anfangsjahren als Chorregent ausschließlich kirchenmusikalische Werke.

\begin{flushleft}
\tablefirsthead{}
\tablehead{}
\tabletail{}
\tablelasttail{}
\begin{supertabular}{m{7.9230003cm}m{7.9230003cm}}
\subparagraph{1897 – 1921}
Veni creator Spiritus B-Dur

Marsch {\textquotedbl}In Treue fest!{\textquotedbl} D-Dur

Ave Maria F-Dur op. 4

Adjuva nos Es-Dur op. 8

\subparagraph{1921 – 1928}
Tantum ergo Nr. 1 Es-Dur op. 11

Kommunionlied Es-Dur op. 12

Cäcilienlied E-Dur op. 12 b

Marienlied Nr. 1 F-Dur op. 13 a

{\textquotedbl}Laurentius{\textquotedbl}-Messe C-Dur op. 14

11 Veni creator Spiritus C-Dur op. 15

8 Adjuva nos op. 15

{\textquotedbl}Mater-Dei{\textquotedbl}-Messe F-Dur op. 16

Marienlied Nr. 2 G-Dur op. 19

Grablied Nr. 4 F-Dur op. 20

Kommunionlied G-Dur op. 21 a

Kommunionlied G-Dur op. 21 b

Marienlied Nr. 3 F-Dur op. 22

Marienlied Nr. 4 G-Dur op. 23

Offertorium D-Dur op. 26

Marienlied Nr. 5 F-Dur op. 28

Offertorium C-Dur op. 30

Tantum ergo Nr. 2 F-Dur op. 32

Marienlied Nr. 9 G-Dur op. 34

\subparagraph{1928 – 1939}
{}-{}-{}-{}- Keine Werke -{}-{}-{}-

(Chorregent Albert Schroll) &
\subparagraph{1939 – 1945}
Grablied für gefallene Soldaten Es-Dur op. 35

Grablied Nr. 1 Es-Dur op. 35

Kommunionlied C-Dur op. 37 b

Marienlied Nr. 6 F-Dur op. 41

Lied von Gotteszell G-Dur op. 42

Pange lingua F-Dur op. 43

Grablied Nr. 3 Es-Dur op. 44

Marienlied Nr. 7 G-Dur op. 45

Pange lingua Es-Dur op. 46

Tantum ergo Nr. 4 A-Dur op. 47

Offertorium G-Dur op. 48

Tantum ergo Nr. 3 Es-Dur op. 49

Libera e-moll op. 50

Benedictus G-Dur op. 50

\ \ Grablied Nr. 2 Es-Dur

\ \ Weihegesang Es-Dur

\subparagraph{1945 – 1953}
Pange lingua Es-Dur op. 51

Fronleichnams-Prozessionsgesänge Es-Dur op. 52

Marienlied Nr. 8 G-Dur op. 54

Marienlied Nr. 10 F-Dur op. 56

Ecce sacerdos F-Dur op. 57

Juravit Dominus B-Dur op. 58

Marienlied Nr. 11 F-Dur op. 59

{\textquotedbl}Josephi{\textquotedbl}-Messe F-Dur op. 62

Marienlied Nr. 12 F-Dur op. 63

\ \ Herz-Jesu-Litanei

Pange lingua (deutsch) G-Dur

\ \ {\textquotedbl}Ehre sei Gott{\textquotedbl} C-Dur

\subparagraph{1953 – 1961}
Marienlied (Nr. 13) C-Dur \\
\end{supertabular}
\end{flushleft}
Da der Beginn des 2. Weltkrieges und der Anfang von Högns 3.
Chorregentenzeit zeitlich eng aneinander liegen, dürfte das Grablied
für gefallene Soldaten Es-Dur op. 35 die erste in seiner 3. Phase als
Chorleiter geschriebene Komposition sein. Das würde bedeuten, dass
zwischen dem vorhergehenden Werk mit der Opuszahl 34, das am 19.6.1928
fertig gestellt wurde, also gegen Ende von Högns 2. Chorregentenzeit,
und dem op. 35 ein Zeitraum von über 10 Jahren liegt. Zu dieser Zeit
war Albert Schroll Chorregent.

Werke mit einer größeren Nummer als op. 51 sind vermutlich kurz nach
Ende des 2. Weltkriegs entstanden. Dieser Zeitpunkt liegt einerseits
nahe, wenn man eine regelmäßige Kompositionstätigkeit Högns in seiner
von 1939 bis 1953 andauernden 3. Chorregentenzeit annimmt. Da das op.
51 die erste nicht nachträglich mit einer Opuszahl versehene
Komposition ist, kann man auf der anderen Seite davon ausgehen, dass
das Werk mit der Opuszahl 51 kurz nach Abschluss der nachträglichen
Nummerierungsarbeiten der bisherigen Werke Högns entstanden ist. Es ist
sehr wahrscheinlich, dass diese Nummerierungsarbeiten in der durch die
Entnazifizierung verursachten Zwangspause von 1945 bis 1947
stattfanden, als Högn vom Schuldienst suspendiert wurde und erstmals
Zeit gefunden hatte, sich um die Pflege seiner bisher entstandenen
Kompositionen zu kümmern.









