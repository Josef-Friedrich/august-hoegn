\section{Schlusswort}

Betrachtet man Högns Leben aus der
Distanz von fast 50 Jahren und im Zeitraffer einer wenige Seiten langen
Biographie, so wird vor allem eines deutlich: Högns Lebensweg ist
maßgeblich durch eine feste Integration in die Gesellschaft
gekennzeichnet. Högn passte sich nicht nur vier grundsätzlich
verschiedenen Gesellschaftsordnungen an, sondern war darüber hinaus in
jeder Phase wichtiger Träger von Aufgaben, die die Gesellschaft an ihn
übertrug. Problemlos ließ er sich von der demokratischen
Gesellschaftsordnung der Weimarer Republik in die Gewaltherrschaft der
Nazis mit einbeziehen. Äußerliches Zeichen dieser Anpassung über die
Systeme hinweg, ist auch eine Übereinstimmung mit der Mode der
jeweiligen Zeit, die bei ihm beobachtet werden kann (Schnurrbart zur
Kaiser-Zeit, Schnauzer zur Hitler-Zeit, kein Bart zur Zeit Adenauers).
Aufgrund seiner Anpassungsfähigkeit lassen sich zeitgeschichtliche
Entwicklungen besonders gut an seiner Biographie ablesen.

Seine gesellschaftliche Anpassung spiegelt sich außerdem im engen
Kontakt seiner Kompositionen zum aktuellen Musikstil wieder. In der
Kirchenmusik war zu Högns Lebzeit der Cäcilianismus die dominierende
Stilrichtung. Die stilistische Entwicklung innerhalb seines Werks
verläuft sich synchron zu den Veränderungen innerhalb des
Cäcilianismus. Högns Werk trägt den Stempel echter Gebrauchsmusik. Högn
war zu den Zeiten Chorregent, wo komponieren von Gebrauchsmusik aus
Mangel an verfügbarem Notenmaterial notwendiger als je zuvor war.
Manches seiner musikalischen Werke ist allein schon durch seine
Entstehung ein Zeuge davon, wie sich beide Weltkriege auf alle Bereiche
des Lebens negativ ausgewirkt haben. Die Kompositionen von Högn können
auch als Früchte der damaligen Lehrerausbildung angesehen werden. Das
kompositorische Handwerkszeug, das sich Högn während seiner
Lehrerausbildung aneignete, war auch ohne jegliche Weiterbildung für
die Erstellung eigener Kompositionen ausreichend.

Hätte es nicht gereicht, nur das Werk von August Högn zu betrachten? War
das Erinnern an einen Lebensweg, der bei vielen Zeitgenossen ähnlich
verlaufen ist, angebracht und der Mühe wert? Der Blick auf die Person
August Högns aus zwei verschiedenen Perspektiven, nämlich aus der
seines Werks und aus der seines Lebens, erbringt zusätzlichen Gewinn
für jede einzelne Facette seines Seins. Das Werk wird aufgewertet durch
eine Reihe von Zusatzinformationen, die aus Forschungsarbeiten um sein
Leben stammen, wenn man beispielsweise an die ungefähre zeitliche
Einordnung der Werke, das Hintergrundwissen zu einzelnen Werken und
aufführungspraktische Bedingungen denkt. Andererseits lassen auch
Rückschlüsse von Werk auf sein Leben ziehen. Allein schon die große
Anzahl der Kompositionen und geschichtlichen Abhandlungen machen
deutlich, dass Högn zeitlebens ein Schaffender war.

Was hat die Erforschung des Lebens und Werks von August Högn gebracht?
Mit Högns Werk als ein Mosaikstein der Mosaikgeschichte im Übergang vom
19. zum 20. Jahrhundert lassen sich Rückschlüsse auf die gesamte Epoche
machen. Nicht nur die Werke von großen Komponisten verschaffen uns
einen musikgeschichtlichen Einblick in eine bestimmte Epoche, sondern
auch Werke unbekannter Vertreter. Ist das beachtliche Werk eines
„einfachen“ Lehrers aus einem kleinen Ort des Bayerischen Waldes nicht
ein unumstößliches Zeugnis des kulturellen Reichtums seiner Zeit? Högns
Lebensweise, die am besten durch „im Strom mitschwimmen“ umschrieben
werden kann, macht ihn zu einem idealtypischen Vertreter vieler
Bereiche. Die Abhandlung über August Högn eignet sich auch besonders
gut zur Verwendung für übergeordnete Forschungsarbeiten als
exemplarischer Verweis, zum Beispiel mit der Thematik „Einfluss des
Cäcilianismus“, „Lehrerkomponisten“ oder „musikalische Lehrerausbildung
und ihrer Auswirkung auf das musikalische Leben im ländlichen Raum.“
Mit der Biographie über Högn wird nicht zuletzt ein kleiner Beitrag zur
kulturellen Identität der Region seines Wirkens geleistet. Högns Werke
stellen ein wichtiges Kulturgut im ansonsten eher „kulturarmen“ Raum am
Land dar und können als ein großer Zugewinn an Eigenständigkeit und
Einzigartigkeit auf kulturellem Gebiet angesehen werden.