\subsection{Bekenntnis zur Gebrauchsmusik}

\hypertarget{RefHeadingToc100333745}{}August Högn hat ausschließlich
Gebrauchsmusik komponiert. Keines seiner erhaltenen Stücke wurde ohne
funktionalen Hintergrund geschrieben. Kirchenmusik ist schon vom Wesen
her Gebrauchsmusik. Auch Högns weltliche Kompositionen wurden nicht
ohne konkreten Verwendungszweck komponiert. Der Weihegesang Es-Dur
übernahm zur NS-Zeit bei einer weltlichen Trauerfeier dieselbe Funktion
wie ein Grablied bei der kirchlichen Beerdigung. Der Marsch „In Treue
fest!“ wurde bei vom Turnverein Ruhmannsfelden veranstalteten bunten
Abenden in einer Fassung für Streichorchester aufgeführt und diente
somit der Unterhaltung. Eine Verwendung des Lieds von Gotteszell G-Dur
op. 42 als Unterhaltungsmusik und somit auch als Gebrauchsmusik ist
ebenfalls vorstellbar. Dass unter Högns Kompositionen kein einziges
Werk zu finden ist, das um seiner selbst Willen geschrieben wurde, sich
also keine Kunstmusik unter seinen Werken befindet, liegt vor allem an
Högns Persönlichkeit und seinem Anspruch, den er an seine Kompositionen
stellte. Zeitgeschichtliche Faktoren und der Bedarf von bestimmten
Kompositionen zu besonderen Anlässen, begünstigten die Entstehung von
Gebrauchsmusik oder machten ihre Entstehung sogar unbedingt notwendig.