\section{Leben}

\subsection{Kindheit in
Deggendorf}


August Högn kam am 2. August 1878
in Deggendorf als Sohn von Andreas und Helene Högn, geborene Zöpfl, auf
die Welt. \footnote{Dokument Nr. 48, Zeitungsartikel aus Viechtacher
Bayerwald-Bote, 2.8.1958} August hatte zwei ältere Geschwister, Theres
(geboren 1869) und Ludwig (geboren 1871). Seine jüngeren Brüder, Joseph
und Otto, kamen 1879 und 1883 jeweils entsprechend auf die
Welt. \footnote{Grabinschrift der Familiengräber Andreas Högn und
Ludwig Högn am Deggendorfer Friedhof}

\begin{figure}
%
\begin{subfigure}[b]{0.5\linewidth}
\centering
\img[height=7cm]{Andreas-Hoegn}
\caption{Andreas Högn}
\end{subfigure}
%
\begin{subfigure}[b]{0.5\linewidth}
\centering
\img[height=7cm]{Helene-Hoegn}
\caption{Helene Högn}
\end{subfigure}
%
\caption{Die Eltern von August Högn}
\end{figure}

\begin{figure}
%
\begin{subfigure}[b]{0.5\linewidth}
\centering
\img[height=7cm]{Ludwig-Hoegn}
\caption{Ludwig Högn}
\end{subfigure}
%
\begin{subfigure}[b]{0.5\linewidth}
\centering
\img[height=7cm]{Joseph-Hoegn}
\caption{Joseph Högn}
\end{subfigure}
%
\caption{Zwei Brüder von August Högn}
\end{figure}

Augusts Vater war von Beruf Buchbinder und eröffnete zusammen mit seiner
Ehefrau 1867 eine Buchbinderei und Buchhandlung im so genannten
„Kerndel´schen-Haus“ am Luitpoldplatz in Deggendorf. Bereits nach 6
Jahren, also 1873, zog die Familie Högn in ein eigenes Haus in die
Pfleggasse 1, wo bis zum heutigen Tag die Buchhandlung Högn zu finden
ist (Abb. 5). Die äußerst günstige Lage der Buchhandlung im Zentrum
Deggendorfs war von bedeutendem wirtschaftlichen Vorteil. Besonders an
Markttagen, so zum Beispiel zum „Saumarkt“ in der Pfleggasse, strömten
viele Menschen aus der Umgebung nach Deggendorf, wovon auch die
Buchhandlung Högn profitierte. Das Sortiment wurde im Laufe der Zeit
erweitert und zusätzlich zu Büchern auch Schreib-, Schul-, Spiel- und
Lederwaren angeboten. Ab 1890 komplettierte ein eigener
Postkarten-Verlag das Angebot.\footnote{
http://www.hoegn.de/ueber/ueb\_gesch.php}

In dem Haus des späteren landgräflichen Magistratsrats, Landrats und
Landtagsabgeordneten \footnote{Grabinschrift des Familiengrabs Andreas
Högn am Deggendorfer Friedhof} Andreas Högn gehörte eine gründliche
Ausbildung und somit eine grundlegende musikalische Schulung der Kinder
allein schon zum guten Ton. Es ist daher kaum verwunderlich, dass alle
fünf Kinder das Klavierspielen erlernten, selbst der von Geburt an fast
taube Joseph Högn. \footnote{Interview Nr. 3, Ida Högn, 29.12.2002,
Absatz 40}

August Högn besuchte die Knabenschule in Deggendorf \footnote{Dokument
Nr. 48, Zeitungsartikel aus Viechtacher Bayerwald-Bote, 2.8.1958} und
war von 1889 bis 1891 Schüler der Unterstufe des Klosters Metten,
genannt Lateinschule, wo er auch im „Klosterseminar“, also im der
Schule angeschlossen Internat wohnte. \footnote{Korrespondenz Nr. 88,
Brief von P. Dr. Michael Kaufmann OSB an Josef Friedrich, 7.1.2005} Mit
dem darauffolgenden Übertritt in die Präparandenschule Deggendorf war
schon viel früher als bei anderen Schularten der endgültige Berufsweg
eingeschlagen. Die Übernahme des elterlichen Geschäfts dürfte für den
zweitgeborenen Sohn nie zur Debatte gestanden haben. Dies war
wahrscheinlich einer der Gründe, weshalb sich August frühzeitig für den
Beruf des Lehrers entschieden hatte. Ludwig Högn, der ältere Bruder von
August, erlernte ganz nach alter Tradition das Buchbinderhandwerk, um
einmal die Stellung seines Vaters einnehmen zu können. Er eröffnete
jedoch in Straubing eine Kunst-, Papier- und Galanteriewarenhandlung.
Somit konnte der jüngste Sohn Otto die Buchhandlung Högn
übernehmen. \footnote{Interview Nr. 26, Eva Ertl, 9.2.2005, Absatz 2}

\begin{figure}
\img{Buchhandlung-Hoegn_2004}
\caption{Buchhandlung Högn, 2004}
\end{figure}
