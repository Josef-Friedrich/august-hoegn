\section{Personenverzeichnis}

\hypertarget{RefHeadingToc100333759}{}\textbf{Ascherl, Anna,}
{\textgreater} Violine, Lehrerin (1937 – 1965);

\textbf{Bartascheck, Richard, }Sänger im Ruhmannsfeldener Männerchor,

\textbf{Beischmied, Rosalia,} {\textgreater} Haushälterin, * 18.5.1901
Ruhmannsfelden, † 28.1.1977 Viechtach;

\textbf{Beischmied, Mathilde,} {\textgreater} Tochter der Haushälterin,
* 1.1.1923 Ruhmannsfelden;

\textbf{Brunner, Josef, }{\textgreater} Aushilfsorganist, * 28.6.1924;

\textbf{Danziger, Franz sen.} {\textgreater} Schüler, Chorsänger,
Instrumentalist, Chorregent, * 27.9.1904 Ruhmannsfelden, † 25.6.1988
Ruhmannsfelden;

\textbf{Ederer, Wilhelm,} {\textgreater} Schüler, * 5.1.1931
Ruhmannsfelden,

\textbf{Essigmann, Barbara,} {\textgreater} Schülerin, Sängerin, *
3.12.1921 Ruhmannsfelden;

\textbf{Fink, Wilhelm,} {\textgreater} Historiker, Benediktiner Pater, *
9.5.1889 Rottenburg an der Laaber, † 13.2.1965 Metten;

\textbf{Freisinger, Johann,} {\textgreater} Schüler, Nachfolger als
Gemeindeschreiber der Gemeinde Zachenberg und als Schriftführer der
Feuerwehr Ruhmannsfelden, * 27.3.1927 Ruhmannsfelden;

\textbf{Freisinger, Maria,} {\textgreater} Sängerin, Schüler, *
18.9.1929 Ruhmannsfelden;

\textbf{Goller, Fritz, }{\textgreater} Kirchenmusiker, Komponist, *
20.11.1914 Deggendorf, 22.7.1986, † Deggendorf;

\textbf{Glasschröder Anna,} {\textgreater} Sängerin, * 23.10.1877 Lam, †
21.3.54 Ruhmannsfelden;

\textbf{Glasschröder, Johann,} {\textgreater} Bekannter von Mathilde
Glasschröder, Sängerin, * 14.8.1936;

\textbf{Glasschröder, Mathilde,} {\textgreater} Chorsängerin, 29.10.1917
Ruhmannsfelden, † 8.6.1965;

\textbf{Graßl, Hedwig, geb. Zellner,} {\textgreater} Chorsängerin, *
23.9.1913 Ruhmannsfelden, † 17.2.1982;

\textbf{Gruber, F.,} {\textgreater} Lehrer und Mitwirkender an der
Kirchenmusik, an der Schule 1939;

\textbf{Haug, Dr.,} {\textgreater} Tierarzt, Violoncello;

\textbf{Heinrich, Ludwig,} {\textgreater} Leiter der Ruhmannsfeldener
Blaskapelle, * 24.8.1916 Zachenberg, † 25.9.1984 Ruhmannsfelden;

\textbf{Hellauer, Mariane, geb. Plank,} {\textgreater} Sängerin, *
4.11.1922 Ruhmannsfelden, † 3.10.2001 Deggendorf;

\textbf{Högn, Andreas,} {\textgreater} Vater, Buchbindemeister,
landgräflicher Magistratsrat, Landrat, Landtagsabgeordneter, *
26.8.1839, † 13.05.1913 Deggendorf;

\textbf{Högn, August,} {\textgreater} Sohn, * 17.1.1912 Ruhmannsfelden,
† 27.2.72 Erlangen;

\textbf{Högn, Elfriede, verh. Schlumprecht, }{\textgreater} Tochter, *
14.8.1906 Eberhardtsreuth, † 9.5.1977 München;

\textbf{Högn, Emma, geb. Gerstl,} {\textgreater} Ehefrau, * 19.9.1887, †
19.6.1926 Ruhmannsfelden;

\textbf{Högn, Helene, geb. Zöpfl,} {\textgreater} Mutter, * 16.3.1840
Geiselhöring, † 4.12.1917 Deggendorf;

\textbf{Högn, Ida,} {\textgreater} Nichte von August Högn, * Jun. 1921
Deggendorf;

\textbf{Högn, Josef,} {\textgreater} Bruder, * 6.12.1879 Deggendorf, †
11.2.1968 Deggendorf;

\textbf{Högn, Ludwig, }{\textgreater} Bruder,\textbf{ }* 1871
Deggendorf, † 18.08.1941 Deggendorf;

\textbf{Högn, Otto,} {\textgreater} Bruder, * 29.6.1883 Deggendorf, †
31.1.1938 Deggendorf;

\textbf{Högn, Theres,} {\textgreater} Schwester, * 1.7.1869 Deggendorf,
† 8.8.1913 Deggendorf;

\textbf{Holler, Max,} {\textgreater} Schüler, * 7.5.1930 Ruhmannsfelden;

\textbf{Holzfurtner,} Schreiner, Sänger;

\textbf{Kestlmeier, Valentin,} {\textgreater} Lehrer (1928 – 1938) und
Mitwirkender an der Kirchenmusik;

\textbf{Kraus, Ludwig,} {\textgreater} Bläser;

\textbf{Kraus, Wolfgang,} {\textgreater} Bläser, Gotteszell;

\textbf{Kraus, Wolfgang,} {\textgreater} Kontrabass;

\textbf{Leitner, Stephan,} {\textgreater} Schüler, Högn schrieb ihm
Briefe nach Australien, * 15.6.1932 Ruhmannsfelden;

\textbf{Raster, Josef,} {\textgreater} Neffe von Theres Raster,
Sängerin, * 26.8.1940;

\textbf{Raster, Theres,} {\textgreater} Sängerin, * 2.5.1897
Ruhmannsfelden, † 30.12.1983 Ruhmannsfelden;

\textbf{Rauscher, Ludwig Theodor,} {\textgreater} Instrumentalist, *
27.12.1905;

\textbf{Rauscher, Max,} {\textgreater} Chorregent, * 17.5.1904,
20.10.1927 Hochzeit mit Anna Zellner in Ruhmannsfelden, 4.12.1932
Hochzeit mit Victoria Mayer in München, Giesing, hl. Kreuz;

\textbf{Rauscher, Siegfried Eugen,} {\textgreater} Instrumentalist, *
18.12.1908, 3.4.1968 aus der Kirche ausgetreten;

\textbf{Schlagintweit, Lorenz,} {\textgreater} Posaunist, * 31.7.1917;

\textbf{Schlumprecht, Gertraud, verh. von Molo,} {\textgreater} Enkelin,
* 29.04.1935 Bayreuth;

\textbf{Schlumprecht, Karl} \textbf{Dr.,} {\textgreater} Schwiegersohn,
* 20.04.1901 Fürth, † 21.03.1970 München;

\textbf{Schlumprecht, Lilo, verh. Leuze,} {\textgreater} Enkelin, *
9.7.1931 Deggendorf;

\textbf{Schlumprecht, Werner,} {\textgreater} Enkel, * 1934 Bayreuth, †
Juni 2002 München;

\textbf{Schröck, Maria, geb. Hetzenecker,} {\textgreater} Schülerin,
Sängerin, * 1918 Arnsdorf;

\textbf{Schroll, Albert, }{\textgreater} Chorregent, Gemeindesekretär
,\textbf{ }* 7.11.1903 Steinbühl,  † 25.1.1940 Deggendorf;

\textbf{Schultz, Friedrich,} {\textgreater} Lehrer (1933 – 1939),
Mitwirkender an der Kirchenmusik;

\textbf{Schwannberger, Centa,} {\textgreater} Schüler, Nachbarin, *
17.2.1913; † 15.2.2005 Viechtach

\textbf{Schwannberger, Rudolf,} {\textgreater} Sänger,
Kreisjägermeister, Nachbar, * 17.4.1891 Ruhmannsfelden, † 11.5.1971
Deggendorf;

\textbf{Schwarz, Anton,} {\textgreater} Seminarmusiklehrer, * 20.4.1858
München, † 27.5.1931 Straubing;

\textbf{Seidl, Emilie,} {\textgreater} Sängerin, * 5.6.1924
Ruhmannsfelden;

\textbf{Stern, Anna, geb. Graßl,} {\textgreater} Sängerin, * 25.10.1923
Ruhmannsfelden, † 5.9.1990 Ruhmannsfelden;

\textbf{Stern, Dr. Josef,} {\textgreater} Hausarzt, Nachbar, * 4.8.1907,
† 15.11.2004 Teisnach;

\textbf{Voit, Vitus,} {\textgreater} Apotheker, Violine, * 1.8.1879
München, 22.9.1955 Ruhmannsfelden;

\textbf{Wiegmann, Dr. Doraliesa,} {\textgreater} Ärztin, Bekannte, *
14.10.1919;

\textbf{Zellner, Anna, verh. Rauscher,} {\textgreater} Sängerin, *
15.9.1906 Ruhmannsfelden, † 1927 – 1932;