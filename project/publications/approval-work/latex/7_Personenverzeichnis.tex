\section{Personenverzeichnis}

\begin{itemize}

\item \textbf{Ascherl, Anna},
Violine, Lehrerin (1937 – 1965);

\item \textbf{Bartascheck, Richard}, Sänger im Ruhmannsfeldener
Männerchor;

\item \textbf{Beischmied, Rosalia}, Haushälterin, * 18.5.1901
Ruhmannsfelden, † 28.1.1977 Viechtach;

\item \textbf{Beischmied, Mathilde}, Tochter der Haushälterin,
* 1.1.1923 Ruhmannsfelden;

\item \textbf{Brunner, Josef}, Aushilfsorganist, * 28.6.1924;

\item \textbf{Danziger, Franz sen.}, Schüler, Chorsänger,
Instrumentalist, Chorregent, * 27.9.1904 Ruhmannsfelden, † 25.6.1988
Ruhmannsfelden;

\item \textbf{Ederer, Wilhelm}, Schüler, * 5.1.1931 Ruhmannsfelden,

\item \textbf{Essigmann, Barbara}, Schülerin, Sängerin, *
3.12.1921 Ruhmannsfelden;

\item \textbf{Fink, Wilhelm}, Historiker, Benediktiner Pater, * 9.5.1889
Rottenburg an der Laaber, † 13.2.1965 Metten;

\item \textbf{Freisinger, Johann}, Schüler, Nachfolger als
Gemeindeschreiber der Gemeinde Zachenberg und als Schriftführer der
Feuerwehr Ruhmannsfelden, * 27.3.1927 Ruhmannsfelden;

\item \textbf{Freisinger, Maria}, Sängerin, Schüler, * 18.9.1929
Ruhmannsfelden;

\item \textbf{Goller, Fritz}, Kirchenmusiker, Komponist, * 20.11.1914
Deggendorf, 22.7.1986, † Deggendorf;

\item \textbf{Glasschröder Anna}, Sängerin, * 23.10.1877 Lam, †
21.3.54 Ruhmannsfelden;

\item \textbf{Glasschröder, Johann}, Bekannter von Mathilde
Glasschröder, Sängerin, * 14.8.1936;

\item \textbf{Glasschröder, Mathilde}, Chorsängerin, 29.10.1917
Ruhmannsfelden, † 8.6.1965;

\item \textbf{Graßl, Hedwig, geb. Zellner}, Chorsängerin, * 23.9.1913
Ruhmannsfelden, † 17.2.1982;

\item \textbf{Gruber, F.}, Lehrer und Mitwirkender an der
Kirchenmusik, an der Schule 1939;

\item \textbf{Haug, Dr.}, Tierarzt, Violoncello;

\item \textbf{Heinrich, Ludwig}, Leiter der Ruhmannsfeldener
Blaskapelle, * 24.8.1916 Zachenberg, † 25.9.1984 Ruhmannsfelden;

\item \textbf{Hellauer, Mariane, geb. Plank}, Sängerin, * 4.11.1922
Ruhmannsfelden, † 3.10.2001 Deggendorf;

\item \textbf{Högn, Andreas}, Vater, Buchbindemeister, landgräflicher
Magistratsrat, Landrat, Landtagsabgeordneter, * 26.8.1839, † 13.05.1913
Deggendorf;

\item \textbf{Högn, August}, Sohn, * 17.1.1912 Ruhmannsfelden,
† 27.2.72 Erlangen;

\item \textbf{Högn, Elfriede, verh. Schlumprecht}, Tochter, * 14.8.1906
Eberhardtsreuth, † 9.5.1977 München;

\item \textbf{Högn, Emma, geb. Gerstl}, Ehefrau, * 19.9.1887, †
19.6.1926 Ruhmannsfelden;

\item \textbf{Högn, Helene, geb. Zöpfl}, Mutter, * 16.3.1840
Geiselhöring, † 4.12.1917 Deggendorf;

\item \textbf{Högn, Ida}, Nichte von August Högn, * Jun. 1921
Deggendorf;

\item \textbf{Högn, Josef}, Bruder, * 6.12.1879 Deggendorf, †
11.2.1968 Deggendorf;

\item \textbf{Högn, Ludwig}, Bruder, * 1871 Deggendorf, † 18.08.1941
Deggendorf;

\item \textbf{Högn, Otto}, Bruder, * 29.6.1883 Deggendorf, †
31.1.1938 Deggendorf;

\item \textbf{Högn, Theres}, Schwester, * 1.7.1869 Deggendorf, †
8.8.1913 Deggendorf;

\item \textbf{Holler, Max}, Schüler, * 7.5.1930 Ruhmannsfelden;

\item \textbf{Holzfurtner}, Schreiner, Sänger;

\item \textbf{Kestlmeier, Valentin}, Lehrer (1928 – 1938) und
Mitwirkender an der Kirchenmusik;

\item \textbf{Kraus, Ludwig}, Bläser;

\item \textbf{Kraus, Wolfgang}, Bläser, Gotteszell;

\item \textbf{Kraus, Wolfgang}, Kontrabass;

\item \textbf{Leitner, Stephan}, Schüler, Högn schrieb ihm Briefe nach
Australien, * 15.6.1932 Ruhmannsfelden;

\item \textbf{Raster, Josef}, Neffe von Theres Raster, Sängerin, *
26.8.1940;

\item \textbf{Raster, Theres}, Sängerin, * 2.5.1897 Ruhmannsfelden, †
30.12.1983 Ruhmannsfelden;

\item \textbf{Rauscher, Ludwig Theodor}, Instrumentalist, * 27.12.1905;

\item \textbf{Rauscher, Max}, Chorregent, * 17.5.1904, 20.10.1927
Hochzeit mit Anna Zellner in Ruhmannsfelden, 4.12.1932 Hochzeit mit
Victoria Mayer in München, Giesing, hl. Kreuz;

\item \textbf{Rauscher, Siegfried Eugen}, Instrumentalist, * 18.12.1908,
3.4.1968 aus der Kirche ausgetreten;

\item \textbf{Schlagintweit, Lorenz}, Posaunist, * 31.7.1917;

\item \textbf{Schlumprecht, Gertraud, verh. von Molo}, Enkelin,
* 29.04.1935 Bayreuth;

\item \textbf{Schlumprecht, Karl Dr.}, Schwiegersohn,
* 20.04.1901 Fürth, † 21.03.1970 München;

\item \textbf{Schlumprecht, Lilo, verh. Leuze}, Enkelin, * 9.7.1931
Deggendorf;

\item \textbf{Schlumprecht, Werner}, Enkel, * 1934 Bayreuth, † Juni 2002
München;

\item \textbf{Schröck, Maria, geb. Hetzenecker}, Schülerin, Sängerin, *
1918 Arnsdorf;

\item \textbf{Schroll, Albert}, Chorregent, Gemeindesekretär, *
7.11.1903 Steinbühl,  † 25.1.1940 Deggendorf;

\item \textbf{Schultz, Friedrich}, Lehrer (1933 – 1939),
Mitwirkender an der Kirchenmusik;

\item \textbf{Schwannberger, Centa}, Schüler, Nachbarin, * 17.2.1913; †
15.2.2005 Viechtach

\item \textbf{Schwannberger, Rudolf}, Sänger, Kreisjägermeister,
Nachbar, * 17.4.1891 Ruhmannsfelden, † 11.5.1971 Deggendorf;

\item \textbf{Schwarz, Anton}, Seminarmusiklehrer, * 20.4.1858 München,
† 27.5.1931 Straubing;

\item \textbf{Seidl, Emilie}, Sängerin, * 5.6.1924 Ruhmannsfelden;

\item \textbf{Stern, Anna, geb. Graßl}, Sängerin, * 25.10.1923
Ruhmannsfelden, † 5.9.1990 Ruhmannsfelden;

\item \textbf{Stern, Dr. Josef}, Hausarzt, Nachbar, * 4.8.1907, †
15.11.2004 Teisnach;

\item \textbf{Voit, Vitus}, Apotheker, Violine, * 1.8.1879 München,
22.9.1955 Ruhmannsfelden;

\item \textbf{Wiegmann, Dr. Doraliesa}, Ärztin, Bekannte, * 14.10.1919;

\item \textbf{Zellner, Anna, verh. Rauscher}, Sängerin, * 15.9.1906
Ruhmannsfelden, † 1927 – 1932;

\end{itemize}