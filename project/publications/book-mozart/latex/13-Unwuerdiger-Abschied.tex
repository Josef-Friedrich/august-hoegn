\section{Unwürdiger Abschied als Chorregent}

Ähnlich große Wellen wie die Entlassung des Chorregenten Max Rauscher im
Jahr 1927 schlug 1953 August Högns Ausscheiden als Kirchenchorleiter.
Dem Chorleiterwechsel ging ein Pfarrerwechsel voraus. Am 17. Februar
1953 verstarb Pfarrer Jakob Bauer. Der neue Pfarrherr Franz Seraph
Reicheneder wurde am 26. Mai 1953 in Ruhmannsfelden empfangen und am
14. Juni schließlich feierlich mit einer Aufführung der Josephi-Messe
installiert. August Högn wollte eigentlich bei Antritt des neuen
Pfarrers seinen Rücktritt als Chorregent und Organist bekannt geben,
doch der neue Pfarrer lehnte sein Ersuchen anstandsgemäß ebenso ab wie
einige Zeit vorher Pfarrer Jakob Bauer und in der Übergangszeit
Pfarrprovisor Georg Huber. Högns Rücktrittsabsichten sprachen sich
sogar herum, so dass Josef Brunner – er war während der Kriegszeit
Aushilfsorganist in Ruhmannsfelden – eine Bewerbung für die
möglicherweise frei werdende Chorregentenstelle bei der
Kirchenverwaltung einreichte.

Umso verwunderlicher ist die Dramatik mit der sich Högns Ausscheiden als
Chorregent dann tatsächlich vollzog: Nachdem seine Bitte um Rücktritt
von Pfarrer Reicheneder abgelehnt worden war, scheint Högn mit einer
längerfristigen Dienstzeit als Chorregent und Organist gerechnet zu
haben, sonst hätte er nicht schon 1953 zwei Marienlieder zum
marianischen Jahr 1954 komponiert. Reicheneder aber, von Högns
Amtsmüdigkeit überzeugt, war wahrscheinlich davon ausgegangen, dass der
75-jährige bald abdanken würde. Vielleicht hatte er auch deswegen der
von vorneherein als Nachfolgerin favorisierten Maria Reisinger eine
baldige Anstellung in Aussicht gestellt. Maria Reisinger, ein
Waisenkind, war Reicheneders Ziehtochter, um deren Ausbildung er sich
gekümmert hatte.

Mit Sicherheit hat Reicheneder die Ablehnung des Rücktrittsgesuchs aus
folgenden zwei Gründen bereut: Zum einen war für seine Ziehtochter
keine Anstellung in absehbarer Zeit greifbar, zum anderen leisteten zum
damaligen Zeitpunkt Högn und sein Chor äußerst schlechte Darbietungen.
Die Aufführung der \textit{Josephi}{}-Messe durch den
\zitat{klangvollen Chor} zu Reicheneders Installation war
nicht repräsentativ für den alltäglichen Kirchenmusikbetrieb, da hier
viele Aushilfen mitwirkten und hatte möglicherweise bei Reicheneder
einen falschen Eindruck hinterlassen.

Der Chor, der während der Woche sang, bestand nur aus vier Personen: Den
drei Sängerinnen Mathilde Glasschröder, Barbara Essigmann, Theres
Raster, und August Högn, der selbst eine Männerstimme übernahm. Vor
allem die Heterogenität der einzelnen Stimmen, die durch die
Kleinstbesetzung hervortrat, hinterließ bei so manchem Kirchenbesucher
einen schlechten Eindruck. Wurde Mathilde Glasschröder als sängerische
Naturbegabung mit großer Stimme, vergleichbar einer Opernsängerin,
beschrieben, blieb Theres Raster als außerordentlich schlechte Sängerin
mit \textit{fürchterlichem} Stimmklang in Erinnerung. Die alternde
Stimme Högns scheint sich ebenso wenig in einen Gesamtklang eingebunden
zu haben. Diese Besetzung lässt sich deutlich am \textit{Marienlied Nr.
12 op. 63}, Högns letztem Werk seiner Chorregentenzeit ablesen. Das
Stück ist größtenteils zweistimmig solistisch und nur in den letzten
vier Takten dreistimmig besetzt, so dass neben den guten Sopranistinnen
Mathilde Glasschröder und Barbara Essigmann die schlechte Alt-Sängerin
Theres Raster kurz zum Einsatz kam.

Neben dem schlechtem Gesang stellte für Reicheneder möglicherweise auch
das kirchenmusikalische Repertoire einen Stein des Anstoßes dar. Ein
und dasselbe Lied, das zum Schluss des Gottesdienstes gesungen wurde,
hatte sich als Standardstück eingebürgert und nicht nur werktags wurden
Teile aus dem Gloria und Credo übersprungen.

Obwohl in Ruhmannsfelden die Bereitschaft zum Singen recht groß war – es
gab neben dem Kirchenchor einen Männerchor und eine weltliche
Liedertafel – gab es lediglich ein Gesangsquartett als Kirchenchor.
Reicheneder muss als Hauptgrund für diesen Missstand die laxe
Probenpraxis in Högns Wohnung angesehen haben, den er durch Ansetzung
von öffentlichen Proben zu beseitigen versuchte – wohlgemerkt mit Högn
als Chorleiter. Es ist höchst ungewöhnlich, dass nicht der Chorleiter
über die Anberaumung der Proben entscheidet. Einerseits wollte
Reicheneder durch diese Maßnahme, das Niveau der musikalischen
Darbietungen erhöhen. Andererseits hoffte er wohl insgeheim, dass
Högn durch diese Bevormundung abdanken und die Stelle für seine
Vertrauensperson Maria Reisinger frei machen würde. Die mit Högn
befreundeten Chorsängerinnen Mathilde Glasschröder und Barbara
Essigmann erschienen nicht zu den festgesetzten Proben und lieferten
somit Reicheneder einen Grund zum Einschreiten.

In seinen Briefen an die zwei Chorsängerinnen und an Högn teilte der
Pfarrer ihnen kurz vor Weihnachten 1953 ihre „Kündigung“ mit. Über den
genauen Wortlaut der „Kündigungsschreiben“ ließe sich natürlich viel
spekulieren. Bekannt ist nur, dass die Briefe große Verärgerung bei den
Betroffenen auslösten. Fraglich ist allerdings, ob diese
„Kündigungsschreiben“ wirklich in Zusammenhang mit Högns schwerem
Schlaganfall stehen, wie von der Sängerin Barbara Essigmann behauptet.
Tatsache ist, dass Högn seitdem nicht mehr in der Lage war, Orgel zu
spielen.

Deshalb wirkt der Brief, den Högn nach seinem Krankenhausaufenthalt am
25. Januar 1954 an Reicheneder schrieb, schon etwas befremdlich. Mit
der \zitat{berechtigten Forderung auf Ruhestand auch im
Kirchenchordienst} verkündete Högn in diesem Schreiben seinen Rücktritt
als Chorregent und Organist und bezog sich dabei nicht auf die Folgen
seines Schlaganfalls oder auf das „Kündigungsschreiben.“ In seinem
Antwortbrief vom 6. Februar 1954 nahm Reicheneder zu Högns Brief
Stellung, in dem dieser von sich aus abdankte, und unterstrich, dass er
Högns \zitat{Standpunkt voll und ganz verstehen} könne,
seinen Rücktritt aber \zitat{sehr bedauere.}

Wären nur diese zwei Briefe und nicht zusätzlich die genauen
Erinnerungen von mehreren Zeitzeugen, könnte man eine ganz andere
Schlussfolgerung daraus ziehen. Sowohl Reicheneder als auch Högn waren
daran interessiert, der Nachwelt eine andere Version des Ausscheidens
Högns als die der Kündigung durch Reicheneder zu überliefern. Nur
zwischen den Zeilen, kann man in beiden Briefen die vorhergehenden
Ereignisse erahnen, wie zum Beispiel an der Stelle, an der Högn die
\zitat{Kündigung seitens des Hochwürdigen Herrn Pfarrers und
der gleichen} als \zitat{blödes Weibergeschwätz} bewertet und
darauf hinweist, dass \zitat{eine vertragliche Abmachung
über den Kirchenchordienst zwischen Pfarramt Ruhmannsfelden} und ihm
 \textit{niemals bestanden} habe.\zitat{\textup{}}Und seine
angebliche \zitat{Stimmbandlähmung,} die Reicheneder als
Entschuldigung anführte, weshalb er Högn keinen Krankenbesuch
abstattete, erscheint in diesem Zusammenhang mehr als eine faule
Ausrede.

Über den wahren Wortlaut der „Kündigungsbriefe“ ließe sich natürlich
viel spekulieren. Allein schon ihr Verschwinden ist ein Beweis für ihre
Brisanz. Reicheneder war ein passionierter Historiker und Archivar. Es
gibt wohl keinen auf die Gegend bezogenen Zeitungsartikel aus
Reicheneders Ruhmannsfeldener Zeit, der nicht in seine über dreißig
prall gefüllte Ordner umfassende \textit{Chronik Ruhmannsfelden}
angefügt wurde. Auch seine Korrespondenz dokumentierte Reicheneder sehr
genau. Viele von ihm verfasste Briefe lassen sich im Pfarrarchiv im
Kohlepapierabdruck nachlesen, wie etwa das Schreiben an Högn vom 6.
Februar 1954. Die „Kündigungsschreiben“ hat Reicheneder ganz bewusst
nicht archiviert, damit kein schlechtes Licht auf ihn fällt. Einen
unwürdigeren Abschied hätte Reicheneder Högn nach 43 Jahren Dienstzeit
an der Kirchenmusik in Ruhmannsfelden kaum bieten können.

Diese Vorgehensweise Reicheneders gegenüber Högn ist kein Einzelfall.
Ein Ereignis wenige Jahre später ist bezeichnend für seine
Problemlösungsstrategie mit der Brechstange: Nachdem Reicheneder bei
einer Sonntagspredigt unter anderem über die
\zitat{Leistungsabzeichen für nächtliche Liebesfahrten}
wetterte, die beim Fußballerball 1957 verliehen wurden, ließen sich die
Beschuldigten nicht zurechtweisen und veranstalten, nach Angaben des
Geistlichen selbst, aus einer Trotzreaktion heraus ein
Faschingsbegräbnis am Aschermittwoch mit Musik und Saufgelage.
\zitat{Bis die Haupträdelsführer beim Pfarramt vorstellig
geworden sind,} wie es in einer Pressemitteilung des Pfarramts hieß,
sollten nun die Glocken in Ruhmannsfelden schweigen. Diese Aktion
machte natürlich nicht nur in der regionalen Presse die Runde. Die
Glocken läuteten erst wieder, als sich der Bürgermeister in den Fall
einschaltete und für eine Lösung des Problems sorgte, die schriftlich
festgehalten wurde.

In der Kirchenverwaltungssitzung vom 21. Februar 1954 wurde Högns
Nachfolge endgültig geregelt. Maria Reisinger erhielt die
Organistenstelle und Franz Danziger übernahm die Chorleitung.