\section{Der „Mitläufer“}

In Folge des Entnazifizierungs-Prozesses wurde August Högn am 20.
Februar 1947 vom Spruchkammergericht Viechtach zu einer Geldstrafe von
300 Reichsmark verurteilt und in die Gruppe der „Mitläufer“ eingestuft.
Die Militärregierung suspendierte Högn am 11. September 1945 aus dem
Schuldienst. Bis sein Fall im Februar 1947 verhandelt wurde, verfügte
Högn über keinerlei Einkünfte. Er lebte von den Bezügen des
Chorregentendienstes und den Nahrungsmitteln, die er als Gegenleistung
für seine Arbeit als Erntehelfer bekam. Seine große Dienstwohnung im
Schulhaus musste er 1946 räumen. Er zog in eine nahe gelegene Wohnung
im Haus des Gemeindesekretärs Hertl.

Obwohl der fünfzigseitige Akt seines Prozesses fast ausschließlich
Entlastendes enthält, genügte allein die Mitgliedschaft in
zugegebenermaßen mehreren NS-Organisationen, dass Högn nach dem
Prozess nicht der Gruppe der „Entlasteten“ angehörte, wie er sich
selbst gerne eingeordnet gesehen hätte, sondern als „Mitläufer“
bezeichnet wurde. Högn war am 1. Mai 1933 in die NSDAP eingetreten,
trug seitdem die Mitglieds-Nummer 2663243 und zahlte monatlich drei
Reichsmark Beitrag. 1934 trat er in den NS-Lehrerbund und den
Reichsluftschutzbund ein, 1935 in die Deutschen Jägerschaft, 1936 in
die NS-Volkswohlfahrt und schließlich 1942 in die Reichsmusikkammer.

\begin{figure}
\centering
\img[width=6cm]{Hoegn-mit-Enkel}
\caption{August Högn mit Enkel Werner Schlumprecht (1939)}
\end{figure}

Högn rechtfertigte sich im Prozess für seine NSDAP-Mitgliedschaft und
führte aus, dass \textit{der Beitritt zur Partei keine persönliche
freie Willensäußerung oder ein offenes Bekenntnis zum Nazi-Programm
oder gar Sympathie für Hitler war, sondern die Folge des Zwanges, der
sich von allen Seiten geltend machte und die Furcht vor einer
beruflichen Benachteiligung.} Diese Aussage wurde auch vom
Spruchkammergericht gestützt, das dem Betroffenen die
\textit{formelle Mitgliedschaft zur NSDAP} bescheinigte und darlegte,
dass er dem \textit{Nationalsozialismus passiv gegenüberstand und sich
politisch nicht aktiv betätigt} hatte.

Högn gab zu seiner Verteidigung, laut Prozess-Akten, an, dass er
beispielsweise nie die Parteiuniform getragen, vielfach mit
\textit{Grüß Gott} gegrüßt habe, den Religions-Unterricht durch das
Einüben religiöser Gesänge unterstützt und in den Klassenzimmern
persönlich die entfernten Kruzifixe wieder an ihren alten Platz gehängt
habe. In einem Brief an das Spruchkammergericht Viechtach nannte er
sich sogar einen \textit{Vertrauensmann aller hiesigen Antifaschisten.}

Tatsächlich gibt es Belege für Högns oppositionelles Verhalten. Ende
1943 erging vom Abschnittsleiter der NSDAP in Cham eine Anweisung an
das Landratsamt Viechtach, dass Högn die Erlaubnis zur Ausübung des
Chorregentendienstes entzogen werden sollte. Bucher beschuldigte
Högn, eigenmächtig in gewissen Klassen zwei, statt wie in mehreren
Rundschreiben vorgeschrieben, nur eine Religionsstunde pro Woche
angesetzt zu haben. Als Bestrafung sollte die enge kirchliche Bindung
Högns in Form seines Chorregentendienstes gelöst werden, den er dann
ab 6. Januar 1944 abgeben musste, ihn aber tatsächlich nur werktags
aufgab.

Die Auseinandersetzungen Högns mit dem örtlichen Jägerverband in den
Jahren 1935 und 1936, der wie alle Jägerverbände im „Reichsbund
Deutsche Jägerschaft“ eingegliedert war, zeigen Högn ebenfalls von
seiner aufmüpfigen Seite. In der Jägerei setzten die Nazis sehr früh
ihre radikalen Reformen um. Kein geringerer als der
\textit{Reichsjägermeister} Hermann Göring verkündete das neue
Jagdgesetz, das ab 18. Januar 1934 galt. Die Jäger mussten demnach den
Wildbestand in ihrem Revier genau schätzen und durften nur mehr einen
gewissen Prozentsatz des geschätzten Bestandes abschießen. Diese
Einschränkungen brachten für viele Jäger finanzielle Einbußen mit sich,
die manche sehr schmerzten. Viele Jäger mussten das Wildfleisch
verkaufen, um erst einmal die für das Jagdrevier zu entrichtende Pacht
– so war es auch in Högns Fall – zu erwirtschaften. Diese
Abschuss-Einschränkung versuchte August Högn zu umgehen, indem er den
Bestand in seinem Revier einfach viel zu hoch angab. Die stolze Zahl
von zwanzig starken Böcken übermittelte Högn 1935 den zuständigen
Behörden, obwohl im Vorjahr nur ein einziger starker Bock in seinem
Revier geschossen werden konnte. Prompt wurde Högn vom zuständigen
Hegeringführer Max Forster auf diese Übertreibung hingewiesen. Högn gab
bei dieser Zurechtweisung keineswegs klein bei, vielmehr muss er in
einem Brief an Forster, der leider nicht erhalten ist, noch ordentlich
nachgelegt haben. Forster fühlte sich von diesem Brief, der
\textit{im ersten Zorn geschrieben} worden sei, so angegriffen, dass er
monierte, Högn wolle ihn \textit{durch persönliche Angriffe im
Vertrauen auf seine gewandte Feder niederkämpfen.} Forster folgerte aus
dem Brief, dass Högn \textit{eine Zusammenarbeit nicht wünsche und die
Tätigkeit der Vollzugsorgane nicht anerkennen wolle.} Auch 1936 ging
Högn auf Konfrontationskurs mit dem Jägerverband. Beim Pflichtschießen
am 30. Mai 1936 verwendeten zwar die Jäger Paukner und Völkl Högns
Gewehre, doch Högn selbst blieb zum anberaumten Termin ohne
Entschuldigung fern. Auch zum Wiederholungstermin am 15. Juni 1936
erschien er nicht.

\begin{figure}
\centering
\img[width=6cm]{Hoegn-2-Weltkrieg.jpg}
\caption{August Högn 1944: Das Bild eines eher im inneren Konflikt
lebenden, gebrochenen Mannes als das eines stolzen, überzeugten Nazis.}
\end{figure}

Es ist deshalb nicht verwundlich, dass viele Bürger bereit waren, Högn
eine oppositionelle Einstellung zu den Nazis für das
Spruchkammergericht zu bescheinigen, darunter waren angesehene
Persönlichkeiten, wie der Pfarrer und die Bürgermeister von
Ruhmannsfelden, Zachenberg und Patersdorf. Auch drei zufällig befragte
Personen, die wahrscheinlich von dem Ermittler des Spruchkammergerichts
bei einer Visite auf der Straße vernommen wurden, schilderten Högn
als Gegner der Nazis. Als treuer Parteianhänger ist August Högn auch
keinem der sechs zu dieser Thematik fast 60 Jahre nach Kriegsende
befragten Zeitzeugen in Erinnerung geblieben. Als fanatische
Anhängerin der NS-Ideologie prägten sich dagegen einige der Befragten
die seit 1942 an der Volksschule unterrichtende Lehrerin Charlotte
Werner ein. In ihrem Unterricht mussten Schüler aus einem Kalender
Vorträge über herausragende Persönlichkeiten des NS-Regimes halten.
Vergleichbares ist aus Högns Unterricht nicht bekannt.

Rückschlüsse auf Högns negative Einstellung zu Hitlers Politik lassen
sich nicht zuletzt aus seinem in der Nachkriegszeit entstandenen
Geschichtswerk ziehen. Folgende Textpassage aus der \textit{Geschichte
von Zachenberg} lässt sogar einen Interpretationsspielraum in Bezug auf
seine anfangs positive Einstellung zur NS-Ideologie, die sich im Lauf
der Zeit zur Ablehnung verändert haben könnte: \textit{Die 1932/33
hereinbrechende Hitlerzeit hat zwar auf der einen Seite dieser
schlimmen Arbeitslosigkeit ein Ende gesetzt, aber auf der anderen Seite
den unheilvollen zweiten Weltkrieg heraufbeschworen, der das größte
Unglück, das je über ein Land und seine Bevölkerung kommen konnte, in
übervollem Maße über Deutschland ausgeschüttet hat.} Als ob er sich
selbst für seine anfängliche Gefolgschaft rechtfertigen wollte, führte
er die Bekämpfung der Arbeitslosigkeit durch Hitler als eine zu
würdigende Leistung an, geht aber sehr deutlich auf Distanz zur
späteren NS-Politik, die die \textit{Kriegsfurie}, – so nennt er den
Weltkrieg – verursacht hat, und dessen Folgen aufzählt. Wie ein
Läuterungsversuch erscheinen die ausgedehnten Passagen in seiner
\textit{Geschichte von Ruhmannsfelden}, in denen er den beiden
Widerständlern, Bürgermeister Sturm und Studienrat Leonhard Donauer,
die nur mit viel Glück der Exekution durch die SS entronnen waren, ein
Denkmal setzt.

Außer Frage steht, dass Högn kein Rassist war. Franz Danziger,
Kirchenchorsänger, Violinist und schließlich Högns Nachfolger als
Chorregent, war ein \textit{Halbjude}, wie es im Jargon der
Nationalsozialisten geheißen hätte. Er verliert in seinen Memoiren
kein schlechtes Wort über Högn. Vielmehr ehrte er seinen ehemaligen
Lehrer, indem er einige seiner Kompositionen aufführte.

Aufgrund dieser Tatsachen könnte man meinen, Högn wäre ein
Widerständler gewesen, er sei vom Spruchkammergericht zu Unrecht als
„Mitläufer“ eingestuft worden und in Wirklichkeit ein „Entlasteter“
gewesen. Einige Details zu seiner Vergangenheit, die Högn von einer
ganz anderen Seite zeigen, hat er dem Spruchkammergericht jedoch
verschwiegen, etwa dass sein Schwiegersohn ein hoher Nazi-Funktionär
war.

Die Bewunderung, die Högn für seinen Schwiegersohn Dr. Karl
Schlumprecht, besonders für dessen politische Karriere zeigte und das
gute Verhältnis zu ihm, sind Indizien dafür, dass Högn die Ideologie
des Nationalsozialismus nicht von Grund auf abgelehnt hat. Schlumprecht
wurde am 1. Oktober 1929 als zweiter Staatsanwalt nach Deggendorf
versetzt. Hier lernte er Högns Tochter Elfriede Kroiss kennen, die zu
diesem Zeitpunkt noch mit dem Deggendorfer Bierbrauer und Gastwirt
Johann Kroiss verheiratet war. Diese Ehe wurde geschieden und am 4.
Juni 1932 heirateten Karl Schlumprecht und Elfriede Högn. Bereits am 1.
Dezember 1930 trat Schlumprecht – er war seit 1920 Mitglied der
deutschnationalen Volkspartei – in die NSDAP ein. Ab 1931 setzte er
sich aktiv für die pro\-pa\-gan\-di\-sti\-schen Zwecke der Partei als
sogenannter „Gauredner“ und ab 1933 als „Reichsredner“ ein. Nach der
Machtübernahme der NSDAP war er zunächst ab 11. März 1933
Personalreferent des bayerischen Innenministers Wagner und von 26.
April 1933 bis April 1937 erster nationalsozialistischer
Oberbürgermeister von Bayreuth. Anschließend arbeitete er als
Ministerialdirektor im Finanzministerium in München und wurde zum
Leiter der Wirtschaftsabteilung beim Chef der Zivilverwaltung der 10.
Armee mit Dienstsitz in Radom/Polen (1941) und daraufhin zum Leiter der
Wirtschaftsabteilung beim Militärbefehlshaber in Belgien und
Nordfrankreich (1941-1943) ernannt. Zurückgekehrt nach München
übernahm er vom 16. Juli 1943 bis 21. April 1944 die Leitung des
bayerischen Wirtschaftsministeriums. Im Laufe seiner
Bilderbuch-Karriere in NS-Staat war er zeitweise Mitglied des
bayerischen Landtags, Mitglied des Reichstags und SS-Brigadeführer.
Vor allem die berufliche Karriere Schlumprechts war der Grund,
weshalb Högn auf seinen Schwiegersohn stolz war. Er wäre es
sicherlich nicht gewesen, hätte er die NS-Ideologie in den Grundzügen
abgelehnt.

\begin{figure}
\img{Karl-Schlumprecht.jpg}
\caption{Dr. Karl Schlumprecht mit Frau Frieda, geborene Högn}
\end{figure}

Högn wusste den guten Draht zu einem hohen NS-Funktionär zu nutzen –
nicht zum eigenen Vorteil, sondern zu dem der Allgemeinheit. Der
damalige Ruhmannsfeldener Bürgermeister Sturm stellte am 1. September
1938 an den Ministerialdirektor Schlumprecht im Finanzministerium einen
Antrag auf Unterstützung bei der Finanzierung des Außenputzes der
Turnhalle. Der Vermerk auf dem Brief \textit{In Abdruck an H.
Oberlehrer Högn, hier zur Kenntnis} zeigt eindeutig, dass diese
Petition durch Högn, den ehemaligen Turnvereinsvorstand und
Veranstalter von Benefizkonzerten zugunsten der Errichtung der
Turnhalle, eingefädelt worden war.

Im Blick auf Högns NS-Vergangenheit gibt uns sein \textit{Weihegesang
Es-Dur} für Chor und eine Blechbläserformation, eine Komposition mit
stark NS-pro\-pa\-gan\-di\-sti\-schen Zügen, ein Rätsel auf. Nur schwer lässt
sich aus heutiger Sicht der \textit{Vertrauensmann aller hiesigen
Antifaschisten,} wie sich Högn selbst bezeichnet hat, mit dem –
überspitzt formuliert – „NS-Komponisten Högn“ vereinbaren. Wie kann ein
Mann, der seine oppositionelle Einstellung zum NS-Staat durchaus
glaubhaft und manche Aspekte auch beweisbar darlegte, eine solche
Propaganda-Musik schreiben? Wäre nicht deshalb eine Einteilung Högns
in die Gruppe der „Aktivisten“ durch das Spruchkammergericht Viechtach
eher gerechtfertig gewesen, als in die Gruppe der „Mitläufer“? Das
Urteil des Spruchkammergerichts Viechtach, dass Högn dem
\textit{Nationalsozialismus passiv gegenüberstand und sich politisch
nicht aktiv betätigt} habe, ist demnach schlichtweg falsch.

\begin{figure}
\img{Weihegesang}
\caption{Sopran-Stimme des Weihegesang Es-Dur}
\end{figure}

Historisch gesehen ist der \textit{Weihegesang} ein kleiner
Sensationsfund. Der sparsame Högn hat nach Kriegsende und dem Untergang
der Nazis auf den Rückseiten der Notenblätter des \textit{Weihegesangs}
pikanterweise das Grablied \textit{Näher mein Gott zu dir!} notiert und
den \textit{Weihegesang} lediglich durchgestrichen. Wahrscheinlich
war nach Kriegsende das Notenpapier so knapp, dass Högn es sich nicht
erlauben konnte, diese brisante Komposition mit noch freien Rückseiten
zu vernichten. Der \textit{Weihegesang} überdauerte so in der Rubrik
„Trauergesänge“ des Notenschranks als \textit{Näher mein Gott zu Dir}
die NS-Zeit um fast 60 Jahre.

Die durch Regentropfen verwischte Tinte des \textit{Weihegesangs}
verrät, dass die Komposition im Freien aufgeführt wurde. Der Einsatzort
war wahrscheinlich die von der Wehrmacht veranstalteten weltlichen
Trauerfeiern für Soldaten, die vor den so genannten christlichen
„Heldengottesdiensten“ stattfanden. Da in den seltensten Fällen der
Leichnam der gefallenen Soldaten überführt werden konnte, zog bei
dieser Trauerfeier eine Abordnung von bis zu zehn Soldaten auf
Fronturlaub nicht zum Grab sondern zum Kriegerdenkmal, wo
wahrscheinlich auch der \textit{Weihegesang} zu hören war. Diese
weltliche Trauerfeier ist ein Beispiel von vielen, wie sich der
Nationalsozialismus christlicher Riten bediente und in seine neu
kreierten Feiern einbaute. Ähnlich eklektizistisch ging Högn bei seiner
Komposition vor: Er lehnte den \textit{Weihegesang} stark an das
christliche Grablied an.

Der \textit{Weihegesang} klingt genauso farbig, lieblich und an manchen
Stellen kitschig wie die vier \textit{Grablieder Nr. 1-4}.

Auch wenn Högn im Unterschied zu den Grabliedern einzelne Phrasen im
Chor im Unisono und die Blechbläser rhythmisierter setzte, lässt sich
die musikalische Ähnlichkeit nicht leugnen. Es scheint, als hätte Högn
seine NS-Musik nicht anders wie in seinem lang eingeübten
Kirchenmusikstil komponieren können. War der \textit{Weihegesang}
deshalb schlechte Propaganda-Musik? Das Gegenteil ist der Fall: Der
belanglose und liebliche Tonfall verleiht dem Gesang eine positive
Grundstimmung, verbreitet Optimismus und suggeriert dem Hörer eine
„heile Welt“. Die rhythmisierte Bläserbegleitung lässt den
\textit{Weihegesang} von fern an einen Marsch erinnern, das für die
propagandistischen Zwecke zuverlässigste und deshalb von den Nazis am
meisten eingesetzte Musikgenre. Ein „propagandistischer Kunstgriff“ ist
Högn bei der Vertonung der vorletzten Textzeile jeder Strophe gelungen.

Högn verzichtet in diesen Takten auf den vierstimmigen Satz und führt
alle Chorstimmen im Unisono. Der Effekt, dass alle Sänger eine Melodie
singen, suggeriert einerseits ein Gemeinschaftsgefühl, denn jeder
Zuhörer könnte sich an diesen Stellen dem Chor anschließen und wie alle
anderen Sänger \textit{mit einer Stimme} singen. Andererseits bewirkt
der plötzliche Übergang vom Unisono in den vierstimmigen Satz von der
vorletzten zur darauf folgenden letzten Textzeile der jeweiligen
Strophen eine große klangliche Steigerung und damit eine besondere
Hervorhebung der letzten Textzeile. So bleiben dem Zuhörer vor allem
die letzte und im Fall des \textit{Weihegesangs} auch die für die zu
übermittelnde Botschaft wichtigsten Zeile einer Strophe, vor allem aber
der Schluss des Textes in Erinnerung: \zitat{Ein Geist
 \textup{[}}NS]\zitat{lässt Deutschland neu erstehn.}

\settowidth{\versewidth}{Wohl droht aufs neu ob unserm Haupt die schwierge Wetternacht.}
\begin{verse}[\versewidth]
\itshape
Die Ihr dereinst fürs Vaterland gezogen in die Schlacht \\
und dort das teure Leben uns zum Opfer habt gebracht. \\
Die Ihr mit Eurem Herzensblut die Wohlstatt habt getränkt. \\
An Euch, die selge Heldenschar, die Heimat treu gedenkt. \\!

Wohl droht aufs neu ob unserm Haupt die schwierge Wetternacht. \\
Verrat und Meineid haben freudlos, leidlos uns gemacht. \\
Es ist, als ob umsonst vermachet ihr den Lebenshauch. \\
Geborsten ist der Ehrenschild und tot der Väter Brauch. \\!

Soll das der schnöde Dank für Euer Lebensopfer sein, \\
dass zaghaft, wehrlos wir dem feigen Untergang uns weihn? \\
Steigt ein, der aus Walhallas Höhe, hoch ist es Zeit zur Tat, \\
Ihr deutschen Recken! Ein Geist lässt Deutschland neu erstehn! \\!
\end{verse}

Högns \textit{Weihegesang} transportierte anhand seines Texts eines
unbekannten Autors folgende Botschaft an die Trauergäste: Angesichts
der schwierigen Kriegslage (\textit{Wohl droht aufs neu … die schwierge
Wetternacht} – Stalingrad, Z. 5) und aufgrund der noch immer zu
geringen Anstrengungen für den Krieg von Seiten der Bevölkerung,
(\textit{Verrat und Meineid haben freudlos, leidlos uns gemacht.} Z. 6;
\textit{zaghaft, wehrlos} Z. 10) besteht die Gefahr, dass Deutschland
den Krieg verliert. (\textit{feiger Untergang} Z. 10). Der Tod des
Angehörigen wäre in diesem Fall umsonst gewesen (\textit{Es ist, als
ob umsonst vermachet ihr den Lebenshauch} Z. 7; \textit{Soll das der
schnöde Dank für Euer Lebensopfer sein} Z. 9). Als Rettung werden
Hitler (\textit{der aus Walhallas Höhe} Z. 11) und die
nationalsozialistischen Bewegung (\textit{ein Geist} Z. 12)
angepriesen, die den Menschen neuen Mut geben (\textit{hoch ist es Zeit
zur Tat} Z. 11) und die wie „Götter“ Deutschland \textit{neu erstehn}
(Z. 12) lassen. Der hochgradig propagandistische Text des
Weihegesangs benützt also den Tod eines Soldaten, um einen
Durchhalteappell an die versammelten Trauergäste zu richten.

Nach dem Inhalt dieses Textes zu urteilen, kann der \textit{Weihegesang}
erst kurz vor Ende des Kriegs geschrieben worden sein. Diese
Propaganda-Musik entstand deshalb möglicherweise um den Zeitpunkt
herum, als Högn die Ausübung des Chorregentendienstes ab dem 6.
Januar 1944 durch die NS-Behörden untersagt wurde. Vielleicht
komponierte Högn den \textit{Weihegesang} nur, um weiterhin den
Chorregentendienst ausüben zu können. Dass seit Högns Entlassung nur
noch stille Messen bei den Heldengottesdiensten gelesen wurden, fiel
auch dem Wehrführer der Feuerlöschpolizei und Stellvertreter des
Ortsgruppenleiters Josef Hengkofer auf, der in seinem Brief vom 12.
Januar 1944 den Regierungspräsidenten von Regensburg bat, Högn die
Ausübung des Chordiensts wieder zu erlauben. Festliche Musik für die
NS-Heldenzeremonie war also durchaus ein wirksames Druckmittel, die
Nazis kooperativ zu machen. Vielleicht versuchte Högn tatsächlich, mit
Hilfe des \textit{Weihegesangs} den Veranwortlichen seinen guten Willen
zu beweisen, damit die Sanktionen rückgängig gemacht wurden oder
härtere Sanktionen, wie etwa eine Entlassung aus dem Schuldienst,
ausblieben.