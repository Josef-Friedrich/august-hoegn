Als \textit{Mozart von Ruhmannsfelden} wurde August Högn in einer
Ansprache zu seinem 80. Geburtstag bezeichnet. Der schmeichelhafte
Vergleich mit dem Salzburger Komponisten war eine Verbeugung vor seinem
jahrzehntelangen engagierten Mitwirken am musikalischen Leben in diesem
kleinen Ort des Bayerischen Waldes und besonders vor dem umfangreichen,
in Ruhmannsfelden noch nie da gewesenen, kompositorischen Schaffen, auf
das er zurückblicken konnte. Auch hatte er ein beachtliches
heimatkundliches Werk vorzuweisen und das rechtfertigte mehr als genug,
die am letzten runden Geburtstag erteilten Ehren. Was die Würdigung
seines Schaffens und die Anerkennung seiner Dienste für die
Allgemeinheit angeht, könnte der Unterschied zwischen seinen Lebzeiten
und der Gegenwart kaum größer sein: Fast 50 Jahre nach seinem Tod
kennt kaum mehr ein Ruhmannsfeldener den Namen August Högn, geschweige
denn sein musikalisches und heimatkundliches Werk. Durch Zufall habe
ich im Notenschrank der Ruhmannsfeldener Pfarrkirche einige seiner
Handschriften entdeckt, die mein Interesse an seinem Leben und Wirken
weckten und zum Verfassen dieser Arbeit führten.

Ich möchte mit dieser Arbeit an das Leben und Werk des Rektors,
Heimatforschers und Komponisten August Högn erinnern, um seine Person
und sein Werk vor dem Vergessen zu bewahren. Da Högn sehr eng mit
meinem Heimatort Ruhmannsfelden verbunden war, wurde diese Arbeit fast
zwangsläufig zu einer „musikalischen Heimatkunde“. Insofern ist diese
Arbeit auch eine „Hommage an meine Heimat“.

Weitere Informationen über August Högn erhalten sie auf der
Internetseite August-Hoegn.de.

Die \textit{Josephi-Messe F-Dur op. 64} wird demnächst im Comes-Verlag,
Piding erscheinen.

Ruhmannsfelden im März 2006\ \ Josef Friedrich