\section{Schnelle Integration in Ruhmannsfelden}

August Högn wurde zum 1. Januar 1910 an die Volksschule in
Ruhmannsfelden versetzt. Der Markt Ruhmannsfelden, in dem damals
ungefähr 1500 Menschen lebten, liegt im Bayerischen Wald und ist etwa
in der Mitte des Dreiecks zu finden, das die drei Städte Viechtach,
Regen und Deggendorf bilden. Das Einzugsgebiet der Volksschule umfasste
auch vor dem Ersten Weltkrieg nicht nur die Gemeinde Ruhmannsfelden,
sondern zusätzlich weite Teile der benachbarten Gemeinde Zachenberg
und Randgebiete der Gemeinde Patersdorf, also in etwa das Gebiet der
Pfarrei St. Laurentius. Die Gemeinde Zachenberg, die geringfügig mehr
Einwohner als Ruhmannsfelden zählte, bestand aus 38 überwiegend
landwirtschaftlich geprägten Kleinstortschaften und hatte mit dem Dorf
Zachenberg kein wirkliches Zentrum, denn dort gab es weder ein Rathaus
noch ein Schulhaus. Nicht nur in Schulangelegenheiten war
Ruhmannsfelden damals wie auch heute Zentrum. Ansässige Ärzte und eine
seit 1910 bestehende Apotheke lieferten auch für die Bewohner der
Nachbargemeinden Achslach und Gotteszell einen Grund, nach
Ruhmannsfelden zu kommen. Man kann davon ausgehen, dass Högn
Ruhmannsfelden als einen fortschrittlichen Ort erlebt hat, da dort kurz
vor seiner Ankunft Investitionen getätigt und Reformen durchgeführt
wurden. So besaß Ruhmannsfelden seit dem Schulhausneubau im Jahr 1908
gleich drei Schulhäuser. Das älteste, 1834 erbaute wurde als
Lehrerwohnhaus genutzt. Auch die junge Familie Högn zog hier ein. In
institutioneller Hinsicht hatte sich an der Volksschule Ruhmannsfelden
kurz vor 1910 einiges geändert. Dem Anwachsen der Schülerzahl wurde
Rechnung getragen: statt früher vier unterrichteten nun sieben Lehrer,
also ein Lehrer pro Schülerjahrgang. Den jahrgangsübergreifenden und
nach Geschlecht getrennten Klassen war somit ein Ende gesetzt. Die
einzelnen Klassen umfassten aber immer noch fast 70 Schüler. 1908
hatten die Anträge der Gemeinden Zachenberg und Patersdorf auf
Einführung einer Sommerschule Erfolg. Damit Kinder vor allem zur
Erntezeit in der Landwirtschaft ihrer Eltern mithelfen konnten, wurde
vom 1. Mai bis 1. Oktober ein auf drei Stunden verkürzter Unterricht
eingeführt.

\begin{figure}
\img{Rathaus-von-Ruhmannsfelden}
\caption{Das Rathaus in Ruhmannsfelden zu Högns Zeit. Es wurde vor
einigen Jahren abgerissen. Seitdem ziert ein Bauzaun den Marktplatz von
Ruhmannsfelden.}
\end{figure}

Mit August Högn traten gleich drei neue Lehrer in Ruhmannsfelden ihren
Schuldienst an. In der Schulchronik wird Högn bald als zweiter. Lehrer
und somit Stellvertreter von Schulleiter Alois Auer aufgeführt. Das
hatte neben seiner schon zehn Jahre langen Berufserfahrung
wahrscheinlich auch den Grund, dass Högn die einzige männliche
Lehrkraft neben Auer war, die in den unruhigen Zeiten des Ersten
Weltkrieges an der Schule bleiben konnte. Högns Kriegseinsatz
beschränkte sich auf einen kurzen Heeresdienst in den Jahren 1915/1916.
Am 11. August 1915 wurde er zum \textit{10. Infanterie Regiment
Ersatzbatallion Straubing 4. Ring} einberufen und am 19. Oktober 1916
zur weiteren dienstlichen Verwendung entlassen. Für diesen
Kriegseinsatz verlieh ihm das NS-Regime 1936 – wie wahrscheinlich
vielen Teilnehmern am Ersten Weltkrieg – das \textit{König Ludwig
Kreuz für Heimatverdienste während der Kriegszeit} und das
\textit{Ehrenkreuz für Kriegsteilnehmer}.

Eine weitere Neuerung in Ruhmannsfelden war, dass die Pfarrkirche St.
Laurentius im selben Jahr, als August Högn nach Ruhmannsfelden kam,
eine neue, pneumatische Orgel mit 22 Registern vom Orgelbaumeister
Ludwig Edenhofer aus Deggendorf bekam. Von Anfang an wirkte Högn an der
Kirchenmusik mit. Der versierte Orgelspieler war auch im
Kirchendienst mehr als ein würdiger Stellvertreter für Auer, der
zusammen mit seiner Frau Anna und Tochter Auguste den
Chorregentendienst versah.

Bereits ein halbes Jahr nach seiner Ankunft wurde Högn zum Vorstand
eines Vereins gewählt, was seine schnelle gesellschaftliche Integration
unterstreicht. Der Turnverein Ruhmannsfelden suchte zu dem Zeitpunkt
händeringend jemanden, der bereit war, den unbesetzten Posten des
Vorstands zu übernehmen. Vom 21. Mai 1910 bis 27. Dezember 1913 hatte
Högn dieses Amt inne. Auch wenn er nach dieser Zeit in keiner Funktion
der Vereinsspitze in Erscheinung tritt, blieb er über Jahrzehnte hinweg
vor allem als Leiter der Sänger- und Orchesterriege dem Verein treu.
1924 versuchte man noch einmal Högn als Schriftführer in die
Vereinsspitze einzubinden, doch er lehnte ab, weil er schon bei der
Feuerwehr die Schriftführertätigkeit ausübte.

August Högn war bereits am 19. September 1902 der Wallersdorfer
Feuerwehr beigetreten und wechselte Anfang Januar 1910 in die
Feuerwehr Ruhmannsfelden. Am 26. Dezember 1910 wurde Högn dann zum
Schriftführer der Feuerwehr gewählt. Dass ein Lehrer diesen Posten
übernahm, hatte Tradition. Die Lehrer Raymund Schinagl und Max Weig
waren lange Zeit vor Högn als Schriftführer tätig gewesen. Der
Mitbegründer der Feuerwehr und letzte Schriftführer Joseph Lukas starb
am 30. August 1910. Högn, als Lehrer im Schreiben souverän wie sonst
kein anderer bei der Feuerwehr, bot sich als Nachfolger für Lukas
geradezu an. Die Protokolle vor Högns Tätigkeit zeichnen sich
dementsprechend durch viele Rechtschreibfehler aus. August Högn blieb
40 Jahre lang Schriftführer der Feuerwehr. Neben dem Verfassen von
Protokollen übernahm er die Korrespondenz der Feuerwehr, hielt
zahlreiche Ansprachen und Vorträge und packte auch mal mit eigenen
Händen an, wie zum Beispiel 1933, als bei strömendem Regen die neue
Feuerwehrspritze vom Zug abgeladen werden musste.

Ein weiteres Aufgabenfeld, das die damalige dörfliche Gemeinschaft für
einen Lehrer bereithielt, übernahm Högn 1913: Von da an bis 1920 war er
Schreiber der Gemeinde Zachenberg. Nach dem Erlass der
Gemeindeverordnung wurden diese Arbeiten hauptsächlich den Lehrern
übertragen. So verwundert es nicht, dass die angehenden Lehrer
speziell im Fach Gemeindeschreiberei unterrichtet wurden und dass vor
Högn die Lehrer Milter, Schinagl, Lechner und Hochstraßer für die
Gemeinde Zachenberg tätig waren. Högns Eifer, sich für das Gemeinwohl
zu engagieren, kommt auch dadurch besonders zum Ausdruck, dass er sogar
ein Zimmer seiner Dienstwohnung abtrat und in eine Art Gemeindekanzlei
umwandelte. Vorher wurden die Gemeindearbeiten, die für Lehrer eine
wichtige Nebeneinkunft darstellten, in einem Raum der Brauerei Rankl in
Ruhmannsfelden erledigt.